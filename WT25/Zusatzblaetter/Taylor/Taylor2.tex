\Aufgabe{}{
	Gegeben sei die Funktion 
	$$
	f(x) = e^{-2x}.
	$$
	\begin{abc}
		\item Bestimmen Sie das Taylor-Polynom 3. Ordnung \( T_3(x) \) von \( f(x) \) um den Punkt \( x_0 = 0 \).
		\item Geben Sie das Restglied \( R_3(x;0) \) an. 
		\item Schätzen Sie das Restglied für \( |x| < \frac{1}{2} \) ab.
	\end{abc}
}

\Loesung{
	\begin{abc}
		\item Die Taylor-Entwicklung von \( f(x) \) um \( x_0 = 0 \) ist:
		\[
		T_3(x) = f(0) + f'(0)x + \frac{f''(0)}{2!}x^2 + \frac{f'''(0)}{3!}x^3 + \dots
		\]
		
		Berechnung der Ableitungen:
		
		\begin{align*}
			f(0) &= 1 \\
			f'(x) &= -2e^{-2x} \quad \Rightarrow \quad f'(0) = -2 \\
			f''(x) &= 4e^{-2x} \quad \Rightarrow \quad f''(0) = 4 \\
			f'''(x) &= -8e^{-2x} \quad \Rightarrow \quad f'''(0) = -8
		\end{align*}
		
		Damit ergibt sich das Taylor-Polynom dritter Ordnung:
		
		\[
		T_3(x) = f(0) + f'(0)x + \frac{f''(0)}{2!}x^2 + \frac{f'''(0)}{3!}x^3
		\]
		
		\[
		T_3(x) = 1 - 2x + 2x^2 - \frac{4}{3}x^3.
		\]
		
		\item Das Restglied wird durch die Lagrange-Formel gegeben:
		
		\[
		R_3(x; 0) = \frac{f^{(4)}(\xi)}{4!}x^4,
		\]
		
		
		
		Die vierte Ableitung von \( f(x) \) ist:
		
		\[
		f^{(4)}(x) = 16e^{-2x}.
		\]
		
		Damit folgt für das Restglied:
		
		\[
		R_3(x; 0) = \frac{16e^{-2\xi}}{24}x^4 = \frac{2}{3}e^{-2\xi} x^4.
		\]
		
		\item Für \( |x| < \frac{1}{2} \) setzen wir die obere Schranke für \( e^{-2\xi} \):
		
		Da \( e^{-2\xi} \) auf \( [-\frac{1}{2}, \frac{1}{2}] \) maximal ist für \( \xi = -\frac{1}{2} \), gilt:
		
		\[
		e^{-2\xi} \leq e.
		\]
		
		Damit ergibt sich die Abschätzung:
		
		\[
		|R_3(x; 0)| \leq \frac{2}{3} e \cdot \left(\frac{1}{2}\right)^4.
		\]
		
		\[
		= \frac{2}{3} e \cdot \frac{1}{16} = \frac{2e}{48} = \frac{e}{24}.
		\]
		
		Also:
		
		\[
		|R_3(x; 0)| \leq \frac{e}{24}.
		\]
		
	\end{abc}
}
