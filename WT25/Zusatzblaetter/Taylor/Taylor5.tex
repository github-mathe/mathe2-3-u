\Aufgabe{}{
Gegeben sei die Funktion $$
f(x) = \cos(x^2),
$$
\begin{abc}
\item Bestimmen Sie das Taylor-Polynom 3. Ordnung $T_3(x)$ von $f (x)$ um den
	  Punkt $x_0 = 0$.
\item Geben Sie das Restglied $R_3(x;0)$ an. 
\item Schätzen Sie das Restglied für $|x| < \frac{1}{2}$ ab.
\end{abc}
\textbf{Hinweis:} Es ist nicht notwendig das Restglied optimal abzuschätzen.
}


\Loesung{

	
	\begin{abc}
		\item
	
	Das Taylor-Polynom \( n \)-ter Ordnung einer Funktion \( f(x) \) um den Punkt \( x_0 \) ist gegeben durch:
	
	\[
	T_n(x) = \sum_{k=0}^n \frac{f^{(k)}(x_0)}{k!} (x - x_0)^k
	\]
	
	Für \( n = 3 \) und \( x_0 = 0 \) lautet das Taylor-Polynom:
	
	\[
	T_3(x) = f(0) + f'(0) \cdot x + \frac{f''(0)}{2!} \cdot x^2 + \frac{f'''(0)}{3!} \cdot x^3
	\]
	
	\subsubsection*{Schritt 1: Berechnung der Ableitungen von \( f(x) = \cos(x^2) \)}
	
	\begin{align*}
	1. \quad & f(x) = \cos(x^2) \\
	2. \quad & f'(x) = -\sin(x^2) \cdot 2x \\
	3. \quad & f''(x) = -\cos(x^2) \cdot (2x)^2 - \sin(x^2) \cdot 2 = -4x^2 \cos(x^2) - 2 \sin(x^2) \\
	4. \quad & f'''(x) = 8x^3 \sin(x^2) - 12x \cos(x^2)
	\end{align*}
	
	\subsubsection*{Schritt 2: Auswertung der Ableitungen an der Stelle \( x_0 = 0 \)}
	
	\begin{align*}
	1. \quad & f(0) = \cos(0) = 1 \\
	2. \quad & f'(0) = -\sin(0) \cdot 0 = 0 \\
	3. \quad & f''(0) = -4 \cdot 0^2 \cos(0) - 2 \sin(0) = 0 \\
	4. \quad & f'''(0) = 8 \cdot 0^3 \sin(0) - 12 \cdot 0 \cos(0) = 0
	\end{align*}
	
	\subsubsection*{Schritt 3: Einsetzen in das Taylor-Polynom}
	
	\[
	T_3(x) = 1 + 0 \cdot x + \frac{0}{2!} \cdot x^2 + \frac{0}{3!} \cdot x^3 = 1
	\]
	
	Das Taylor-Polynom 3. Ordnung von \( f(x) = \cos(x^2) \) um \( x_0 = 0 \) lautet also:
	
	\[
	T_3(x) = 1
	\]
	
		
	\item Abschätzung des Restglieds für \( |x| < \frac{1}{2} \)
		
		Wir schätzen das Restglied \( R_3(x;0) \) für \( |x| < \frac{1}{2} \) ab. Dazu betrachten wir den Betrag des Restglieds:
		
		\[
		|R_3(x;0)| = \left| \frac{f^{(4)}(\xi)}{24} x^4 \right|
		\]
		
		Da \( |x| < \frac{1}{2} \), gilt \( x^4 < \left( \frac{1}{2} \right)^4 = \frac{1}{16} \). Um \( |f^{(4)}(\xi)| \) abzuschätzen, betrachten wir den maximalen Wert der 4. Ableitung im Intervall \( |\xi| < \frac{1}{2} \).
		
		\subsubsection*{Schritt 1: Abschätzung von \( |f^{(4)}(\xi)| \)}
		
		Für \( |\xi| < \frac{1}{2} \):
		
		\[
		|f^{(4)}(\xi)| = \left| 16\xi^4 \cos(\xi^2) - 48\xi^2 \sin(\xi^2) - 12 \cos(\xi^2) \right|
		\]
		
		Da \( |\cos(\xi^2)| \leq 1 \) und \( |\sin(\xi^2)| \leq 1 \), gilt:
		
		\[
		|f^{(4)}(\xi)| \leq 16 \cdot \left( \frac{1}{2} \right)^4 \cdot 1 + 48 \cdot \left( \frac{1}{2} \right)^2 \cdot 1 + 12 \cdot 1 = 16 \cdot \frac{1}{16} + 48 \cdot \frac{1}{4} + 12 = 1 + 12 + 12 = 25
		\]
		
		\subsubsection*{Schritt 2: Abschätzung des Restglieds}
		
		\[
		|R_3(x;0)| \leq \frac{25}{24} \cdot \frac{1}{16} = \frac{25}{384} \approx 0,0651
		\]
		
		Für \( |x| < \frac{1}{2} \) gilt also:
		
		\[
		|R_3(x;0)| \leq \frac{25}{384}
		\]
		
		
	\end{abc}
	


}
