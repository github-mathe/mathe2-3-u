\Aufgabe{}{
Gegeben sei die Funktion $$
f(x) = \frac{1}{\sqrt{x^2+1}},
$$
\begin{abc}
\item Bestimmen Sie das Taylor-Polynom 2.Ordnung $T_21(x)$ von $f (x)$ um den
	  Punkt $x_0 = 1$.
\item Geben Sie das Restglied $R_2(x;1)$ an.
\item Schätzen Sie das Restglied für $\frac{1}{2} < x < \frac{3}{2}$ ab.
\end{abc}
\textbf{Hinweis:} Es ist nicht notwendig das Restglied optimal abzuschätzen.
}


\Loesung{
\begin{abc}
\item Die Taylor-Entwicklung von \( f(x) \) um \( x_0 = 1\) ist:
\[
f(x) = f(1) + f'(1)(x-1) + \frac{f''(1)}{2!}(x-1)^2 \cdots
\]

\begin{align*}
f(1)  &= \frac{1}{\sqrt{2}} \\
f'(x) &= -\frac{x}{\sqrt{(x^2+1)^3}}  \quad \Rightarrow \quad f'(1) = -\frac{1}{2\sqrt{2}}\\
f''(x) &= \frac{2x^2-1}{\sqrt{(x^2+1)^5}}  \quad \Rightarrow \quad f''(1) = \frac{1}{4\sqrt{2}}
\end{align*}


Damit ergibt sich das Taylor-Polynom zu
\[
T_2(x) = f(1) + f'(1)(x-1) + \frac{f''(1)}{2!}(x-1)^2 \\
= \frac{1}{\sqrt{2}}-\frac{1}{2\sqrt{2}}(x-1)+\frac{1}{8\sqrt{2}}(x-1)^2.
\]

\item Das Restglied wird wie folgt bestimmt:
\[
R_2(x; 1) = \frac{f'''(\xi)}{3!}(x-1)^3,
\]
wobei $\frac{1}{2}\leq \xi \leq \frac{3}{2}$ gilt.

\[
f'''(x) = -\frac{6x^3-9x}{\sqrt{(x^2+1)^7}}.
\]

Das Restglied wird damit 
\[
R_2(x; 1) = \frac{f'''(\xi)}{3!}(x-1)^3
=-\frac{2\xi^3-3\xi}{2\sqrt{(\xi^2+1)^7}}(x-1)^3.
\]


\item F\"ur \( \frac{1}{2} <x  < \frac{3}{2} \) ergibt sich \( \frac{1}{2} \leq \xi \leq \frac{3}{2} \) :
\begin{align*}
|R_2(x; 1)|& = \left|-\frac{2\xi^3-3\xi}{2\sqrt{(\xi^2+1)^7}}(x-1)^3\right|\\
&=\left|-\frac{\xi(2\xi^2-3)}{2\sqrt{(\xi^2+1)^7}}(x-1)^3\right|\\
&\leq \left|-\frac{\frac{3}{2}(2\frac{3}{2}^2-3)}{2\sqrt{(\frac{3}{2}^2+1)^7}}(\frac{3}{2}-1)^3\right|\\
&= \frac{144}{\sqrt{13^7}}\frac{1}{8}\\
&= \frac{18}{\sqrt{13^7}}\\
&\leq \frac{18}{\sqrt{9^7}}\\
&= \frac{2}{3^5} \approx 0.008
\end{align*}

D.h.
\[
|R_2(x; 1)| \leq \frac{2}{3^5} .
\]

	  
\end{abc}

}
