\Aufgabe{}{
Gegeben sei die Funktion $$
f(x) = \frac{1}{(x+1)^2},
$$
\begin{abc}
\item Bestimmen Sie das Taylor-Polynom 3. Ordnung $T_3(x)$ von $f (x)$ um den
	  Punkt $x_0 = 0$.
\item Geben Sie das Restglied $R_3(x;0)$ an. 
\item Schätzen Sie das Restglied für $|x| < \frac{1}{2}$ ab.
\end{abc}
\textbf{Hinweis:} Es ist nicht notwendig das Restglied optimal abzuschätzen.
}


\Loesung{
\begin{abc}
\item Die Taylor-Entwicklung von \( f(x) \) um \( x_0 = 0 \) ist:
\[
f(x) = f(0) + f'(0)x + \frac{f''(0)}{2!}x^2 + \frac{f'''(0)}{3!}x^3 \cdots
\]

\begin{align*}
f(0)  &= 1 \\
f'(x) &= -\frac{2}{(x+1)^3} \\
f'(0) &= -2 \\
f''(x) &= \frac{6}{(x+1)^4} \\
f''(0) &= 6 \\
f'''(x) &= -\frac{24}{(x+1)^5} \\
f'''(0) &= -24
\end{align*}


Damit ergibt sich das Taylor-Polynom zu
\[
T_3(x) = f(0) + f'(0)x + \frac{f''(0)}{2!}x^2 +  \frac{f'''(0)}{3!}x^3 \\
= 1-2x+3x^2-4x^3.
\]

\item Das Restglied wird wie folgt bestimmt:
\[
R_3(x; 0) = \frac{f^{(4)}(\xi)}{4!}x^4,
\]
wobei $-\frac{1}{2}\leq \xi \leq \frac{1}{2}$ gilt.

\[
f^{(4)}(x) = \frac{120}{(x+1)^6}.
\]

Das Restglied wird damit 
\[
R_3(x; 0) = \frac{f^{(4)}(\xi)}{4!}x^4 = \frac{120}{24(\xi+1)^6}x^4.
\]

\item For \( |x| < \frac{1}{2} \):
\[
|R_3(x; 0)| = \left|\frac{120}{24(\xi+1)^6}x^4\right| = \frac{5}{(\xi+1)^6}x^4
\leq \frac{5}{(\frac{1}{2})^6} \left( \frac{1}{2}\right) ^4 = \frac{5 \cdot 64}{16} = 20.
\]

D.h.
\[
|R_3(x; 0)| \leq 20.
\]

	  
\end{abc}

}
