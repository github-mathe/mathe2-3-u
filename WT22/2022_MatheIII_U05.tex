
\documentclass[a4paper,12pt]{article}
\usepackage[ngerman]{babel}
% \usepackage[ngerman]{babel}
\usepackage[utf8]{inputenc}
\usepackage[T1]{fontenc}
\usepackage{lmodern}
\usepackage{titling}
\usepackage{geometry}
\geometry{left=3cm, right=3cm, top=2cm, bottom=3cm}
\usepackage{setspace}
\onehalfspacing

\usepackage{ifthen}

\usepackage{graphicx}
% \usepackage{pstricks}
% \usepackage{relsize}
% \usepackage[decimalsymbol=comma,exponent-product = \cdot, per=frac]{siunitx}
% \sisetup{range-phrase=\,bis\,}

\usepackage{xargs}
\usepackage{calc}
\usepackage{amsmath}
\usepackage{amsfonts}
\usepackage{mathtools}
\usepackage{amssymb}

\usepackage{cancel}
\usepackage{trfsigns}
\usepackage{array}
\usepackage{enumerate}
\usepackage{enumitem}

\usepackage{caption}
\usepackage{subcaption}

\usepackage{multicol}

\usepackage{pdflscape}
\usepackage[table]{xcolor}

\usepackage{float}

%%%%%%%%%%%%%%%%%%%%%%%%%%%%%%%%%%%%%%%%%%%%%%%%%%%%%%%%%%%%%%%%%%%%%%%%%%%%%%%%
\newboolean{WITHPSTRICKS}
\setboolean{WITHPSTRICKS}{false}


\newcommand{\PROFESSOR}{Prof.\ Dr.\ Thomas Carraro}
\newcommand{\ASSISTANT}{\setlength{\tabcolsep}{0pt}\begin{tabular}{l}Dr.\ Frank Gimbel\\Janna Puderbach\end{tabular}}

\newcommand{\Jahr}{2022}
% \newcommand{\Trimester}{HT}
\newcommand{\Trimester}{FT}
\newcommand{\Kurs}{Mathematik III}
\newcommand{\TYPE}{Aufgabenblatt}
\newcommand{\BLATT}{5}
\newcommand{\TOPIC}{Integration, Differentialgleichungen}

%%%%%%%%%%%%%%%%%%%%%%%%%%%%%%%%%%%%%%%%%%%%%%%%%%%%%%%%%%%%%%%%%%%%%%%%%%%%%%%%
\newboolean{mitLoes}
\setboolean{mitLoes}{false}
%\setboolean{mitLoes}{true}

\newboolean{withHints}
\setboolean{withHints}{false}
\setboolean{withHints}{true}

%%%%%%%%%%%%%%%%%%%%%%%%%%%%%%%%%%%%%%%%%%%%%%%%%%%%%%%%%%%%%%%%%%%%%%%%%%%%%%%%

%\setboolean{WITHPSTRICKS}{false}
\setboolean{WITHPSTRICKS}{true}


\usepackage{tikz}
\usetikzlibrary{arrows,automata,backgrounds,calendar,decorations.pathmorphing,fadings,shadings,calc,intersections}
\usetikzlibrary{decorations.pathreplacing}
\usetikzlibrary{decorations.shapes}
\usetikzlibrary{decorations.footprints}
\usetikzlibrary{decorations.text}
\usetikzlibrary{positioning}
\usetikzlibrary{through}

\ifthenelse{\boolean{WITHPSTRICKS}}{%
\usepackage{auto-pst-pdf}
\usepackage{pstricks,pst-plot,pst-text}
}{}

\usepackage{pgfplots}

%%%%%%%%%%%%%%%%%%%%%%%%%%%%%%%%%%%%%%%%%%%%%%%%%%%%%%%%%%%%%%%%%%%%%%%%%%%%%%%%
\usepackage{mbdefAufgaben}

%%%%%%%%%%%%%%%%%%%%%%%%%%%%%%%%%%%%%%%%%%%%%%%%%%%%%%%%%%%%%%%%%%%%%%%%%%%%%%%%
\newboolean{mitErg}
\setboolean{mitErg}{false}

%%%%%%%%%%%%%%%%%%%%%%%%%%%%%%%%%%%%%%%%%%%%%%%%%%%%%%%%%%%%%%%%%%%%%%%%%%%%%%%%
\newcounter{Aufg}
\setcounter{Aufg}{0}
\newcounter{Blatt}
\setcounter{Blatt}{\BLATT}

%%%%%%%%%%%%%%%%%%%%%%%%%%%%%%%%%%%%%%%%%%%%%%%%%%%%%%%%%%%%%%%%%%%%%%%%%%%%%%%%
\usepackage{KopfEnglish}

% Seitenraender
\textwidth = 285mm
\textheight = 180mm
\leftmargin 5mm
\oddsidemargin = -20mm
\evensidemargin = -20mm
\topmargin = -25mm
\parindent 0cm
\columnsep 2cm

% % % Aufgabenstellung
% % % Schwierungkeitsgrad mit "e" , "f" oder "v" angeben
% % % "e" Einführung   
% % % "f" Festigung
% % % "v" Vertiefung  

\newcommand{\Aufgabe}[3][]{
\stepcounter{Aufg}
\subsubsection*{Aufgabe 
\arabic{Blatt}.\arabic{Aufg}\ifthenelse{\equal{#1}{e}}{}{\ifthenelse{\equal{#1}{f}}{
$\!\!{}^\star$}{\ifthenelse{\equal{#1}{v}}{$^{\star\star}$}{}}}{: #2}}
{#3}
}
% % % Ergebnisse jeweils am Ende des Aufgabenblattes Anzeigen
\newcommand{\Ergebnisse}{}
\makeatletter
\newcommand{\Ergebnis}[1]{
	\g@addto@macro{\Ergebnisse}{#1}
}
\makeatother
\makeatletter
\newcommand{\ErgebnisC}[2]{
\@ifundefined{c@#1}
{\newcounter{#1}}
{}
\setcounter{#1}{\theAufg}

\ifthenelse{\boolean{mitErg}}{	\g@addto@macro{\Ergebnisse}{\subsubsection*{Ergebnisse zu Aufgabe \arabic{Blatt}.\arabic{#1}:}
}%
	\g@addto@macro{\Ergebnisse}{#2}}{}
}
\makeatother


% % % Lösungen
\newcommand{\Loesung}[1]{
	\ifthenelse{\boolean{mitLoes}}
	{\subsubsection*{Lösung \arabic{Blatt}.\arabic{Aufg}:}
		#1}
	{}
}
% % % % % % % % % % % % % % % % % % % % % % % % % % % % % % % % % % % % % % % % % % % % % % % % % % % % % %
% % % % % % % % % % % % % % % % % % % % % % % % % % % % % % % % % % % % % % % % % % % % % % % % % % % % % %
% % % % % % % % % % % % % % % % % % % % % % % % % % % % % % % % % % % % % % % % % % % % % % % % % % % % % %
\begin{document}
\begin{twocolumn}
% % % % % % % % % % % % % % % % % % % % % % % % % % %

%%%%%%%%%%%%%%%%%%%%%%%%%%%%%%%%%%%%%%%%%%%%%%%%%%%%%%%%%%%%%%%%%%%%%%%%%%%%%%%%
% Set the TITLE of the sheet here:
%\uebheader{\Kurs}{\arabic{Blatt}}{\Trimester\,\Jahr}{\TOPIC}
%\uebheader{\Kurs}{\arabic{Blatt}}{\Trimester\,\Jahr}{\TOPIC}
\uebheader{\Kurs}{\arabic{Blatt}}{\Trimester\,\Jahr}{\TOPIC}
\ruleBig

\setboolean{mitErg}{false}
\setboolean{mitErg}{true}

% % % Hinweise
\newcommand{\Hints}[1]
{
\ifthenelse{\boolean{withHints}}
{\definecolor{gruen}{rgb}{0.0,.5,.2}
\color{gruen}
\subsubsection*{\textcolor{gruen}{Zus\"atzliche Hinweise zu Aufgabe \arabic{Blatt}.\arabic{Aufg}:}}
		#1

\ruleBig
\color{black}
}
	{}
}


%%%%%%%%%%%%%%%%%%%%%%%%%%%%%%%%%%%%%%%%%%%%%%%%%%%%%%%%%%%%%%%%%%%%%%%%%%%%%%%%
% Set the INTRODUCTION section of the sheet here:
% \input{introduction.tex}
\renewcommand{\d }{\mathrm{d}}

\textbf{Einführende Bemerkungen}

\begin{itemize}
\item Vermeiden Sie die Verwendung von Taschenrechnern oder Online-Ressourcen.
\item Die mit einem Stern *) markierten (Teil-)Aufgaben entfallen in diesem Trimester. Stattdessen werden einzelne Online-Aufgaben im ILIAS-Kurs kenntlich gemacht, zu denen Sie dort Ihre L\"osungswege zur Korrektur hochladen k\"onnen. 
\item Die mit zwei Sternen  **) markierten (Teil-)Aufgaben richten sich an Studierende, die die \"ubrigen Aufgaben bereits gel\"ost haben und die Inhalte weiter vertiefen m\"ochten. 
\end{itemize}

\ruleBig

%Mathe III Blatt 5






\Aufgabe[e]{Inhomogene lineare Differentialgleichungen}
{
Bestimmen Sie von folgenden inhomogenen linearen Differentialgleichungen mit konstanten Koeffizienten jeweils die allgemeine reelle L\"osung, indem Sie zun\"achst die zugeh\"orige homogene lineare Differentialgleichung allgemein l\"osen und eine spezielle (partikul\"are) L\"osung der inhomogenen linearen Differentialgleichung mit Hilfe von geeigneten Ans\"atzen bestimmen.


\begin{abc}
\item[e] \ \ \ \ $y''(x)-5\,y'(x)+6\,y(x)=r_{k}(x)$ \ \ mit 
\[
\mathbf{i)}\;\;r_{1}=108\,x^{2}\;,\;\;\;\;\;\mathbf{ii)}\;\;r_{2}=7\,\EH{3x}\;,\;\;\;\;\;\mathbf{iii)}\;\;r_{3}=18+14\,\EH{3x}\;. 
\]

\item[e]  \ \ \ \ $y'''(x)+25\,y'(x)=s_{k}(x)$ \ \ mit 
\begin{align*}
\mathbf{i)} &  & s_{1}=&150\,x\;, & \;\;\;  
\mathbf{ii)} &  & s_{2}=&\sin(x)\;, \\ 
\mathbf{iii)} &  & s_{3}=&\sin (5x)-200\,x\;, &  
\mathbf{iv)} &  & 
s_{4}=&6\,\sin (3x)\,\cos (2x)\;.
\end{align*}
\textbf{Hinweis}: F\"ur $\alpha,\beta\in\R$ gilt $\sin(\alpha+\beta)+\sin(\alpha-\beta)=2\sin\alpha\cos\beta$. 
\item[e]  \ \ \ \ $y''(x)-2\,y'(x)=t_{k}(x)$ \ \ mit 
\[
\mathbf{i)}\;\;t_{1}=4\,\EH{2x}\;,\;\;\;\;\;\mathbf{ii)}\;\;t_{2}=\cosh (2x)\;. 
\]
%\item[e]  Geben Sie einfache Inhomogenit\"aten (St\"orfunktionen) an, f\"ur die es keine Faustregelans\"atze gibt.
\end{abc}
}
\Loesung{
\begin{enumerate}
\item[\textbf{a)}]  Zun\"achst die zugeh\"orige homogene lineare Differentialgleichung:

Die charkt. Gl. ist:\ \ \ \ $\lambda ^{2}-5\lambda +6=0 \,\Rightarrow\,  \lambda_{1}=2\;,\;\;\lambda_{2}=3\;.$%
\[
\Rightarrow \;\;\;\;\underline{\;y_{\text{h}}(x)=a\,\text{e}^{2x}+b\,\EH{3x}\;,\;\;a,b\in \Bbb{R}\;}\;. 
\]

\textbf{i)} \ Faustregelansatz: 
\[
y_{\text{p}}=A+B\,x+C\,x^{2} \,\Rightarrow\,  ;y_{\text{p}}'=B+2C\,x \text{ und } y_{\text{p}}''=2C\;. 
\]

Einsetzen in die Differentialgleichung: 
\[
2C-5\cdot \left( B+2C\,x\right) +6\cdot \left( A+B\,x+C\,x^{2}\right)=108\,x^{2}\;. 
\]

Koeffizientenvergleich: 
\[
\begin{array}{rrrrrrrrrrr}
1: &  & 2\,C -  5\,B  +  6\,A  =  0 &  &  \\ 
x: &  & -10\,C  +  6\,B   =  0 &  &  \\ 
x^{2}: &   &6\,C =  108 &  & \,\Rightarrow\,  C=18\;,\;\;B=30\;\;A=19\;.
\end{array}
\]

Partikul\"are L\"osung der inhomogen linearen Differentialgleichung: 
\[
\Rightarrow \;\;\;\;\underline{\;y_{\text{p}}(x)=19+30\,x+18\,x^{2}\;}\;. 
\]

Allgemeine L\"osung der inhomogen linearen Differentialgleichung: 
\[
\,\Rightarrow\,  \underline{\underline{\;y(x)=y_{\text{h}}+y_{\text{p}}=a\,\EH{2x}+b\,\EH{3x}+19+30\,x+18\,x^{2}\;,\;\;a,b\in \Bbb{R}\;}}\;. 
\]


\textbf{ii)}\ \ Faustregelansatz:\ \ \ $y_{\text{p}}=A\,x\,\EH{3x}$\ \ \ \glqq$x$--spendieren\grqq.

Eingesetzt: \ \ $$\big((6A+9A\,x)-5\cdot (A+3A\,x)+6\cdot A\,x\big)\cdot \,\EH{3x}=7\,\EH{3x} \,\Rightarrow\,  A=7 \text{ und } 0=0\ .$$

Partikul\"are L\"osung: \ \ $\underline{\;y_{\text{p}}(x)=7x\,\EH{3x}\;}\;.$

Allgemeine L\"osung: \ \ $\underline{\underline{\;y(x)=y_{\text{h}}+y_{\text{p}} = a\,\EH{2x}+b\,\EH{3x}+7x\,\EH{3x}\;,\;\;a,b\in \mathbb{R}\;}}\;.$


\textbf{iii)}\ \ Faustregelans\"atze f\"ur beide Summanden der Inhomogenit\"at einzeln.

Ansatz f\"ur \ $r=18$\ :\ \ \ \ $y_{\text{p}_{1}}=A \,\Rightarrow\,  \underline{\;y_{\text{p}_{1}}(x)=3\;}\;.$

F\"ur\ \ $r=14\,\EH{3x}$\ \ ergibt sich nach ii) : \ \ \ $\underline{\;y_{\text{p}_{2}}(x)=14x\,\EH{3x}\;}\;.$

Allgemeine L\"osung: \ \ \ $\underline{\underline{\;y(x)=y_{\text{h}}+y_{\text{p}_{1}}+y_{\text{p}_{2}} = a\,\EH{2x}+b\,\EH{3x}+3+14x\,\EH{3x}\;,\;\;a,b\in \mathbb{R}\;}}\;.$


\item[\textbf{b)}]  Das charakteristische Polynom der Differentialgleichung ist $\lambda^3+25\lambda$ mit den Nullstellen $\lambda_1=0$, $\lambda_{2/3}=\pm 5\imag$. Die zugeh\"orige homogene lineare Differentialgleichung hat damit die allgemeine (reelle) L\"osung 
\[
\underline{\;y_{\text{h}}(x)=a\,\cos (5x)+b\,\sin (5x)+c\;,\;\;\;a,b,c\in \mathbb{R}\;}\;. 
\]

\textbf{i)} \ Faustregelansatz:\ \ \ $y_{\text{p}}=A\,x+B\,x^{2}$ \ (\glqq$x$--spendieren\grqq). Einsetzen in die Differentialgleichung ergibt
\begin{align*}
&&&y'''(x)+25y'(x)\\
&&=& 0 + 25(A+2Bx) \overset!= 150 x\\
\Rightarrow && A=&0,\, B=3\\
\Rightarrow&& y_{\text{p}}=&3\,x^{2}\\
\Rightarrow&& y(x)=& y_{\text{h}}+y_{\text{p}} = 
a\,\cos(5x)+b\,\sin(5x)+c+3\,x^{2}\;,\;\;\;a,b,c\in \mathbb{R}
\end{align*}

\textbf{ii)} 
% 
% \textbf{1. L\"osungsweg} 
% 
% \ Faustregelansatz:\ \ \ $y_{\text{p}}(x)=\operatorname{Im}(A\cdot\EH{ix}) $
% 
% Einsetzen in die Differentialgleichung: 
% \[A\cdot(i^3+25i)\cdot\EH{ix}=\EH{ix}\Rightarrow\, A=\dfrac{1}{-i+25i}=\dfrac{1}{24i}=\dfrac{-i}{24}\]
% 
% Somit gilt 
% \[y_{\text{p}}=\operatorname{Im}\left(\dfrac{-i}{24}\cdot(\cos(x)+i\sin(x))\right)=-\dfrac{1}{24}\,\cos (x)\]
% 
% \textbf{2. L\"osungsweg} 

\ Faustregelansatz:\ \ \ $y_{\text{p}}=A\,\cos(x)+B\,\sin(x) \,\Rightarrow\,  \underline{\;y_{\text{p}}(x)=-\dfrac{1}{24}\,\cos (x)\;}\;.$
Einsetzen in die Differentialgleichung ergibt
\begin{align*}
&&&y'''(x)+25y'(x)\\
&&=& A\sin(x) -B\cos(x) + 25(-A\sin(x)+B\cos(x)) \overset!= \sin(x)\\
\Rightarrow && -24A=&1,\, 24B=0\\
\Rightarrow&& y_{\text{p}}=&-\frac 1{24}\cos(x)
\end{align*}

Beide Wege liefern dann die allgemeine L\"osung

\[
\Rightarrow \;\;\;\;\underline{\underline{\;y(x)=y_{\text{h}}+y_{\text{p}} = 
a\,\cos(5x)+b\,\sin(5x)+c-\dfrac{1}{24}\,\cos (x)\;,\;\;\;a,b,c\in \mathbb{R}\;}}\;. 
\]

\textbf{iii)} \ Faustregelans\"atze f\"ur beide Summanden einzeln:

Ansatz f\"ur \ $s=\sin(5x)=\operatorname{Im}(\EH{i5x})$ :
\ \ $y_{\text{p}_{1}}=\operatorname{Im}(Ax\EH{i5x})$\ (\glqq$x$--spendieren\grqq) liefert nach Einsetzen in die DGL

\[A\cdot(3\cdot(5i)^2\cdot\EH{i5x}+x\cdot(5i)^3\cdot\EH{i5x}+25\cdot(\EH{i5x}+x\cdot5i\cdot\EH{i5x}))=\EH{i5x}\]
\[A\cdot(-75-125ix+25+125ix)=1\Rightarrow\ A=\dfrac{1}{-50}\]
\[\Rightarrow\ y_{\text{p}_{1}}=\operatorname{Im}\left(\dfrac{1}{-50}x\EH{i5x}\right)=-\dfrac{1}{50}\,x\,\sin(5x)\]

Alternativ kann mann den Ansatz \ \ $Ax\,\cos(5x)+Bx\,\sin (5x)$ \ benutzen.
Einsetzen in die Differentialgleichung ergibt
\begin{align*}
&&&y'''(x)+25y'(x)\\
&&=& -3\cdot 25A\cos(5x)+125Ax\sin(5x) -3\cdot 25 B\sin(5x) -125 B x\cos(5x)+ \\
&& & + 25(A\cos(5x)-5Ax\sin(5x)+ B\sin(5x)+5Bx\cos(5x)) \overset!= \sin(5x)\\
\Rightarrow &&\sin(5x)=& (-75A+25 A) \cos(5x) + (-75B+25B)\sin(5x)\\
\Rightarrow && A=&0,\, B=-\frac 1{50}\\
\Rightarrow&& y_{\text{p1}}=&-\frac 1{50}\sin(x)
\end{align*}

F\"ur \ $s=-200$ \ ergibt sich die spezielle L\"osung nach i) zu \ \ $\underline{\;y_{\text{p}_{2}}=-4\,x^{2}\;}\;.$
\[
\underline{\underline{\;y(x)=y_{\text{h}}+y_{\text{p}_{1}}+y_{\text{p}_{2}} = a\,\cos(5x)+b\,\sin(5x)+c-\dfrac{1}{50}\,x\,\sin(5x)-4\,x^{2}\;,\;\;\;a,b,c\in \mathbb{R}\;}}\;. 
\]

\textbf{iv)} \ Die Inhomogenit\"at ist \ \ $s_{4}=6\,\sin (3x)\,\cos(2x)=3\,\big(\sin (x)+\sin (5x)\big)\;.$

Nach ii) und iii) ist damit die spezielle L\"osung:\ \ \ \ $\underline{\;y_{\text{p}} = -\dfrac{1}{8}\,\cos (x)-\dfrac{3}{50}\,x\,\sin(5x)\;}\;.$%
\[
\underline{\underline{\;y(x)=y_{\text{h}}+y_{\text{p}}=a\,\cos (5x)+b\,\sin (5x)+c-\dfrac{1}{8}\,\cos (x)-\dfrac{3}{50}\,x\,\sin(5x)\;,\;\;\;a,b,c\in \Bbb{R}\;}}\;. 
\]

\item[\textbf{c)}]  Das charakteristische Polynom $\lambda^2-2\lambda$ hat die Nullstellen $\lambda_1=0,\, \lambda_2=2$. Damit ist die allgemeine L\"osung der zugeh\"origen homogenen linearen Differentialgleichung 
\[
\underline{\;y_{\text{h}}(x)=a\,+b\,\EH{2x}\;,\;\;\;a,b\in \mathbb{R}\;}\ . 
\]


\textbf{i)} \ Faustregelansatz \ \ $y_{\text{p}}=A\,x\;\EH{2x}$ \ (\glqq$x$--spendieren\grqq)$ \,\Rightarrow\,  \underline{\;y_{\text{p}} = 2\,x\,\EH{2x}\;}\;.$%
\[
\Rightarrow \;\;\;\;\underline{\underline{\;y(x)=y_{\text{h}}+y_{\text{p}} = 
a\,+b\,\EH{2x}+2\,x\,\EH{2x}\;,\;\;\;a,b\in \mathbb{R}\,.}} 
\]

\textbf{ii)}\ \ Die Inhomogenit\"at ist \ \ $t_{2}= \cosh(2x)=\dfrac{1}{2}\,\EH{2x}+\dfrac{1}{2}\,\EH{-2x}\;.$

Faustregelans\"atze f\"ur beide Summanden einzeln:

F\"ur \ $t=\dfrac{1}{2}\,\EH{2x}$\ \ ergibt sich nach i) \ \underline{\;$y_{\text{p}_{1}}(x)=\dfrac{1}{4}\,x\,\EH{2x}\;$}$\;.$

Ansatz f\"ur \ $t=\dfrac{1}{2}\,\EH{-2x}$ : \ \ $y=A\,\EH{-2x}\;\;$(\textbf{kein} \glqq$x$--spendieren\grqq\,!)$ \,\Rightarrow\,  \underline{\;y_{\text{p}_{2}}=\dfrac{1}{16}\,\EH{-2x}\;}\;.$%
\[
\Rightarrow \;\;\;\;\underline{\underline{\;y(x) = y_{\text{h}}+y_{\text{p}_{1}}+y_{\text{p}_{2}} = a\,+b\,\EH{2x}+\dfrac{1}{4}\,x\,\EH{2x}+\dfrac{1}{16}\,\EH{-2x}\;,\;\;\;a,b\in \mathbb{R}\;}} 
\]


%\item[\textbf{d)}]  Keine Faustregelans\"atze gibt es z.\,B. f\"ur Terme wie 
%\[
%\ln (x)\;,\;\;\;\sqrt{x}\;,\;\;\;\dfrac{1}{x}\;,\;\;\;\sin(x^{2})\;,\;\;\;\tan (x)\;. 
%\]
\end{enumerate}
}


\ErgebnisC{AufggewdglIhomLine001}
{
L\"osungen der homogenen Gleichungen: 
a) $a\EH{2x}+b\EH{3x}$\,,  
b) $a\cos(5x)+b\sin(5x)+c$\,,\\ 
c) $a+b\EH{2x}$\,.
}
 %1
\Aufgabe[e]{LR-Kreis}
{
Ein Stromkreis habe einen Widerstand von \ $R=0.8$ Ohm \ und eine Selbstinduktion von \ $L=4$ Henry\,. Bis zur Zeit \ $t_0=0$ \ fließe kein Strom. Dann wird eine Spannung von \ $U=5$ Volt \ angelegt. Nach $5$ Sekunden wird
die Spannung abgeschaltet. Berechnen Sie den Stromverlauf \
$I(t)$ \ für \ $0 \le t \le 5$ \ und \ $t > 5$. \\ 
\textbf{Hinweis}: In diesem Stromkreis gilt $L\dot I(t) + RI(t)=U(t)$
}

\Loesung{
Es gilt gilt die Differentialgleichung
\[
  L\dot{I}(t)+RI(t)=U(t)
\]

Für \ $0\leq t\leq 5$ \ gilt \ $U(t)=5$\,. Die Trennung der Variablen führt zu
\[
  \int \frac{dI}{U-R I}=\int \frac{1}{L} \,dt,\Rightarrow
   -\frac{1}{R}\ln | U-RI(t)| =\frac{1}{L}t + c_1\,, \;\;    c_1 \in \R\,. 
\]
Auflösen nach $I(t)$ liefert (mit $c_2=\EH{c_1}$)
\[
  U-RI(t)=c_2 e^{-Rt/L}
\] 
und somit
$$
   I(t)=\frac{1}{R} \big( U-c_2 e^{-Rt/L} \big)\,.
$$

Einsetzen der Anfangsbedingung \ $I(0)=0$ \ ergibt \ $c_2=U$ \ und
$$
  I(t)=\frac{U}{R}\big(1-e^{-R/Lt}\big)\,.
$$
Einsetzen der gegebenen Zahlenwerte ergibt die Lösung
$$ 
  I(t) = 6.25\, \big(1-e^{-0.2 t}\big) \text{ f\"ur } 0 < t < 5\,. 
$$


Im Zeitraum \ $t>5$ \ ist $U(t)=0$ \ und der Anfangsstrom ist
$$
I(5)=I_0=6.25\,\big(1-e^{-1}\big)\ .
$$
Die Lösung der Differentialgleichung ist
$$
  \int \frac{dI}{I}=-\int\limits \frac{R}{L}dt\Rightarrow  \ln |I(t)|=-\frac{R}{L}t+c_3
$$ 
und damit
$$
  I(t)= c_4 e^{-Rt/L}\ .$$
Aus \ $I(t_0)=I_0$ \ folgt 
$$
 I(t) = I_0 e^{-R(t-t_0)/L}\ .
$$
Einsetzen der Zahlenwerte ergibt die Lösung 
$$ 
I(t) = 6.25 \big(1-e^{-1}\big) e^{-0.2 (t-5)} \text{ f\"ur } t > 5\ . 
$$
}


\ErgebnisC{gewdglStrmLrkr001}{
$I(t)=\left\{\begin{array}{ll}
6.25\, \big(1-e^{-0.2 t}\big)& \text{ f\"ur } 0 < t < 5\\
6.25 \big(1-e^{-1}\big) e^{-0.2 (t-5)}& \text{ f\"ur } t > 5
\end{array}\right.$
}
 %2

%\Aufgabe[e]{Transformationsformel f\"ur Volumenintegrale} {

Berechnen Sie 
$$
\int_{-\infty}^{+\infty}\int_{-\infty}^{+\infty}\int_{-\infty}^{+\infty}\frac{\d x \d y  \d  z}{(x^2+y^2+z^2+1)^2} \text{.}
$$
\textit{Hinweis:} Integrieren Sie \"uber eine Kugel $K$ mit Radius $R$ und lassen Sie dann $R$ gegen unendlich wachsen.

}


\Loesung{

Die Integration \"uber eine Kugel legt die Einf\"uhrung von Kugelkoordinaten nahe:
$$
x= r \sin \vartheta \cos \varphi\,, \quad y = r \sin \vartheta \sin \varphi\,, \quad z = r \cos \vartheta
$$
mit $0\leq r\leq R$, $0\leq \varphi \leq 2\pi$ und $0\leq \vartheta \leq \pi$. Es gilt (vgl.\ Vorlesung oder einfaches Nachrechnen):
$$
\left|\det\left(\frac{\partial(x,y,z)}{\partial(r,\vartheta,\varphi)}\right)\right| = r^2 \sin \vartheta\,.
$$
Damit folgt
\begin{align*}
& \int_0^R \left(\int_0^{\pi}\left(\int_0^{2\pi}\frac{1}{(r^2+1)^2}r^2\sin \vartheta \d \varphi\right)\d\vartheta\right) \d r\\[1ex]
 &\qquad = \int_0^R \frac{2\pi r^2}{(r^2+1)^2}\underbrace{\int_0^{\pi}\sin\vartheta\d \vartheta}_{=2} \d r\\[1ex]
 &\qquad = 4\pi \int_0^R \frac{r^2 \d r}{(r^2+1)^2}\\[1ex]
 &\qquad = 4\pi \left[-\frac{r}{2(r^2+1)}+\frac{1}{2}\arctan r\right]_{r=0}^{r=R}\\[1ex]
 &\qquad = 4\pi \Big[\frac{1}{2}
 \underbrace{\arctan R}_{\substack{\to \frac{\pi}{2}\\ \text{f\"ur} R\to \infty}}
 - \underbrace{\frac{R}{2(R^2+1)}}_{\substack{\to 0\\ \text{f\"ur} R\to \infty}}
 \Big]
 \xrightarrow[R\to \infty]{} \pi^2\,.
\end{align*}
}


%{\Huge \textbf{Aufgabe 3,4,5 zusammenfassen, nur eine Resonanz lassen}\\
%}

\Aufgabe[e]{Lineare Differentialgleichungen $n$-ter Ordnung}
{
Bestimmen Sie die allgemeinen L\"osungen der folgenden
Differentialgleichungen. Falls Anfangswerte gegeben sind, ermitteln Sie auch die L\"osung des Anfangswertproblems. 
\begin{abc}
\item $y^{\prime\prime} + 6y^\prime + 8y = 0$,
\item $y^{\prime\prime} + 2 y^\prime + 5 y  = 17 \, \sin (2x)$.
\item $y^{\prime\prime}(x) - 2 y^\prime(x) - 3y(x) = 4 \EH{x}, \quad y(0) = 0, 
         \quad y'(0) = 6,$
%\item \textbf{R} $y^{\prime\prime}(x)+4y^\prime(x)+4y(x)=4 \EH{-2x}, \quad y(0)=1 ,\quad
%  y^\prime(0)=0$.
  \item $y^{\prime\prime}(x)+5y^\prime(x)+6y(t)=3\EH{3x}$, 
       % (x+2)(x+3) keine Resonanz
%  \item \textbf{R} $y^{\prime\prime}(x)-2y^\prime(x)+y(x)=4\EH x$, 
%	% (x-1)^2 Resonanz
  \item $y^{\prime\prime}(x)-y^\prime(x)-2y(x)= 4x\EH{x}$. 
	%(x+1)(x-2) keine Resonanz
\item  $y^{\prime\prime\prime}+y^{\prime\prime}-y^\prime-y=3\EH{-2x}$,

\end{abc}
}

\Loesung{
\begin{abc}
\item Das charakteristische Polynom 
$p(\lambda) = \lambda^2+6\lambda+8$ hat die Nullstellen $\lambda_1=-2$ und $\lambda_2=-4$.
Damit ist
$$ y(x) = c_1 \EH{-2x} + c_2 \EH{-4x} \ \text{ mit } c_1,c_2 \in \R . $$

\item Das charakteristische Polynom 
$p(\lambda)=\lambda^2+2\lambda+5$ hat die Nullstellen $\lambda_{1/2}=-1\pm\sqrt{1-5}
=-1\pm 2i$. Damit hat man das reelle Fundamentalsystem
$$ \big\{ \EH{-x} \cos(2x), \EH{-x} \sin(2x) \big\}. $$
Um die Partikul\"arl\"osung der inhomogenen Gleichung zu finden gibt es zwei Möglichkeiten:

\textbf{Reeller Ansatz}\\
Ein Ansatz f\"ur eine Partikul\"arl\"osung ist 
$$y_p(x)=A\cos(2x)+B\sin(2x).$$
Einsetzen in die Differentialgleichung liefert: 
\begin{align*}
17\sin(2x)\overset !=& 
-4A\cos(2x)-4B\sin(2x)-4A\sin(2x)+4B\cos(2x)+5A\cos(2x)+5B\sin(2x)\\
=& (-4A+4B+5A)\cos(2x) + (-4B-4A+5B)\sin(2x).
\end{align*}
Koeffizientenvergleich führt dann zum linearen Gleichungssystem f\"ur $A$ und $B$: 
$$\begin{array}{rrl}
\cos(2x):&A+4B&=0\\
\sin(2x):&-4A+B&=17\end{array}
$$
Mit der L\"osung $B=1$ und $A=-4$ haben wir die Partikul\"arl\"osung 
$$y_p(x)=-4\cos(2x)+\sin(2x)$$
 
\textbf{Alternativ: Komplexer Ansatz}: \\
Der Ansatz f\"ur eine Partikul\"arl\"osung ist 
$$y_p(x)=\operatorname{Im}(b\EH{i2x})$$

Einsetzen in die Differentialgleichung liefert: 
$$b((-4)\EH{i2x}+4i\EH{i2x}+5\EH{i2x})\overset != 17\EH{i2x}
$$
Daraus folgt $b=\dfrac{17}{1+4i}=1-4i.$
Damit haben wir die Partikul\"arl\"osung 
$$y_p(x)=\operatorname{Im}((1-4i)\cdot(\cos(2x)+i\sin(2x)))=-4\cos(2x)+\sin(2x)$$


Die Gesamtl\"osung lautet also
$$ y(x) = \sin(2x)-4\cos(2x) + c_1 \EH{-x} \cos(2x) + c_2  \EH{-x} \sin(2x). $$
\item Die Nullstellen des charakteristischen Polynoms
$p(\lambda)=\lambda^2-2\lambda-3$ sind $\lambda_1=-1$ und $\lambda_2=3$. 
Eine Partikul\"arl\"osung der inhomogenen Gleichung berechnet man mit
dem Ansatz $y_p(x)=ae^x$, es folgt
$ -4 a e^x \stackrel{!}{=}4e^x$ und damit $a=-1$. Die allgemeine L\"osung
ist 
$$ y_{allg}(x) = - e^x + c_1 \EH{-x} + c_2 \EH{3x} \ \text{ mit } c_1,c_2 \in \R .$$
Aus den Anfangsbedingungen $y(0)=-1+c_1+c_2 \stackrel{!}{=} 0$
und $y^\prime(0) = -1 - c_1 + 3c_2 \stackrel{!}{=} 6$ folgt das lineare 
Gleichungssystem
$$ \begin{array}{rcrl}
   c_1 & + & c_2 & = 1, \\
   -c_1 & + & 3c_2 & = 7 
   \end{array} $$
mit L\"osung $c_2=2$ und $c_1=-1$. Damit ist 
$$ y_{AWP}(x) = -e^x-\EH{-x}+2\EH{3x}. $$

%\item Die Nullstelle von $p(\lambda)=\lambda^2+4\lambda+4$ ist
%$\lambda=-2$, dies ist eine doppelte Nullstelle. Als Ansatz f\"ur die
%Partikul\"arl\"osung muss man 
%$y_p(x)=ax^2 \EH{-2x}$ nehmen, denn man hat Resonanz der Ordnung 2. 
%Mit $y_p^\prime(x) = a \EH{-2x} \big( 2x-2x^2\big)$
%und $y_p^{\prime\prime}(x) = a \EH{-2x} \big(2-8x+4x^2\big)$
%folgt
%$2a \EH{-2x} \stackrel{!}{=} 4 \EH{-2x}$ und damit $a=2$. 
%Dies liefert die allgemeine L\"osung
%$$ y_{allg}(x) = \big( 2x^2 + c_1 x + c_2 \big) \EH{-2x}
%   \ \text{ mit } c_1,c_2 \in \R. $$
%Die Anfangsbedingungen  $y(0)=c_2 \stackrel{!}{=} 1$ und
%$y^\prime(0) = c_1-2c_2 \stackrel{!}{=} 0$ liefern $c_2=1$ und $c_1=2$ 
%und damit die L\"osung
%%$$ y_{AWP}(x) = \big( 2x^2 + 2x + 1 \big) \EH{-2x} . $$ 
\item Man berechnet zuerst die L\"osungen der homogenen Gleichung
$$ y^{\prime\prime}+5y^\prime+6y=0. $$
Das charakteristische Polynom
$p(\lambda)=\lambda^2+5\lambda+6$ hat die Nullstellen $\lambda_1=-2$ und 
$\lambda_2=-3$. Ein Fundamentalsystem ist 
$\{ \EH{-2x},\EH{-3x}\}$. Nun braucht man noch eine spezielle L\"osung
der inhomogenen Gleichung. Diese berechnet man mit dem Ansatz
$y_p(x)=a \EH{3x}$ mit $a \in \R$. 
Einsetzen in die inhomogene DGl liefert
$(9+15+6)a\EH{3x}=3\EH{3x}$, also $a=1/10$.
Die allgemeine L\"osung der
Gleichung ist
$$ y(x) = c_1 \EH{-2x}+c_2\EH{-3x}+\frac{1}{10} \EH{3x}
   \text{ mit } c_1,c_2 \in \R . $$
%\item Aus $p(\lambda)=\lambda^2-2\lambda+1=0$ folgt $\lambda_{1/2}=1$ (doppelte 
%Nullstelle). Damit hat man das Fundamentalsystem
%$\{ e^x, xe^x \}$.  F\"ur die partikul\"are L\"osung muss man nun den
%Ansatz $y_p(x)=a x^2 e^x$ machen, denn es liegt Resonanz der Ordnung 
%$2$ vor. Es ist $y_p^\prime(x)=a(2x+x^2)e^x$ und
%$y_p^{\prime\prime}(x)=a(2+4x+x^2)e^x$.
%Eingesetzt in die inhomogene DGL erhält man 
%$$ ae^x(2+4x+x^2-2(2x+x^2)+x^2 )= 4 e^x, $$
%$$2ae^x=4 e^x$$
%woraus $a=2$ folgt. Die allgemeine L\"osung der Gleichung ist
%$$ y(x) = \big(c_1 + c_2 x + 2x^2\big) e^x  \text{ mit } c_1,c_2 \in \R . $$
\item Das Polynom $p(\lambda)=\lambda^2-\lambda-2$ hat die Nullstellen
$\lambda_1=2$ und $\lambda_2=-1$, dies ergibt das Fundamentalsystem
$\{\EH{2x},\EH{-x}\}$. Der Ansatz f\"ur die Partikul\"arl\"osung ist
$y_p(x)=(ax+b) e^x$. Mit $y_p^\prime(x)=(ax+a+b)e^x$
und $y_p^{\prime\prime}=(ax+2a+b)e^x$ folgt
$$ \EH{x}(ax+2a+b-(ax+a+b)-2(ax+b))=4x\EH{x} $$
$$ -2ax+a-2b=4x $$
Koeffizientenvergleich liefert dann $a=-2$ und $2b=a$, $b=-1$. Damit hat man die allgemeine 
L\"osung 
$$ y(x) = c_1 \EH{2x}+c_2\EH{-x}-(2x+1)e^x  \text{ mit } c_1,c_2 \in \R. $$
\item Das charakteristische Polynom ist 
$p(\lambda) = \lambda^3+\lambda^2-\lambda-1$. Eine Nullstelle kann man raten, 
zum Beispiel $\lambda_1=1$. Polynomdivision oder Anwendung des
Horner--Schemas liefert dann
$$ p(\lambda) = (\lambda-1)\big(\lambda^2+2\lambda+1\big), $$
damit ist $\lambda_2=-1$ eine weitere, und zwar doppelte, Nullstelle.
Folglich hat die homogene Gleichung das Fundamentalsystem
$$ \big\{ e^x, \EH{-x}, x\EH{-x} \big\}. $$
Zur Berechnung einer Partikul\"arl\"osung benutzt man den Ansatz
$y_p(x) = a \EH{-2x}$. Einsetzen in die Differentialgleichung liefert
$$ a \EH{-2x} \big(-8+4+2-1 \big) \stackrel{!}{=} 3 \EH{-2x} $$
und damit $a=-1$. Die allgemeine L\"osung ist also
$$ y(x) = - \EH{-2x} + c_1 \EH{x} + c_2 \EH{-x} + c_3 x \EH{-x} 
   \ \text{ mit } c_1,c_2,c_3 \in \R. $$

\end{abc} 
}


\ErgebnisC{AufggewdglLinenOrd001}
{

b) $ y(x) = \sin(2x)-4\cos(2x) + c_1 \EH{-x} \cos(2x) + c_2  \EH{-x} \sin(2x)$
c) $y_{AWP}(x) = -e^x-\EH{-x}+2\EH{3x}$\\
%e) $y(x) = \big( 2x^2 + 2x + 1 \big) \EH{-2x}$
d) $y(x) = c_1 \EH{-2x}+c_2\EH{-3x}+\frac{1}{10} \EH{3x}$
%g) $y(x) = \big(c_1 + c_2 x + 2x^2\big) e^x$
e) $y(x) = c_1 \EH{2x}+c_2\EH{-x}-(2x+1)e^x$
f) $y(x) = - \EH{-2x} + c_1 \EH{x} + c_2 \EH{-x} + c_3 x \EH{-x}$\\
}
 %3
%\input{../A/GewDgln/gewdgl_Line_nOrd_002.tex} %4
%\Aufgabe[e]{Anfangswertprobleme zu linearen Differentialgleichungen $n$-ter Ordnung}
{
Gegeben seien die folgenden Anfangswertprobleme: 
\begin{abc}
\item $y^{\prime\prime}(t) - 2 y^\prime(t) - 3y(t) = 4 \EH{t}, \quad y(0) = 0, 
         \quad y'(0) = 6,$
\item $y^{\prime\prime}(t)+4y^\prime(t)+4y(t)=4 \EH{-2t}, \quad y(0)=1 ,\quad
  y^\prime(0)=0$.
\end{abc}
Bestimmen Sie die L\"osungen jeweils mit Hilfe des Exponentialansatzes \textbf{und} zus\"atzlich mit Hilfe der Laplace-Transformation. 
}

\Loesung{
Zun\"achst die L\"osung mittels Exponentialansatz: \\
\begin{abc}
\item Die Nullstellen des charakteristischen Polynoms
$p(\lambda)=\lambda^2-2\lambda-3$ sind $\lambda_1=-1$ und $\lambda_2=3$. 
Eine Partikul\"arl\"osung der inhomogenen Gleichung berechnet man mit
dem Ansatz $y_p(t)=ae^t$, es folgt
$ -4 a e^t \stackrel{!}{=}4e^t$ und damit $a=-1$. Die allgemeine L\"osung
ist 
$$ y_{allg}(t) = - e^t + c_1 \EH{-t} + c_2 \EH{3t} \ \text{ mit } c_1,c_2 \in \R .$$
Aus den Anfangsbedingungen $y(0)=-1+c_1+c_2 \stackrel{!}{=} 0$
und $y^\prime(0) = -1 - c_1 + 3c_2 \stackrel{!}{=} 6$ folgt das lineare 
Gleichungssystem
$$ \begin{array}{rcrl}
   c_1 & + & c_2 & = 1, \\
   -c_1 & + & 3c_2 & = 7 
   \end{array} $$
mit L\"osung $c_2=2$ und $c_1=-1$. Damit ist 
$$ y_{AWP}(t) = -e^t-\EH{-t}+2\EH{3t}. $$

\item Die Nullstelle von $p(\lambda)=\lambda^2+4\lambda+4$ ist
$\lambda=-2$, dies ist eine doppelte Nullstelle. Als Ansatz f\"ur die
Partikul\"arl\"osung muss man 
$y_p(t)=at^2 \EH{-2t}$ nehmen, denn man hat Resonanz der Ordnung 2. 
Mit $y_p^\prime(t) = a \EH{-2t} \big( 2t-2t^2\big)$
und $y_p^{\prime\prime}(t) = a \EH{-2t} \big(2-8t+4t^2\big)$
folgt
$2a \EH{-2t} \stackrel{!}{=} 4 \EH{-2t}$ und damit $a=2$. 
Dies liefert die allgemeine L\"osung
$$ y_{allg}(t) = \big( 2t^2 + c_1 t + c_2 \big) \EH{-2t}
   \ \text{ mit } c_1,c_2 \in \R. $$
Die Anfangsbedingungen  $y(0)=c_2 \stackrel{!}{=} 1$ und
$y^\prime(0) = c_1-2c_2 \stackrel{!}{=} 0$ liefern $c_2=1$ und $c_1=2$ 
und damit die L\"osung
$$ y_{AWP}(t) = \big( 2t^2 + 2t + 1 \big) \EH{-2t} . $$ 
\end{abc}
Nun die L\"osung mit Hilfe der Laplace-Transformation: 
\begin{abc}
\item Die Laplace-Transformation der Differentialgleichung ergibt
\begin{align*}
&&\sL\{4\EH{t}\}=&\sL\{y''(t)-2y'(t)-3y(t)\}\\
\Rightarrow&&\frac{4}{s-1}=&s^2Y(s)-y'(0)-sy(0)-2(sY(s)-y(0))-3Y(s)\\
&&=&s^2Y(s)-6-2sY(s)-3Y(s)
\end{align*}
Die L\"osung im Bildbereich ist dann
\begin{align*}
Y(s)=&\frac 1{s^2-2s-3}\cdot \left( \frac 4{s-1}+6\right)\\
=& \frac{6s-2}{(s-1)(s-3)(s+1)}
\end{align*}
Diese l\"asst sich mittels Partialbruchzerlegung darstellen als 
\begin{align*}
Y(s)=& \frac{-1}{s-1} + \frac{2}{s-3} + \frac{-1}{s+1}
\end{align*}
und die R\"ucktransformation ergibt die L\"osung des Anfangswertproblems: 
\begin{align*}
y(t)=& -\sL^{-1}\left\{\frac{1}{s-1}\right\} + 2 \sL^{-1}\left\{\frac 1{s-3}\right\} -\sL^{-1}\left\{\frac 1{s+1}\right\}\\
=& - \EH{t} + 2 \EH{3t}-\EH{-t}
\end{align*}
\item Die Laplace-Transformation der Differentialgleichung ergibt
\begin{align*}
&&\sL\{4\EH{-2t}\}=&\sL\{y''(t)+4y'(t)+4y(t)\}\\
\Rightarrow&&\frac{4}{s+2}=&s^2Y(s)-y'(0)-sy(0)+4(sY(s)-y(0))+4Y(s)\\
&&=&s^2Y(s)-s+4sY(s)-4+4Y(s)
\end{align*}
Die L\"osung im Bildbereich ist dann
\begin{align*}
Y(s)=&\frac 1{s^2+4s+4}\cdot \left( \frac 4{s+2}+s+4\right)\\
=& \frac{s^2+6s+12}{(s+2)^3}
\end{align*}
Diese l\"asst sich mittels Partialbruchzerlegung darstellen als 
\begin{align*}
Y(s)=& \frac{1}{s+2} + \frac{2}{(s+2)^2} + \frac{4}{(s+2)^3}
\end{align*}
und die R\"ucktransformation ergibt die L\"osung des Anfangswertproblems: 
\begin{align*}
y(t)=& \EH{-2t}+2t\EH{-2t}+4\frac{t^2\EH{-2t}}{2}
= \EH{-2t}(1+2t+2t^2)
\end{align*}
\end{abc}
 
}


\ErgebnisC{gewdglLineAWPr001}
{
a) $y(t) = -e^t-\EH{-t}+2\EH{3t}$\\
b) $y(t) = \big( 2t^2 + 2t + 1 \big) \EH{-2t}$
}
 %5
\Aufgabe[e]{Alte Klausuraufgabe}{
\begin{abc}
\item Berechnen Sie das Integral $\iint_D\frac{x^2}{y^2}\d x\d y$, wobei 
$D$ den von den Geraden $x=2$, $y=x$ und der Hyperbel $xy=1$ begrenzten Bereich 
des $\R^2$ bezeichne.
 
%\item Berechnen Sie das Integral 
%$$
%I = \int_{B} x^2 \d B\,.
%$$
%Dabei ist $B$ jener Bereich, der von den beiden Zylindern $x^2+z^2=1$ und $x^2 
%+ y^2 = 1$ \ eingeschlossen wird. 

\item Gegeben sei der K\"orper
$$ 
K = \big\{ (x,y,z) \in \R^3 \, | \, x^2+y^2-z^2 \le 1, \ -1 \le z \le 2 \big\} 
\,. 
$$
Skizzieren Sie den K\"orper und berechnen Sie dessen Volumen.

\end{abc}

}

\Loesung{
\begin{abc}
\item Der Integrationsbereich hat die folgende Gestalt: \\
\end{abc}
\begin{center}
\begin{pspicture}(-1,-1)(3,3)
\psline{->}(0,-1)(0,3)
\psline{->}(-1,0)(3,0)
\put(.1,2.6){$y$}
\put(2.6,.1){$x$}

\psplot[plotpoints=100, plotstyle=curve, fillstyle=solid, fillcolor=gray]
{1}{2}
{
1 x div
}

\psline[fillstyle=solid, fillcolor=gray, linecolor=gray](1,1)(2,.5)(2,2)(1,1)

\psplot[plotpoints=100, plotstyle=curve]
{.35}{3}
{
1 x div
}
\put(.5,2.5){$xy=1$}
\psline(2,-1)(2,3)
\put(2.1,1.6){$x=2$}
\psline(-1,-1)(3,3)
\put(2.6,2.2){$x=y$}

\end{pspicture}
\end{center}

$D$ ist Normalbereich bez\"uglich $x$,
$$
D =\left\{\begin{pmatrix}x\\y\end{pmatrix}\in\R^2: \frac{1}{x}\leq y \leq x\text{,}\quad 1\leq x\leq 
2\right\}\,.
$$
Es gilt
\begin{align*}
\iint_D\frac{x^2}{y^2}\d x\d y &= \int_1^2\left(\int_{1/x}^x 
\frac{x^2}{y^2}\d 
y\right)\d x\\[1ex]
 &= \int_1^2 x^2 \cdot \left[\frac{-1}{y}\right]_{y = 1/x}^{y=x} \d x\\[1ex]
 &= \int_1^2 \left(-x+x^3\right)\d x = \frac{9}{4}\,.
\end{align*}
\begin{abc}\setcounter{enumi}{1}
%\item Der Bereich $B$ ist ein Normalbereich mit
%$$
%-\sqrt{1-x^2}\leq z \leq \sqrt{1-x^2}\,, \qquad -\sqrt{1-x^2}\leq y \leq 
%\sqrt{1-x^2}\,, \qquad -1 \leq x \leq 1\,.
%$$
%Damit folgt
%\begin{align*}
%I & = \int_{x=-1}^{1}  \int_{y=-\sqrt{1-x^2}}^{y=\sqrt{1-x^2}} 
%\int_{z=-\sqrt{1-x^2}}^{z=\sqrt{1-x^2}} x^2 \d z \d y\d x\\[3ex] 
%& =  2 \int_{x=-1}^{1}  \int_{y=-\sqrt{1-x^2}}^{y=\sqrt{1-x^2}} x^2 
%\sqrt{1-x^2} 
%\d y\d x \\[3ex]
%& = 4 \int_{x=-1}^{1} x^2 \sqrt{1-x^2} \sqrt{1-x^2}\d x = 4 \int_{x=-1}^{1} 
%(x^2-x^4) \d x\\[3ex]
%& = 8 \left(\dfrac{1}{3}- \dfrac{1}{5}\right) = \boxed{\dfrac{16}{15}\,.}
%\end{align*}

\item Die Ungleichung in der Definition des Integrationsgebietes
 l\"asst sich schreiben als 
$$ x^2+y^2 \le 1 + z^2. $$
Skizze: 

\end{abc}

\begin{center}
\begin{pspicture}(-3,-2)(3,3)
\psline{->}(-3,0)(3,0)
\put(2.6,.1){$x$}
\psline{->}(0,-2)(0,3)
\put(.1,2.7){$z$}

\psline{->}(-3,-.8)(3,.8)
\put(2.75,.9){$y$}

\psparametricplot[plotpoints=100, plotstyle=curve]
{-1}{2}
{
1 t t mul add sqrt
t
}
\psparametricplot[plotpoints=100, plotstyle=curve]
{-1}{2}
{
1 t t mul add sqrt neg
t
}

\psellipse(0,-1)(1.44,.4)
\psellipse(0,2)(2.24,.6)
\psellipse(0,0)(1,.27)
\end{pspicture}
\end{center}
Das Volumen berechnet man in Zylinderkoordinaten, 
$$ x = r \cos \varphi, \quad \ y = r \sin \varphi, \quad z = z . $$
Die zugeh\"orige Funktionaldeterminante ist $ \left| 
\dfrac{\partial(x,y,z)}{\partial(r,\varphi,z)}\right| = r$. Das Volumen ist:
\begin{align*}
  \int_{z=-1}^2 \int_{\varphi = 0}^{2\pi} \int_{r=0}^{\sqrt{1+z^2}}
	r \, \d r \d\varphi \d z
  & =  \int_{z=-1}^2 2\pi \frac{1+z^2}{2} \d z\\[2ex]
  & = \pi \left( 3 + \left[ \frac{z^3}{3} \right]_{z=-1}^2 \right) \\[2ex]
  & =  \pi \left( 3 + \frac{8}{3} + \frac{1}{3} \right) =  6 \pi\,.
\end{align*}
}

\ErgebnisC{analysIntgNdim020}
{
a) $9/4$, b) %$16/15$, c) 
$6\pi$
}
 %6
\Aufgabe[e]{Lineare Differentialgleichungen $n$-ter Ordnung}
{
Bestimmen Sie die allgemeinen L\"osungen der folgenden
Differentialgleichungen:
\begin{abc}
% \item $y^{(4)} + 4y = 0$,
\item $y^{(4)} + 2y''' + y'' = 12 x$,
\item $y^{\prime\prime}+4y^\prime+5y=8\sin t$,
\item $y^{\prime\prime}-4y^\prime+4y=\EH{2x}$.
\end{abc}

}

\Loesung{
\begin{abc}
% \item Es ist $p(\lambda)=\lambda^4+4$, die Nullstellen sind\ 
% $\lambda_\ell = \sqrt{2} \EH{i (2\ell+1) \pi / 4}$ f\"ur $\ell=0,1,2,3$,
% also $$\lambda_0=1+i, \lambda_1=-1+i, \lambda_2=-1-i, \lambda_3=1-i$$
% Ein reelles Fundamentalsystem ist
% $$\{ e^x \cos x, e^x \sin x, \EH{-x} \cos x, \EH{-x} \sin x \}$$ 
% Die allgemeine L\"osung lautet also
% $$ y(x) = c_1 e^x \cos x + c_2 e^x \sin x + c_3 \EH{-x} \cos x 
%    + c_4 \EH{-x} \sin x \text{ mit } c_1,c_2,c_3,c_4 \in \R . $$

\item Das charakteristische Polynom $p(\lambda)=\lambda^4+2\lambda^3+\lambda^2$ hat die Nullstellen\newline
$\lambda_{1/2}=0$ (doppelte Nullstelle), $\lambda_{3/4}=-1$ (ebenfalls doppelt).\newline
Ein Fundamentalsystem ist also $\{ 1,x,\EH{-x},x\EH{-x}\}$.\newline
F\"ur die Partikul\"arl\"osung ist der Ansatz $y_p(x)=(ax+b)x^2=ax^3+bx^2$
sinnvoll.  $$y_p^\prime(x)=3ax^2+2bx,
y_p^{\prime\prime}(x)=6ax+2b, y_p^{(3)}(x)=6a \text{ und } 
y_p^{(4)}(x)=0.$$ 
Eingesetzt in die Dgl. ergibt
$$ 0 + 12a + 6ax+2b
   = 12x .$$
\noindent
Koeffizientenvergleich liefert dann $a=2$, $b=-6a=-12$. Die allgemeine L\"osung der Gleichung
ist 
$$ y(x)=c_1 + c_2 x + c_3 \EH{-x} + c_4 x \EH{-x} - 12 x^2 + 2x^3
  \text{ mit } c_1,c_2,c_3,c_4 \in \R. $$

\item Das charakteristische Polynom $p(\lambda)=\lambda^2+4\lambda+5$ hat die Nullstellen
$$\lambda_1=-2+i, \lambda_2=-2-i.$$ \noindent
Die homogene Lösung ist dann
$$y_h=C_1\EH{-2t+it}+C_2\EH{-2t-it}= \EH{-2t}(C_1(\cos t+i\sin t)+C_2(\cos t-i\sin t))$$
$$= \EH{-2t}(c_1\cos t+c_2 \sin t ),\text{ wobei } C_1=\overline{C_2}=\dfrac{ic_1+c_2}{2}.$$
\noindent
Eine Partikul\"arl\"osung berechnet man f\"ur die rechte Seite
$8\sin t=\operatorname{Im}(8 \EH{it})$ mit dem Ansatz $y_p(t) =\operatorname{Im}( a \EH{it})$.\newline
Man erh\"alt durch Einsetzen in die Dgl.
$$a(-1+4i+5)\EH{it}=8\EH{it}\Rightarrow a=\dfrac{2}{1+i}=1-i .$$
Damit ist $$y_p(t)=\operatorname{Im}((1-i) \EH{it})= \operatorname{Im}\big((1-i) (\cos t+i\sin t)\big) = \sin t -\cos t$$
eine Partikul\"arl\"osung.
Die allgemeine L\"osung der Gleichung ist
$$ y(t) = \EH{-2t} ( c_1 \cos t + c_2 \sin t) + \sin t - \cos t. $$

\item Es ist $p(\lambda)=\lambda^2-4\lambda+4=(\lambda-2)^2$.\newline
Das Polynom hat die doppelte Nullstelle $\lambda=2$.\newline
Ein Fundamentalsystem ist daher
$\{ \EH{2x}, x \EH{2x}\}$.\newline
Mit dem Ansatz $y_p(x)=ax^2 \EH{2x}$ folgt
$$y_p^\prime(x)=a (2x^2+2x) \EH{2x},\ y_p^{\prime\prime}(x)=a (4x^2+8x+2) \EH{2x}.$$
Das Einsetzen in die Dgl. liefert somit
$$a\EH{2x}(4x^2+8x+2-8x^2-8x+4x^2)=\EH{2x}.$$
Daraus folgt $a=1/2$. \newline
Die allgemeine L\"osung lautet
$$ y(x) = \left(c_1+c_2x+\frac{1}{2}x^2\right) \EH{2x} 
  \text{ mit } c_1,c_2 \in \R. $$
\end{abc} 
}


\ErgebnisC{AufggewdglLinenOrd003}
{
Allgemeine L\"osungen:
%a) $y(x) = c_1 e^x \cos x + c_2 e^x \sin x + c_3 \EH{-x} \cos x 
%    + c_4 \EH{-x} \sin x $
a) $y(x)=c_1 + c_2 x + c_3 \EH{-x} + c_4 x \EH{-x} - 12 x^2 + 2x^3$, \,
b) $y(t) = \EH{-2t} ( c_1 \cos t + c_2 \sin t) + \sin t - \cos t$, \, 
c) $y(x) = \left(c_1+c_2x+\frac{1}{2}x^2\right) \EH{2x}$\, 
}
 %7



\Aufgabe[e]{Online Aufgabe}{
Bearbeiten Sie die aktuelle Online-Aufgabe im ILIAS-Kurs. \\
Beachten Sie, dass Sie dort auch die L\"osungswege zu einzelnen Aufgaben zur Korrektur hochladen k\"onnen. 
}

% \Loesung{
% }
% 
% \ErgebnisC{Online Aufgabe}
% {
% }



%\Aufgabe[e]{Uneigentliche Integrale}{
\"Uberpr\"ufen Sie, ob die folgenden uneigentlichen Integrale existieren (d. h. einen endlichen Wert annehmen). Berechnen Sie f\"ur das dritte Beispiel den Wert des Integrals. 
\begin{align*}
  I_1&=\int\limits_{1}^\infty \frac{\sin \frac 1{x^2}}{x^2}\d x\\
  I_2&=\int\limits_{0}^1\frac{\cos x^2}{1-x}\d x \\
  I_3&= \int\limits_{0}^\infty\frac{\arctan x}{x^2+1}\d x
\end{align*}
}

\Loesung{
\begin{itemize}
\item Das erste Integral hat einen endlichen Wert, man kann seinen Wert nach oben absch\"atzen, indem man den Integranden durch eine gr\"oßere Funktion ersetzt: 
\begin{align*}
I_1&= \int\limits_{1}^\infty \frac {\sin \frac 1{x^2}}{x^2}\d x 
\leq \int\limits_1^\infty \left|\frac{\sin \frac 1{x^2}}{x^2}\right|\d x \\
&\leq \int\limits_1^\infty \left|\frac{1}{x^2}\right|\d x \text{ \qquad (weil $|\sin(1/x^2)|\leq 1$)}\\
&= \left[\frac{-1}{x}\right]_1^\infty = \lim_{b\to\infty} \frac{-1}{b}- \frac{-1}{1}
= 1
\end{align*}
Damit hat $I_1$ einen endlichen Wert, der an dieser Stelle jedoch nicht berechnet werden soll. 
\item Das zweite Integral wird nach unten abgesch\"atzt, indem man die Integrandenfunktion durch eine geringere Funktion absch\"atzt: 
\begin{align*}
I_2&= \int\limits_0^1\frac{\cos{x^2}}{1-x}\d x\\
&\geq \int\limits_0^1\frac{\cos 1}{1-x}\d x\text{ \qquad (weil $\cos$ auf dem Intervall $[0,1]$ monoton f\"allt)}\\
&= \cos 1\Bigl[ -\ln|1-x|\Bigr]_0^1 = \cos 1\cdot \left(- \lim_{b\to 1}\ln|1-b| - (-\ln|1-0|)\right)=\infty
\end{align*}
Damit hat auch das Integral $I_2$ keinen endlichen Wert. 
\item F\"ur den Integranden $f(x)=\frac{\arctan x}{1+x^2}$ des dritten Integrals kann man mittels partieller Integration eine Stammfunktion ermittelt werden: 
\begin{align*}
&&F(x)&= \int\limits_0^x f(t)\d t 
= \int\limits_0^x \underbrace{\arctan t}_{u(t)} \underbrace{\frac 1{1+t^2}}_{v'(t)}\d t\\
&&&= \Bigl.\underbrace{\arctan t}_{u(t)} \underbrace{\arctan t}_{v(t)}\Bigr|_{0}^x - \int\limits_0^x \underbrace{\frac 1 {1+t^2}}_{u'(t)} \underbrace{\arctan t}_{v(t)}\d t\\
&&&= \arctan^2 x - F(x)\\
\Rightarrow && 2F(x)&= \arctan^2 x\\
\Rightarrow && F(x)&= \frac{\arctan^2 x}2
\end{align*}
Den Wert des Integrals $I_3$ erh\"alt man dann durch den Grenz\"ubergang: 
$$I_3= \lim_{b\to \infty} F(b)-F(0) = \frac 12 \left( \frac \pi 2\right)^2 - 0 = \frac{\pi^2}8.$$
\end{itemize}
}

\ErgebnisC{analysPartBrch006}
{
$I_3= \lim_{b\to \infty} F(b)-F(0) = \frac 12 \left( \frac \pi 2\right)^2 - 0 = \frac{\pi^2}8$
}





%\Aufgabe[e]{~}{
 Es sei 
$$
\mathrm{sign} \,(x) := \left\{\begin{array}{@{}r@{\,}c@{\,}l} 1 &\quad \text{f\"ur }\,& x>0\\[1ex] 0& \quad\text{f\"ur }\,& x= 0\\[1ex] -1 &\quad\text{f\"ur }\,& x<0  \end{array}\right.
$$
Gegeben sei die Funktion $f\in { Abb}\,(\R,\R)$ mit $f(x) = \dfrac{x}{2} + \mathrm{sign}\, (x)$. Skizzieren Sie $f$. Ist $f$ auf $I= [-1,2]$ Riemann-integrierbar? Begr\"unden Sie Ihre Antwort und bestimmen Sie gegebenenfalls das zugeh\"orige Riemann-Integral.
% \end{itemize}
}

\Loesung{
\begin{center}
\begin{pspicture}(-2,-2)(3,3)
%\psgrid(-2,-2)(3,3)
\psline[fillstyle=solid, fillcolor=lightgray, linecolor=lightgray](0,0)(-1,0)(-1,-1.5)(0,-1)(0,0)
\psline[fillstyle=solid, fillcolor=lightgray, linecolor=lightgray](0,0)(2,0)(2,2)(0,1)(0,0)
\psplot[plotpoints=100,plotstyle=curve]
{-2}{0}{
.5 x mul -1 add
}
\psplot[plotpoints=100,plotstyle=curve]
{0}{3}{
.5 x mul 1 add
}

\psline{->}(-2,0)(3,0)
\psline{->}(0,-2)(0,3)
\psline(-.1,1)(.1,1)
\psline(1,-.1)(1,.1)
\put(2.7, .1){$x$}
\put(1,-.3){$1$}
\put(-.3, 1){$1$}
\put(.1,2.7){$f(x)$}
\end{pspicture}
\end{center}
Nach einem Satz aus der Vorlesung (Satz 3.96) ist $f$ auf $[-1,2]$ Riemann-integrierbar, da $f$ auf $[-1,0]$ und auf $[0,2]$ jeweils Riemann-integrierbar ist. Es gilt
$$
I = \int_{-1}^0 \left(\dfrac{x}{2} -1 \right) \: \mathrm{d}x + \int_{0}^2 \left(\dfrac{x}{2} +1 \right) \: \mathrm{d}x =   \left.\left(\dfrac{x^2}{4} -x \right)\right|_{-1}^0 +  \left.\left(\dfrac{x^2}{4} +x \right)\right|_{0}^2 = \dfrac{7}{4} \,.
$$
}


% \ErgebnisC{b-2013-K-A1}{
% 
% }



 
% \Aufgabe[e]{Taylor-Entwicklung} {
Gegeben seien die beiden Funktionen $\R\overset{\vec r}\rightarrow \R^2\overset f \rightarrow \R$
mit
\begin{align*}
f(x,y)=& -\EH{y+1-x^2}\qquad\text{ und }\qquad \vec r(t)=\begin{pmatrix} \cos(\pi\cdot t)\\\ln t
- \sin^2(\pi\cdot t)\end{pmatrix}.
\end{align*}
\begin{abc}
\item Geben Sie die Taylorentwicklung zweiter Ordnung $T_{2,f}$ der Funktion $f$ um den Punkt $\vec x_0=(1,\,
0)^\top$ an. 
\item Geben Sie die Taylorentwicklung zweiter Ordnung $\vec T_{2,r}$ der Funktion $\vec r$ um den Punkt $t_0=1$
an. (Entwickeln Sie dazu die Komponenten $r_1$ und $r_2$ der Funktion $\vec r=(r_1,r_2)^\top$
separat.)
\item Bilden Sie durch Verkettung beider Funktionen die Funktion 
$$g(t):=f\circ \vec r(t).$$
\item Verketten Sie ebenso die beiden Taylor-Polynome
$$\tilde g(t):=T_{2,f}\circ \vec T_{2,r}(t).$$
\item Vergleichen Sie $g$ und $\tilde g$ und die Taylorpolynome erster Ordung dieser beiden Funktionen. 
\end{abc}
}


\Loesung{
\begin{abc}
\item Die ersten beiden Ableitungen von $f$ ergeben sich zu 
\begin{align*}
f(x,y)=& -\EH{y+1-x^2}                                       &\quad \Rightarrow&\quad&f(1,0)=&-1\\
\vec J_f(x,y)=& -\EH{y+1-x^2}(-2x,\, 1)                      &\Rightarrow&&\vec J_f(1,0)=&(2,-1)\\
\vec H_f(x,y)=& -\EH{y+1-x^2}\begin{pmatrix} 4x^2-2& -2x\\ -2x  &  1\end{pmatrix}&\Rightarrow&&\vec
H_{f}(1,0)=&\begin{pmatrix} -2 & 2\\ 2 & -1\end{pmatrix}
\end{align*}
und damit ist das Taylorpolynom zweiter Ordnung um den Entwicklungspunkt $(1,0)^\top$: 
\begin{align*}
T_{2,f}(x,y)=&f(1,0)+\vec J_f(1,0)\begin{pmatrix}x-1\\y-0\end{pmatrix} + \frac 12 (x-1,\, y-0) \vec
H_f(1,0) \begin{pmatrix} x-1\\y-0\end{pmatrix}\\
=& -1 +2(x-1)-y + \frac 12 \left( -2 (x-1)^2 + 2\cdot 2 (x-1)y -1\cdot y^2\right)\\
=& -1 + 2(x-1)-y - (x-1)^2 + 2 (x-1)y -\frac {y^2}2.
\end{align*}
\item F\"ur $\vec r$ ergeben sich die Ableitungen zu
\begin{align*}
\vec r(t)=& \begin{pmatrix}\cos(\pi t)\\ \ln t - \sin^2(\pi
t)\end{pmatrix}&\Rightarrow&\ &\vec r(1)=&\begin{pmatrix}-1\\0 \end{pmatrix}\\
\vec r'(t)=& \begin{pmatrix} -\pi\sin(\pi t)\\ \frac 1 t -2\pi\sin(\pi t)\cos(\pi t)\end{pmatrix} = \begin{pmatrix} -\pi\sin(\pi t)\\ \frac 1 t - \pi \sin(2\pi t) \end{pmatrix}&\Rightarrow&&\vec r'(1)=&\begin{pmatrix}0\\1 \end{pmatrix}\\
\vec r''(t)=& \begin{pmatrix} \pi^2\cos(\pi t)\\ -\frac 1{t^2} -2\pi^2\cos(2\pi t)\end{pmatrix}&\Rightarrow&&\vec r''(t)=& \begin{pmatrix}-\pi^2 \\ -1-2\pi^2\end{pmatrix}
\end{align*}
und das Taylorpolynom zu: 
\begin{align*}
T_{2,\vec r}=& \vec r(1) + (t-1)\vec r'(1) + \frac 12 (t-1)^2 \vec r''(1)\\
=& \begin{pmatrix}-1 -\frac 12 \pi^2(t-1)^2\\(t-1)-\frac 12(1+2\pi^2)(t-1)^2\end{pmatrix}.
\end{align*}
\item Es ist 
$$g(t)=-\EH{\ln t - \sin^2(\pi t) + 1 - \cos^2(\pi t)}=-\EH{\ln t }=-t.$$
\item Die Verkettung der beiden Taylor-Polynome hingegen ergibt: 
\begin{align*}
\tilde g(t)=& -1 + 2 \left( -2 -\frac 12 \pi^2 (t-1)^2\right) 
- (t-1)\left( 1-\frac 12(1+2\pi^2)(t-1)\right) + \\
&- \left( -2-\frac 12 \pi^2(t-1)^2\right)^2 +\\
&+ 2 \left( -2-\frac
12 \pi^2(t-1)^2\right)(t-1)\left( 1 - \frac 12 (1+2\pi^2)(t-1)\right) + \\
&- \frac 12 (t-1)^2 \left(1-\frac 12(1+2\pi^2)(t-1)\right)^2\\
=& -9-5(t-1)+\left( 2\pi^2+2\right)(t-1)^2 + \frac12 (t-1)^3 + \left( \frac14 \pi^4 -\frac 18\right)(t-1)^4
\end{align*}
\item Anders als man vermuten k\"onnte, ist die exakt ermittelte Funktion $g(t)$ weit weniger
kompliziert als die N\"aherungsfunktion $\tilde g(t)$. Auch die beiden linearen Taylorpolynome um
den Punkt $t=1$ unterscheiden sich deutlich: 
$$T_{1,g}(t)=g(t)=-t,\qquad T_{1,\tilde g}(t)=-9-5(t-1).$$
% Anders w\"are dies, wenn man die Funktion $f$ statt um $\vec x_0=(1,0)^\top$ um den Punkt $\vec
% r(t_0)=(-1,0)^\top$ entwickeln w\"urde. Dann w\"aren zumindest $T_{1,g}$ und $T_{1,\tilde g}$
% identisch, denn 
% \begin{align*}
% T_{2,f}(x,y)=&f(-1,0)+\vec J_f(-1,0)\begin{pmatrix}x+1\\y-0\end{pmatrix} + \frac 12 (x+1,\, y-0) \vec H_f(-1,0) \begin{pmatrix} x+1\\y-0\end{pmatrix}\\
% =& -1 -2(x+1)-y + \frac 12 \left( -2 (x+1)^2 - 2\cdot 2 (x+1)y -1\cdot y^2\right)\\
% =& -1 - 2(x+1)-y - (x+1)^2 - 2 (x+1)y -\frac {y^2}2.
% \end{align*}
% und f\"ur die Verkettung der beiden Polynome h\"atten wir dann
% \begin{align*}
% \tilde g(t)=& -1 - 2 \left( -\frac 12 \pi^2 (t-1)^2\right) 
% - (t-1)\left( 1-\frac 12(1+2\pi^2)(t-1)\right) + \\
% &- \left( -\frac 12 \pi^2(t-1)^2\right)^2 +\\
% &+ 2 \left( -\frac 12 \pi^2(t-1)^2\right)(t-1)\left( 1 - \frac 12 (1+2\pi^2)(t-1)\right) + \\
% &- \frac 12 (t-1)^2 \left(1-\frac 12(1+2\pi^2)(t-1)\right)^2\\
% =& -1-(t-1)+ 2\pi^2(t-1)^2 + \left(2\pi^2+\frac12\right) (t-1)^3+\\
% &- \left(\pi^2+\frac74 \pi^4 +\frac 18\right)(t-1)^4\,.
% \end{align*}
% Somit ist $T_{1,\tilde g} = -1 - (t-1) = -t = T_{1,g}$.
\end{abc}
}

\ErgebnisC{AufganalysTaylNdim005}
{
$T_{2,f}(x,y)= -1 + 2(x-1)-y - (x-1)^2 + 2 (x-1)y -\frac {y^2}2$\\
$T_{2,\vec r}= \begin{pmatrix}-1 -\frac 12 \pi^2(t-1)^2\\(t-1)-\frac 12(1+2\pi^2)(t-1)^2\end{pmatrix}$


}



%\Aufgabe[e]{Taylor--Polynom  \& Extrema in 2 Dimensionen}{
\begin{itemize}
	\item[ a)] Berechnen Sie das Taylor--Polynom 2. Grades der Funktion $f(x,y) = (2x-3y)\cdot\sin(3x-2y)$ zum Entwicklungspunkt $ \vec x_0=(0,0)^{\text {T} }$ \,.
	
	\item[ b)] Ermitteln Sie die Extrema der Funktion \ $f(x,y)=2x^3-3xy+2y^3-3$\,.
\end{itemize}
}

\Loesung{
\begin{abc}
%NOTE: Die Sytnax der Lösung muss überarbeitet werden.

\item
\begin{align*}
 	f & = & (2x-3y)\cdot\sin(3x-2y) & \Rightarrow & f(0,0) & = & 0\\[1ex]
 	f_x & = & 2\,\sin(3x-2y)+3\,(2x-3y)\cdot\cos(3x-2y) &  \Rightarrow & f_x(0,0) & = & 0 \\
 	f_y & = & -3\,\sin(3x-2y)-2\,(2x-3y)\cdot\cos(3x-2y) &  \Rightarrow & f_y(0,0) & = & 0 \\[1ex]
 	f_{xx} & = & 12\,\cos(3x-2y)-9\,(2x-3y)\cdot\sin(3x-2y) & \Rightarrow & f_{xx}(0,0) & = & 12 \\
 	f_{xy} & = & -13\,\cos(3x-2y)+6\,(2x-3y)\cdot\sin(3x-2y) &  \Rightarrow & f_{xy}(0,0) & = & -13 \\
 	f_{yy} & = & 12\,\cos(3x-2y)-4\,(2x-3y)\cdot\sin(3x-2y) & \Rightarrow & f_{yy}(0,0) & = & 12 \\
\end{align*}
 Damit erhält man für das gesuche Taylor--Polynom die Darstellung:
\[
 	T_2(x,y) = 6\,x^2 -13\,xy+6\,y^2\ .
 \]
 
 % \textbf{2. Lösungsweg:}
 % 
 % Da der Entwicklungspunkt der Ursprung des Koordinatensystems ist, bietet sich ein sehr viel einfacheres Vorgehen an:
 % Mit
 % 	\[
 % 	\sin(t)=t-\dfrac{t^3}{6}+... \UND t=(3x-2y) \PFEIL \sin(3x-2y)=(3x-2y)-\dfrac{(3x-2y)^3}{6}+...
 % \]
 % Das gesuchte Taylor--Poynom der Funktion \ $f$ \ erhält man jetzt als das Produkt von dem Faktor \ $(2x-3y)$ \ mit dem Anfang der Taylor--Entwicklung der Sinus--Funktion, wobei nur Terme bis zur zweiten Ordnung berücksichtigt werden müssen:
 % 	\[
 % 	T_2(x,y) = (2x-3y)\cdot (3x-2y) = 6\,x^2 -13\,xy+6\,y^2\ .
 % \]
 
\item Die stationären Punkte erhält man aus
 \begin{align*}
 	\textbf{grad}\,f(x,y) &= \begin{pmatrix} 6\,x^2 -3\,y \\ -3\,x+6y^2 \end{pmatrix} = \textbf{0}\\ 
 	\Rightarrow \begin{cases}y=2x^2 \\ -3x+24x^4=0 \end{cases} 
 	&\Rightarrow \begin{cases} y = 2x^2\\ 3x\left(8x^3-1\right)=0 \end{cases}
 	\Rightarrow  \begin{cases}y = 2x^2\\ x = 0  \vee x = \frac12 \end{cases}
 \end{align*}
 zu \ $P_1 = (0,0)$ \ und \ $P_2 = \big(\frac 12,\frac 12\big)$\ .
 
 Die Hesse--Matrix ist
 	\[
 	H(x,y) = \begin{pmatrix}12x & -3 \\ -3 & 12y \end{pmatrix}.
 \]
 
 Für \ $P_1 = (0,0)$ \ erhält man
 	\[
 	H(0,0) = \begin{pmatrix} 0 & -3 \\ -3 & 0 \end{pmatrix} \Rightarrow \lambda_1 \cdot \lambda_2 = \det(H(0,0)) = -9 < 0\ .
 \]
 Damit haben die Eigenwerte verschiedene Vorzeichen und es handelt sich um einen Sattelpunkt.
  
 Für \ $P_2 = \big(\frac 12,\frac 12\big)$ \ erhält man
 	\[
 	H\big(\tfrac 12,\tfrac 12\big) = \begin{pmatrix} 6 & -3 \\ -3 & 6 \end{pmatrix} \Rightarrow \lambda_1 \cdot \lambda_2 = \det(H\big(\tfrac 12,\tfrac 12\big)) = 27 > 0
 \]
% sowie 
% \[
% f_{xx}\big(\tfrac 12,\tfrac 12\big)= 6 > 0\,.
% \]
 Damit sind beide Eigenwerte positiv und es handelt sich um ein Minimum.

\end{abc}
}

\ErgebnisC{Aufgwz-2013-8-2}
{
\[
T_2(x,y) = 6\,x^2 -13\,xy+6\,y^2\,.
\]
Kritische Punkte: $P_1 = (0,0)$ und $P_2 = \big(\frac 12,\frac 12\big)$.
 }


%\Aufgabe[e]{Regel von L'Hospital}{
Berechnen Sie mit Hilfe der Regel von L'Hospital die Grenzwerte
\begin{tabbing}
\hspace*{1em} \= a) $ \ \underset{x\to 0}\lim \dfrac{x^2 \sin x}{\tan x-x}$\,,
\hspace*{8em} \= b) 
    $\underset{x\rightarrow 0}\lim \dfrac{\ln(\EH{x}-x)}{\ln(\cos x)}$\,, \\[1ex]
\> c) $\underset{x\rightarrow \infty}\lim x(2\arctan x - \pi)$\,,
\> d) $\underset{x\rightarrow 1}\lim \dfrac{x^2-1}{x^{x}-1}$\,. 
\end{tabbing}

}

\Loesung{
\begin{abc}
\item Da sowohl Z\"ahler, als auch Nenner gegen Null konvergieren
$$\underset{x\to 0}\lim (x^2\sin x)=0=\underset{x\to 0}\lim (\tan x - x),$$
darf der Satz von L'Hospital angewendet werden: 
\begin{align*}
\underset{x\to 0}\lim \frac{x^2 \sin x}{\tan x-x} =& \underset{x\to 0}\lim \frac{2x\sin x + x^2\cos
x}{\frac 1{\cos^2 x}-1} = \underset{x\to 0}\lim \frac{2x \sin x \cos^2 x + x^2\cos^3 x}{1-\cos^2
x}\\
=&\underset{x\to 0}\lim \frac{2\sin x + 2x\cos x + 2x \cos x - x^2\sin x}{2(\cos x)^{-3}\sin x}\\
&\qquad\text{ (Erneut gehen Z\"ahler und Nenner gegen 0)}\\
=& \underset{x\to 0}\lim \frac{2\cos x + 4\cos x -4x\sin x -2x\sin x -x^2\cos x}{6(\cos
x)^{-4}\sin^2 x + 2(\cos x)^{-2}}
\end{align*}
Der Z\"ahler dieses Bruches geht gegen $6$, der Nenner gegen $2$, also ist insgesamt
$$\underset{x\to 0}\lim \frac{x^2\sin x}{\tan x-x} = \frac 6 2 =3$$

\item Auch hier kann die Regel von L'Hospital angewendet werden, da Z\"ahler und Nenner gegen Null gehen:
\begin{align*}
\lim_{x \to 0} \frac{\ln(\EH{x}-x)}{\ln(\cos x)}=& \lim_{x \to 0} \frac{\frac{\EH{x}-1}{\EH{x} - x}
}{\frac{-\sin x}{\cos x}}\\
=&\lim_{x\to 0} \frac{\EH{x} \cos x - \cos x}{-\EH{x} \sin x + x \sin x }\\
&\qquad\text{ (Erneut gehen Z\"ahler und Nenner gegen 0)}\\
=&\lim_{x\to 0} \frac{\EH{x} \cos x -\EH{x} \sin x + \sin x}{-\EH{x} \sin x - \EH{x} \cos x + \sin x +
x\cos x}=\frac 1 {-1}=-1
\end{align*}

\item Das Produkt $x(2\arctan x - \pi)$ kann in einen Quotienten umgeformt werden, dessen Z\"ahler
und Nenner jeweils gegen Unendlich gehen, danach kann die Regel von L'Hospital angewendet werden: 
\begin{align*}
 \lim\limits_{x \to \infty} x(2\arctan x - \pi)
  =& \lim\limits_{x \to \infty} \frac{2\arctan x - \pi}{x^{-1}}
  = \lim\limits_{x \to \infty} \frac{\dfrac{2}{1+x^2}}{-x^{-2}}\\
  =& -\lim\limits_{x \to \infty} \frac{2x^2}{1+x^2}\quad\text{(Z\"ahler und Nenner gehen gegen
  $\infty$)}\\
=& -\lim\limits_{x\to \infty} \frac{4x}{2x}=-2
\end{align*}


\item Da Z\"ahler und Nenner gegen Null konvergieren kann man die Regel von L'Hospital
  anwenden. Danach folgt mit  $x^x = \EH{x \ln x}$:
$$ \lim_{x \to 1} \dfrac{x^2-1}{x^{x}-1} 
   = \lim_{x \to 1} \dfrac{2x}{ (1+\ln x) \EH{x \ln x}} = 2\,.$$
\end{abc}
}

\ErgebnisC{AufganalysReglLhos001}
{
\textbf{a)} $3$, \textbf{b)} $-1$, \textbf{c)} $-2$, \textbf{d)} $2$
}
 

%\Aufgabe[e]{Newton-Verfahren}
{
Gegeben sei die Funktion 
$$f(x)=x^3+2x^2-15x-36.$$
\begin{abc}
\item Bestimmen Sie alle lokalen Extrema der Funktion $f(x)$. 
\item Begr\"unden Sie, weshalb $f$ zwei Nullstellen besitzt. 
\item F\"uhren Sie das Newton-Verfahren mit der Funktion $f$ zwei mal durch.  
W\"ahlen Sie im ersten Durchlauf den Startwert $x_0=1$ und
im  zweiten Durchlauf $x_0=-1$. F\"uhren Sie jeweils drei Iterationsschritte durch. 
\end{abc}
}

\Loesung{
\begin{abc}
\item Die Ableitung der Funktion $f$ ist
$$f'(x)=3x^2+4x-15.$$
Ihre Nullstellen 
$$x_{1/2}=\frac{-2\pm\sqrt{49}}3=\left\{\begin{array}{l}5/3\\-3\end{array}\right.$$
sind die station\"aren Punkte der Funktion $f$. \\
Die zweite Ableitung der Funktion ist $f''(x)=6x+4$ und wegen 
$$f''(x_1)=14>0\text{ und } f''(x_2)=-14<0$$
liegt bei $(x_1,f(x_1))=(5/3,-1372/27)$ ein Minimum und bei 
$(x_2,f(x_2))=(-3,0)$ ein Maximum der Funktion $f$ vor. 
\item Da $f$ links der Maximalstelle $x_2=-3$ (die auch Nullstelle ist) sowie rechts der
Minimalstelle $x_1=5/3$ streng monoton steigt, und sonst streng monoton f\"allt, besitzt $f$
lediglich die Nullstelle $x_2$ sowie (wegen $f(x_1)<0$) eine weitere Nullstelle rechts von $x_1$. 
\item Mit der oben berechneten Ableitung $f'(x)=3x^2+4x-15$  lautet die Iterationsvorschrift f\"ur das Newton-Verfahren: 
$$x_{k+1}=x_k-(f'(x_k))^{-1}\cdot f(x_k).$$
F\"ur die gegebenen Startwerte sind die ersten drei Iterationsschritte:\\


\begin{tabular}{r|r|r}
$k$&$x_k$    & $f(x_k)$\\\hline
 0 & 1.0000  &  -48.0000 \\
 1 & -5.0000 &  -36.0000 \\
 2 & -4.1000 &   -9.8010 \\
 3 & -3.5850 &   -2.5955 \\
\end{tabular}
\qquad\qquad
\begin{tabular}{r|r|r}
$k$&$x_k$    & $f(x_k)$\\\hline
    0 &  -1.00000 &  -20.00000 \\
    1 &  -2.25000 &  -3.51562  \\
    2 &  -2.64894 &  -0.81945  \\
    3 &  -2.82923 &  -0.19916  \\
\end{tabular}\\

Obwohl der erste Startwert $x_0=1$ dichter an der Nullstelle $x=4$ der Funktion liegt, konvergiert
das Verfahren trotzdem gegen die Nullstelle $x=-3$. \\
Um die Nullstelle $x=4$ dennoch zu korrekt zu
finden, muss $x_0$ geeignet gew\"ahlt werden. 
\end{abc}
}

\ErgebnisC{analysNewtVerf002}{
Es ist $f(-3)=f(4)=0$.} 
 
 
  %noch erweitern um Taylor-Teilaufgabe 

% \Aufgabe[e]{Kurvendiskussion}{\label{KurvDisk002}
% \begin{abc}
% \item  
Gegeben sei die Funktion
$$	f(x) = \dfrac{x^2+3\,x}{1-x}.$$
\begin{iii}
	\item Geben Sie den maximalen Definitionsbereich der Funktion \ $f$ \ an.
	
	\item Bestimmen Sie die Nullstellen der Funktion.
	
	\item Bestimmen Sie die kritischen Punkte der Funktion und deren Funktionswerte. Klassifizieren Sie alle kritischen Punkte als Minimum, Maximum oder Wendepunkt.
	
	\item Untersuchen Sie das Monotonieverhalten der Funktion. 
	
	\item Bestimmen Sie alle Asymptoten der Funktion.

	\item Bestimmen Sie den Wertebereich der Funktion.

	\item Skizzieren Sie die Funktion.
\end{iii}
% \item Gegeben sei die Funktion
% $$	g\,:\,\R\rightarrow\R\,:\,x\rightarrow \big(x^2-4\big)^2\cdot\EH{-x} .$$
% Bestimmen Sie alle relativen Minima und Maxima der Funktion \ $g$ \ \textbf{ohne} die zweite Ableitung zu berechnen.
% \end{abc}
}

\Loesung{
\begin{abc}
\item
\begin{iii}
	\item Der maximale Definitionsbereich ist \ $\mathcal D = \R\backslash\left\{1\right\}$\,.
	
	\item Die Nullstellen der Funktion sind \ $x_{N_1} = -3 \text{ und } x_{N_2} = 0$ \,.
	
	\item Die kritischen Punkte sind die Nullstellen der ersten Ableitung. Aus
$$	f'(x) = \frac{(2x+3)(1-x)+(x^2+3x)}{(1-x)^2} = \dfrac{-x^2+2\,x+3}{(1-x)^2}=0  $$
  folgt
$$
  x_{K_1}= -1\text{ mit } f(-1)=-1 \text{ und } x_{K_2}= 3\text{ mit } f(3)=-9\ .
$$
  	Die zweite Ableitung ist:
  	$$
		f''(x) = \frac{8}{(1-x)^3}.
  	$$
  	Für $x_{K_1}= -1$ ist $f''(-1)=1 > 0$. Es handelt sich also um ein Minimum.
  	Für $x_{K_2}= 3$ ist $f''(3)=-1 < 0$. Es handelt sich also um ein Maximum.
  	
	\item
	Aus den stationären Punkten un der Definitionslücke ergeben sich die Monotonieintervalle.
	In $(-\infty, -1)$ und $(3,\infty)$ ist die Funktion monoton fallend. In $(-1,1)$ und 
	$(1,3)$ ist die Funktion monoton steigend. 
	\item Aus  $f(x) = -x-4+\dfrac{4}{1-x}$  folgt, dass  $g(x)=-x-4$  die Asymptote ist.
	Wegen $\lim_{x \to 1^-} f(x) = \infty$ und $\lim_{x \to 1^+} f(x) = -\infty$ gibt es 
	eine senkrechte Asymptote bei $x=1$.
	\item Der Wertebereich ist $\mathcal{W} = (-\infty,-9] \cup [-1,\infty)$. 
	\item \quad\\
\end{iii}
\end{abc}
\begin{center}	
\psset{unit=0.5cm}
\begin{pspicture}(-5,-16)(7,6)

\psgrid[griddots=8,subgriddiv=0](-5,-16)(7,6)
\psline[linestyle=dashed](-5,1)(7,-11)
\psline[linestyle=dashed](1,-16)(1,6)

\pscircle*(-3,0){0.1}
\pscircle*(-1,-1){0.1}
\pscircle*(-0,0){0.1}
\pscircle*(3,-9){0.1}
\psplot[plotpoints=100, plotstyle=curve]
{-5}{.63}
{
x x mul 3 x mul add 1 x neg add div
}
\psplot[plotpoints=100, plotstyle=curve]
{1.37}{7}
{
x x mul 3 x mul add 1 x neg add div
}

\end{pspicture}

\end{center}

Bei \ $(-1,-1)$ \ handelt es sich also um ein (lokales) Minimum, bei \ $(3,-9)$ \ um ein Maximum.
% \begin{abc}\setcounter{enumi}{1}
% \item Da die Funktion nirgends negativ ist, sind die Nullstellen automatisch lokale Minima: \ $x_{\text {min}_{1,2}} = \pm 2$\,. 
% 
% Die Nullstellen der ersten Ableitung sind die kritischen Punkte. Aus
% $$
% 	g'(x) = 4x\,(x^2-4)\,\EH{-x}-(x^2-4)^2\,\EH{-x} = (-x^2+4x+4)(x^2-4)\,\EH{-x}=0  
% $$
% folgt
% $$
% 	x_{1,2} =\pm 2 \text{ und } x_{3,4} = 2\pm\sqrt 8.
% $$
% Die ersten beiden kritischen Punkte sind die schon bekannten Nullstellen und die beiden anderen sind
% Maxima, da ein einfacher kritischer Punkt zwischen zwei Minima nur ein Maximum sein kann und die
% Funktion für \ $x\rightarrow\infty$ \ gegen \ 0 \ geht.
% \end{abc}
}

\ErgebnisC{AufganalysKurvDisk002}
{
\textbf{a)}\textbf{ii)} $0,\, -3$, 
\textbf{iii)} -1,\, 3, 
\textbf{iv)} $g(x)=-x-4$
}


% \Aufgabe[e]{}
{
% \begin{abc}
%  
% \item 
Gegeben sei die Funktion $f(x) = \cos \left(\dfrac{\pi}{2} \sin x\right)$.

\begin{iii}
 
 \item Bestimmen Sie den Definitionsbereich und den Wertebereich von $f$. 
 
 \item Zeigen Sie, dass die Funktion $f$ die Periodizität $\pi$ besitzt, d.h.\ zeigen Sie, dass $f(x+\pi) = f(x)$ für alle $x\in \R$ gilt.
 
 \item Bestimmen Sie alle Nullstellen von $f$.\\[0.5ex]
 \textbf{Hinweis:} Beachten Sie die Periodizität von $f$.
 
 \item Bestimmen Sie alle Extrema von $f$ und charakterisieren Sie diese.\\[0.5ex]
 \textbf{Hinweis:} Beachten Sie die Periodizität von $f$.
 
 \item Skizzieren Sie den Graphen von $f$ im Intervall $[-\pi,2\pi]$. 
 
\end{iii}
% 
% 
% 
% \item Bestimmen Sie den Grenzwert 
% \[
% \lim_{x\rightarrow \infty} (x+3)\left(\operatorname e^{2/x}-1\right)\,.
% \]
% 
% \end{abc}
}


\Loesung{
\textbf{Zu a)} 
\begin{iii}
\item  Der Wertebereich der Sinusfunktion ist $[-1,1]$. Auf $[-\pi/2,\pi/2]$ 
nimmt der Kosinus alle Werte von $0$ bis $1$ an. Somit gilt 
\[
D(f) =\R \qquad \text{ und } \qquad W(f) = [0,1]\,.
\]

\medskip
\item  Es gilt
\begin{align*}
 f(x+\pi) &  = \cos \left(\frac{\pi}{2} \sin (x+\pi)\right)\\[1ex]
 & = \cos \left(-\frac{\pi}{2} \sin x \right)\\[1ex]
 & = \cos \left(\frac{\pi}{2} \sin x\right)\\[1ex]
 & = f(x) 
\end{align*}
für alle $x\in \R$. Somit hat $f$ die Periode $\pi$. Hinweis: Es lässt sich zeigen, dass $f$ keine kleinere Periode als $\pi$ besitzt. 

\medskip
\item Aufgrund der Periodizität reicht es aus, die Nullstellen im Intervall $[0,\pi]$ zu bestimmen. Es gilt 
\begin{align*}
 f(x) & = \cos \left(\frac{\pi}{2} \sin x\right) = 0 \\[1ex]
 \Longleftrightarrow \quad \frac{\pi}{2} \sin x & = \frac{\pi}{2} \quad \text{oder} \quad \frac{\pi}{2} \sin x = \frac{3\pi}{2}\,.
\end{align*}
Der erste Fall liefert $x=\pi/2$. Der zweite Fall würde $\sin x = 3$ ergeben, was unmöglich ist. Die Menge aller Nullstellen ist somit
\[
N = \left\{\frac{\pi}{2} + k\pi \;\; \Big| \;\; k\in \mathbb Z \right\}\,.
\]

\medskip
\item Es gilt 
\[
f'(x) = - \sin \left(\frac{\pi}{2}\sin x\right)\frac{\pi}{2}\cos x\,.
\]
Es folgt 
\begin{align*}
 f'(x) & = - \sin \left(\frac{\pi}{2}\sin x\right)\frac{\pi}{2}\cos x = 0 \\[1ex]
 \Longleftrightarrow \quad \sin \left(\frac{\pi}{2}\sin x\right) & = 0 \quad \text{oder} \quad \cos x = 0\,.
\end{align*}
Der erste Fall liefert $x=0$ und $x=\pi$. Der zweite Fall ergibt $x=\pi/2$. Da $f$ nur Werte zwischen $0$ und $1$ annimmt, sind $x_1=(0,1)$ und $x_3=(\pi,1)$ Maxima, während $x_2 = (\pi/2,0)$ Minimum ist. Die Menge aller Maxima ist 
\[
E_{\max} = \{(k\pi,1) \mid k\in \mathbb Z \}\,.
\]
Die Menge aller Minima ist
\[
E_{\max} = \left\{ \left(\frac{\pi}{2}+k\pi,0\right) \mid k\in \mathbb Z \right\}\,.
\]

\medskip
\item
\end{iii}
\begin{center}
	\begin{pspicture}(-4,-1)(7,2)
	\psgrid[griddots=8,subgriddiv=0](-4,-1)(7,2)
	\psline[linewidth=1.2pt]{->}(-4,0)(7,0)
	\psline[linewidth=1.2pt]{->}(0,-1)(0,2)
        \psplot[plotpoints=100, plotstyle=curve]{-4}{7}
        {
        x 180 mul 3.14159 div sin 90 mul cos
        }
        \psdot(-1.5708,0)
        \psdot(1.5708,0)
        \psdot(4.7124,0)
        \psdot(  -3.14159,   1.00000)
        \psdot(   0.00000,   1.00000)
        \psdot(   3.14159,   1.00000)
        \psdot(   6.283  ,   1.00000)
        \psdot(  -1.57080,   0.00000)
        \psdot(   1.57080,   0.00000)
        \psdot(   4.71239,   0.00000)

%\cos \left(\dfrac{\pi}{2} \sin x\right)$.



	\end{pspicture} 

%\boxed{\includegraphics[width =0.25\textwidth]{./fig_f.eps}}
\end{center}

\bigskip
\textbf{Zu b)} Es gilt 
\[
(x+3)\left(\operatorname e^{2/x}-1\right) = \frac{\operatorname 
e^{2/x}-1}{\dfrac{1}{x+3}} =: \frac{f(x)}{g(x)}
\]
mit $\lim_{x\rightarrow \infty} f(x) = 0$ und $\lim_{x\rightarrow \infty} g(x) = 0$. Mit 
dem Satz von L'Hospital folgt dann
\begin{align*}
 \lim_{x\rightarrow \infty} \dfrac{f(x)}{g(x)} & = \lim_{x\rightarrow \infty }
\dfrac{f'(x)}{g'(x)} = \lim_{x\rightarrow \infty} \dfrac{2\operatorname e^{2/x} 
(x+3)^2}{x^2} \\[1ex]
& = \lim_{x\rightarrow \infty} 2\operatorname e^{2/x} \, \lim_{x\rightarrow \infty} 
\dfrac{(x+3)^2}{x^2} = 2\,.
\end{align*}
}

 

%\Aufgabe[v]{Interpolation}{
Gesucht ist ein Polynom vierten Grades 
$$p(x)=a_0 + a_1 x + a_2x^2+a_3x^3+a_4 x^4$$
mit folgenden Eigenschaften:
\begin{itemize}
\item Der Funktionswert bei $0$ ist $p(0)=0$. 
\item $p$ hat ein Minimum bei $(1,-1)$. 
\item $p$ hat einen Sattelpunkt bei $(2,0)$. 
\end{itemize}
\begin{abc}
\item Geben Sie die Bedingungen, die der Koeffizientenvektor  $(a_0,\hdots,\, a_4)^\top$ erf\"ullen
muss, in Form eines linearen Gleichungssystems an. (Es sollten sich sechs lineare Gleichungen mit
f\"unf Unbekannten ergeben.)
\item Berechnen Sie den Rang der Koeffizientenmatrix und der erweiterten Koeffizientenmatrix. Ist
das Gleichungssystem l\"osbar?
\item Welchen Wert $p(0)$ muss die Funktion bei $0$ annehmen, damit das System doch l\"osbar ist?
\item L\"osen Sie das so ge\"anderte Gleichungssystem. 
\item Skizzieren Sie die Funktion $p(x)$. 
\end{abc}

}


\Loesung{
\begin{abc}
\item Die angegebenen Eigenschaften der Funktion ergeben folgende Bedingungen:
\begin{align*}
p(0)=0                   &               &                          &\,\Rightarrow\,& a_0
          =& 0\\
p(1)=-1 \text{ minimal } &\,\Rightarrow\,& p(1)=-1,\, p'(1)=0       &\,\Rightarrow\,& a_0+a_1+a_2+a_3+a_4  =& -1\\
                         &               &                          &               & a_1+2a_2+3a_3+4a_4       =& 0\\
p(2)=0 \text{ Sattelpkt.}&\,\Rightarrow\,& p(2)=0,\, p'(2)=p''(2)=0 &\,\Rightarrow\,&
a_0+2a_1+4a_2+8a_3+16a_4 =& 0\\
                         &               &                          &               & a_1+4a_2+12a_3+32a_4     =& 0\\
                         &               &                          &               &
2a_2+12a_3+48a_4     =& 0
\end{align*}
Als Matrixgleichung ergibt sich 
$$\vec A \vec a = \vec b$$
mit 
$$\vec A = \begin{pmatrix}
1  &  0&  0&  0&  0 \\
1  &  1&  1&  1&  1 \\
0  &  1&  2&  3&  4 \\
1  &  2&  4&  8& 16 \\
0  &  1&  4& 12& 32 \\
0  &  0&  2& 12& 48 \end{pmatrix}
\text{ und }\vec b = \begin{pmatrix}
0\\-1\\0\\0\\0\\0\end{pmatrix}.$$
\item Der Rang von $\vec A$ ist h\"ochstens 5, da die Matrix nur f\"unf Spalten hat. Die Determinante
der ersten f\"unf Zeilen der Matrix ist
\begin{align*}
\det \begin{pmatrix}
1  &  0&  0&  0&  0 \\
1  &  1&  1&  1&  1 \\
0  &  1&  2&  3&  4 \\
1  &  2&  4&  8& 16 \\
0  &  1&  4& 12& 32 \end{pmatrix}
=& \det\begin{pmatrix}
  1&  1&  1&  1 \\           
  1&  2&  3&  4 \\           
  2&  4&  8& 16 \\           
  1&  4& 12& 32 \end{pmatrix}=\det \begin{pmatrix}
 1 & 0 & 0 & 0\\
 1 & 1 & 2 & 3\\
 1 & 2 & 6 &14\\
 1 & 3 & 11&31\end{pmatrix}\\
=& \det \begin{pmatrix}
  1 & 2 & 3\\             
  2 & 6 &14\\             
  3 & 11&31\end{pmatrix} = \det\begin{pmatrix}
  1 & 0 & 0\\           
  2 & 2 & 8\\           
  3 &  5&22\end{pmatrix}\\
=& \det\begin{pmatrix} 2 & 8 \\ 5 & 22\end{pmatrix} = 4\neq 0
\end{align*}
Also haben die ersten f\"unf Zeilen den Rang 5 und damit auch 
$$\Rang \vec A = 5.$$
Die Determinante der erweiterten Systemmatrix $(\vec A|\vec b)$ ist 
\begin{align*}
\det(\vec A|\vec b)=& \det\begin{pmatrix}
1  &  0&  0&  0&  0 &  0 \\
1  &  1&  1&  1&  1 & -1 \\
0  &  1&  2&  3&  4 &  0 \\
1  &  2&  4&  8& 16 &  0 \\
0  &  1&  4& 12& 32 &  0 \\
0  &  0&  2& 12& 48 &  0 \end{pmatrix}= \det \begin{pmatrix}
  1&  1&  1&  1 & -1 \\           
  1&  2&  3&  4 &  0 \\           
  2&  4&  8& 16 &  0 \\           
  1&  4& 12& 32 &  0 \\           
  0&  2& 12& 48 &  0 \end{pmatrix}\\
=& -\det \begin{pmatrix}
  1&  2&  3&  4  \\           
  2&  4&  8& 16  \\           
  1&  4& 12& 32  \\           
  0&  2& 12& 48  \end{pmatrix} = -\det\begin{pmatrix}
  1&  2&  3&  4  \\           
  0&  0&  2&  8  \\           
  0&  2&  9& 28  \\           
  0&  2& 12& 48  \end{pmatrix} \\
=& -\det\begin{pmatrix}
0&  2&  8  \\              
2&  9& 28  \\              
2& 12& 48  \end{pmatrix} = -\det \begin{pmatrix}
0&  2&  8  \\              
2&  9& 28  \\              
0&  3& 20  \end{pmatrix} \\
=& 2\cdot \det \begin{pmatrix} 2 & 8 \\ 3 & 20\end{pmatrix} = 32\neq 0
\end{align*}
Also hat $(\vec A | \vec b)$ Vollrang, $\Rang (\vec A|\vec b)=6$ und wegen 
$$\Rang \vec A \neq \Rang (\vec A|\vec b)$$
ist das Gleichungssystem $\vec A \vec a=\vec b$ nicht l\"osbar. 
\item Der Funktionswert $p(0)=c$ taucht in der oberen rechten Ecke der Matrix $(\vec A|\tilde {\vec  b})$
auf. Dabei ist $\tilde {\vec b} = (c,-1,0,0,0,0)^\top$. \\

Die Determinante \"andert sich dadurch wie folgt: 
\begin{align*}
\det (\vec A|\tilde {\vec b})=& \det(\vec A|\vec b)-c\cdot \det \begin{pmatrix}
1  &  1&  1&  1&  1  \\
0  &  1&  2&  3&  4  \\
1  &  2&  4&  8& 16  \\
0  &  1&  4& 12& 32  \\
0  &  0&  2& 12& 48  \end{pmatrix}\\
=&32-c\cdot \det \begin{pmatrix}
  1&  2&  3&  4  \\             
  2&  4&  8& 16  \\             
  1&  4& 12& 32  \\             
  0&  2& 12& 48  \end{pmatrix} - c \cdot \det\begin{pmatrix}
  1&  1&  1&  1  \\             
  1&  2&  3&  4  \\             
  1&  4& 12& 32  \\             
  0&  2& 12& 48  \end{pmatrix}\\
=& 32 -c\cdot (-32) - c \cdot \det\begin{pmatrix}
  1&  1&  1&  1  \\             
  0&  1&  2&  3  \\             
  0&  3& 11& 31  \\             
  0&  2& 12& 48  \end{pmatrix}= 32+32c -c\cdot \det\begin{pmatrix}
  1&  2&  3  \\           
  3& 11& 31  \\           
  2& 12& 48  \end{pmatrix}\\
=& 32+32c-c\cdot \begin{pmatrix} 5 & 22\\ 8 & 42\end{pmatrix}=32-2c
\end{align*}
Die Determinante verschwindet also f\"ur $p(0)=c= 16$. Damit gilt dann
$$\Rang \vec A = \Rang (\vec A|\tilde{\vec b})=5$$
und das System besitzt eine eindeutige L\"osung. 
\item Die L\"osung dieses Systems ist
$$\begin{array}{rrrrr|r|l}
1  &  0&  0&  0&  0 & 16 & \text{                     }\\
1  &  1&  1&  1&  1 & -1 & \text{ -1. Zeile           }\\
0  &  1&  2&  3&  4 &  0 & \text{                     }\\
1  &  2&  4&  8& 16 &  0 & \text{ -1. Zeile           }\\
0  &  1&  4& 12& 32 &  0 & \text{                     }\\
0  &  0&  2& 12& 48 &  0 & \text{                     }\\\hline

1  &  0&  0&  0&  0 & 16 & \text{                     }\\
0  &  1&  1&  1&  1 &-17 & \text{                     }\\
0  &  1&  2&  3&  4 &  0 & \text{ -2. Zeile           }\\
0  &  2&  4&  8& 16 &-16 & \text{ -2$\cdot$ 2. Zeile  }\\
0  &  1&  4& 12& 32 &  0 & \text{ -2. Zeile           }\\
0  &  0&  2& 12& 48 &  0 & \text{                     }\\\hline

1  &  0&  0&  0&  0 & 16 & \text{                     }\\
0  &  1&  1&  1&  1 &-17 & \text{                     }\\
0  &  0&  1&  2&  3 & 17 & \text{                     }\\
0  &  0&  2&  6& 14 & 18 & \text{ -2$\cdot$ 3. Zeile  }\\
0  &  0&  3& 11& 31 & 17 & \text{ -3 $\cdot$ 3. Zeile }\\
0  &  0&  2& 12& 48 &  0 & \text{ -2$\cdot$ 3. Zeile  }\\\hline

1  &  0&  0&  0&  0 & 16 & \text{                     }\\
0  &  1&  1&  1&  1 &-17 & \text{                     }\\
0  &  0&  1&  2&  3 & 17 & \text{                     }\\
0  &  0&  0&  2&  8 &-16 & \text{ $\cdot 1/2$         }\\
0  &  0&  0&  5& 22 &-34 & \text{-5/2$\cdot$ 4. Zeile }\\
0  &  0&  0&  8& 42 &-34 & \text{-4 $\cdot$ 4. Zeile  }\\\hline

1  &  0&  0&  0&  0 & 16 & \text{                     }\\
0  &  1&  1&  1&  1 &-17 & \text{                     }\\
0  &  0&  1&  2&  3 & 17 & \text{                     }\\
0  &  0&  0&  1&  4 & -8 & \text{                     }\\
0  &  0&  0&  0&  2 &  6 & \text{    $\cdot 1/2$      }\\
0  &  0&  0&  0& 10 & 30 & \text{-5 $\cdot$ 5. Zeile  }\\\hline

1  &  0&  0&  0&  0 & 16 & \text{                     }\\
0  &  1&  1&  1&  1 &-17 & \text{                     }\\
0  &  0&  1&  2&  3 & 17 & \text{                     }\\
0  &  0&  0&  1&  4 & -8 & \text{                     }\\
0  &  0&  0&  0&  1 &  3 & \text{                     }\\
0  &  0&  0&  0&  0 &  0 & \text{                     }
\end{array}$$
Es ergibt sich die L\"osung 
$$\vec a =( 16,\, -48,\, 48,\, -20,\, 3)^\top.$$
\item \quad\\
\begin{minipage}{.4\textwidth}
\psset{xunit=1cm, yunit=.5cm, runit=1cm}
\begin{pspicture}(-1,-1)(3,17)
\psgrid[subgriddiv=1,griddots=10,gridlabels=.3](-1,-1)(3,17)
\psplot[plotpoints=200, plotstyle=curve]
{0}{3}
{16 -48 x mul add 48 x mul x mul add -20 x mul x mul x mul add 3 x mul x mul x mul x mul add}
\psdot(0,16)
\psdot(1,-1)
\psdot(2,0)
\end{pspicture}
\end{minipage}

\end{abc}
}

\ErgebnisC{AufganalysIntrPoly001}
{
{\textbf{a)}} Die Systemmatrix ist $\Vek A = \begin{pmatrix} 
1 & 0 & 0 & 0 & 0 \\  
1 & 1 & 1 & 1 & 1 \\  
0 & 1 & 2 & 3 & 4 \\  
1 & 2 & 4 & 8 & 16\\  
0 & 1 & 4 & 12& 32\\  
0 & 0 & 2 & 12& 48\\  
\end{pmatrix}$, die rechte Seite des Systems $\vec b=(0,-1,0,0,0,0)^\top$. 
\textbf{d)} $\vec a=(16,\, -48,\, 48,\, -20,\, 3)^\top$
}






\ifthenelse{\boolean{mitLoes}}{\cleardoublepage}{}
\ifthenelse{\boolean{mitErg}}{
\ruleBig
\Ergebnisse}{}


\end{twocolumn}
\end{document}
