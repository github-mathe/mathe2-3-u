
\documentclass[a4paper,12pt]{article}
\usepackage[ngerman]{babel}
% \usepackage[ngerman]{babel}
\usepackage[utf8]{inputenc}
\usepackage[T1]{fontenc}
\usepackage{lmodern}
\usepackage{titling}
\usepackage{geometry}
\geometry{left=3cm, right=3cm, top=2cm, bottom=3cm}
\usepackage{setspace}
\onehalfspacing

\usepackage{ifthen}

\usepackage{graphicx}
% \usepackage{pstricks}
% \usepackage{relsize}
% \usepackage[decimalsymbol=comma,exponent-product = \cdot, per=frac]{siunitx}
% \sisetup{range-phrase=\,bis\,}

\usepackage{xargs}
\usepackage{calc}
\usepackage{amsmath}
\usepackage{amsfonts}
\usepackage{mathtools}
\usepackage{amssymb}

\usepackage{cancel}
\usepackage{trfsigns}
\usepackage{array}
\usepackage{enumerate}
\usepackage{enumitem}

\usepackage{caption}
\usepackage{subcaption}

\usepackage{multicol}

\usepackage{pdflscape}
\usepackage[table]{xcolor}

\usepackage{float}

%%%%%%%%%%%%%%%%%%%%%%%%%%%%%%%%%%%%%%%%%%%%%%%%%%%%%%%%%%%%%%%%%%%%%%%%%%%%%%%%
\newboolean{WITHPSTRICKS}
\setboolean{WITHPSTRICKS}{false}


\newcommand{\PROFESSOR}{Prof.\ Dr.\ Thomas Carraro}
\newcommand{\ASSISTANT}{\setlength{\tabcolsep}{0pt}\begin{tabular}{l}Dr.\ Frank Gimbel\\Janna Puderbach\end{tabular}}

\newcommand{\Jahr}{2022}
% \newcommand{\Trimester}{HT}
\newcommand{\Trimester}{WT}
\newcommand{\Kurs}{Mathematik II}
\newcommand{\TYPE}{Aufgabenblatt}
\newcommand{\BLATT}{6}
\newcommand{\TOPIC}{Differentialrechnung in $\R^n$}

%%%%%%%%%%%%%%%%%%%%%%%%%%%%%%%%%%%%%%%%%%%%%%%%%%%%%%%%%%%%%%%%%%%%%%%%%%%%%%%%
\newboolean{mitLoes}
\setboolean{mitLoes}{false}
 \setboolean{mitLoes}{true}

%%%%%%%%%%%%%%%%%%%%%%%%%%%%%%%%%%%%%%%%%%%%%%%%%%%%%%%%%%%%%%%%%%%%%%%%%%%%%%%%

%\setboolean{WITHPSTRICKS}{false}
\setboolean{WITHPSTRICKS}{true}


\usepackage{tikz}
\usetikzlibrary{arrows,automata,backgrounds,calendar,decorations.pathmorphing,fadings,shadings,calc,intersections}
\usetikzlibrary{decorations.pathreplacing}
\usetikzlibrary{decorations.shapes}
\usetikzlibrary{decorations.footprints}
\usetikzlibrary{decorations.text}
\usetikzlibrary{positioning}
\usetikzlibrary{through}

\ifthenelse{\boolean{WITHPSTRICKS}}{%
\usepackage{auto-pst-pdf}
\usepackage{pstricks,pst-plot,pst-text}
}{}

\usepackage{pgfplots}

%%%%%%%%%%%%%%%%%%%%%%%%%%%%%%%%%%%%%%%%%%%%%%%%%%%%%%%%%%%%%%%%%%%%%%%%%%%%%%%%
\usepackage{mbdefAufgaben}

%%%%%%%%%%%%%%%%%%%%%%%%%%%%%%%%%%%%%%%%%%%%%%%%%%%%%%%%%%%%%%%%%%%%%%%%%%%%%%%%
\newboolean{mitErg}
\setboolean{mitErg}{false}

%%%%%%%%%%%%%%%%%%%%%%%%%%%%%%%%%%%%%%%%%%%%%%%%%%%%%%%%%%%%%%%%%%%%%%%%%%%%%%%%
\newcounter{Aufg}
\setcounter{Aufg}{0}
\newcounter{Blatt}
\setcounter{Blatt}{\BLATT}

%%%%%%%%%%%%%%%%%%%%%%%%%%%%%%%%%%%%%%%%%%%%%%%%%%%%%%%%%%%%%%%%%%%%%%%%%%%%%%%%
\usepackage{KopfEnglish}

% Seitenraender
\textwidth = 285mm
\textheight = 180mm
\leftmargin 5mm
\oddsidemargin = -20mm
\evensidemargin = -20mm
\topmargin = -25mm
\parindent 0cm
\columnsep 2cm

% % % Aufgabenstellung
% % % Schwierungkeitsgrad mit "e" , "f" oder "v" angeben
% % % "e" Einführung   
% % % "f" Festigung
% % % "v" Vertiefung  

\newcommand{\Aufgabe}[3][]{
\stepcounter{Aufg}
\subsubsection*{Aufgabe 
\arabic{Blatt}.\arabic{Aufg}\ifthenelse{\equal{#1}{e}}{}{\ifthenelse{\equal{#1}{f}}{
$\!\!{}^\star$}{\ifthenelse{\equal{#1}{v}}{$^{\star\star}$}{}}}{: #2}}
{#3}
}
% % % Ergebnisse jeweils am Ende des Aufgabenblattes Anzeigen
\newcommand{\Ergebnisse}{}
\makeatletter
\newcommand{\Ergebnis}[1]{
	\g@addto@macro{\Ergebnisse}{#1}
}
\makeatother
\makeatletter
\newcommand{\ErgebnisC}[2]{
\@ifundefined{c@#1}
{\newcounter{#1}}
{}
\setcounter{#1}{\theAufg}

\ifthenelse{\boolean{mitErg}}{	\g@addto@macro{\Ergebnisse}{\subsubsection*{Ergebnisse zu Aufgabe \arabic{Blatt}.\arabic{#1}:}
}%
	\g@addto@macro{\Ergebnisse}{#2}}{}
}
\makeatother


% % % Lösungen
\newcommand{\Loesung}[1]{
	\ifthenelse{\boolean{mitLoes}}
	{\subsubsection*{Lösung \arabic{Blatt}.\arabic{Aufg}:}
		#1}
	{}
}
% % % % % % % % % % % % % % % % % % % % % % % % % % % % % % % % % % % % % % % % % % % % % % % % % % % % % %
% % % % % % % % % % % % % % % % % % % % % % % % % % % % % % % % % % % % % % % % % % % % % % % % % % % % % %
% % % % % % % % % % % % % % % % % % % % % % % % % % % % % % % % % % % % % % % % % % % % % % % % % % % % % %
\begin{document}
\begin{twocolumn}
% % % % % % % % % % % % % % % % % % % % % % % % % % %

%%%%%%%%%%%%%%%%%%%%%%%%%%%%%%%%%%%%%%%%%%%%%%%%%%%%%%%%%%%%%%%%%%%%%%%%%%%%%%%%
% Set the TITLE of the sheet here:
%\uebheader{\Kurs}{\arabic{Blatt}}{\Trimester\,\Jahr}{\TOPIC}
%\uebheader{\Kurs}{\arabic{Blatt}}{\Trimester\,\Jahr}{\TOPIC}
\uebheader{\Kurs}{\arabic{Blatt}}{\Trimester\,\Jahr}{\TOPIC}
\ruleBig

\setboolean{mitErg}{false}
%\setboolean{mitErg}{true}


%%%%%%%%%%%%%%%%%%%%%%%%%%%%%%%%%%%%%%%%%%%%%%%%%%%%%%%%%%%%%%%%%%%%%%%%%%%%%%%%
% Set the INTRODUCTION section of the sheet here:
% \input{introduction.tex}
\renewcommand{\d }{\mathrm{d}}

\textbf{Einführende Bemerkungen}

\begin{itemize}
\item Vermeiden Sie die Verwendung von Taschenrechnern oder Online-Ressourcen.
\item Die mit einem Stern *) markierten (Teil-)Aufgaben entfallen in diesem Trimester. Stattdessen werden einzelne Online-Aufgaben im ILIAS-Kurs kenntlich gemacht, zu denen Sie dort Ihre L\"osungswege zur Korrektur hochladen k\"onnen. 
\item Die mit zwei Sternen  **) markierten (Teil-)Aufgaben richten sich an Studierende, die die \"ubrigen Aufgaben bereits gel\"ost haben und die Inhalte weiter vertiefen m\"ochten. 
\end{itemize}

\ruleBig

%Mathe II Blatt 6


\Aufgabe[e]{Vektor- und Matrixwertige Funktionen}{
Gegeben seien
\begin{align*}
\vec A(t) =& \begin{pmatrix} t&t^2\\\sqrt t& t^5\end{pmatrix},& \quad &&\vec B(t)=& \begin{pmatrix}\sin
t & \cos t\\ \EH{t}& \cosh t\end{pmatrix}\,,\\
\vec c(t)=& \left(t^3,\, \sqrt{t^5},\, \frac 1t\right)^\top,&&&\vec d(t) =
& \left( \EH{-t^2},\, \sin(\sqrt t),\, \tanh(t/2)\right)^\top.
\end{align*}
Berechnen Sie 
\begin{abc}
\item $\frac d{dt}\Bigl( \vec A(t)\vec B(t)\Bigr)$ und
\item $\frac d{dt}\Bigl( \vec c(t)\times \vec d(t)\Bigr)$
\end{abc}
auf die folgenden beiden Arten: 
\begin{itemize}
\item Indem sie vor der Differentiation die Produkte $\vec A\vec B$ bzw. $\vec c\times \vec d$
bilden und die Ergebnisfunktionen ableiten. 
\item Indem Sie zun\"achst die Ableitungen der einzelnen Funktionen bilden und diese dann gem\"aß
einer Produktregel verrechnen. 
\end{itemize}

}

\Loesung{
\begin{abc}
\item Zun\"achst ist
\begin{align*}
&&\Vec A \Vec B =& \begin{pmatrix} t\sin t +t^2\EH{t} & t\cos t +t^2 \cosh t\\
\sqrt t \sin t + t^5\EH{t} & \sqrt t \cos t + t^5 \cosh t\end{pmatrix}\\
\Rightarrow && &\frac{d}{dt}\left( \vec A(t)\vec B(t)\right) \\
&&=& \begin{pmatrix}
\sin t + t\cos t + (2t+t^2)\EH{t} & \cos t -t\sin t + 2t \cosh t + t^2 \sinh t\\
\frac {\sin t}{2\sqrt t} + \sqrt t \cos t + (5t^4+t^5)\EH{t} & \frac{\cos t}{2\sqrt t} -\sqrt t \sin
t +5t^4\cosh t + t^5 \sinh t \end{pmatrix}\,.
\end{align*}
Ebenso ist 
\begin{align*}
\vec A'(t)=& \begin{pmatrix} 1 & 2t\\\frac 1 {2\sqrt t} & 5t^4\end{pmatrix}\\
\vec B'(t)=& \begin{pmatrix} \cos t & -\sin t\\ \EH t & \sinh t\end{pmatrix}\\
\frac{d}{dt}\left( \vec A(t)\vec B(t)\right)=&\vec A'(t)\vec B(t)+\vec A(t) \vec B'(t)\\
=& \begin{pmatrix} \sin t + 2t\EH t & \cos t + 2t\cosh t \\
\frac{\sin t }{2\sqrt t} + 5t^4\EH t & \frac{\cos t }{2\sqrt t} + 5t^4\cosh t\end{pmatrix}\\
&+ \begin{pmatrix}
t\cos t + t^2 \EH t &      -t\sin t +t^2\sinh t\\
\sqrt t \cos t + t^5\EH t & -\sqrt t \sin t + t^5\sinh t\end{pmatrix}\\
=& \begin{pmatrix}
\sin t + t \cos t + (2t+t^2)\EH t    & \cos t - t \sin t +2t \cosh t + t^2 \sinh t\\
\frac{\sin t }{2\sqrt t} + \sqrt t \cos t + (5t^4+t^5)\EH t & \frac{\cos t }{2\sqrt t } -\sqrt
t \sin t + 5 t^4 \cosh t + t^5 \sinh t \end{pmatrix}\,.
\end{align*}

\item Das Vektorprodukt von $\vec c$ und $\vec d$ ist
\begin{align*}
&&\vec c(t)\times \vec d(t) =& \begin{pmatrix}t^{5/2}\tanh(t/2)-\frac{\sin(\sqrt t)}t\\
\frac{\EH {-t^2}}t-t^3\tanh\frac t2\\
t^3\sin(\sqrt t)-t^{5/2}\EH {-t^2}\end{pmatrix}\\
\Rightarrow&&\frac d{dt} (\vec c(t)\times \vec d(t)) =& \begin{pmatrix}
\frac 12 t^{3/2}\left( 5\tanh\frac t2 + \frac t{\cosh^2(t/2)}\right) +\frac{\sin(\sqrt
t)}{t^2}-\frac{\cos(\sqrt t)}{2\sqrt t \, t}\\
\EH{-t^2}\left( -2 - \frac 1{t^2}\right) -3t^2\tanh \frac t 2 - \frac {t^3}{2\cosh^2\frac t2}\\
3t^2\sin(\sqrt t) + \frac{t^{5/2}}{2}\cos(\sqrt t) - \EH{-t^2}\left( \frac 52 t^{3/2}-2t^{7/2}\right)
\end{pmatrix}.
\end{align*}
Durch separates Differenzieren ergibt sich 
\begin{align*}
&&\vec c'(t) =& \begin{pmatrix}3t^2\\\frac 52 t^{3/2}\\-\frac 1{t^2}\end{pmatrix},\, 
\vec d'(t)=  \begin{pmatrix} 
-2t\EH{-t^2}\\
\frac{\cos(\sqrt t)}{2\sqrt t}\\ 
\frac{1}{2\cosh^2\frac t 2}\end{pmatrix}\\
\Rightarrow && &\vec c'(t)\times \vec d(t) + \vec c (t) \times \vec d'(t)\\
&&= & \begin{pmatrix}
\frac {5t^{3/2}\tanh\frac t2}{2}  +\frac{\sin(\sqrt t)}{t^{2}}\\
\frac{-\EH{-t^2}}{t^2}-3t^2\tanh\frac t2\\
3t^2\sin(\sqrt t) -\frac 52t^{3/2}\EH {-t^2}
\end{pmatrix} + \begin{pmatrix}
\frac{t^{5/2}}{2\cosh^2\frac t2} - \frac{\cos(\sqrt t)}{2t^{3/2}}\\
-2\EH{-t^2}-\frac{t^3 }{2\cosh^2\frac t2} \\
\frac{t^{5/2}\cos\sqrt t}{2} +2t^{7/2}\EH{-t^2}\end{pmatrix}\,.
\end{align*}
\end{abc}
}

\ErgebnisC{AufganalysMatrFnkt001}
{
{ a)} $\begin{pmatrix}
t\cos t + t^2 \EH{t}+\sin t + 2t\EH{t} & -t\sin t + t^2 \sinh t + \cos t + 2t\cosh t\\
\sqrt t \cos t + t^5\EH{t} + \frac {\sin t}{2\sqrt t } + 5t^4\EH{t}&
-\sqrt t \sin t + t^5\sinh t + \frac{\cos t}{2\sqrt t}  + 5 t^4\cosh t\end{pmatrix}.$\\
{ b)} $\begin{pmatrix} \frac 52 t^{3/2}\tanh \frac t2 + \frac{t^{5/2}}{2\cosh^2 \frac t2} - \frac{\cos \sqrt
t}{2t^{3/2}}+\frac{\sin\sqrt t}{t^2}\\
-2\EH{-t^2}-\frac{\EH{-t^2}}{t^2}-3t^2\tanh\frac t2-\frac{t^3}{2\cosh^2 \frac t2}\\
 3t^2\sin\sqrt t +\frac{t^{5/2}\cos\sqrt t}{2} - \frac 52 t^{3/2}\EH{-t^2}+2t^{7/2}\EH{-t^2}
\end{pmatrix}$

}
 

 \Aufgabe[e]{Kurvendiskussion}{
Bestimmen Sie den maximalen Definitionsbereich, die Symmetrie, alle Nullstellen, sowie Art und Lage
der kritischen Punkte und Wendepunkte der rellen Funktion
$$  f(x) =x \sqrt{16-x^2}.$$

}

\Loesung{
\begin{iii}
\item Der maximale Definitionsbereich ist \ $D = [-4,4]$ .
\item Die Funktion \ $f$ \ ist ungerade bzw. punktsymmetrisch zum Ursprung:
$$
  f(x) = x\sqrt{16-x^2} = -(-x)\sqrt{16-(-x)^2}=-f(-x)\ .
$$
\item  Die Nullstellen sind 
\begin{align*}
&&0=& x \sqrt{16-x^2}\\
\Leftrightarrow&&x=&0\,\text{ oder }\, 16=x^2\\
\Leftrightarrow&&x\in& \{0,\, 4,\, -4\}
\end{align*}
\item Kritische Punkte liegen bei $x\in D$ mit: 
\begin{align*}
&&0=& f'(x) = \sqrt{16-x^2}-\frac{x\cdot x}{\sqrt{16-x^2}}=\frac{16-2x^2}{\sqrt{16-x^2}}\\
\Leftrightarrow &&\pm 4 =& \sqrt 2 x \quad\Leftrightarrow \quad x=\pm 2\sqrt{2}
\end{align*}
Die Funktion $f$ selbst hat an den Grenzen des Definitionsbereiches $D$ sowie im Ursprung den Wert
$f(0)=f(\pm 4)=0$ 
F\"ur alle anderen $x>0$ ist $f(x)>0$. Also muss im Punkt $x=+2\sqrt 2$ das absolute (und damit auch
ein relatives) Maximum $f(2\sqrt 2)=8$ der Funktion liegen. \\
Mit der Symmetrie der Funktion folgt, dass in $x=-2\sqrt 2$ ein Minimum $f(-2\sqrt 2)=-8$ liegt. 
\item  Wendepunkte und Konvexität:\\
Zun\"achst ist 
\begin{align*}
f''(x)=& \frac{-4x\sqrt{16-x^2}-(16-2x^2)\cdot (-2x)\frac 12(16-x^2)^{-1/2}}{16-x^2}\\
=& \frac{-4x(16-x^2)+x(16-2x^2)}{(16-x^2)^{3/2} }\\
=& \frac{2x^3-48x}{(16-x^2)^{3/2}}
\end{align*}
Die einzige reelle Nullstelle des Z\"ahlers im Definitionsbereich $]-4,4[$ ist $x=0$ und es gilt
$$
   f''(x)=	\begin{cases}
					>0 & \text{für }x \in ]-4,0[\\
					<0 & \text{für }x \in ]0,4[ 
				\end{cases}
$$
Also liegt in $(0,0)$ ein Wendepunkt, links davon ist $f$ konvex und rechts davon konkav. \\

\unitlength1cm
\begin{picture}(2.5,9)
\put(5.5,4){\begin{picture}(0,0)\unitlength1cm
\thinlines
\put( -4.5,0){\vector(1,0){10}}
\multiput( -4 , -0.1)(1,0){9}{\line(0,1){0.2}}
\put(.9,-0.4){{ 1}}
\put( 0,-4.25){\vector(0,1){8.75}}
\multiput( -0.1,-4)(0,1){9}{\line(1,0){0.2}}
\put(-0.5,.9){{ 2}}

\thicklines
\bezier{200}( -4.000, -0.002)( -4.000, -2.748)( -3.333, -3.685)
\bezier{200}( -3.333, -3.685)( -3.047, -4.089)( -2.667, -3.975)
\bezier{200}( -2.667, -3.975)( -2.365, -3.885)( -2.000, -3.464)
\bezier{200}( -2.000, -3.464)( -1.697, -3.114)( -1.333, -2.514)
\bezier{200}( -1.333, -2.514)( -1.041, -2.032)( -0.667, -1.315)
\bezier{200}( -0.667, -1.315)( -0.445, -0.891)(  0.000,  0.000)
\bezier{200}(  0.000,  0.000)(  0.445,  0.891)(  0.667,  1.315)
\bezier{200}(  0.667,  1.315)(  1.041,  2.032)(  1.333,  2.514)
\bezier{200}(  1.333,  2.514)(  1.697,  3.114)(  2.000,  3.464)
\bezier{200}(  2.000,  3.464)(  2.365,  3.885)(  2.667,  3.975)
\bezier{200}(  2.667,  3.975)(  3.047,  4.089)(  3.333,  3.685)
\bezier{200}(  3.333,  3.685)(  4.000,  2.748)(  4.000,  0.002)

\end{picture}}
\end{picture}

\end{iii}
}

\ErgebnisC{AufganalysKurvDisk001}
{
$D(f)= [-4,4]$, $f$ ist ungerade, Nullstellen: $x=0,\pm 4$, Extrema bei $x=\pm 2\sqrt 2$, \\
Wendepunkt
bei $x=0$

}


\Aufgabe[e]{}{
\begin{abc}
\item Bestimmen Sie die folgenden Integrale
\begin{align*}
&\int (x^2+3x-4)\d x&,\qquad & 
\int\limits_{x=-4}^4 (x^3-x)\d x\\
&\int\limits_{x=-1}^3 \left(5 x^4 + \frac{x^3}3 +2\right) \d x&,\qquad&
\int \left( x^2+7x-\frac{x^5}5\right)\d x.
\end{align*} 
\item Bestimmen Sie desweiteren
\begin{align*}
&\int \cos(x)\d x&,\qquad & 
\int\limits_{x=2}^8 \frac 1x\d x\\
&\int\limits_{x=0}^{2\pi} \sin(x) \d x&,\qquad&
\int\limits_{x=0}^{\pi/2}\cos(x)\d x.
\end{align*} 
\end{abc}
}

\Loesung{
\begin{abc}
\item \begin{align*}
\int\left( x^2+3x-4\right)\d x =& \frac{x^3}3+\frac {3x^2}2 - 4x + C\\
\int\limits_{x=-4}^4\left(x^3-x\right)\d x=& \left[\frac{x^4}4-\frac{x^2}2\right]_{x=-4}^4=0\\
&\begin{array}{l}
\text{(Dasselbe Ergebnis kann man auch ohne Rechnung}\\
\text{ begr\"unden, da eine ungerade Funktion auf }\\
\text{ einem symmetrischen Intervall integriert wird.)}\end{array}\\
\int\limits_{x=-1}^3\left(5x^4+\frac{x^3}3+2\right) \d x =& \left[x^5+\frac{x^4}{12}+2x\right]_{x=-1}^3\\
=& 3^5+\frac{3^4}{12}+2\cdot 3-\left( (-1)^5+\frac{(-1)^4}{12}+2\cdot(-1)\right)\\
=& 243+\frac{27}4+6+1-\frac 1{12}+2 \\
=& 252+\frac{81-1}{12} = \frac{776}3\\
\int\left(x^2+7x-\frac{x^5}5\right)\d x=& \frac{x^3}3+\frac{7x^2}2-\frac{x^6}{30}+C
\end{align*}
\item \begin{align*}
\int \cos(x)\d x=&\sin(x)+C\\
\int\limits_{x=2}^8 \frac 1x\d x=& \Bigl.\ln|x|\Bigr|_{x=2}^8=\ln(8)-\ln(2)=\ln\frac{8}2=\ln(4)\\
\int\limits_{x=0}^{2\pi} \sin(x) \d x=& \Bigl.-\cos(x)\Bigr|_{x=0}^{2\pi}=-\cos(2\pi)+\cos(0)=0\\
&\begin{array}{l}
\text{(Auch hier h\"atte man mit der Symmetrie }\\
\text{ der Sinusfunktion argumentieren k\"onnen.)}\end{array}\\
\int\limits_{x=0}^{\pi/2}\cos(x)\d x=& \Bigl.\sin(x)\Bigr|_{x=0}^{\pi/2}=\sin\left(\frac\pi 2\right)-\sin(0)=1-0=1.
\end{align*}

\end{abc}
}

 \ErgebnisC{analysInteGral013}
 {
\textbf{a) }$\frac{x^3}3+\frac {3x^2}2 - 4x + C$, 
$0$, 
$\frac{776}3$,
$\frac{x^3}3+\frac{7x^2}2-\frac{x^6}{30}+C$\\
\textbf{b) }$\sin(x)+C$, 
$\ln(4)$,
$0$,
$1$
 }




\Aufgabe[e]{~}{
Bestimmen Sie den Fl\"acheninhalt der Fl\"ache, die durch die Gerade $f_1(x)=x+2$ und die Parabel $f_2(x)=4-x^2$ begrenzt wird. 
}

\Loesung{
Zun\"achst bestimmt man die Schnittpunkte der beiden Funktionsgraphen, indem man \(f_1(x) = f_2(x)\) l\"ost:
\[
 x+2=4-x^2 \Rightarrow x^2+x-2=0 \Rightarrow x=-2 \text{ oder } x=1
\]
Die beiden Funktionsgraphen schneiden sich in \(x=-2\) und \(x=1\). 
Skizziert man die Funktionsgraphen, erkennt man, wie die Fl\"ache von den beiden Funktionsgraphen begrenzt wird, \(f_2(x)\) liegt hierbei \"uber \(f_1(x)\).


\begin{center}
\psset{yunit=1cm}
\begin{pspicture}(-3,-1)(3,4)
\psgrid(-3,-1)(3,4)
\psplot[plotpoints=100,plotstyle=curve]
{-2.21}{2.21}
{4 x neg x mul add}
\psplot[plotpoints=100,plotstyle=curve, fillstyle=solid, fillcolor=lightgray]
{-2}{1}
{4 x neg x mul add}
\psline[showpoints=true](-2,0)(1,3)
\psline(-3,-1)(2,4)
\put(2.1,.1){$f_2(x)$}
\put(1.6,3.3){$f_1(x)$}
\end{pspicture}
\end{center}

Die Funktionen schneiden sich lediglich in zwei Punkten, also gibt es nur ein zu
ber\"ucksichtigendes Fl\"achenst\"uck. Dessen Fl\"acheninhalt wird berechnet, indem man die Differenz der Integrale der beiden Funktionen bestimmt.
\begin{align*}
A & = \int\limits_{-2}^1 f_2(x)\,\d x- \int_{-2}^1 f_1(x) \,\d x = \int_{-2}^1 \left(f_2(x)-f_1(x) \right) \,\d x\\[2ex]
&= \int\limits_{-2}^1(-x^2-x+2) \,\d x = \left[-\dfrac{1}{3} x^3-\dfrac{1}{2}x^2 +2x\right]_{-2}^1
= \frac 92\,.
\end{align*}

}


\ErgebnisC{b-2011-K2-A5}{
 $9/2$
}
 

\Aufgabe[e]{Iterierte Grenzwerte}{
Berechnen Sie  für folgende Funktionen $f(x,y)$ die Grenzwerte
  \[
  \lim\limits_{x\to 0}\ \lim\limits_{y\to 0}\ f(x,y)\,;\quad
  \lim\limits_{y\to 0}\ \lim\limits_{x\to 0}\ f(x,y)\,;\quad
  \lim\limits_{\vec{x} \to \vec{0}}\ f(\vec{x}) \text{ mit } \vec{x}:= (x,y)^{\top}\,,
\] 
(falls diese existieren):
\begin{abc}
\item $f(\vec{x}):= \dfrac{x-y}{x+y}\,, \qquad$
\item $f(\vec{x}):= \dfrac{x^2y^2}{x^2y^2+(x-y)^2}\qquad$
\item $f(\vec{x}):= (x+y)\sin\left(\frac 1x\right)\sin\left(\frac 1y\right)\,.$
\end{abc}

\textbf{Hinweis zu b):} \ Betrachten Sie auch den Fall \ $y=\alpha x$\,, $\alpha \in \R$ .
}

\Loesung{
\begin{abc}
\item Es gilt
\begin{eqnarray*}
\lim_{x \to 0}\ \left(\lim_{y \to 0} \frac{x-y}{x+y}\right) &=&\lim_{x \to 0} \frac{x}{x} = \lim_{x \to 0} 1 = 1\,,\\[3ex]
\lim_{y \to 0}\ \left(\lim_{x \to 0} \frac{x-y}{x+y}\right) &=& \lim_{y \to 0} \frac{-y}{y} = -1\,.
\end{eqnarray*}
Die iterierten Grenzwerte sind also verschieden und \ $\lim\limits_{\vec{x} \to \vec{0}} f(\vec{x})$ \ existiert nicht.
\item Es gilt
\begin{eqnarray*}
\lim_{x \to 0}\ \left(\lim_{y \to 0} \frac{x^2y^2}{x^2y^2+(x-y)^2}\right) & = & \lim_{x \to 0} \frac{0}{x^2}\ = \ \lim_{x \to 0} 0 =\ 0\ , \ \  \text{da}\ \ x\neq 0\ \ \\[3ex]
\lim_{y \to 0}\ \left(\lim_{x \to 0} f(x,y)\right) &=& 0\ , \quad \text{da \ } f(x,y) = f(y,x) \text{\ \ symmetrisch}\,.
\end{eqnarray*}

Es sei \ $y = \alpha x$\,. Wir n\"ahern uns also dem Punkt \(x=0\) entlang einer Geraden \(y= \alpha x\) an. Dann gilt
\begin{equation*}
	\lim_{x \to 0} f(x, \alpha x)= \lim_{x \to 0} \frac{\alpha^2 x^4}{\alpha^2 x^4 + x^2(1-\alpha)^2} =
						\begin{cases}
							 1  & \text{für }\alpha=1\,,\\
  							0 & \text{für } \alpha \neq 1\,.
						\end{cases}
\end{equation*}
Abh\"angig von dem Weg, auf dem wir uns dem Punkt \(\vec{x}=\vec{0}\) ann\"ahern, erh\"alt man unterschiedliche Grenzwerte. Deshalb existiert \ $\lim\limits_{\vec{x} \to \vec{0}} f(\vec{x})$ \ nicht.

\item Es gilt 
\begin{equation*}
\lim_{x \to 0}\ \left(\lim_{y \to 0} f(x,y)\right) =\lim_{x \to 0} \sin\left(\frac{1}{x}\right)\left[\lim_{y \to 0} (x+y) \sin\left(\frac{1}{y}\right) \right]\,.
\end{equation*}
Der Grenzwert \ $\lim\limits_{y \to 0} (x+y) \sin\left(\frac{1}{y}\right) $ \  existiert nicht und  damit auch \  $\lim\limits_{x \to 0}\ \left( \lim\limits_{y \to 0} f(x,y)\right)$ \ nicht. 

Wegen \ $f(x,y) = f(y,x)$ \ gilt dasselbe für den Grenzwert $ \lim\limits_{y \to 0}\ \left( \lim\limits_{x \to 0} f(x,y)\right)$\,. 


Sei \ $\vec x_n=(x_n,y_n)$ \ eine Folge mit \ $\vec{0} \neq \vec{x}_n \in \mathbb{R}^2$\,, dann gilt:
\[	\vec{x}_n \to \vec{0} \Leftrightarrow \lim_{n \to \infty} \sqrt{x_n^2 + y_n^2} = 0\\[3ex]\]

Mit
\begin{align*}|x_n+y_n| \leq |x_n| + |y_n| &= \sqrt{(|x_n| + |y_n|)^2}
= \sqrt{|x_n|^2+|y_n|^2+2|x_n||y_n|}\\[2ex]
&\leq \sqrt{|x_n|^2+|y_n|^2+|x_n|^2+|y_n|^2} = \sqrt{2}\sqrt{x_n^2+y_n^2}
\end{align*}
erhalten wir
\begin{align*} 0 \leq \lim\limits_{n\to \infty} |x_n+y_n| \leq \sqrt{2} \lim\limits_{n\to\infty} \sqrt{x_n^2+y_n^2} = 0 
\Rightarrow \lim\limits_{n \to \infty} |x_n+y_n| = 0\,.
\end{align*}

Somit ist
\begin{eqnarray*}
% 	\vec{x}_n \to \vec{0} &\Leftrightarrow& \lim_{n \to \infty} \sqrt{x_n^2 + y_n^2} = 0\\[3ex]
	&\lim_{n \to \infty} |f(\vec{x}_n)|\ =\  \lim_{n \to \infty} \underbrace{|x_n+y_n|}_{\text{Nullfolge}} \underbrace{|\sin \frac{1}{x_n}|}_{\leq 1} \underbrace{|\sin \frac{1}{y_n}|}_{\leq 1}=0\\[1ex]
		&\Rightarrow  \lim_{\vec{x} \to \vec{0}} f(\vec{x}) = 0\,.
\end{eqnarray*}
\end{abc}
}

% \ErgebnisC{dummy}
% {
% 
% }



%Identisch mit analys_Diff_Ndim_001.tex
\Aufgabe[e]{Partielle Ableitungen}{
Bestimmen Sie alle ersten und zweiten partiellen Ableitungen der folgenden reellen Funktionen und geben Sie jeweils die ersten und zweiten Ableitungen in Matrixschreibweise an: 
\[
\begin{array}{rclcrcl}
\mathbf{i)} &  & f(x,y)=\,\text{e}^{2\,xy^2} & \;\;\;\;\; & \mathbf{ii)} & & g(x,y)=x^2\,\sin (2x+y) \\ 
&  &  &  &  &  &  \\ 
\mathbf{iii)} &  & h(x,y)=\sin (2x)\,\cos (3y) &  & \mathbf{iv)} &  & k(x,y,z)=x\,y^2\,z^3\\
&  &  &  &  &  &  \\ 
\mathbf{v)} &  & j(u,v)=\dfrac{uv}{uv-1}
\end{array}
\]
}

\Loesung{
\textbf{Anmerkungen:} 

\textbf{1.} \ Die Ableitung (Gradient) einer Funktion mehrer Variablen ist ein Vektor (Spaltenvektor): 
\begin{align*}
   f(x,y,...) &\Rightarrow f'(x,y,...) = \Big(\partial_x f\ ,\ \partial_y\,f\ ,\ ...\,\Big)=(\nabla
   f(x,y,\hdots))^\top\\
&\Rightarrow  f''(x,y,\hdots)=\vec H(x,y,\hdots)=\begin{pmatrix}
\partial_{xx} f & \partial_{xy} f & \ldots \\ 
\partial_{yx} f & \partial_{yy} f & \ldots \\ 
\ldots& \ldots & \ldots
\end{pmatrix}
\end{align*}

\textbf{2.} \ Wenn die 2. partiellen Ableitungen \textbf{stetig} sind, dann ist die Hesse--Matrix \
   $\vec H(x,y,...)$ \ symmetrisch.


\begin{iii}
\item \begin{align*}
\nabla f(x,y)=&\left( 2\,y^2\,e^{2xy^2}\;,\;4\,xy\,e^{2xy^2}\right)^\top\\
\vec H_f(x,y)=&\begin{pmatrix}
4\,y^4\,e^{2xy^2} & (4\,y+8\,x\,y^3)\,e^{2xy^2} \\ 
(4\,y+8\,x\,y^3)\,e^{2xy^2} & (4\,x+16\,x^2\,y^2)\,e^{2xy^2}
\end{pmatrix}
\end{align*}
\item \begin{align*}
\nabla g(x,y)=&\Big( 2\,x\,\sin (2x+y)+2\,x^2\,\cos(2x+y)\;,\;x^2\,\cos (2x+y)\Big)^\top \\
\vec H_g(x,y)=&\small{\begin{pmatrix}
(2-4\,x^2)\,\sin (2x+y)+8\,x\,\cos (2x+y)) & 2\,x\,\cos (2x+y)-2\,x^2\,\sin
(2x+y) \\ 
2\,x\,\cos (2x+y)-2\,x^2\,\sin (2x+y) & -x^2\,\sin (2x+y)
\end{pmatrix}}
\end{align*}
\item \begin{align*}
\nabla h(x,y)=&\Big( 2\,\cos (2x)\,\cos (3y)\;,\;-3\,\sin(2x)\,\sin (3y)\Big)^\top\\
\vec H_h(x,y)=&\begin{pmatrix}
-4\,\sin (2x)\,\cos (3y) & -6\;\cos (2x)\,\sin (3y) \\ 
-6\;\cos (2x)\,\sin (3y) & -9\,\sin (2x)\,\cos (3y)
\end{pmatrix}
\end{align*}
\item \begin{align*}
\nabla k(x,y,z)=&\big(y^2\,z^3\;,\;2\,x\,y\,z^3\;,\;3\,x\,y^2\,z^2\big)^\top\\
\vec H_k(x,y,z)=&
\begin{pmatrix}
0 & 2\,y\,z^3 & 3\,y^2\,z^2 \\ 
2\,y\,z^3 & 2\,x\,z^3 & 6\,x\,y\,z^2 \\ 
3\,y^2\,z^2 & 6\,x\,y\,z^2 & 6\,x\,y^2\,z
\end{pmatrix}
\end{align*}
\item \begin{align*}
\nabla j(u,v) =& \dfrac{1}{(uv-1)^2}\cdot\big( -v\ ,\ -u\big)^\top\ ,\qquad\\
\vec H_j(u,v) =&  \dfrac{1}{(uv-1)^3}\cdot\begin{pmatrix}
2\,v^2 & uv+1 \\
uv +1  & 2\,u^2 
\end{pmatrix}
\end{align*}
\end{iii}
}

% \ErgebnisC{dummy}
% {
% 
% }



\Aufgabe[e]{Richtungsableitungen} {
\begin{abc}
\item
Gegeben seien die skalarwertigen Funktionen
$$f(x,y,z)=y^2-xz\qquad \text{ und } \qquad g(x,y,z)=x^2\sin(y)+\cos(z).$$
Berechnen Sie die Richtungsableitung beider Funktionen in Richtung $\vec h:=(-2,3,4)^\top$.
% \item
% Gegeben sei die folgende vektorwertige Funktion $$\vec f(x,y) = (\operatorname{e}^{xy}, x^2+y^3)^\top.$$ 
% \begin{iii}
% \item Berechnen Sie die Jacobi-Matrix von $\vec f$ in dem Punkt $\vec P_1=(1,2)^\top$.
% \item Berechnen Sie die Richtungsableitung von $\vec f$ in Richtung $\vec v=(1,0)^\top$ in dem Punkt $\vec P_2=(1,1)^\top$.
% \end{iii}
\end{abc}
}
% \Aufgabe[e]{Directional derivatives} {
% Given are the functions
% $$f(x,y,z)=y^2-xz\qquad \text{ and } \qquad g(x,y,z)=x^2\sin(y)+\cos(z).$$
% Compute the directional derivative of both functions in direction $\vec h:=(-2,3,4)^T$.
% }


\Loesung{
\begin{abc}
\item
Zun\"achst berechnen wir die Gradienten der beiden Funktionen:
\begin{align*}
\nabla f(x,y,z)=& (-z,\, 2y,\, -x)^\top\\
\nabla g(x,y,z)=& (2x\sin(y),\, x^2\cos(y),\, -\sin(z))^\top.
\end{align*}
Desweiteren ben\"otigen wir den Normalenvektor in Richtung $\vec h$:
$$\vec{\hat h}=\frac 1{\sqrt{4+9+16}}(-2,\, 3,\, 4)^\top=\frac 1{\sqrt{29}}(-2,\, 3,\, 4)^\top.$$
Damit ergeben sich dann die Richtungsableitungen:
\begin{align*}
\frac{\partial f}{\partial \vec{\hat h}}(x,y,z)=& \skalar{\vec{\hat h},\, \nabla f(x,y,z)}=\frac
1{\sqrt{29}} \left( 2z+6y-4x\right)\\
\frac{\partial g}{\partial \vec{\hat h}}(x,y,z)=& \skalar{\vec{\hat h},\, \nabla g(x,y,z)}
=\frac 1{\sqrt{29}} \left( -4x\sin(y)+3x^2\cos(y)-4\sin(z)\right).
\end{align*}
% \item
% \begin{iii}
% \item Die Jacobi-Matrix ist
% $$
% \vec J(\vec x) = 
% \begin{pmatrix} 
% f_{1,x} & f_{1,y}\\
% f_{2,x} & f_{2,y}
% \end{pmatrix}
% = \begin{pmatrix} 
% y\operatorname{e}^{xy} & x\operatorname{e}^{xy}\\
% 2x & 3y^2
% \end{pmatrix}
% $$
% 
% Die Jacobi-Matrix ausgewertet in dem Punkt $\vec P_1=(1,2)^\top$ ist
% $$
% \vec J(\vec P_1) 
% = \begin{pmatrix}
% 2\operatorname{e}^{2} & \operatorname{e}^{2}\\
% 2 & 12
% \end{pmatrix}
% $$
% \item Um die Richtungsableitung zu bestimmen, ben\"otigen wir einen Einheitsvektor. Da der gegebene Vektor $\vec v$ 
% bereits Einheitsl\"ange hat, kann die Richtungsableitung berechnet werden durch
% $$
% \frac{\partial \vec f}{\partial \vec h}(\vec x) = \vec J(x) \vec v = (y\operatorname{e}^{xy}, 2x)^\top
% $$
% 
% Die Richtungsableitung ausgewertet in dem Punkt $\vec P_2$ ist
% $$
% \frac{\partial \vec f}{\partial \vec h}(\vec P_2) = (\operatorname{e}, 2)^\top.
% $$
% \end{iii}
\end{abc}
}
% \Loesung{
% First, we compute the gradient of both functions:
% \begin{align*}
% \nabla f(x,y,z)=& (-z,\, 2y,\, -x)^T\\
% \nabla g(x,y,z)=& (2x\sin(y),\, x^2\cos(y),\, -\sin(z))^T.
% \end{align*}
% Further, we need the normal vector in direction $\vec h$:
% $$\vec{\hat h}=\frac 1{\sqrt{4+9+16}}(-2,\, 3,\, 4)^T=\frac 1{\sqrt{29}}(-2,\, 3,\, 4)^T.$$
% This gives the directional derivatives:
% \begin{align*}
% \frac{\partial f}{\partial \vec{\hat h}}(x,y,z)=& \skalar{\vec{\hat h},\, \nabla f(x,y,z)}=\frac
% 1{\sqrt{29}} \left( 2z+6y-4x\right)\\
% \frac{\partial g}{\partial \vec{\hat h}}(x,y,z)=& \skalar{\vec{\hat h},\, \nabla g(x,y,z)}
% =\frac 1{\sqrt{29}} \left( -4x\sin(y)+3x^2\cos(y)-4\sin(z)\right).
% \end{align*}
% }

\ErgebnisC{AufganalysRichAblt003}
{
$\frac{-4x+6y+2z}{\sqrt{29}}$,\qquad $\frac{-4x\sin y + 3 x^2\cos y - 4\sin z}{\sqrt{29}}$
}
 

\Aufgabe[e]{Online Aufgabe}{
Bearbeiten Sie die aktuelle Online-Aufgabe im ILIAS-Kurs. \\
Beachten Sie, dass Sie dort auch die L\"osungswege zu einzelnen Aufgaben zur Korrektur hochladen k\"onnen. 
}

% \Loesung{
% }
% 
% \ErgebnisC{Online Aufgabe}
% {
% }


%\Aufgabe[e]{Fehlersuche}{
Behauptung:
	\[
	\int\limits_{-2}^{2} x^2\;\text dx = 0\;
\]
Beweis: \ Mit der Substitution \ \ $t = x^2 \,\Rightarrow\, \text dt = 2x\,\text dx$ \ \ gilt:
	\[
	\int\limits_{-2}^{2} x^2\;\text dx = \dfrac 12 \int\limits_{-2}^{2} x\cdot 2x\;\text dx = \dfrac 12 \int\limits_{4}^{4} \sqrt{t}\;\text dt = 0\;.
\]
Wo steckt der Fehler?
}

\Loesung{
Der Fehler liegt bei der \glqq naiven\grqq\ Ersetzung von \ $x$ \ durch \ $\sqrt{t}$\;. Die Umkehrung von \ $t=x^2$ \ ist:
	\[
	\begin{array}{lcl}
	x = -\sqrt t & \text{für} & x<0 \\
	x = +\sqrt t & \text{für} & x\geq 0 
	\end{array} \;.
\]
Damit gilt
	\[
	\begin{array}{rcl}
	\int\limits_{-2}^{2} x^2\;\text dx & = & \dfrac 12\int\limits_{-2}^{0} x\cdot 2x\;\text dx + \dfrac 12\int\limits_{0}^{2} x\cdot 2x\;\text dx \\
	& & \\
	& = & \dfrac 12\int\limits_{4}^{0} -\sqrt{t}\;\text dt + \dfrac 12\int\limits_{0}^{4} +\sqrt{t}\;\text dt \\
	& & \\
	& = &  \int\limits_{0}^{4} \sqrt{t}\;\text dt \;=\; \left[\dfrac 23\,t^{3/2}\right]_0^4 \;=\; \dfrac 23\cdot 8 \;=\; \dfrac{16}{3}\ ,
	\end{array}
\]
was das richtige Ergebnis ist.
}

% \ErgebnisC{wz-2013-1-1}
% {
% 
% }


%\Aufgabe[e]{Regel von L'Hospital}{
Berechnen Sie mit Hilfe der Regel von L'Hospital die Grenzwerte
\begin{tabbing}
\hspace*{1em} \= a) $ \ \underset{x\to 0}\lim \dfrac{x^2 \sin x}{\tan x-x}$\,,
\hspace*{8em} \= b) 
    $\underset{x\rightarrow 0}\lim \dfrac{\ln(\EH{x}-x)}{\ln(\cos x)}$\,, \\[1ex]
\> c) $\underset{x\rightarrow \infty}\lim x(2\arctan x - \pi)$\,,
\> d) $\underset{x\rightarrow 1}\lim \dfrac{x^2-1}{x^{x}-1}$\,. 
\end{tabbing}

}

\Loesung{
\begin{abc}
\item Da sowohl Z\"ahler, als auch Nenner gegen Null konvergieren
$$\underset{x\to 0}\lim (x^2\sin x)=0=\underset{x\to 0}\lim (\tan x - x),$$
darf der Satz von L'Hospital angewendet werden: 
\begin{align*}
\underset{x\to 0}\lim \frac{x^2 \sin x}{\tan x-x} =& \underset{x\to 0}\lim \frac{2x\sin x + x^2\cos
x}{\frac 1{\cos^2 x}-1} = \underset{x\to 0}\lim \frac{2x \sin x \cos^2 x + x^2\cos^3 x}{1-\cos^2
x}\\
=&\underset{x\to 0}\lim \frac{2\sin x + 2x\cos x + 2x \cos x - x^2\sin x}{2(\cos x)^{-3}\sin x}\\
&\qquad\text{ (Erneut gehen Z\"ahler und Nenner gegen 0)}\\
=& \underset{x\to 0}\lim \frac{2\cos x + 4\cos x -4x\sin x -2x\sin x -x^2\cos x}{6(\cos
x)^{-4}\sin^2 x + 2(\cos x)^{-2}}
\end{align*}
Der Z\"ahler dieses Bruches geht gegen $6$, der Nenner gegen $2$, also ist insgesamt
$$\underset{x\to 0}\lim \frac{x^2\sin x}{\tan x-x} = \frac 6 2 =3$$

\item Auch hier kann die Regel von L'Hospital angewendet werden, da Z\"ahler und Nenner gegen Null gehen:
\begin{align*}
\lim_{x \to 0} \frac{\ln(\EH{x}-x)}{\ln(\cos x)}=& \lim_{x \to 0} \frac{\frac{\EH{x}-1}{\EH{x} - x}
}{\frac{-\sin x}{\cos x}}\\
=&\lim_{x\to 0} \frac{\EH{x} \cos x - \cos x}{-\EH{x} \sin x + x \sin x }\\
&\qquad\text{ (Erneut gehen Z\"ahler und Nenner gegen 0)}\\
=&\lim_{x\to 0} \frac{\EH{x} \cos x -\EH{x} \sin x + \sin x}{-\EH{x} \sin x - \EH{x} \cos x + \sin x +
x\cos x}=\frac 1 {-1}=-1
\end{align*}

\item Das Produkt $x(2\arctan x - \pi)$ kann in einen Quotienten umgeformt werden, dessen Z\"ahler
und Nenner jeweils gegen Unendlich gehen, danach kann die Regel von L'Hospital angewendet werden: 
\begin{align*}
 \lim\limits_{x \to \infty} x(2\arctan x - \pi)
  =& \lim\limits_{x \to \infty} \frac{2\arctan x - \pi}{x^{-1}}
  = \lim\limits_{x \to \infty} \frac{\dfrac{2}{1+x^2}}{-x^{-2}}\\
  =& -\lim\limits_{x \to \infty} \frac{2x^2}{1+x^2}\quad\text{(Z\"ahler und Nenner gehen gegen
  $\infty$)}\\
=& -\lim\limits_{x\to \infty} \frac{4x}{2x}=-2
\end{align*}


\item Da Z\"ahler und Nenner gegen Null konvergieren kann man die Regel von L'Hospital
  anwenden. Danach folgt mit  $x^x = \EH{x \ln x}$:
$$ \lim_{x \to 1} \dfrac{x^2-1}{x^{x}-1} 
   = \lim_{x \to 1} \dfrac{2x}{ (1+\ln x) \EH{x \ln x}} = 2\,.$$
\end{abc}
}

\ErgebnisC{AufganalysReglLhos001}
{
\textbf{a)} $3$, \textbf{b)} $-1$, \textbf{c)} $-2$, \textbf{d)} $2$
}
 

%\Aufgabe[e]{Newton-Verfahren}
{
Gegeben sei die Funktion 
$$f(x)=x^3+2x^2-15x-36.$$
\begin{abc}
\item Bestimmen Sie alle lokalen Extrema der Funktion $f(x)$. 
\item Begr\"unden Sie, weshalb $f$ zwei Nullstellen besitzt. 
\item F\"uhren Sie das Newton-Verfahren mit der Funktion $f$ zwei mal durch.  
W\"ahlen Sie im ersten Durchlauf den Startwert $x_0=1$ und
im  zweiten Durchlauf $x_0=-1$. F\"uhren Sie jeweils drei Iterationsschritte durch. 
\end{abc}
}

\Loesung{
\begin{abc}
\item Die Ableitung der Funktion $f$ ist
$$f'(x)=3x^2+4x-15.$$
Ihre Nullstellen 
$$x_{1/2}=\frac{-2\pm\sqrt{49}}3=\left\{\begin{array}{l}5/3\\-3\end{array}\right.$$
sind die station\"aren Punkte der Funktion $f$. \\
Die zweite Ableitung der Funktion ist $f''(x)=6x+4$ und wegen 
$$f''(x_1)=14>0\text{ und } f''(x_2)=-14<0$$
liegt bei $(x_1,f(x_1))=(5/3,-1372/27)$ ein Minimum und bei 
$(x_2,f(x_2))=(-3,0)$ ein Maximum der Funktion $f$ vor. 
\item Da $f$ links der Maximalstelle $x_2=-3$ (die auch Nullstelle ist) sowie rechts der
Minimalstelle $x_1=5/3$ streng monoton steigt, und sonst streng monoton f\"allt, besitzt $f$
lediglich die Nullstelle $x_2$ sowie (wegen $f(x_1)<0$) eine weitere Nullstelle rechts von $x_1$. 
\item Mit der oben berechneten Ableitung $f'(x)=3x^2+4x-15$  lautet die Iterationsvorschrift f\"ur das Newton-Verfahren: 
$$x_{k+1}=x_k-(f'(x_k))^{-1}\cdot f(x_k).$$
F\"ur die gegebenen Startwerte sind die ersten drei Iterationsschritte:\\


\begin{tabular}{r|r|r}
$k$&$x_k$    & $f(x_k)$\\\hline
 0 & 1.0000  &  -48.0000 \\
 1 & -5.0000 &  -36.0000 \\
 2 & -4.1000 &   -9.8010 \\
 3 & -3.5850 &   -2.5955 \\
\end{tabular}
\qquad\qquad
\begin{tabular}{r|r|r}
$k$&$x_k$    & $f(x_k)$\\\hline
    0 &  -1.00000 &  -20.00000 \\
    1 &  -2.25000 &  -3.51562  \\
    2 &  -2.64894 &  -0.81945  \\
    3 &  -2.82923 &  -0.19916  \\
\end{tabular}\\

Obwohl der erste Startwert $x_0=1$ dichter an der Nullstelle $x=4$ der Funktion liegt, konvergiert
das Verfahren trotzdem gegen die Nullstelle $x=-3$. \\
Um die Nullstelle $x=4$ dennoch zu korrekt zu
finden, muss $x_0$ geeignet gew\"ahlt werden. 
\end{abc}
}

\ErgebnisC{analysNewtVerf002}{
Es ist $f(-3)=f(4)=0$.} 
 
 
  %noch erweitern um Taylor-Teilaufgabe 
 
% \Aufgabe[e]{Kurvendiskussion}{\label{KurvDisk002}
% \begin{abc}
% \item  
Gegeben sei die Funktion
$$	f(x) = \dfrac{x^2+3\,x}{1-x}.$$
\begin{iii}
	\item Geben Sie den maximalen Definitionsbereich der Funktion \ $f$ \ an.
	
	\item Bestimmen Sie die Nullstellen der Funktion.
	
	\item Bestimmen Sie die kritischen Punkte der Funktion und deren Funktionswerte. Klassifizieren Sie alle kritischen Punkte als Minimum, Maximum oder Wendepunkt.
	
	\item Untersuchen Sie das Monotonieverhalten der Funktion. 
	
	\item Bestimmen Sie alle Asymptoten der Funktion.

	\item Bestimmen Sie den Wertebereich der Funktion.

	\item Skizzieren Sie die Funktion.
\end{iii}
% \item Gegeben sei die Funktion
% $$	g\,:\,\R\rightarrow\R\,:\,x\rightarrow \big(x^2-4\big)^2\cdot\EH{-x} .$$
% Bestimmen Sie alle relativen Minima und Maxima der Funktion \ $g$ \ \textbf{ohne} die zweite Ableitung zu berechnen.
% \end{abc}
}

\Loesung{
\begin{abc}
\item
\begin{iii}
	\item Der maximale Definitionsbereich ist \ $\mathcal D = \R\backslash\left\{1\right\}$\,.
	
	\item Die Nullstellen der Funktion sind \ $x_{N_1} = -3 \text{ und } x_{N_2} = 0$ \,.
	
	\item Die kritischen Punkte sind die Nullstellen der ersten Ableitung. Aus
$$	f'(x) = \frac{(2x+3)(1-x)+(x^2+3x)}{(1-x)^2} = \dfrac{-x^2+2\,x+3}{(1-x)^2}=0  $$
  folgt
$$
  x_{K_1}= -1\text{ mit } f(-1)=-1 \text{ und } x_{K_2}= 3\text{ mit } f(3)=-9\ .
$$
  	Die zweite Ableitung ist:
  	$$
		f''(x) = \frac{8}{(1-x)^3}.
  	$$
  	Für $x_{K_1}= -1$ ist $f''(-1)=1 > 0$. Es handelt sich also um ein Minimum.
  	Für $x_{K_2}= 3$ ist $f''(3)=-1 < 0$. Es handelt sich also um ein Maximum.
  	
	\item
	Aus den stationären Punkten un der Definitionslücke ergeben sich die Monotonieintervalle.
	In $(-\infty, -1)$ und $(3,\infty)$ ist die Funktion monoton fallend. In $(-1,1)$ und 
	$(1,3)$ ist die Funktion monoton steigend. 
	\item Aus  $f(x) = -x-4+\dfrac{4}{1-x}$  folgt, dass  $g(x)=-x-4$  die Asymptote ist.
	Wegen $\lim_{x \to 1^-} f(x) = \infty$ und $\lim_{x \to 1^+} f(x) = -\infty$ gibt es 
	eine senkrechte Asymptote bei $x=1$.
	\item Der Wertebereich ist $\mathcal{W} = (-\infty,-9] \cup [-1,\infty)$. 
	\item \quad\\
\end{iii}
\end{abc}
\begin{center}	
\psset{unit=0.5cm}
\begin{pspicture}(-5,-16)(7,6)

\psgrid[griddots=8,subgriddiv=0](-5,-16)(7,6)
\psline[linestyle=dashed](-5,1)(7,-11)
\psline[linestyle=dashed](1,-16)(1,6)

\pscircle*(-3,0){0.1}
\pscircle*(-1,-1){0.1}
\pscircle*(-0,0){0.1}
\pscircle*(3,-9){0.1}
\psplot[plotpoints=100, plotstyle=curve]
{-5}{.63}
{
x x mul 3 x mul add 1 x neg add div
}
\psplot[plotpoints=100, plotstyle=curve]
{1.37}{7}
{
x x mul 3 x mul add 1 x neg add div
}

\end{pspicture}

\end{center}

Bei \ $(-1,-1)$ \ handelt es sich also um ein (lokales) Minimum, bei \ $(3,-9)$ \ um ein Maximum.
% \begin{abc}\setcounter{enumi}{1}
% \item Da die Funktion nirgends negativ ist, sind die Nullstellen automatisch lokale Minima: \ $x_{\text {min}_{1,2}} = \pm 2$\,. 
% 
% Die Nullstellen der ersten Ableitung sind die kritischen Punkte. Aus
% $$
% 	g'(x) = 4x\,(x^2-4)\,\EH{-x}-(x^2-4)^2\,\EH{-x} = (-x^2+4x+4)(x^2-4)\,\EH{-x}=0  
% $$
% folgt
% $$
% 	x_{1,2} =\pm 2 \text{ und } x_{3,4} = 2\pm\sqrt 8.
% $$
% Die ersten beiden kritischen Punkte sind die schon bekannten Nullstellen und die beiden anderen sind
% Maxima, da ein einfacher kritischer Punkt zwischen zwei Minima nur ein Maximum sein kann und die
% Funktion für \ $x\rightarrow\infty$ \ gegen \ 0 \ geht.
% \end{abc}
}

\ErgebnisC{AufganalysKurvDisk002}
{
\textbf{a)}\textbf{ii)} $0,\, -3$, 
\textbf{iii)} -1,\, 3, 
\textbf{iv)} $g(x)=-x-4$
}


% \Aufgabe[e]{}
{
% \begin{abc}
%  
% \item 
Gegeben sei die Funktion $f(x) = \cos \left(\dfrac{\pi}{2} \sin x\right)$.

\begin{iii}
 
 \item Bestimmen Sie den Definitionsbereich und den Wertebereich von $f$. 
 
 \item Zeigen Sie, dass die Funktion $f$ die Periodizität $\pi$ besitzt, d.h.\ zeigen Sie, dass $f(x+\pi) = f(x)$ für alle $x\in \R$ gilt.
 
 \item Bestimmen Sie alle Nullstellen von $f$.\\[0.5ex]
 \textbf{Hinweis:} Beachten Sie die Periodizität von $f$.
 
 \item Bestimmen Sie alle Extrema von $f$ und charakterisieren Sie diese.\\[0.5ex]
 \textbf{Hinweis:} Beachten Sie die Periodizität von $f$.
 
 \item Skizzieren Sie den Graphen von $f$ im Intervall $[-\pi,2\pi]$. 
 
\end{iii}
% 
% 
% 
% \item Bestimmen Sie den Grenzwert 
% \[
% \lim_{x\rightarrow \infty} (x+3)\left(\operatorname e^{2/x}-1\right)\,.
% \]
% 
% \end{abc}
}


\Loesung{
\textbf{Zu a)} 
\begin{iii}
\item  Der Wertebereich der Sinusfunktion ist $[-1,1]$. Auf $[-\pi/2,\pi/2]$ 
nimmt der Kosinus alle Werte von $0$ bis $1$ an. Somit gilt 
\[
D(f) =\R \qquad \text{ und } \qquad W(f) = [0,1]\,.
\]

\medskip
\item  Es gilt
\begin{align*}
 f(x+\pi) &  = \cos \left(\frac{\pi}{2} \sin (x+\pi)\right)\\[1ex]
 & = \cos \left(-\frac{\pi}{2} \sin x \right)\\[1ex]
 & = \cos \left(\frac{\pi}{2} \sin x\right)\\[1ex]
 & = f(x) 
\end{align*}
für alle $x\in \R$. Somit hat $f$ die Periode $\pi$. Hinweis: Es lässt sich zeigen, dass $f$ keine kleinere Periode als $\pi$ besitzt. 

\medskip
\item Aufgrund der Periodizität reicht es aus, die Nullstellen im Intervall $[0,\pi]$ zu bestimmen. Es gilt 
\begin{align*}
 f(x) & = \cos \left(\frac{\pi}{2} \sin x\right) = 0 \\[1ex]
 \Longleftrightarrow \quad \frac{\pi}{2} \sin x & = \frac{\pi}{2} \quad \text{oder} \quad \frac{\pi}{2} \sin x = \frac{3\pi}{2}\,.
\end{align*}
Der erste Fall liefert $x=\pi/2$. Der zweite Fall würde $\sin x = 3$ ergeben, was unmöglich ist. Die Menge aller Nullstellen ist somit
\[
N = \left\{\frac{\pi}{2} + k\pi \;\; \Big| \;\; k\in \mathbb Z \right\}\,.
\]

\medskip
\item Es gilt 
\[
f'(x) = - \sin \left(\frac{\pi}{2}\sin x\right)\frac{\pi}{2}\cos x\,.
\]
Es folgt 
\begin{align*}
 f'(x) & = - \sin \left(\frac{\pi}{2}\sin x\right)\frac{\pi}{2}\cos x = 0 \\[1ex]
 \Longleftrightarrow \quad \sin \left(\frac{\pi}{2}\sin x\right) & = 0 \quad \text{oder} \quad \cos x = 0\,.
\end{align*}
Der erste Fall liefert $x=0$ und $x=\pi$. Der zweite Fall ergibt $x=\pi/2$. Da $f$ nur Werte zwischen $0$ und $1$ annimmt, sind $x_1=(0,1)$ und $x_3=(\pi,1)$ Maxima, während $x_2 = (\pi/2,0)$ Minimum ist. Die Menge aller Maxima ist 
\[
E_{\max} = \{(k\pi,1) \mid k\in \mathbb Z \}\,.
\]
Die Menge aller Minima ist
\[
E_{\max} = \left\{ \left(\frac{\pi}{2}+k\pi,0\right) \mid k\in \mathbb Z \right\}\,.
\]

\medskip
\item
\end{iii}
\begin{center}
	\begin{pspicture}(-4,-1)(7,2)
	\psgrid[griddots=8,subgriddiv=0](-4,-1)(7,2)
	\psline[linewidth=1.2pt]{->}(-4,0)(7,0)
	\psline[linewidth=1.2pt]{->}(0,-1)(0,2)
        \psplot[plotpoints=100, plotstyle=curve]{-4}{7}
        {
        x 180 mul 3.14159 div sin 90 mul cos
        }
        \psdot(-1.5708,0)
        \psdot(1.5708,0)
        \psdot(4.7124,0)
        \psdot(  -3.14159,   1.00000)
        \psdot(   0.00000,   1.00000)
        \psdot(   3.14159,   1.00000)
        \psdot(   6.283  ,   1.00000)
        \psdot(  -1.57080,   0.00000)
        \psdot(   1.57080,   0.00000)
        \psdot(   4.71239,   0.00000)

%\cos \left(\dfrac{\pi}{2} \sin x\right)$.



	\end{pspicture} 

%\boxed{\includegraphics[width =0.25\textwidth]{./fig_f.eps}}
\end{center}

\bigskip
\textbf{Zu b)} Es gilt 
\[
(x+3)\left(\operatorname e^{2/x}-1\right) = \frac{\operatorname 
e^{2/x}-1}{\dfrac{1}{x+3}} =: \frac{f(x)}{g(x)}
\]
mit $\lim_{x\rightarrow \infty} f(x) = 0$ und $\lim_{x\rightarrow \infty} g(x) = 0$. Mit 
dem Satz von L'Hospital folgt dann
\begin{align*}
 \lim_{x\rightarrow \infty} \dfrac{f(x)}{g(x)} & = \lim_{x\rightarrow \infty }
\dfrac{f'(x)}{g'(x)} = \lim_{x\rightarrow \infty} \dfrac{2\operatorname e^{2/x} 
(x+3)^2}{x^2} \\[1ex]
& = \lim_{x\rightarrow \infty} 2\operatorname e^{2/x} \, \lim_{x\rightarrow \infty} 
\dfrac{(x+3)^2}{x^2} = 2\,.
\end{align*}
}

 

%\Aufgabe[v]{Interpolation}{
Gesucht ist ein Polynom vierten Grades 
$$p(x)=a_0 + a_1 x + a_2x^2+a_3x^3+a_4 x^4$$
mit folgenden Eigenschaften:
\begin{itemize}
\item Der Funktionswert bei $0$ ist $p(0)=0$. 
\item $p$ hat ein Minimum bei $(1,-1)$. 
\item $p$ hat einen Sattelpunkt bei $(2,0)$. 
\end{itemize}
\begin{abc}
\item Geben Sie die Bedingungen, die der Koeffizientenvektor  $(a_0,\hdots,\, a_4)^\top$ erf\"ullen
muss, in Form eines linearen Gleichungssystems an. (Es sollten sich sechs lineare Gleichungen mit
f\"unf Unbekannten ergeben.)
\item Berechnen Sie den Rang der Koeffizientenmatrix und der erweiterten Koeffizientenmatrix. Ist
das Gleichungssystem l\"osbar?
\item Welchen Wert $p(0)$ muss die Funktion bei $0$ annehmen, damit das System doch l\"osbar ist?
\item L\"osen Sie das so ge\"anderte Gleichungssystem. 
\item Skizzieren Sie die Funktion $p(x)$. 
\end{abc}

}


\Loesung{
\begin{abc}
\item Die angegebenen Eigenschaften der Funktion ergeben folgende Bedingungen:
\begin{align*}
p(0)=0                   &               &                          &\,\Rightarrow\,& a_0
          =& 0\\
p(1)=-1 \text{ minimal } &\,\Rightarrow\,& p(1)=-1,\, p'(1)=0       &\,\Rightarrow\,& a_0+a_1+a_2+a_3+a_4  =& -1\\
                         &               &                          &               & a_1+2a_2+3a_3+4a_4       =& 0\\
p(2)=0 \text{ Sattelpkt.}&\,\Rightarrow\,& p(2)=0,\, p'(2)=p''(2)=0 &\,\Rightarrow\,&
a_0+2a_1+4a_2+8a_3+16a_4 =& 0\\
                         &               &                          &               & a_1+4a_2+12a_3+32a_4     =& 0\\
                         &               &                          &               &
2a_2+12a_3+48a_4     =& 0
\end{align*}
Als Matrixgleichung ergibt sich 
$$\vec A \vec a = \vec b$$
mit 
$$\vec A = \begin{pmatrix}
1  &  0&  0&  0&  0 \\
1  &  1&  1&  1&  1 \\
0  &  1&  2&  3&  4 \\
1  &  2&  4&  8& 16 \\
0  &  1&  4& 12& 32 \\
0  &  0&  2& 12& 48 \end{pmatrix}
\text{ und }\vec b = \begin{pmatrix}
0\\-1\\0\\0\\0\\0\end{pmatrix}.$$
\item Der Rang von $\vec A$ ist h\"ochstens 5, da die Matrix nur f\"unf Spalten hat. Die Determinante
der ersten f\"unf Zeilen der Matrix ist
\begin{align*}
\det \begin{pmatrix}
1  &  0&  0&  0&  0 \\
1  &  1&  1&  1&  1 \\
0  &  1&  2&  3&  4 \\
1  &  2&  4&  8& 16 \\
0  &  1&  4& 12& 32 \end{pmatrix}
=& \det\begin{pmatrix}
  1&  1&  1&  1 \\           
  1&  2&  3&  4 \\           
  2&  4&  8& 16 \\           
  1&  4& 12& 32 \end{pmatrix}=\det \begin{pmatrix}
 1 & 0 & 0 & 0\\
 1 & 1 & 2 & 3\\
 1 & 2 & 6 &14\\
 1 & 3 & 11&31\end{pmatrix}\\
=& \det \begin{pmatrix}
  1 & 2 & 3\\             
  2 & 6 &14\\             
  3 & 11&31\end{pmatrix} = \det\begin{pmatrix}
  1 & 0 & 0\\           
  2 & 2 & 8\\           
  3 &  5&22\end{pmatrix}\\
=& \det\begin{pmatrix} 2 & 8 \\ 5 & 22\end{pmatrix} = 4\neq 0
\end{align*}
Also haben die ersten f\"unf Zeilen den Rang 5 und damit auch 
$$\Rang \vec A = 5.$$
Die Determinante der erweiterten Systemmatrix $(\vec A|\vec b)$ ist 
\begin{align*}
\det(\vec A|\vec b)=& \det\begin{pmatrix}
1  &  0&  0&  0&  0 &  0 \\
1  &  1&  1&  1&  1 & -1 \\
0  &  1&  2&  3&  4 &  0 \\
1  &  2&  4&  8& 16 &  0 \\
0  &  1&  4& 12& 32 &  0 \\
0  &  0&  2& 12& 48 &  0 \end{pmatrix}= \det \begin{pmatrix}
  1&  1&  1&  1 & -1 \\           
  1&  2&  3&  4 &  0 \\           
  2&  4&  8& 16 &  0 \\           
  1&  4& 12& 32 &  0 \\           
  0&  2& 12& 48 &  0 \end{pmatrix}\\
=& -\det \begin{pmatrix}
  1&  2&  3&  4  \\           
  2&  4&  8& 16  \\           
  1&  4& 12& 32  \\           
  0&  2& 12& 48  \end{pmatrix} = -\det\begin{pmatrix}
  1&  2&  3&  4  \\           
  0&  0&  2&  8  \\           
  0&  2&  9& 28  \\           
  0&  2& 12& 48  \end{pmatrix} \\
=& -\det\begin{pmatrix}
0&  2&  8  \\              
2&  9& 28  \\              
2& 12& 48  \end{pmatrix} = -\det \begin{pmatrix}
0&  2&  8  \\              
2&  9& 28  \\              
0&  3& 20  \end{pmatrix} \\
=& 2\cdot \det \begin{pmatrix} 2 & 8 \\ 3 & 20\end{pmatrix} = 32\neq 0
\end{align*}
Also hat $(\vec A | \vec b)$ Vollrang, $\Rang (\vec A|\vec b)=6$ und wegen 
$$\Rang \vec A \neq \Rang (\vec A|\vec b)$$
ist das Gleichungssystem $\vec A \vec a=\vec b$ nicht l\"osbar. 
\item Der Funktionswert $p(0)=c$ taucht in der oberen rechten Ecke der Matrix $(\vec A|\tilde {\vec  b})$
auf. Dabei ist $\tilde {\vec b} = (c,-1,0,0,0,0)^\top$. \\

Die Determinante \"andert sich dadurch wie folgt: 
\begin{align*}
\det (\vec A|\tilde {\vec b})=& \det(\vec A|\vec b)-c\cdot \det \begin{pmatrix}
1  &  1&  1&  1&  1  \\
0  &  1&  2&  3&  4  \\
1  &  2&  4&  8& 16  \\
0  &  1&  4& 12& 32  \\
0  &  0&  2& 12& 48  \end{pmatrix}\\
=&32-c\cdot \det \begin{pmatrix}
  1&  2&  3&  4  \\             
  2&  4&  8& 16  \\             
  1&  4& 12& 32  \\             
  0&  2& 12& 48  \end{pmatrix} - c \cdot \det\begin{pmatrix}
  1&  1&  1&  1  \\             
  1&  2&  3&  4  \\             
  1&  4& 12& 32  \\             
  0&  2& 12& 48  \end{pmatrix}\\
=& 32 -c\cdot (-32) - c \cdot \det\begin{pmatrix}
  1&  1&  1&  1  \\             
  0&  1&  2&  3  \\             
  0&  3& 11& 31  \\             
  0&  2& 12& 48  \end{pmatrix}= 32+32c -c\cdot \det\begin{pmatrix}
  1&  2&  3  \\           
  3& 11& 31  \\           
  2& 12& 48  \end{pmatrix}\\
=& 32+32c-c\cdot \begin{pmatrix} 5 & 22\\ 8 & 42\end{pmatrix}=32-2c
\end{align*}
Die Determinante verschwindet also f\"ur $p(0)=c= 16$. Damit gilt dann
$$\Rang \vec A = \Rang (\vec A|\tilde{\vec b})=5$$
und das System besitzt eine eindeutige L\"osung. 
\item Die L\"osung dieses Systems ist
$$\begin{array}{rrrrr|r|l}
1  &  0&  0&  0&  0 & 16 & \text{                     }\\
1  &  1&  1&  1&  1 & -1 & \text{ -1. Zeile           }\\
0  &  1&  2&  3&  4 &  0 & \text{                     }\\
1  &  2&  4&  8& 16 &  0 & \text{ -1. Zeile           }\\
0  &  1&  4& 12& 32 &  0 & \text{                     }\\
0  &  0&  2& 12& 48 &  0 & \text{                     }\\\hline

1  &  0&  0&  0&  0 & 16 & \text{                     }\\
0  &  1&  1&  1&  1 &-17 & \text{                     }\\
0  &  1&  2&  3&  4 &  0 & \text{ -2. Zeile           }\\
0  &  2&  4&  8& 16 &-16 & \text{ -2$\cdot$ 2. Zeile  }\\
0  &  1&  4& 12& 32 &  0 & \text{ -2. Zeile           }\\
0  &  0&  2& 12& 48 &  0 & \text{                     }\\\hline

1  &  0&  0&  0&  0 & 16 & \text{                     }\\
0  &  1&  1&  1&  1 &-17 & \text{                     }\\
0  &  0&  1&  2&  3 & 17 & \text{                     }\\
0  &  0&  2&  6& 14 & 18 & \text{ -2$\cdot$ 3. Zeile  }\\
0  &  0&  3& 11& 31 & 17 & \text{ -3 $\cdot$ 3. Zeile }\\
0  &  0&  2& 12& 48 &  0 & \text{ -2$\cdot$ 3. Zeile  }\\\hline

1  &  0&  0&  0&  0 & 16 & \text{                     }\\
0  &  1&  1&  1&  1 &-17 & \text{                     }\\
0  &  0&  1&  2&  3 & 17 & \text{                     }\\
0  &  0&  0&  2&  8 &-16 & \text{ $\cdot 1/2$         }\\
0  &  0&  0&  5& 22 &-34 & \text{-5/2$\cdot$ 4. Zeile }\\
0  &  0&  0&  8& 42 &-34 & \text{-4 $\cdot$ 4. Zeile  }\\\hline

1  &  0&  0&  0&  0 & 16 & \text{                     }\\
0  &  1&  1&  1&  1 &-17 & \text{                     }\\
0  &  0&  1&  2&  3 & 17 & \text{                     }\\
0  &  0&  0&  1&  4 & -8 & \text{                     }\\
0  &  0&  0&  0&  2 &  6 & \text{    $\cdot 1/2$      }\\
0  &  0&  0&  0& 10 & 30 & \text{-5 $\cdot$ 5. Zeile  }\\\hline

1  &  0&  0&  0&  0 & 16 & \text{                     }\\
0  &  1&  1&  1&  1 &-17 & \text{                     }\\
0  &  0&  1&  2&  3 & 17 & \text{                     }\\
0  &  0&  0&  1&  4 & -8 & \text{                     }\\
0  &  0&  0&  0&  1 &  3 & \text{                     }\\
0  &  0&  0&  0&  0 &  0 & \text{                     }
\end{array}$$
Es ergibt sich die L\"osung 
$$\vec a =( 16,\, -48,\, 48,\, -20,\, 3)^\top.$$
\item \quad\\
\begin{minipage}{.4\textwidth}
\psset{xunit=1cm, yunit=.5cm, runit=1cm}
\begin{pspicture}(-1,-1)(3,17)
\psgrid[subgriddiv=1,griddots=10,gridlabels=.3](-1,-1)(3,17)
\psplot[plotpoints=200, plotstyle=curve]
{0}{3}
{16 -48 x mul add 48 x mul x mul add -20 x mul x mul x mul add 3 x mul x mul x mul x mul add}
\psdot(0,16)
\psdot(1,-1)
\psdot(2,0)
\end{pspicture}
\end{minipage}

\end{abc}
}

\ErgebnisC{AufganalysIntrPoly001}
{
{\textbf{a)}} Die Systemmatrix ist $\Vek A = \begin{pmatrix} 
1 & 0 & 0 & 0 & 0 \\  
1 & 1 & 1 & 1 & 1 \\  
0 & 1 & 2 & 3 & 4 \\  
1 & 2 & 4 & 8 & 16\\  
0 & 1 & 4 & 12& 32\\  
0 & 0 & 2 & 12& 48\\  
\end{pmatrix}$, die rechte Seite des Systems $\vec b=(0,-1,0,0,0,0)^\top$. 
\textbf{d)} $\vec a=(16,\, -48,\, 48,\, -20,\, 3)^\top$
}


%\Aufgabe[e]{}{
\begin{abc}
\item \textbf{zur partiellen Integration}
$$
	\int u(t)\cdot v'(t)\ \mathrm{d} t\ =\ u(t)\cdot v(t) - \int u'(t)\cdot v(t)\  \mathrm{d} t 
$$


Berechnen Sie Stammfunktionen der beiden Funktionen
$$
	\textbf{i)} \  (2t-1)\,\cos(t), \qquad \textbf{ii)} \  \big(t^2+t-5\big)\,\EH {t/2}. 
$$
\item \textbf{zur Substitution}
$$
	\int f\big(g(t)\big)\cdot g'(t)\  \mathrm{d} t\ =\ \int f(z)\ \mathrm{d} z \text{ mit } z =
        g(t)\ ,\ \ \mathrm{d} z = g'(t)\,\mathrm{d} t\
$$


Berechnen Sie Stammfunktionen von 
$$	
	\textbf{i)} \  4\,t\,\EH{t^2}\  \qquad \textbf{ii)} \  \dfrac{1}{\sqrt t}\,\EH{\sqrt t}\  .
$$
\end{abc}
}

\Loesung{
\begin{abc}
\item \begin{iii}
\item Mit \ \ $u(t)=(2t-1)\,,\ u'(t)=2$ \ \ und \ \ $v'(t)=\cos(t)\,,\ v(t)=\sin(t)$ \ \ erhält man
$$
	\int (2t-1)\,\cos(t)\,\mathrm{d} t\ =\ (2t-1)\sin(t) - \int 2\sin(t)\ \mathrm{d} t\ = \ (2t-1)\sin(t) + 2\cos(t) + C\ .
$$

\item Mit \ $u(t)=t^2+t-5\,,\ u'(t)=2t+1$ \ und \ $v'(t)=\EH {t/2}\,,\ v(t)=2\,\EH {t/2}$ \ f\"ur die erste partielle Integration und mit \ $u(t)=4t+2\,,\ u'(t)=4$\ und \ $v'(t)=\EH {t/2}\,,\ v(t)=2\,\EH {t/2}$\ f\"ur die zweite erhält man
	\[\begin{array}{rcl}
	  \int \big(t^2+t-5\big)\,\EH {t/2}\ \mathrm{d} t & = & \big(t^2+t-5\big)\,2\,\EH {t/2} - \int
          (2t+1)\,2\,\EH {t/2}\ \mathrm{d} t \\[2ex]
	  & = & \big(2\,t^2+2\,t-10\big)\,\EH {t/2}-(4t+2)\,2\,\EH {t/2} + \int 4\cdot 2\,\EH
               {t/2}\ \mathrm{d} t \\[2ex]
	  & = & \big(2\,t^2-6\,t-14\big)\,\EH {t/2} + 16\,\EH {t/2} + C\\[2ex]
	  & = & \big(2\,t^2-6\,t+2\big)\,\EH {t/2} + C\ .
	\end{array}
\]
\end{iii}

\item\begin{iii} 
\item Mit \ \ $z=t^2$ \ \ und \ \ $\mathrm{d} z=2\,t\ \mathrm{d}t$ \ \ erhält man
	\[
	\int 4\,t\,\EH{t^2}\ \mathrm{d} t\ =\ \int 2\,\EH z\ \mathrm{d} z\ =\ 2\,\EH z + C\ =\ 2\,\EH{t^2} + C\ .
\]

\item Mit \ \ $z=\sqrt t$ \ \ und \ \ $\mathrm{d} z= \dfrac{1}{2\sqrt t}\,\mathrm{d} t$ \ \ erhält man
	\[
	\int \dfrac{1}{\sqrt t}\,\EH{\sqrt t}\ \mathrm{d} t\ =\ 2\int \EH z\ \mathrm{d} z\ =\ 2\EH z + C\ =\ 2\EH{\sqrt t} + C\ .
\]
\end{iii}
\end{abc}
}

 \ErgebnisC{analysInteGral010}
 {
 \textbf{a)i)} $(2t-1)\sin(t) + 2\cos(t)$, 
 \textbf{a)ii)} $2(t^2-3t+1)\EH{t/2}$, 
 \textbf{b)i)} $2\EH{t^2}$, 
 \textbf{b)ii)} $2\EH{\sqrt t}$
 }


%6

%\Aufgabe[e]{Integration}{
\begin{abc}
\item Berechnen Sie folgende Integrale mittels partieller Integration: 
\begin{multicols}{2}
\begin{iii}
\item $\int x\cdot \sin x \,\mathrm{d} x$,
\item $\int \sin^2(x) \,\mathrm{d} x$,
%  \int\frac{x^2+1}{\sqrt x}\,\mathrm{d} x
\item $\int x^2\EH{1-x} \,\mathrm{d} x$,
\item $\int \frac{x}{\cos^2 x}\,\mathrm{d} x$,
\item $\int\limits_{0}^{\pi/4} \frac x{\cos^2 x}\,\mathrm{d} x$
\end{iii}
\end{multicols}
\item Berechnen Sie folgende Integrale mittels einer geeigneten Substitution: 
\begin{multicols}{2}
\begin{iii}
\item $\int\limits_1^2 \frac{3x^2+7}{x^3+7x-2}\,\mathrm{d} x$,
\item $\int\limits_\pi^{3\pi/2}x^2\cos(x^3+2)\,\mathrm{d} x$,
\item $\int\limits_1^2 \frac1x \EH{1+\ln x}\,\mathrm{d} x $,
\item $\int\limits_{1/4}^1 \EH{\sqrt x}\,\mathrm{d} x,$
\item $\int\cosh^2 x \sinh x \,\mathrm{d} x$
% ,&&\int\limits_{-1}^32x|x|\,\mathrm{d} x
\end{iii}
\end{multicols}
\end{abc}
}


\Loesung{
\begin{abc}
\item \begin{iii}
\item \begin{align*}
\int \underbrace{x}_{u}\underbrace{\sin x}_{v'} \,\mathrm{d} x =& \underbrace{x}_{u} \underbrace{(-\cos
x)}_{v}- \int \underbrace{1}_{u'} \cdot \underbrace{(-\cos x)}_{v} \mathrm{d} x\\
=&-x \cos x + \int \cos x \mathrm{d} x = -x\cos x + \sin x + C
\end{align*}
\item \begin{align*}
&&\int\underbrace{\sin x}_u \underbrace{\sin x}_{v'}  \,\mathrm{d} x=&\underbrace{\sin
x}_{u}\underbrace{(-\cos x)}_{v} - \int \underbrace{\cos x}_{u'}\underbrace{(-\cos x)}_{v}\,\mathrm{d} x\\
&&=&-\sin x\cos x +\int\underbrace{\cos x\cos x}_{=1-\sin^2 x}\,\mathrm{d} x\\
&&=& -\sin x \cos x +\int 1 \,\mathrm{d} x - \int \sin^2 x\,\mathrm{d} x\\
\Rightarrow &&2\int\sin^2 x \,\mathrm{d} x =& -\sin x \cos x + x + 2C\\
\Rightarrow &&\int\sin^2 x \,\mathrm{d} x =& \frac {x-\sin x \cos x}{2} + C
\end{align*}
Alternativ kann man die Beziehung \(\sin^2(x) = \frac 12 (1-\cos(2x))\) (siehe Formelsammlung) nutzen. Damit bekommt man:
\begin{align*}
 \int \sin^2(x) \,\mathrm{d} x = \frac 12 \int (1-\cos(2x)) \,\mathrm{d} x = \frac 12\left(x-\frac 12 \sin(2x)\right) + C
\end{align*}
Dies l\"asst sich mithilfe von \(\sin(x) \cos(x) = \frac 12 \sin(x)\) in die andere Darstellung der L\"osung umwandeln.


% \item \begin{align*}
% \int\underbrace{(x^2+1)}_u\underbrace{\frac 1{\sqrt x}}_{v'}\,\d x
% =& \underbrace{(x^2+1)}_u\underbrace{2\sqrt x}_{v}-\int\underbrace{2x}_{u'=u_2}\cdot \underbrace{2\sqrt
% x}_{v=v_2'}\,\d x\\
% =& 2(x^2+1)\sqrt x -\underbrace{2x}_{u_2}\cdot \underbrace{\frac 43
% x^{3/2}}_{v_2}+\int\underbrace{2}_{u_2'}\cdot \underbrace{\frac 43 x^{3/2}}_{v_2}\,\d x\\
% =&2\sqrt x\left(x^2+1-\frac 43 x^2\right) + \frac{16}{15}x^{5/2} + C
% =\frac 2{5}\sqrt x\left( x^2+5\right) + C
% \end{align*}

\item \begin{align*}
\int \underbrace{x^2}_{u}\underbrace{\EH{1-x}}_{v'}\,\mathrm{d}x=& \underbrace{x^2}_u\underbrace{(-\EH{1-x})}_{v}-\int\underbrace{2x}_{u'=u_2}\underbrace{(-\EH{1-x})}_{v=v_2'}\,\mathrm{d}
x\\
=&
-x^2\EH{1-x}-\underbrace{2x}_{u_2}\underbrace{\EH{1-x}}_{v_2}+\int\underbrace{2}_{u_2'}\underbrace{\EH{1-x}}_{v_2}\,\mathrm{d}x\\
=& -(x^2+2x)\EH{1-x}-2\EH{1-x} + C =-(x^2+2x+2)\EH{1-x} + C
\end{align*}
\item \begin{align*}
\int \underbrace{x}_u\underbrace{\frac 1{\cos^2 x}}_{v'}\,\mathrm{d} x = & \underbrace{x}_u\underbrace{\tan
x}_{v}-\int\underbrace{1}_{u'}\underbrace{\tan x }_v\,\mathrm{d} x\\
=&x\tan x + \ln |\cos x| + C
\end{align*}
\item $\bigl[x\tan x+\ln |\cos x|\bigr]_{x=0}^{\pi/4}=\frac \pi 4 + \ln \frac 1{\sqrt 2} -
0=\frac \pi 4 -\frac {\ln 2}{2}$
\end{iii}
\item\begin{iii}
\item Mit $y=x^3+7x-2$ und $\,\mathrm{d} y = (3x^2+7)\,\mathrm{d} x$ hat man: 
\begin{align*}
\int\limits_1^2\frac{3x^2+7}{x^3+7x-2}\,\mathrm{d} x=&\int\limits_{y(1)}^{y(2)}\frac 1 y \,\mathrm{d} y= \Bigl[ \ln
|y|\Bigr]_6^{20}=\ln \frac {20}6=\ln \frac {10}3
\end{align*}
\item Hier w\"ahlt man $y=x^3+2$ und  erh\"alt daraus $\,\mathrm{d} y = 3x^2 \, \mathrm{d} x$ und setzt ein: 
\begin{align*}
\frac 13 \int\limits_{\pi}^{3\pi/2} 3x^2\cos(x^3+2)\,\mathrm{d} x=& \frac
13 \int\limits_{y(\pi)}^{y(3\pi/2)}\cos(y)\,\mathrm{d} y\\
=&\frac 13\left( \sin\left( \left( \frac
{3\pi}2\right)^3+2\right)-\sin (\pi^3+2)\right)
\end{align*}
\item Mit $y=1+\ln x$ und $\,\mathrm{d} y = \frac{\,\mathrm{d} x}x$ hat man
\begin{align*}
\int\limits_1^2 \frac 1x \EH{1+\ln x} \,\mathrm{d} x =& \int\limits_{y(1)}^{y(2)} \EH{y}\,\mathrm{d} y=\EH{1+\ln
2}-\EH{1}=\EH{}
\end{align*}
\item Wir w\"ahlen $y=\sqrt x$. Damit folgt (f\"ur $x>0$):
$$x=y^2\,\Rightarrow \, \,\mathrm{d} x = 2y\,\mathrm{d} y$$
und schlie\ss{}lich
\begin{align*}
\int\limits_{1/4}^1\EH{\sqrt x}\,\mathrm{d} x=& \int\limits_{y(1/4)}^{y(1)} \EH y \cdot 2y \,\mathrm{d}
y= 2\int\limits_{1/2}^1 y\EH y\,\mathrm{d} y\\
=& \bigl[2y\EH y\bigr]_{1/2}^1 -2 \int\limits_{1/2}^1\EH y\,\mathrm{d} y\text{ (partielle Integration)}\\
=&2\EH{ }-\sqrt{\EH{ }}-2(\EH{ }-\sqrt{\EH { }})=\sqrt{\EH{ }}
\end{align*}
\item Mit $y=\cosh x$ und $\,\mathrm{d} y = \sinh x \,\mathrm{d} x$ hat man
\begin{align*}
\int\cosh^2 x \sinh x \,\mathrm{d} x=& \int y^2\,\mathrm{d} y = \frac{y^3}3 + C = \frac{\cosh^3 x}3 + C
\end{align*}
% \item Wir stellen den Betrag dar als $|x|=\sqrt{x^2}$ und substituieren $y=x^2$. Die Ableitung ist
% $$\frac{\,\d y}{\,\d x} = 2x,$$
% sie wechselt ihr Vorzeichen bei $x=0$. Also muss das Integral aufgeteilt werden, um die Substitution
% doch zu verwenden: 
% $$\int\limits_{-1}^3 2x|x|\,\d x=\int\limits_{-1}^02x|x|\,\d x + \int\limits_0^3 2x|x|\,\d x.$$
% Auf jedem Teilintervall ist $y(x)$ monoton. Nun kann man substituieren: 
% \begin{align*}
% \int\limits_{-1}^3 2x|x|\,\d x =& \int\limits_{y(-1)}^{y(0)}\sqrt{y}\,\d y
% + \int\limits_{y(0)}^{y(3)}\sqrt y  \,\d y\\
% =& \left[\frac 23 y^{3/2}\right]_1^0 +\left[ \frac 23 y^{3/2}\right]_0^9=\frac 23 \left(
% 0-1+27-0\right)=\frac{52}3
% \end{align*}
\end{iii}
\end{abc}
}

\ErgebnisC{AufganalysInteGral001}
{
{ a)} $-x\cos x + \sin x + C$, $\frac {x-\sin x \cos x}{2} + C$, 
$-(x^2+2x+2)\EH{1-x} + C$, $x\tan x + \ln |\cos x| + C$, $\frac{\pi}4-\frac{\ln 2}2$\\
{ b)} $\ln \frac{10}3$, $\frac 13\left( \sin \frac{27\pi^3+16}8-\sin (\pi^3+2)\right)$, $\EH{1+\ln
2}-\EH{ }$, $\sqrt{\text{e}}$, $\frac 13 \cosh^3 x + C$
}


%7

%\Aufgabe[e]{Fr\"uhere Klausuraufgabe }{
Berechnen Sie die Integrale
\begin{iii}
\item $I_1={\displaystyle \int\limits_0^1 }(2x-1)\cosh(x)\d x$
\item $I_2={\displaystyle \int}\dfrac{\sin(x) \EH{\tan x}}{\cos^3(x)}\d x$ %\qquad f\"ur $-\pi/2<x<\pi/2$.
\end{iii}
}

\Loesung{
 \begin{iii}
\item Partielles Integrieren ergibt
\begin{align*}
I_1=& \int\limits_0^1(2x-1)\cosh(x)\d x\\
=& \left[(2x-1)\sinh(x)\right]_0^1-\int\limits_0^1 2\sinh(x)\d x\\
=& \sinh(1)-2\cosh(1)+2
= \frac 12\left( \EH{}- \EH{-1}-2\EH{}-2\EH{-1}+4\right)\\
=& \frac {4-\EH{}- 3\EH{-1}}2.
\end{align*}
\item Wir substituieren zun\"achst $u=\tan x$, $\d u = \frac{\d x}{\cos^2 x}$ und integrieren dann partiell: 
\begin{align*}
I_2=& \int u\EH{u}\d u = u\EH u - \int \EH u\d u\\
=& (u-1)\EH u +C = (\tan(x)-1)\EH{\tan(x)} + C.
\end{align*}
\end{iii}
}


  

%\Aufgabe[e]{~}{
 Es sei 
$$
\mathrm{sign} \,(x) := \left\{\begin{array}{@{}r@{\,}c@{\,}l} 1 &\quad \text{f\"ur }\,& x>0\\[1ex] 0& \quad\text{f\"ur }\,& x= 0\\[1ex] -1 &\quad\text{f\"ur }\,& x<0  \end{array}\right.
$$
Gegeben sei die Funktion $f\in { Abb}\,(\R,\R)$ mit $f(x) = \dfrac{x}{2} + \mathrm{sign}\, (x)$. Skizzieren Sie $f$. Ist $f$ auf $I= [-1,2]$ Riemann-integrierbar? Begr\"unden Sie Ihre Antwort und bestimmen Sie gegebenenfalls das zugeh\"orige Riemann-Integral.
% \end{itemize}
}

\Loesung{
\begin{center}
\begin{pspicture}(-2,-2)(3,3)
%\psgrid(-2,-2)(3,3)
\psline[fillstyle=solid, fillcolor=lightgray, linecolor=lightgray](0,0)(-1,0)(-1,-1.5)(0,-1)(0,0)
\psline[fillstyle=solid, fillcolor=lightgray, linecolor=lightgray](0,0)(2,0)(2,2)(0,1)(0,0)
\psplot[plotpoints=100,plotstyle=curve]
{-2}{0}{
.5 x mul -1 add
}
\psplot[plotpoints=100,plotstyle=curve]
{0}{3}{
.5 x mul 1 add
}

\psline{->}(-2,0)(3,0)
\psline{->}(0,-2)(0,3)
\psline(-.1,1)(.1,1)
\psline(1,-.1)(1,.1)
\put(2.7, .1){$x$}
\put(1,-.3){$1$}
\put(-.3, 1){$1$}
\put(.1,2.7){$f(x)$}
\end{pspicture}
\end{center}
Nach einem Satz aus der Vorlesung (Satz 3.96) ist $f$ auf $[-1,2]$ Riemann-integrierbar, da $f$ auf $[-1,0]$ und auf $[0,2]$ jeweils Riemann-integrierbar ist. Es gilt
$$
I = \int_{-1}^0 \left(\dfrac{x}{2} -1 \right) \: \mathrm{d}x + \int_{0}^2 \left(\dfrac{x}{2} +1 \right) \: \mathrm{d}x =   \left.\left(\dfrac{x^2}{4} -x \right)\right|_{-1}^0 +  \left.\left(\dfrac{x^2}{4} +x \right)\right|_{0}^2 = \dfrac{7}{4} \,.
$$
}


% \ErgebnisC{b-2013-K-A1}{
% 
% }



 

%\Aufgabe[e]{~}{
\begin{itemize}
\item[a)]  Bestimmen Sie drei verschiedene (reelle) Nullstellen der Ableitung der Funktion 
\[
f\,:\,\mathbb{R\longrightarrow R}\,,\, f(x)=\int_{0}^{x^{3}}\text{e}^{t^{4}}\cdot (t^{2}-t-2)\;{d}t\;. 
\]
\textbf{Hinweis}: Das Integral \textbf{nicht} berechnen!
\item[b)] Gegeben seien die Funktionen (Das Integral \textbf{nicht} berechnen!) 
$$
F(x):=\int_0^x\frac{\sinh t^2}{t^2}\,\text{e}^{t^2}\,dt,\qquad G(x):=\text{e}^{-2x^2},\quad x\in \R \,.
$$
Berechnen Sie den Grenzwert $\displaystyle\lim_{x\to\infty}G(x)\cdot F(x)$. \\
\textbf{Hinweis}: Regel von L'Hospital.
\item[c)] Bestimmen Sie die \textbf{reelle} Partialbruchzerlegung von $$\ f(x)=\dfrac{1}{(x-1)(x^2+x+1)}\,.$$
\end{itemize}
}

\Loesung{
\textbf{L\"osung}\\[1ex]
\textbf{Zu a)} Mit dem Hauptsatz der Differential-- und Integralrechnung und der Kettenregel folgt: 
\[
\begin{array}{lll}
\displaystyle \left( \int_{0}^{\;x^{3}}\text{e}^{t^{4}}\cdot (t^{2}-t-2)\;\text{d}t\right)' & = & 
\left( \text{e}^{(x^{3})^{4}}\cdot \left( (x^{3})^{2}-(x^{3})-2\right)
\right) \cdot 3\,x^{2} \\ 
&  &  \\ 
& = & 3\,\text{e}^{x^{12}}\cdot x^{2}\cdot \left( x^{6}-x^{3}-2\right)
\end{array}
\]
Der erste Term wird nie Null, d.h. die Nullstellen sind die Nullstellen der beiden letzten Terme: 
\[
x=0\;\text{(doppelt)}\;,\;\;x=-1\;,\;\;x=\sqrt[3]{2}\;. 
\]


\bigskip
\textbf{Zu b)} Wir betrachten als Erstes 
\[f(t)=\frac{\sinh t^2}{t^2}\,\text{e}^{t^2}= \frac{\left(\text{e}^{t^2}-\text{e}^{-t^2}\right)\cdot \text{e}^{t^2}}{2t^2} = \frac{\text{e}^{t^2} -1}{2t^2}\]
\[\lim_{x\to0} f(t) \overset{\text{l'Hosp.}}{=} \lim_{x\to0} \frac{4t\text{e}^{2t^2}}{4t} = 1\]
\[\lim_{x\to\infty} f(t) \overset{\text{l'Hosp.}}{=} \lim_{x\to\infty} \text{e}^{2t^2} = + \infty\]
Die Funktion \(f(t)\) ist positiv f\"ur \(t\in(0,\infty)\). Dadurch entspricht \(F(x)\) der Fl\"ache zwischen dem Funktionsgraphen von \(f(t)\), der \(t\)-Achse und den Geraden \(t=0\) und \(t=x\).
Da f\"ur \(t\to\infty\) die Funktion \(f(t)\) uneingeschr\"ankt w\"achst, gilt auch f\"ur die Fl\"ache \(\underset{{x\to\infty}}\lim F(x) = \infty\).
\begin{center}
\begin{pspicture}(-2,0)(3,4)
%\psgrid(-2,0)(3,4)
\psline[fillstyle=solid, fillcolor=lightgray, linecolor=lightgray, linewidth=0.0](0,0)(2.2,0)(2.2,1.3064)(0,.1)(0,0)
\psplot[plotpoints=100,plotstyle=curve, fillstyle=solid, fillcolor=white, linecolor=white]
{-.01}{2.2}{
2.7183 x x mul exp 2.7183 x x mul neg exp neg add 20 div x x mul div
}
\psplot[plotpoints=100,plotstyle=curve]
{-.01}{2.5}{
2.7183 x x mul exp 2.7183 x x mul neg exp neg add 20 div x x mul div
}

\psline{->}(-2,0)(3,0)
\psline{->}(0,0)(0,4)
\psline(2.2,-.1)(2.2,.1)
\put(2.7, .1){$t$}
\put(2.2,-.3){$x$}
\put(.1,3.7){$f(t)$}
\put(2.6,0.6){$F(x)$}
\psline(2,.3)(2.6,.7)
\end{pspicture}
\end{center}
Mit dem Grenzwert
\[\lim_{x\to\infty}G(x) = \text{e}^{-2x^2} = \lim_{x\to\infty} \frac{1}{\text{e}^{2x^2}} = 0\]
ist der Ausdruck \(\underset{x\to\infty}\lim G(x)\cdot F(x)= 0 \cdot \infty\) unbestimmt. Wir behandeln ihn mit der Regel von l'Hospital:
\begin{align*}
\lim_{x\to\infty}G(x)\cdot F(x) & =\lim_{x\to\infty}\frac{F(x)}{\frac1{G(x)}}
=\lim_{x\to\infty}\frac{F'(x)}{4x\text{e}^{2x^2}}\\[2ex]
& =\lim_{x\to\infty}\frac{\sinh x^2}{4x^3\text{e}^{x^2}}
= \lim_{x\to\infty}\frac{\frac{1}{2}\,(\text{e}^{x^2}-\text{e}^{-x^2})}{4x^3\text{e}^{x^2}}\\[2ex]
& =\lim_{x\to\infty}\frac{1-\text{e}^{-2x^2}}{8x^3}=0\,.
\end{align*}


\bigskip
\textbf{Zu c)} Der Faktor $x^2+x+1$ hat keine reellen Nullstellen und kann deshalb nicht in Linearfaktoren zerlegt werden (zumindest nicht in $\R$). Deshalb verwendet man den Ansatz
$$ 
f(x)=\dfrac{1}{(x-1)(x^2+x+1)} = \dfrac{A}{x-1} + \dfrac{Bx+C}{x^2+x+1}. 
$$
Durch Multiplikation mit dem Hauptnenner erh\"alt man
\[ 1 = A(x^2+x+1)+(Bx+C)(x-1)\]

Einsetzen der reellen Nullstelle des Hauptnenners in die Gleichung liefert 
f\"ur \(x=1\) die Gleichung \(1 = 3\cdot A \Rightarrow A = \frac13\) und Koeffizientenvergleich f\"ur \(x^2\) liefert \(0 = A+B \Rightarrow B = -\frac13\). Nun w\"ahlen wir noch \(x = 0\) und erhalten \( 1=A-C  \Rightarrow C =A-1=-\frac23\,.\)\\

Damit ist
$f(x)= \dfrac{1}{3(x-1)}-\dfrac{x+2}{3(x^2+x+1)}$. 

}


% \ErgebnisC{b-2013-K-A1}{
% 
% }



 

%\Aufgabe[e]{Bogenl\"ange} {
Die Bogenl\"ange des Graphen einer stetig differenzierbaren Funktion $f:\, [a,b]\rightarrow \R$ auf dem Interval $[a,b]$ wird
definiert als 
$$L=\int\limits_a^b\sqrt{1+(f'(x))^2}\,\d x.$$
Berechnen Sie die Bogenl\"ange der folgenden Funktionen:
\begin{abc}
\item $ f_1(x)=\sqrt{4-x^2}$ auf $[-2,2]$
\item $f_2(x)= x^2$ auf $[0,b]$
\end{abc}

}


\Loesung{
\begin{abc}
\item \begin{align*}
L_1&= \int\limits_{-2}^2\sqrt{1+\left( \frac{-x}{\sqrt{4-x^2}}\right)^2}\,\d x
=\int\limits_{-2}^2\sqrt{\frac{4-x^2+x^2}{4-x^2}}\,\d x\\
&= \int\limits_{-2}^2\frac 2{\sqrt{4-x^2}}\,\d x = 2\lim_{a\to2}\int\limits_{0}^a\frac 2{\sqrt{4-x^2}}\,\d x
\end{align*}
Wir behandeln dabei das uneigentliche Integral als Grenzwert und nutzen die Symmetrie aus.

Weiter berechnen wir das unbestimmte Integral \(\int\frac 2{\sqrt{4-x^2}}\,\d x\) mit der Substitution $x=2\sin \varphi$ f\"ur $-\frac\pi 2 \leq \varphi\leq \frac \pi 2, \,\d x=2\cos\varphi \,\d \varphi$.
\begin{align*}
\int\frac{2}{\sqrt{4-4\sin^2\varphi}}2\cos\varphi\,\d \varphi
&=\int \frac 1{|\cos \varphi|}\cdot
2\cos \varphi \,\d \varphi\\
& = \int 2 \,\d \varphi \quad(\cos\varphi\geq 0\text{ f\"ur }-\frac\pi 2 \leq \varphi\leq \frac \pi 2)\\
&= 2\varphi+C= 2 \arcsin\left(\frac x2\right)+ C
\end{align*}

\begin{align*}
 L_1 &= 2\lim_{a\to2}\int\limits_{0}^a\frac 2{\sqrt{4-x^2}}\,\d x = 2\lim_{a\to2}2 \left[\arcsin\left(\frac x2\right)\right]_0^a \\
 & = 4 \lim_{a\to2}\arcsin\left(\frac a2 \right) = 4 \arcsin(1) = 2 \pi
\end{align*}


\item \begin{align*}
L_2&=\int\limits_0^b\sqrt{1+(2x)^2}\,\d x \quad\text{(substituiere $2x=\sinh t$, $\,\d x = \frac12\cosh t\,\d t $)}\\
&= \int\limits_0^{t_0}\sqrt{1+\sinh^2t}\cdot \frac 12 \cosh t \,\d t \quad(t_0=\text{Arsinh }(2b))\\
&=\frac 12 \int\limits_0^{t_0} \cosh^2 t \,\d t \overset{\text{(*)}}{=}\frac 12\cdot \left.\frac{t
+\sinh t \cosh t}2\right|_{0}^{t_0}=\frac{t_0 + \sinh t_0\cdot  \sqrt{1+\sinh^2 t_0}}4\\
&= \frac{\text{Arsinh }(2b) + 2b \cdot\sqrt{1+4b^2}}4
\end{align*}
\end{abc}
(*) wegen 
\begin{align*}\int \cosh^2(t) \,\d t &= \int \left(\dfrac{\text{e}^{t}+\text{e}^{-t}}{2}\right)^2 \,\d t = \int \dfrac{\text{e}^{2t}+2+\text{e}^{-2t}}{4} \,\d t \\
&= \dfrac 14 \left( \dfrac{\text{e}^{2t}}{2}+ \dfrac{\text{e}^{-2t}}{2} +2t\right)+C 
= \dfrac 12 \left(\dfrac{\left(\text{e}^t -\text{e}^{-t}\right)\cdot \left(\text{e}^t +\text{e}^{-t}\right)}{4} +t \right) + C\\
&= \dfrac 12 \left(\sinh(t) \cdot \cosh(t) +t\right) + C.\end{align*}
}

%\newcounter{AufganalysIntgBogl001}
%\setcounter{AufganalysIntgBogl001}{\theAufg}
%\Ergebnis{\subsubsection*{Ergebnisse zu Aufgabe \arabic{Blatt}.\arabic{AufganalysIntgBogl001}:}
%
%}
 


\ifthenelse{\boolean{mitLoes}}{\cleardoublepage}{}
\ifthenelse{\boolean{mitErg}}{
\ruleBig
\Ergebnisse}{}


\end{twocolumn}
\end{document}
