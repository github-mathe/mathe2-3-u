\Aufgabe[e]{Harmonischer Oszillator mit Resonanz}
{
Man betrachte die Differentialgleichung des harmonischen Oszillators
\[
y'' + 2\rho y' + \omega^2 y = r(t), \quad \text{mit } \rho, \omega \in \mathbb R_+
\]
aus Aufgabe 16.4, %harmonischer_oszillator.tex
mit den drei Fällen
\begin{enumerate}
  \item \textbf{Überdämpft (\( \rho > \omega \)):} Das System kehrt ohne Oszillation langsam zum Gleichgewicht zurück.
  $$
  r(t) = \operatorname{e}^{-\rho + \sqrt{\rho^2 - \omega^2}}
  $$
  \item \textbf{Kritisch gedämpft (\( \rho = \omega \)):} Das System kehrt so schnell wie möglich ohne Oszillation zum Gleichgewicht zurück mit
  $$
  r(t) = \operatorname{e}^{-\omega t}
  $$
  \item \textbf{Untergedämpft (\( \rho < \omega \)):} Das System oszilliert mit einer Amplitude, die allmählich auf null abnimmt.
  $$
  r(t) = \operatorname{e}^{-\rho t} \cos(\sqrt{\rho^2 - \omega}t)
  $$
\end{enumerate}
Bestimmen Sie die Lösung der DGl für alle drei Fälle: überdämpft, kritisch gedämpft und untergedämpft.

}


\Loesung{
Fall 1: Kritische Dämpfung (\(\rho = \omega\))
Die charakteristische Gleichung vereinfacht sich zu:
\[
\lambda^2 + 2\omega \lambda + \omega^2 = 0
\]
mit einer doppelten Wurzel \(\lambda = -\omega\). Die homogene Lösung ist:
\[
y(t) = (A + Bt)e^{-\omega t}
\]
Für die partikuläre Lösung wählen wir den Ansatz


Fall 2: Untergedämpft (\(\rho < \omega\))
Die Wurzeln sind komplex:
\[
\lambda = -\rho \pm i\sqrt{\omega^2 - \rho^2}
\]
was zu der homogenen Lösung führt:
\[
y(t) = e^{-\rho t}(A \cos(\sqrt{\omega^2 - \rho^2} t) + B \sin(\sqrt{\omega^2 - \rho^2} t))
\]

Fall 3: Überdämpft (\(\rho > \omega\))
Die Wurzeln sind reell und unterschiedlich:
\[
\lambda = -\rho \pm \sqrt{\rho^2 - \omega^2}
\]
mit der homogenen Lösung:
\[
y(t) = Ae^{(-\rho + \sqrt{\rho^2 - \omega^2})t} + Be^{(-\rho - \sqrt{\rho^2 - \omega^2})t}
\]


}



