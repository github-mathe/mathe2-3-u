\Aufgabe[e]{Trennung der Ver\"anderlichen und Anfangswertproblem}
{
\begin{abc}
 \item Bestimmen Sie alle L\"osungen der Differentialgleichung
 $$
 t f^\prime(t)=f(t)+te^{f(t)/t}
 $$
 \item Bestimmen Sie die L\"osung der Anfangswertaufgabe
 $$
 y''(x) - 2 y'(x) - 3y(x) = 0, \quad y(0) = 0, \quad y'(0) = 6.
 $$
\end{abc}

}

\Loesung{

\begin{abc}
%
\item 
Die Gleichung lautet
$$ f^\prime(t)=\frac{f(t)}{t}+e^{f(t)/t} $$
und hat somit die Form
 $f^\prime(t)=F\left(f(t)/t\right)$.  \\
Mit der Substitution $z(t)=\dfrac{f(t)}{t}$,\; $f(t)=tz(t),\;f^\prime(t)=z(t)+tz^\prime(t)$ erh\"alt man:
$$ z+tz^\prime=z+e^{z}\, , \text{ also }  \dfrac{\d z}{\d t} =\frac{1}{t}e^{z}. $$
Trennung der Variablen liefert
\[\int e^{-z}\d z =\int \dfrac{1}{t}\, dt,\]
also $-e^{-z}=\ln|t|+C$ und damit
$z=-\ln(-\ln|t|-C)$. R\"ucktransformation auf die alten Variablen ergibt
$$ f(t)=tz(t)=-t\ln(-\ln|t|-C) \text{ mit } C\in\mathbb{R} . $$
%
\item Die Nullstellen des charakteristischen Polynoms
$p(\lambda)=\lambda^2-2\lambda-3$ sind $\lambda_1=-1$ und $\lambda_2=3$. 

Die allgemeine L\"osung ist 
$$ y(x) = c_1 e^{-x} + c_2 e^{3x} \ \text{ mit } c_1,c_2 \in \R .$$

Aus den Anfangsbedingungen $y(0)=c_1 + c_2 \stackrel{!}{=} 0$ und $y^\prime(0) = -c_1 + 3c_2 \stackrel{!}{=} 6$ folgt das lineare 
Gleichungssystem
$$ \begin{array}{rcrl}
   c_1 & + & c_2 & = 0, \\
  -c_1 & + & 3c_2 & = 6 
   \end{array} $$
mit L\"osung $c_2=3/2$ und $c_1=-3/2$. Damit ist 
$$ y(x) = -\frac32  e^{-x} + \frac32 e^{3x}. $$
\end{abc}

}
\ErgebnisC{gewdglDglTdvAWP099}{
\textbf{a)} $f(t)=tz(t)=-t\ln(-\ln|t|-C)$\\
\textbf{b)} $y(x) = -\frac32  e^{-x} + \frac32 e^{3x}$
}
