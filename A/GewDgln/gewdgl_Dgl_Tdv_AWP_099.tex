\Aufgabe[e]{Trennung der Veränderlichen und Anfangswertproblem}

{
Klassifizeren die folgende Differentialgleichung und bestimmen Sie die Lösung des Anfangswertproblems mit $y(0) = 0$ und $y'(0) = 6$:
 $$
 y''(x) - 2 y'(x) - 3y(x) = 0.
 $$
}

\Loesung{
Die Nullstellen des charakteristischen Polynoms
$p(\lambda)=\lambda^2-2\lambda-3$ sind $\lambda_1=-1$ und $\lambda_2=3$.

Die allgemeine Lösung ist
$$ y(x) = c_1 \operatorname{e}^{-x} + c_2 \operatorname{e}^{3x}  \text{ mit } c_1,c_2 \in \mathbb{R} .$$

Mit den Anfangswertbedingungen $y(0)=c_1 + c_2 \stackrel{!}{=} 0$ und 
$y^\prime(0) = -c_1 + 3c_2 \stackrel{!}{=} 6$ erhalten wir das lineare System
% $$ \begin{array}{rcrl}
$$ \begin{array}{rcrl}
   c_1 & + & c_2 & = 0, \\
  -c_1 & + & 3c_2 & = 6 
   \end{array} $$
mit Lösung $c_2=3/2$ und $c_1=-3/2$. Damit ist 
$$ y(x) = -\frac32  \operatorname{e}^{-x} + \frac32 \operatorname{e}^{3x}. $$
%    \end{array} $$
% with solution $c_2=3/2$ and $c_1=-3/2$. This yields
% $$ y(x) = -\frac32  \operatorname{e}^{-x} + \frac32 \operatorname{e}^{3x}. $$

}

