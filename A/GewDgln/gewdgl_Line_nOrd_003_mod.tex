\Aufgabe[e]{Lineare Differentialgleichungen $n$-ter Ordnung}
{
Bestimmen Sie die allgemeinen L\"osungen der folgenden
Differentialgleichungen:
% \begin{abc}
% \item $y^{(4)} + 4y = 0$,
% \item $y^{(4)} + 2y''' + y'' = 12 x$,
% \item $y^{\prime\prime}+4y^\prime+5y=8\sin t$,
% \item $y^{\prime\prime}-4y^\prime+4y=\EH{2x}$.
% \Aufgabe[e]{Linear ODEs of the 4th order}
% {
% Compute the general solutions of the following
% differential equations:
\begin{abc}
\item $y^{(4)} + 4y = 0$,
\item $y^{(4)} - 18 y'' + 81y=0$.
\end{abc}
}

\Loesung{
\begin{abc}
% \item Es ist $p(\lambda)=\lambda^4+4$, die Nullstellen sind 
% $\lambda_\ell = \sqrt{2} \EH{i (2\ell+1) \pi / 4}$ für $\ell=0,1,2,3$,
% also $$\lambda_0=1+i, \lambda_1=-1+i, \lambda_2=-1-i, \lambda_3=1-i$$
% Ein reelles Fundamentalsystem ist
% $$\{ e^x \cos x, e^x \sin x, \EH{-x} \cos x, \EH{-x} \sin x \}$$ 
% Die allgemeine Lösung lautet also
% $$ y(x) = c_1 e^x \cos x + c_2 e^x \sin x + c_3 \EH{-x} \cos x 
%    + c_4 \EH{-x} \sin x \text{ mit } c_1,c_2,c_3,c_4 \in \R . $$
\item Das charakteristische Polynom ist $p(\lambda)=\lambda^4+4$, mit den Nullstellen
%$\lambda_\ell = \sqrt{2} \EH{i (2\ell+1) \pi / 4}$ for $\ell=0,1,2,3$,
$\lambda_\ell = \sqrt{2} \EH{i (\pi/2+k\pi) / 2}$ for $k=0,1,2,3$,
$$\lambda_0=1+i, \lambda_1=-1+i, \lambda_2=-1-i, \lambda_3=1-i.$$
Das Fundamentalsystem ist dann
$$\{ e^x \cos x, e^x \sin x, \EH{-x} \cos x, \EH{-x} \sin x \}$$ 
Die allgemeine Lösung ist also gegeben als
% The general solution is therefore given as
$$ y(x) = c_1 e^x \cos x + c_2 e^x \sin x + c_3 \EH{-x} \cos x 
   + c_4 \EH{-x} \sin x \text{ with } c_1,c_2,c_3,c_4 \in \R . $$
\item
Von der gegebenen Gleichung $$y^{(4)} - 18 y'' + 81y=0,$$
ist das charakteristische Polynom
$$
\lambda^4-18\lambda^2 + 81 = 0
$$
und kann geschrieben werden als
$$
(x^2-9)^2 = (x-3)^2\cdot(x+3)^2 = 0
$$
mit den zwei doppelten Nullstellen: $\lambda_1 = -3$ und $\lambda_2 = 3$.
Die allgemeine Lösung der hommogenen Differentialgleichung ist
% The general solution of the homogeneous ODE is
$$
y(x) = (c_1+c_2x) e^{-3 x} + (c_3 + c_4x) e^{3 x}.
$$
Die Koeffizienten vor der Exponentialfunktion sind lineare Funktionen, weil es sich um doppelte Nulstellen handelt.
% % Here we observe that the coefficients in front of the exponential functions are linear functions because the multiplicity of the zeros is 2.
\end{abc} 
}

% 
\ErgebnisC{gewdgl_Line_nOrd_003_mod}
 {
\begin{abc}
\item $y(x) = c_1 e^x \cos x + c_2 e^x \sin x + c_3 \EH{-x} \cos x 
   + c_4 \EH{-x} \sin x$
\item $y(x) = (c_1+c_2x) e^{-3 x} + (c_3 + c_4x) e^{3 x}$
\end{abc}
 }

