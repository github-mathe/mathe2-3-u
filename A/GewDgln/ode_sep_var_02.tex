\Aufgabe[e]{Anfangswertproblem}{
Gegeben sei das Anfangswertproblem:
% Given the initial value problem:
$$y^\prime=\sqrt y,\quad y(0)=0.$$
\begin{abc}
\item Bestimmen sie eine Lösung des gegebenen Anfangswertproblems.
% Find a solution of the IVP.
\item Hat das Anfangswertproblem eine eindeutige Lösung?
% Does the IVP have a unique solution?
\end{abc}
}

\Loesung{
\begin{abc}
\item Die L\"osung kann mit Trennung der Variablen berechnet.
% The solution can be computed by the separation of variables
\begin{align*}
\frac{1}{\sqrt{y}}\d y &= \d x\\
\int \frac{1}{\sqrt{y}}\d y &= \int \d x\\
2\sqrt{y} &= x +C\\
y &= \frac{1}{4}(x+C)^2
\end{align*}
Mit dem Anfangswert erhalten wir $C=0$ und die L\"osung
% Using the initial condition we get $C=0$ and a solution is
$$
y=\frac{1}{4} x^2,
$$
Die L\"osung ist nicht eindeutig, da offensichtlich die konstante Null-Funktion eine Lösung ist,
die die Anfangswertbedingung erfüllt. 
% but this is not the unique solution since it is clear that also the constant zero function is a solution that satisfies the initial condition. 
\item 
Das Problem hat unendlich viele L\"osungen der Form
% This problem has inifinitely many solutions, which are of the form
\begin{align*}
f(x) = 
\begin{cases}
0, \quad 0 < x < \bar x,\\
\frac{1}{4} x^2 -\frac{1}{4}\bar x^2, \quad x > \bar x.
\end{cases}
\end{align*}
Das kann man daraus herleiten, dass die Differentialgleichung  autonom und daher invariant 
gegenüber einer Verschiebung bez\"uglich $x$ ist. Also falls $\bar y(t)$ eine L\"osung 
der obigen Differentialgleichung ist, dann ist auch $\hat y(t)= \overline y(t+t_0)$ eine 
L\"osung des Problems 
$$y^\prime=\sqrt y,\quad y(t_0)=0.$$

% This can be deduced by the fact that the ODE is autonomous and hence invariant with respect to a translation with respect to $x$. So if $\bar y(t)$ is a solution of the above ODE then $\hat y(t)= \overline y(t+t_0)$ is solution of the problem
% $$y^\prime=\sqrt y,\quad y(t_0)=0.$$
\end{abc}
}


\ErgebnisC{sepvar02}
{
\begin{abc}
\item $y(x) =\frac{1}{4} x^2$
\item Das Problem hat unendlich viele Lösungen.
\end{abc}
}