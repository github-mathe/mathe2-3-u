\Aufgabe[e]{Bernoulli'sche Differentialgleichung}

Eine Differentialgleichung der Form 
\[
u^{\prime }(x)=f(x)\cdot u(x)+g(x)\cdot \Big(u(x)\Big)^{n} 
\]
hei\ss t \textbf{Bernoulli'sche Differentialgleichung}. 
Sie läßt sich mit Hilfe der Substitution 
\[
z(x)=\Big(u(x)\Big)^{1\,-\,n} 
\]
in eine lineare Differentialgleichung für $z(x)$ überführen.\\
Bestimmen Sie die allgemeine Lösung der Differentialgleichung für $y(x)$%
\[
y'=\frac{-2}x\cdot y+x^2\cdot y^2\;. 
\]

\Loesung{
Bei der Differentialgleichung
\[
y'=\frac{-2}x\cdot y+x^2\cdot y^2
\]
handelt es sich um eine Bernoulli'sche Differentialgleichung mit $n=2$, $f(x)= -\frac{2}{x}$ und $g(x) = x^2$.
Es wird mit $z(x) = \frac{1}{y(x)}$ substituiert. Es ergibt sich:
\begin{align*}
z'(x) &= -\frac{1}{y^2(x)} y'(x) \\
      &= -z^2(x) y'(x) \\
\Rightarrow y'(x) &= -\frac{1}{z^2(x)} z'(x).
\end{align*}
Einsetzen in die Dgl. ergibt:
\begin{align*}
-\frac{1}{z^2} z'(x) &= -\frac{2}{x} \frac{1}{z(x)} + x^2 \frac{1}{z^2(x)} \\
\Rightarrow z'(x) &= \frac{2}{x} z(x) - x^2.
\end{align*}
Es resultiert eine inhomogene lineare Dgl. f\"ur $z(x)$.
Durch Trennung der Veränderlichen erh\"alt man die L\"osung der homogenen lin. Dgl. $z'(x) = \frac{2}{x} z(x)$
$$
z_{\text{h}}(x) = C x^2.
$$

Mit Variation der Konstanten wird der Ansatz $z_{\text{p}}(x) = C(x) x^2$ gew\"ahlt und man erh\"alt f\"ur die partikul\"are L\"osung
$$
z_{\text{p}}(x) = -x^3.
$$
Damit erhält man die folgende allgemeine Lösung
\[
z(x)=C\cdot x^2-x^3\ . 
\]
Die Rücksubstitution ergibt die gesuchte Lösung für\ \ $y(x)$\ : 
\[
y(x)=\frac 1{C\cdot x^2-x^3}\ . 
\]

%Die Bernoulli'sche Differentialgleichung teilt man zuerst durch $\Big(u(x)\Big)^{n} $:
%\[
%\dfrac{u^{\prime }(x)}{\Big(u(x)\Big)^{n}}=f(x)\cdot\dfrac{1}{u(x)}+g(x)
%\]
%Nun substituiert man $z(x)=\dfrac{1}{u(x)}=\Big(u(x)\Big)^{1\,-\,n}$.
%Es gilt
%\[
%z'(x)=(1-n)\cdot\Big(u(x)\Big)^{-\,n}\cdot u'(x)=(1-n)\cdot\dfrac{u'(x)}{\Big(u(x)\Big)^{n}}\ .
% \]
%Somit wird die Bernoulli'sche Differentialgleichung in die folgende inhomogene lineare Dgl. überführt:
%\[
%\dfrac{z'(x)}{1-n}=f(x)\cdot z(x)+g(x)
%\]
%Für die gegebene Dgl. $\quad y'(x)=\dfrac{-2}x\,y+x^2\cdot y^2\quad $ gilt 
%\[
%u(x)=y(x),\;n=2,\;f(x)=\frac{-2}x,\;g(x)=x^2.
%\]
%Diese Gleichung geht also durch die Substitution $z(x)=(y(x))^{-1}=\dfrac 1{y(x)}$
%in eine inhomogene lineare Dgl. für\ \ $z(x)$\ \ 
%\[
%-z'(x)=-\frac 2x\cdot z+x^2
%\]
%über.
%Durch Trennung der Veränderlichen und Variation der Konstanten erhält man die folgende Lösung
%\[
%z(x)=C\cdot x^2-x^3\ . 
%\]
%Die Rücksubstitution ergibt die gesuchte Lösung für\ \ $y(x)$\ : 
%\[
%y(x)=\frac 1{C\cdot x^2-x^3}\ . 
%\]
}


\ErgebnisC{AufggewdglBernDglg001}{
Es gilt $(1-n)u'(x)=u(x)^n\cdot z'(x)$. Damit ist die Lösung $y(x)=\frac 1{C\cdot x^2-x^3}$.
}

