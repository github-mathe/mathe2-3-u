\Aufgabe[e]{Komplexe Nullstellen des charakteristischen Problems}
{
Bestimmen Sie die Lösung der folgenden Differentialgleichung:
% Determine the solution of the following differential equation:
$$y^{\prime\prime}+4y^\prime+5y=0$$
}

\Loesung{

Das charakteristische Polynom $p(\lambda)=\lambda^2+4\lambda+5$ hat die Nullstellen
$$\lambda_1=-2+i, \lambda_2=-2-i$$
die Lösung ist dann
\begin{align*}
y&=C_1\EH{-2t+\imag t}+C_2\EH{-2t-\imag t}\\
&= \EH{-2t}(C_1(\cos t+\imag \sin t)+C_2(\cos t-\imag \sin t)) \\
&= \EH{-2t}(c_1\cos t+c_2 \sin t ),
\end{align*}
wobei $c_1=C_1+C_2$ und $c_2 = i(C_1-C_2)$ die reellen Konstanten sind. \\
Die komplexen Konstanten können durch die reellen Konstanten wie folgt ausgedrückt werden:
$C_1=\overline{C_2}=\dfrac{c_1-\imag c_2}{2}$.
}


\ErgebnisC{AufggewdglLinenOrd004}
{
$y(x) = \EH{-2t}(c_1\cos t+c_2 \sin t )$
}

