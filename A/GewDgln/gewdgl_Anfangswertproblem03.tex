\Aufgabe[e]{Anfangswertproblem}
{
Gegeben ist die Differentialgleichung
$$
y'(x)+y(x)\sin(x)=\sin(2x).
$$

\begin{abc}
\item Bestimmen Sie die allgemeine L\"osung.
\item Bestimmen Sie die L\"osung f\"ur den Anfangswert $y(0)=1$.
\end{abc}

}

\Loesung{
\begin{abc}
\item Wir l\"osen zun\"achst die hom. lin. Dgl. mit Trennung der Variablen:
\begin{align*}
y'+y\sin(x)=0 & \Rightarrow  \int \frac{\mathrm{d}y}{y} =  \int - \sin(x) \mathrm{d}x  \\
              & \Rightarrow \ln|y(x)| = \cos(x) + \tilde{C} \\
              & \Rightarrow |y(x)| = e^{\tilde{C}} e^{\cos(x)} \\
              & \Rightarrow y(x) = \pm e^{\tilde{C}} e^{\cos(x)}
\end{align*}
Die allgemeine hom. L\"osung lautet:
$$
y_{\text{h}}(x) =  C e^{\cos(x)}.
$$
Zur Bestimmung der speziellen L\"osung der inhom. Dgl. verwenden wir den Produktansatz:
$$
y(x) = c(x) e^{\cos(x)} \Rightarrow y'(x) = c'(x) e^{\cos(x)}-c \sin(x) e^{\cos(x)}.
$$
Einsetzen in die Dgl. ergibt:
$$
c'(x) e^{\cos(x)}  -c \sin(x) e^{\cos(x)} +c e^{\cos(x)} \sin(x) = \sin(2x)
$$
$$
\Rightarrow c'(x) = \sin(2x)  e^{-\cos(x)}.
$$
Um $c(x)$ zu bestimmen wird die Trennung der Variablen durchgef\"uhrt:
$$
\int \mathrm{d}C = \int \sin(2x)  e^{-\cos(x)}  \mathrm{d}x = \int 2 \cos(x)\sin(x)e^{-\cos(x)}   \mathrm{d}x.
$$
Das rechte Integral l\"asst sich mit Substitution l\"osen:
$$
\Rightarrow c(x) = (2\cos(x)+2) e^{-\cos(x)}.
$$
Damit ist die partikul\"are L\"osung:
$$
y_{\text{p}}(x) = c(x) e^{\cos(x)} = 2\cos(x)+2.
$$
Die allgemeine L\"osung der Dgl. ist damit
$$
y(x) = C  e^{\cos(x)} +2\cos(x)+2.
$$

\item 
$$
y(0)=1 = C e^{\cos(1)}+2\cos(1)+2
$$
$$
\Rightarrow C = (-1-2\cos(1))e^{-\cos(1)}
$$
Einsetzen in die allgemeine L\"osung ergibt
$$
y(x) = -3 e^{\cos(x)-1}+2\cos(x)+2.
$$
\end{abc}
}


\ErgebnisC{gewdgl_Anfangswertproblem03}{
\begin{abc}
\item $y(x) = C  e^{\cos(x)} +2\cos(x)+2$
\item $y(x) = -3 e^{\cos(x)-1}+2\cos(x)+2$

\end{abc}
}

