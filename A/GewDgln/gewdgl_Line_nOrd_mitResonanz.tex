\Aufgabe[e]{Lineare Differentialgleichungen $n$-ter Ordnung mit Resonanz}
{
Bestimmen Sie die allgemeinen L\"osungen der folgenden
Differentialgleichungen:
\begin{abc}
\item $y^{(4)} + 2y''' + y'' = 12 x$,
\item $y^{\prime\prime}-4y^\prime+4y=\EH{2x}$,
\item $y'''+6y''+12y'+8y=x e^{-2x}$.
\end{abc}

}

\Loesung{
\begin{abc}

\item Das charakteristische Polynom $p(\lambda)=\lambda^4+2\lambda^3+\lambda^2$ hat die Nullstellen\newline
$\lambda_{1/2}=0$ (doppelte Nullstelle), $\lambda_{3/4}=-1$ (ebenfalls doppelt).\newline
Ein Fundamentalsystem ist also $\{ 1,x,\EH{-x},x\EH{-x}\}$.\newline
F\"ur die Partikul\"arl\"osung ist der Ansatz $y_p(x)=(ax+b)x^2=ax^3+bx^2$
sinnvoll.  $$y_p^\prime(x)=3ax^2+2bx,
y_p^{\prime\prime}(x)=6ax+2b, y_p^{(3)}(x)=6a \text{ und } 
y_p^{(4)}(x)=0.$$ 
Eingesetzt in die Dgl. ergibt
$$ 0 + 12a + 6ax+2b
   = 12x .$$
\noindent
Koeffizientenvergleich liefert dann $a=2$, $b=-6a=-12$. Die allgemeine L\"osung der Gleichung
ist 
$$ y(x)=c_1 + c_2 x + c_3 \EH{-x} + c_4 x \EH{-x} - 12 x^2 + 2x^3
  \text{ mit } c_1,c_2,c_3,c_4 \in \R. $$

\item Es ist $p(\lambda)=\lambda^2-4\lambda+4=(\lambda-2)^2$.\newline
Das Polynom hat die doppelte Nullstelle $\lambda=2$.\newline
Ein Fundamentalsystem ist daher
$\{ \EH{2x}, x \EH{2x}\}$.\newline
Mit dem Ansatz $y_p(x)=ax^2 \EH{2x}$ folgt
$$y_p^\prime(x)=a (2x^2+2x) \EH{2x},\ y_p^{\prime\prime}(x)=a (4x^2+8x+2) \EH{2x}.$$
Das Einsetzen in die Dgl. liefert somit
$$a\EH{2x}(4x^2+8x+2-8x^2-8x+4x^2)=\EH{2x}.$$
Daraus folgt $a=1/2$. \newline
Die allgemeine L\"osung lautet
$$ y(x) = \left(c_1+c_2x+\frac{1}{2}x^2\right) \EH{2x} 
  \text{ mit } c_1,c_2 \in \R. $$
  
\item Es ist $p(\lambda)=\lambda^3+6\lambda^2+12\lambda+8=(\lambda+2)^3$.\newline
Das Polynom hat die dreifache Nullstelle $\lambda=-2$.\newline
Ein Fundamentalsystem ist daher
$\{ e^{-2x}, x  e^{-2x}, x^2 e^{-2x}\}$.\newline
Mit dem Ansatz $y_p(x)=(ax^3+bx^4) \EH{-2x}$ folgt
\begin{align*}
y_p^\prime(x) &= \EH{-2x}(a(-2x^3+3x^2)+b(4x^3-2x^4)), \\
y_p^{\prime\prime}(x) &= \EH{-2x}(a(4x^3-12x^2+6x)+b(4x^4-16x^3+12x^2)), \\
y_p^{\prime\prime\prime}(x) &= \EH{-2x}(a(-8x^3+36x^2-36x+6)+b(-8x^4+48x^3-72x^2+24x))).
\end{align*}
Das Einsetzen in die Dgl. liefert somit
$$
6a + 24 b x = x.
$$
Daraus folgt $b=1/24$ und $a=0$. \newline
Die allgemeine L\"osung lautet
$$ y(x) = \left(c_1+c_2x+c_3x^2+\frac{1}{24}x^4\right) \EH{-2x} 
  \text{ mit } c_1,c_2,c_3 \in \R. $$
\end{abc} 

}


\ErgebnisC{AufggewdglLinenOrdmitResonanz}
{
Allgemeine L\"osungen:
a) $y(x)=c_1 + c_2 x + c_3 \EH{-x} + c_4 x \EH{-x} - 12 x^2 + 2x^3$, \,
b) $y(x) = \left(c_1+c_2x+\frac{1}{2}x^2\right) \EH{2x}$\, 
c) $y(x) = \left(c_1+c_2x+c_3x^2+\frac{1}{24}x^4\right) \EH{-2x} $
}
