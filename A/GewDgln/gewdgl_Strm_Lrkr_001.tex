\Aufgabe[e]{LR-Kreis}
{
Ein Stromkreis habe einen Widerstand von \ $R=0.8$ Ohm \ und eine Selbstinduktion von \ $L=4$ Henry\,. Bis zur Zeit \ $t_0=0$ \ fließe kein Strom. Dann wird eine Spannung von \ $U=5$ Volt \ angelegt. Nach $5$ Sekunden wird
die Spannung abgeschaltet. Berechnen Sie den Stromverlauf \
$I(t)$ \ für \ $0 \le t \le 5$ \ und \ $t > 5$. \\  \\
\noindent
\textbf{Hinweis}: In diesem Stromkreis gilt $L\dot I(t) + RI(t)=U(t).$
}

\Loesung{
Es gilt gilt die Differentialgleichung
\[
  L\dot{I}(t)+RI(t)=U(t).
\]
Für \ $0\leq t\leq 5$ \ gilt \ $U(t)=5$\,. Die Trennung der Variablen führt zu
\[
  \int \frac{dI}{U-R I}=\int \frac{1}{L} \,dt,\Rightarrow
   -\frac{1}{R}\ln | U-RI(t)| =\frac{1}{L}t + c_1\,, \;\;    c_1 \in \R\,. 
\]
Auflösen nach $I(t)$ liefert (mit $c_2=\EH{c_1}$)
\[
  U-RI(t)=c_2 e^{-Rt/L}
\] 
und somit
$$
   I(t)=\frac{1}{R} \big( U-c_2 e^{-Rt/L} \big)\,.
$$
Einsetzen der Anfangsbedingung \ $I(0)=0$ \ ergibt \ $c_2=U$ \ und
$$
  I(t)=\frac{U}{R}\big(1-e^{-R/Lt}\big)\,.
$$
Einsetzen der gegebenen Zahlenwerte ergibt die Lösung
$$ 
  I(t) = 6.25\, \big(1-e^{-0.2 t}\big) \text{ f\"ur } 0 < t < 5\,. 
$$
Im Zeitraum \ $t>5$ \ ist $U(t)=0$ \ und der Anfangsstrom ist
$$
I(5)=I_0=6.25\,\big(1-e^{-1}\big)\ .
$$
Die Lösung der Differentialgleichung ist
$$
  \int \frac{dI}{I}=-\int\limits \frac{R}{L}dt\Rightarrow  \ln |I(t)|=-\frac{R}{L}t+c_3
$$ 
und damit
$$
  I(t)= c_4 e^{-Rt/L}\ .$$
Aus \ $I(t_0)=I_0$ \ folgt 
$$
 I(t) = I_0 e^{-R(t-t_0)/L}\ .
$$
Einsetzen der Zahlenwerte ergibt die Lösung 
$$ 
I(t) = 6.25 \big(1-e^{-1}\big) e^{-0.2 (t-5)} \text{ f\"ur } t > 5\ . 
$$
}


\ErgebnisC{gewdglStrmLrkr001}{
$I(t)=\left\{\begin{array}{ll}
6.25\, \big(1-e^{-0.2 t}\big)& \text{ f\"ur } 0 < t < 5\\
6.25 \big(1-e^{-1}\big) e^{-0.2 (t-5)}& \text{ f\"ur } t > 5
\end{array}\right.$
}
