\Aufgabe[e]{First order differential equations}
{
\begin{itemize}
\item[\textbf{1)}] Classify the following ordinary differential equations (ODEs) of the first order as
\begin{enumerate}
\item[\textbf{a)}] Linear or nonlinear.
\item[\textbf{b)}] In case of linear ODEs classify them as 
\begin{itemize}
\item homogeneous or inhomogeneous.
\item ODE with constant or non-constant coefficients.
\end{itemize}
\item[\textbf{c)}] Classify them according to the type as in the lecture notes:
\begin{enumerate}
\item $y'= f(x)\cdot g(y)$, solved by separation of variables,
\item $y'=g(y/x)$, homogeneous, solved with substitution $u=y/x$,
\item $y'=f(ax+by+c)$, right hand side with bilinear argument, solved with substitution $u=ax+by+c$,
\item $y'+p(x) \cdot y = q(x)$, linear ODE.
\end{enumerate}
\end{enumerate}
\begin{multicols}{2}
\begin{iii}
\item $y' + 2 y = 3x$.
\item $y'y + x = 0$.
\item $y'= \frac{x^2+y^2}{xy}$.
\item $y'=(x+y+1)^2$.
\item $x^2y'= x y + 2y^2$.
\item $y'=\frac{y(x-y)}{x^2}$.
\item $y'=\frac{x-y}{x+y}$.
\item $y'= \ln(y+2x+1)^2$.
\end{iii}
\end{multicols}
\item[\textbf{2)}] Compute the general solution of equations \textbf{i)} and \textbf{ii)}.
\end{itemize}
}


\Loesung{
\begin{itemize}
\item[\textbf{1)}] 
\begin{iii}
\item $y' + 2 y = 3x$. Linear, inhomogeneous, with constant coefficients, type: right hand side with bilinear argument.
\item $y'y + x = 0$. Nonlinear, type: separation of variables.
\item $y'= \frac{x^2+y^2}{xy}$. Nonlinear, type: homogeneous with substitution $u=y/x$.
%\item $x^2y' = 2y + 1$. Linear, inhomogeneous, non-constant coefficients, type D.
\item $y'=(x+y+1)^2$. Nonlinear, type: right hand side with bilinear argument.
\item $x^2y'= x y + 2y^2$. Nonlinear, type: homogeneous with substitution $u=y/x$.
%\item $y' = \cos(x)y$. Linear, homogeneous, non-constant coefficients, type D.
\item $y'=\frac{y(x-y)}{x^2}$. Nonlinear, type: homogeneous with substitution $u=y/x$.
\item $y'=\frac{x-y}{x+y}$. Nonlinear, type: homogeneous with substitution $u=y/x$.
\item $y'= \ln(y+2x+1)^2$. Nonlinear, type: right hand side with bilinear argument.
\end{iii}
\item[\textbf{2)}] Compute the general solution of equations \textbf{i)} and \textbf{ii)}.

\noindent To \textbf{i)}

\underline{The equation can be solved as linear equation:} it is linear, first order, with constant coefficients and inhomogeneous.
The solution can be determined as the sum of the solution of the homogeneous equation and a particular solution: 
$$
y(x) = y_h(x) + y_p(x).
$$
The solution of the homogeneous equation can be determined following the general solution procedure for the case with non-constant coeffcients as we do here to show the method. It could be also determined computing the zeros of the characteristic polynomial and this will be shown later.

\underline{The equation can be also interpreted as type: right hand side with bilinear argument.}
$$
y'=f(ax+by+c)=3x-2y
$$
and solved with a substitution as shown below.

Let's start with the general method. The solution of the homogeneous problem is
$$
y_h(x) = C \operatorname{e}^{-P(x)},
$$
where 
$$
P(x) = \int^x p(t) \d t
$$
and $p(x)$ is 2 in this case, so it is $P(x) = 2x$ and
$$
y_h(x) = C\operatorname{e}^{-2x}.
$$
For the particular solution with use the method of variations of constants
$$
y_p(x) = C(x) \operatorname{e}^{P(x)} = C(x) \operatorname{e}^{-2x}.
$$
We use the same ansatz function as in the homogeneous part but multiplied by the function $C(x)$ instead of the constant $C$.
To determine the expression for $C(x)$ we derive $y_p(x)$
$$
y_p'(x) = C'(x) \operatorname{e}^{-2x} -2 C(x) \operatorname{e}^{-2x}
$$
and insert $y$ and $y'$ into the ODE
\begin{align*}
C'(x) \operatorname{e}^{-2x} -2 C(x) \operatorname{e}^{-2x} + 2 C(x) \operatorname{e}^{-2x} &= 3x\\
C'(x) \operatorname{e}^{-2x} &= 3x\\
C'(x) = 3x \operatorname{e}^{2x}\\
\int \d C = 3 \int x \operatorname{e}^{2x} \d x.
\end{align*}
The integral on the right hand side is computed by integration by parts with
\begin{align*}
u=x, \quad u'&=1,\\
v'=\operatorname{e}^{2x}, \quad v&=\frac{1}{2}\operatorname{e}^{2x}.
\end{align*}
It is
\begin{align*}
\int x \operatorname{e}^{2x} \d x &= \frac{1}{2} \operatorname{e}^{2x} - \int \frac{1}{2} \operatorname{e}^{2x} \d x\\
&= \frac{1}{2} \operatorname{e}^{2x} - \frac{1}{4} \operatorname{e}^{2x} + \tilde C\\
&= \frac{1}{2}\operatorname{e}^{2x} ( x - \frac{1}{2}) + \tilde C.
\end{align*}
Coming back to the integral
$$
\int \d C = 3 \int x \operatorname{e}^{2x} \d x,
$$
we have
$$
C(x) = \frac{3}{2} \operatorname{e}^{2x} (x-\frac{1}{2}) + \tilde C.
$$
The constant $\tilde C$ can be set to zero, because it is already present in the solution of the homogeneous equation. 

The particular solution is
\begin{align*}
y_p(x) &= C(x) \operatorname{e}^{-2x} = \left( \frac{3}{2} \operatorname{e}^{2x} (x-\frac{1}{2})\right) \operatorname{e}^{-2x}\\
&=\frac{3}{2} (x-\frac{1}{2}).
\end{align*}
The general solution is
$$
y(x) = y_h(x) + y_p(x) = C \operatorname{e}^{-2x} + \frac{3}{2} (x-\frac{1}{2}).
$$

\underline{Now we solve the equation as a type: right hand side with bilinear argument}
$$
y'=f(ax+by+c)=3x-2y.
$$
With the substitution $u=3x-2y$, it is
$$
u'= 3 -2y'.
$$
Since $y'=3x-2y=u$ it is 
$$
u'= 3-2u.
$$
This can be solved by separation of variables
\begin{align*}
\frac{\d u}{\d x} &= 3-2u\\
 \int \frac{\d u}{3-2u} &= \int \d x\\
 -\frac{1}{2} \ln |3-2u| &= x + C\\
 -\frac{1}{3-2u} &= C\operatorname{e}^{2x},
\end{align*}
by back substitution it is
\begin{align*}
\frac{1}{3-6x+4y} &= C \operatorname{e}^{2x}\\
3-6x+4y &= C \operatorname{e}^{-2x}\\
y &= C\operatorname{e}^{-2x} +\frac{3}{2}x - \frac{3}{4}.
\end{align*}

\newpage
\underline{The equation can be also solved as a linear ODE with constant coeffcients:}

We derive the characteristic polynomial
$$
p(\lambda) = \lambda + 2.
$$
It has the root $\lambda = -2$, so the solution of the homogeneous equation is
$$
y_h(x) = C\operatorname{e}^{\lambda x} = C \operatorname{e}^{-2x}.
$$
A guess for the particular solution is 
$$
y_p(x) = A_1x + A_0.
$$
Differentiating this we get
$$
y_p'(x) = A_1.
$$
Inserting $y_p$ and $y_p'$ into the ODE we get
$$
A_1+2A_1x+2A_0 = 3x
$$
and equating the coeffcients we get
\begin{align*}
2A_1 &= 3\\
A_1+2A_0 &= 0
\end{align*}
which gives the values $A_0 = -\frac{3}{4}$ and $A_1 = \frac{3}{2}$ and the particular solution
$$
y_p(x) = \frac{3}{2}x-\frac{3}{4}.
$$
The general solution is again
$$
y(x) = y_h(x)+y_p(x) = C\operatorname{e}^{-2x} + \frac{3}{2}x-\frac{3}{4}.
$$
\noindent To \textbf{ii)}

The equation
$$
y'y = -x,
$$
is of \underline{type: separation of variables}
\begin{align*}
\int y \d y &= -\int x \d x\\
\frac{y^2}{2} &= -\frac{x^2}{2}+\widetilde C\\
y^2 &= C-x^2, \quad C = 2\widetilde C,\\
y &= \pm \sqrt{C-x^2}.
\end{align*}
The equation $y'y-x=0$ can be also interpreted as\\
\underline{type: nonlinear homogeneous with substitution $u=y/x$}
$$
y' = -\frac{x}{y}
$$ 
and solved by the substitution $u=\frac{y}{x}$.

From the relation $ux = y$, we differentiate on both sides to get
$$
u' x + u = y',
$$
where we have used the product rule for the product $ux$.
Substituting $y'=-\frac{1}{u}$ we get
\begin{align*}
u'x + u &= - \frac{1}{u}\\
u' &= -\frac{1}{x} (u+\frac{1}{u})\\
\int \frac{u}{u^2+1}\d u &= \int -\frac{1}{x}\d x\\
\frac{1}{2}\ln (u^2+1) &= -\ln x + \ln \widetilde C\\
u^2+1 &= \frac{C}{x^2}, \quad C=\widetilde C^2\\
u^2 &= \frac{C}{x^2}-1,
\end{align*}
with the back substitution it is
\begin{align*}
u^2 &= \frac{C}{x^2}-1,\\
\frac{y^2}{x^2} &= \frac{C}{x^2}-1,\\
y^2 & = C -x^2\\
y&=\pm \sqrt{C-x^2}.
\end{align*}

\end{itemize}
}



