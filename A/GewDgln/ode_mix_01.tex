\Aufgabe[e]{Differentialgleichungen erster Ordnung}
{
\begin{itemize}
\item[\textbf{1)}] Klassifizieren Sie die follgenden gew\"ohnlichen Differentialgleichungen 
                   erster Ordnung als 

\begin{enumerate}
\item[\textbf{a)}] Linear oder nicht-linear.
\item[\textbf{b)}] In dem Fall einer linearen Differentialgleichung klassifizieren Sie 
                   die Gleichung zus\"atzlich als
\begin{itemize}
\item homogen oder inhomogen.
\item Differentialgleichung mit konstanten oder nicht-konstanten Koeffizienten.
\end{itemize}
\item[\textbf{c)}]  Nutzen Sie die Vorlesungsunterlagen, um die Differentialgleichung als
                    einen der folgenden Typen zu klassifizieren:
\begin{enumerate}
\item $y'= f(x)\cdot g(y)$, zu lösen mittels Trennung der Variablen,
\item $y'=g(y/x)$, homogen, zu l\"osen mittels Substitution mit $u=y/x$,
\item $y'=f(ax+by+c)$, rechte Seite mit bilinearen Argumenten, zu lösen mit der 
      Substitution $u=ax+by+c$,
\item $y'+p(x) \cdot y = q(x)$, lineare Differentialgleichung.
\end{enumerate}
\end{enumerate}
\begin{multicols}{2}
\begin{iii}
\item $y' + 2 y = 3x$.
\item $y'y + x = 0$.
\item $y'= \frac{x^2+y^2}{xy}$.
\item $y'=(x+y+1)^2$.
\item $x^2y'= x y + 2y^2$.
\item $y'=\frac{y(x-y)}{x^2}$.
\item $y'=\frac{x-y}{x+y}$.
\item $y'= \ln(y+2x+1)^2$.
\end{iii}
\end{multicols}
\item[\textbf{2)}] Berechnen Sie die allgemeine L\"osung der Gleichungen \textbf{i)} 
                   bis \textbf{iv)}.
\end{itemize}
}


\Loesung{
\begin{itemize}
\item[\textbf{1)}] 
\begin{iii}
\item $y' + 2 y = 3x$. Linear, inhomogen, mit konstanten Koeffizients, Typ: rechte 
      Seite mit bilinearen Koeffizienten.
\item $y'y + x = 0$. Nicht-linear, Typ: Trennung der Variablen.
\item $y'= \frac{x^2+y^2}{xy}$. Nicht-linear, Typ: homogen mit Substitution $u=y/x$.
%\item $x^2y' = 2y + 1$. Linear, inhomogeneous, non-constant coefficients, type D.
\item $y'=(x+y+1)^2$. Nicht-linear, Typ: rechte Seite mit bilinearen Argumenten.
\item $x^2y'= x y + 2y^2$. Nicht-linear, Typ: homogen mit Substitution $u=y/x$.
%\item $y' = \cos(x)y$. Linear, homogeneous, non-constant coefficients, type D.
\item $y'=\frac{y(x-y)}{x^2}$. Nicht-linear, Typ: homogen mit Substitution $u=y/x$.
\item $y'=\frac{x-y}{x+y}$. Nicht-linear, Typ: homogen mit Substitution $u=y/x$.
\item $y'= \ln(y+2x+1)^2$. Nicht-linear, Typ: rechte Seite mit bilineren Argumenten.
\end{iii}
\item[\textbf{2)}] Berechnen Sie die allgemeine L\"osung der Gleichungen \textbf{i)} und \textbf{ii)}.

\noindent Zu \textbf{i)}

\underline{Die Gleichung kann als lineare Gleichung gel\"ost werden:}
Die Gleichung ist linear, erster Ordnung, mit konstanten Koeffizienten und inhomogen.
% it is linear, first order, with constant coefficients and inhomogeneous.
Die L\"osung kann als Summe aus der L\"osung der homogenen Gleichung und einer partikul\"aren 
L\"osung bestimmt werden:
% The solution can be determined as the sum of the solution of the homogeneous equation and a particular solution: 
$$
y(x) = y_h(x) + y_p(x).
$$
Die Lösung der homogenen Gleichung mit dem allgemeinen Lösungsverfahren für den Fall mit 
nicht-konstanten Koeffizienten, den wir hier zeigen. Man kann die Lösung auch durch Berechnung 
der Nullstellen des charakteristischen Polynoms bestimmen. Dieser Methode wird später gezeigt.

% The solution of the homogeneous equation can be determined following the general solution procedure for the case with non-constant coeffcients as we do here to show the method. It could be also determined computing the zeros of the characteristic polynomial and this will be shown later.

\underline{Die Gleichung kann interpretiert werden vom Typ rechte Seite mit bilinearen}\\
\underline{Argumenten.}
% The equation can be also interpreted as type: right hand side with bilinear argument.}
$$
y'=f(ax+by+c)=3x-2y
$$
und wird mit der Substitution, wie unten gezeigt, gelöst.
% and solved with a substitution as shown below.

Wir beginnen mit der allgemeinen Methode. Die Lösung des homogenen Problems ist
% Let's start with the general method. The solution of the homogeneous problem is
$$
y_h(x) = C \operatorname{e}^{-P(x)},
$$
wobei
$$
P(x) = \int^x p(t) \d t
$$
und $p(x)$ in diesem Fall 2 ist, sodass $P(x) = 2x$ gilt und
$$
y_h(x) = C\operatorname{e}^{-2x}.
$$
Für die partikuläre Lösung nutzen wir die Methode der Variation der Konstanten
% For the particular solution with use the method of variations of constants
$$
y_p(x) = C(x) \operatorname{e}^{P(x)} = C(x) \operatorname{e}^{-2x}.
$$
Wir nutzen dieselbe Ansatzfunktion wie im homogenen Teil aber multipliziert mit der 
Funktion $C(x)$ statt der Konstanten $C$.
% We use the same ansatz function as in the homogeneous part but multiplied by the function $C(x)$ instead of the constant $C$.
Um den Ausdruck für $C(x)$ zu bestimmen, leiten wir $y_p(x)$ ab
% To determine the expression for $C(x)$ we derive $y_p(x)$
$$
y_p'(x) = C'(x) \operatorname{e}^{-2x} -2 C(x) \operatorname{e}^{-2x}
$$
und setzen $y$ und $y'$ in die Differentialgleichung ein
\begin{align*}
C'(x) \operatorname{e}^{-2x} -2 C(x) \operatorname{e}^{-2x} + 2 C(x) \operatorname{e}^{-2x} &= 3x\\
C'(x) \operatorname{e}^{-2x} &= 3x\\
C'(x) = 3x \operatorname{e}^{2x}\\
\int \d C = 3 \int x \operatorname{e}^{2x} \d x.
\end{align*}
Das Integral auf der rechten Seite  wird mit partieller Integration berechnet
% The integral on the right hand side is computed by integration by parts with
\begin{align*}
u=x, \quad u'&=1,\\
v'=\operatorname{e}^{2x}, \quad v&=\frac{1}{2}\operatorname{e}^{2x}.
\end{align*}
Es gilt
\begin{align*}
\int x \operatorname{e}^{2x} \d x &= \frac{1}{2} \operatorname{e}^{2x} - \int \frac{1}{2} \operatorname{e}^{2x} \d x\\
&= \frac{1}{2} \operatorname{e}^{2x} - \frac{1}{4} \operatorname{e}^{2x} + \tilde C\\
&= \frac{1}{2}\operatorname{e}^{2x} ( x - \frac{1}{2}) + \tilde C.
\end{align*}
Zurück zu dem Integral
% Coming back to the integral
$$
\int \d C = 3 \int x \operatorname{e}^{2x} \d x,
$$
erhalten wir
$$
C(x) = \frac{3}{2} \operatorname{e}^{2x} (x-\frac{1}{2}) + \tilde C.
$$
Die Konstante $\tilde C$ kann null gesetzt werden, weil sie bereits in der Lösung der 
homogenen Gleichung berücksichtigt wurde.
% The constant $\tilde C$ can be set to zero, because it is already present in the solution of the homogeneous equation. 

Die partikuläre Lösung ist
% The particular solution is
\begin{align*}
y_p(x) &= C(x) \operatorname{e}^{-2x} = \left( \frac{3}{2} \operatorname{e}^{2x} (x-\frac{1}{2})\right) \operatorname{e}^{-2x}\\
&=\frac{3}{2} (x-\frac{1}{2}).
\end{align*}
Die allgemeine Lösung ist
% The general solution is
$$
y(x) = y_h(x) + y_p(x) = C \operatorname{e}^{-2x} + \frac{3}{2} (x-\frac{1}{2}).
$$

\underline{Wir lösen die Gleichung nun als Typ: rechte Seite mit bilinearen}\\
\underline{Argumenten.}
$$
y'=f(ax+by+c)=3x-2y.
$$
Mit der Substitution $u=3x-2y$, erhalten wir
$$
u'= 3 -2y'.
$$
Da $y'=3x-2y=u$ gilt 
$$
u'= 3-2u.
$$
Diese lösen wir mit Trennung der Variablen
% This can be solved by separation of variables
\begin{align*}
\frac{\d u}{\d x} &= 3-2u\\
 \int \frac{\d u}{3-2u} &= \int \d x\\
 -\frac{1}{2} \ln |3-2u| &= x + C\\
 \frac{1}{3-2u} &= C\operatorname{e}^{2x},
\end{align*}
mit der Rücksubstitution erhalten wir
% by back substitution it is
\begin{align*}
\frac{1}{3-6x+4y} &= C \operatorname{e}^{2x}\\
3-6x+4y &= C \operatorname{e}^{-2x}\\
y &= C\operatorname{e}^{-2x} +\frac{3}{2}x - \frac{3}{4}.
\end{align*}

\newpage
\underline{Die Differentialgleichung gelöst werden als lineare Differentialgleichung mit}\\
\underline{ konstanten Koeffizienten:}

Wir bestimmen das charakteristische Polynom
% We derive the characteristic polynomial
$$
p(\lambda) = \lambda + 2
$$
mit den Nullstellen $\lambda = -2$. Damit ist die Lösung der homogenen Gleichung 
$$
y_h(x) = C\operatorname{e}^{\lambda x} = C \operatorname{e}^{-2x}.
$$
Der Ansatz für die partikuläre Lösung ist
$$
y_p(x) = A_1x + A_0.
$$
Durch Ableiten erhalten wir
$$
y_p'(x) = A_1.
$$
Wir setzen $y_p$ und $y_p'$ in die Differentialgleichung ein und erhalten
% Inserting $y_p$ and $y_p'$ into the ODE we get
$$
A_1+2A_1x+2A_0 = 3x.
$$
Ein Koeffizientenvergleich liefert
% and equating the coeffcients we get
\begin{align*}
2A_1 &= 3\\
A_1+2A_0 &= 0
\end{align*}
wodurch wir die Werte $A_0 = -\frac{3}{4}$ und $A_1 = \frac{3}{2}$ erhalten und 
die partikuläre Lösung
$$
y_p(x) = \frac{3}{2}x-\frac{3}{4}.
$$
Die allgemeine Lösung ist wieder
$$
y(x) = y_h(x)+y_p(x) = C\operatorname{e}^{-2x} + \frac{3}{2}x-\frac{3}{4}.
$$
\noindent Zu \textbf{ii)}

Die Gleichung
$$
y'y = -x,
$$
ist vom \underline{Typ: Trennung der Variablen}
\begin{align*}
\int y \d y &= -\int x \d x\\
\frac{y^2}{2} &= -\frac{x^2}{2}+\widetilde C\\
y^2 &= C-x^2, \quad C = 2\widetilde C,\\
y &= \pm \sqrt{C-x^2}.
\end{align*}
Die Gleichung $y'y-x=0$ kann auch interpretiert werden als\\
\underline{Typ: nicht-linear, homogen mit der Substitution $u=y/x$}
$$
y' = -\frac{x}{y}
$$ 
und kann mit der Substitution $u=\frac{y}{x}$ gelöst werden.

Aus der Beziehung $ux = y$, erhalten wir durch differenzieren beider Seiten
$$
u' x + u = y',
$$
wobei wir die Produktregel benutzen.
% where we have used the product rule for the product $ux$.
Mit der Substitution $y'=-\frac{1}{u}$ erhalten wir
\begin{align*}
u'x + u &= - \frac{1}{u}\\
u' &= -\frac{1}{x} (u+\frac{1}{u})\\
\int \frac{u}{u^2+1}\d u &= \int -\frac{1}{x}\d x\\
\frac{1}{2}\ln (u^2+1) &= -\ln x + \ln \widetilde C\\
u^2+1 &= \frac{C}{x^2}, \quad C=\widetilde C^2\\
u^2 &= \frac{C}{x^2}-1,
\end{align*}
durch die Rücksubstitution erhalten wir
\begin{align*}
u^2 &= \frac{C}{x^2}-1,\\
\frac{y^2}{x^2} &= \frac{C}{x^2}-1,\\
y^2 & = C -x^2\\
y&=\pm \sqrt{C-x^2}.
\end{align*}

\noindent Zu \textbf{iii)}
Wir schreiben die Differentialgleichung als
\begin{align*}
y' = \frac{x^2+y^2}{xy} = \frac{x^2}{xy} + \frac{y^2}{xy} = \frac{x}{y} +\frac{y}{x}.
\end{align*}
Wir nutzen nun die Substitution $u = \frac{y}{x}$. 
Wir berechnen die Ableitung von $y = u(x)x$.
\begin{align*}
y' = u'x+u.
\end{align*}
Durch Einsetzen erhalten wir:
\begin{align*}
u'x+u = \frac{1}{u}+u.
\end{align*}
Dies vereinfacht sich zu 
\begin{align*}
u'x= \frac{1}{u}.
\end{align*}
Diese Gleichung können wir mit Trennung der Variablen lösen
\begin{align*}
u \d u &= \frac{1}{x} \d x\\
\int u \d u &= \int \frac{1}{x} \d x \\
\frac{1}{2}u^2 &= \ln|x| + C \\
u &= \pm\sqrt{2\ln|x| +C}.
\end{align*}
Mit Rücksubstitution erhalten wir die Lösung
\begin{align*}
y = \pm x\sqrt{2\ln|x| +C}.
\end{align*}

\noindent Zu \textbf{iv)}
\begin{align*}
y'= (x+y+1)^2
\end{align*}
Wir lösen die Gleichung mittels der Substitution $u = x + y +1$.
Es gilt 
\begin{align*}
y' = u'-1 
\end{align*}
Damit erhalten wir 
\begin{align*}
u' - 1 = u^2
\end{align*}
Mit der Trennung der Variablen erhalten wir
\begin{align*}
\frac{\d u}{u^2 +1} &= 1\d x\\
\int\frac{\d u}{u^2 +1} &=\int 1\d x\\
\arctan(u) &= x + C \\
u &= \tan(x + C)
\end{align*}
Wir erhalten die Lösung durch Rücksubstitution
\begin{align*}
y &= \tan(x + C) -x-1
\end{align*}


\end{itemize}
}



