\Aufgabe[e]{Definitionsbereich der Lösung einer Dgl.}
{
\begin{abc}
\item Bestimmen Sie die allgemeine Lösung der Differentialgleichung 
\[
y'(x)=2\,x\,y^2 
\]
und die spezielle Lösung für den Anfangswert \ $y(0)=4$\ .

Wie groß ist der maximale Definitionsbereich dieser Lösung\thinspace ?\\
\item Bestimmen Sie die allgemeine Lösung der Differentialgleichung
	\[
	\cos(x)\cdot y'(x) = \sin(x)\cdot y(x).
\]
\end{abc}
}

\Loesung{
\begin{abc}
\item Die Dgl. ist vom trennbaren Typ 
\[
	\int\frac{\text dy}{y^2} = \int2x\ \text dx \,\Rightarrow\, \frac{y^{-1}}{-1} = x^2+ C \quad\vee\quad y=0
\]
und hat die allgemeine Lösung 
\[
\underline{\;y(x)=\frac{-1}{x^2+C}\;\;,\;\;\;\;C\in\mathbb{R}\;\quad \text{bzw. } y=0} 
\]

Der Anfangswert \ $y(0)=4$ \ ergibt mit \ $C=-\frac{1}{4}$ \ die Lösung 
\[
\underline{\;y_{\text{AWP}}(x)=\frac{-1}{x^{2}-\frac{1}{4}}\;}\;. 
\]
Die Lösung ist nur im Bereich \ $\frac{-1}{2}<x<\frac{1}{2}$ \ definiert und hat an den Rändern bei \ $x=\pm \frac{1}{2}$ \
Polstellen.

\item L\"osen der hom. lin. DGl. \ 
$$
\cos(x)\cdot y'(x) = \sin(x)\cdot y(x)$$ 
durch Trennung der Veränderlichen:
	\[
	\int\frac{\text dy}{y} = \int\frac{\sin(x)}{\cos(x)}\ \text dx \,\Rightarrow\, \ln(|y|) = -\ln\big(|\cos(x)|\big)+\tilde C \quad\vee\quad y=0
\]
also
  \[ y(x) = \frac{C}{\cos(x)},C\in\mathbb{R}\ .
\]
\end{abc}
}
\ErgebnisC{gewdglDglnDefb001}{
\textbf{a)} $y_{\text{AWP}}(x)=\frac{-1}{x^{2}-\frac{1}{4}}$,\, \textbf{b)} $y(x) = \frac{C}{\cos(x)}$
}
