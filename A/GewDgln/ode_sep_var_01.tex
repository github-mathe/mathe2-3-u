\Aufgabe[e]{Trennung der Variablen}{
Lösen die folgenden Anfangswertprobleme und bestimmen Sie den maximalen Definitionsbereich 
der Lösung
% Solve the following initial value problems and determine the interval of validity for the solution
\begin{abc}
\item $y'(x) = 6y^2(x)x, \quad y(1) = \dfrac{1}{6}$.
\item $y'(x) = \dfrac{{3{x^2} + 4x - 4}}{{2y(x) - 4}},\quad y(1) = 3$.
\item $y'(x) = {\operatorname{e}^{ - y(x)}}\left( {2x - 4} \right), \quad y(5) = 0$.
\item $y'(x) = \dfrac{1}{x^2}, \quad y(x_0)=0$.
\item $y'(x) = x^2, \quad y(0)=y_0$.
\end{abc}
}

\Loesung{
\begin{abc}
\item $y'(x) = 6y^2(x)x, \quad y(1) = \dfrac{1}{6}$.
Mit Trennung der Variablen gilt
% With separation of variables it is
\begin{align*}
\frac{\d y}{y^2} &= 6x \d x\\
\int \frac{\d y}{y^2} &= \int 6x \d x\\
-\frac{1}{y} &= 3x^2+ C.
\end{align*}
Mit der Anfangsbedingung $y(1) = \frac{1}{6}$ erhalten wir
\begin{align*}
-6 &= 3 + C\\
-9 & = C.
\end{align*}
Die Lösung ist dann
$$
y = \frac{1}{9-3x^2}.
$$
Wir bestimmen nun den Gültigkeitsbereich der Lösung. Es muss gelten
$$9 - 3x^2 \neq 0,$$
daher
$$
x\neq \pm \sqrt{3}.
$$
Die Werte $x=\pm \sqrt{3}$ müssen vermieden werden, damit erhalten wir die 
folgenden möglichen Gültigkeitsbereiche: 
$$
-\infty < x < -\sqrt{3}, \quad -\sqrt{3} < x < \sqrt{3}, \quad \sqrt{3} < x < \infty.
$$
Da die Lösung in $x=1<\sqrt{3}$ positiv ist (siehe Anfangswert) ist der Gültigkeitsbereich
in dem Intervall
$$
-\sqrt{3} < x < \sqrt{3}.
$$
\item $y'(x) = \dfrac{{3{x^2} + 4x - 4}}{{2y(x) - 4}},\quad y(1) = 3$.
Es gilt 
\begin{align*}
(2y-4) \d y &= (3x^2+4x-4) \d x\\
\int (2y-4) \d y &= \int(3x^2+4x-4) \d x\\
y^2-4y & = x^3 +2x^2-4x+C.
\end{align*}
Durch Anwenden der der Anfangsbedingung, gilt
\begin{align*}
9 -12 & = 1 + 2 -4 +C\\
-2 &= C.
\end{align*}
Mit $d=-x^3-2x^2+4x+2$ gilt
$$y^2-4y+d = 0$$
welches eine quadratische Gleichung mit der Lösungsmenge
\begin{align*}
y &= 2\pm \sqrt{4-d}\\
&=2\pm \sqrt{x^3+2x^2-4x+2}.
\end{align*}
Von den zwei Kandidaten für die Lösung ist nur eine eine gültige Lösung. Das 
kann mit Hilfe der Anfangsbedingung nachgewiesen werden $y(1) = 3$. 
In der Tat gilt
\begin{align*}
3 &= 2+\sqrt{1+2-4+2},\\
3 &\neq 2-\sqrt{1+2-4+2},
\end{align*}
daher ist die Lösung mit dem negativen Term $2-\sqrt{4-d}$ nicht gültig.

Um den Gültigkeitsbereich der Lösung zu untersuchen, nutzen wir  
$$
4-d = x^3+2x^2-4x+2 \geq 0.
$$
Durch Einsetzen verschiener Werte für $x$, können wir überprüfen, dass für $x=-3$ 
die Funktion $x^3+2x^2-4x+2$ positiv und für $x=-4$ negativ ist.
Da die Funktion stetig ist, muss die Nullstelle zwischen -4 und -3 liegen. Wir bezeichnen 
diesen Wert mit $\bar x$, der Gültigkeitsbereich ist dann 
$$
x\geq \bar x \approx -3.36.
$$
\item $y'(x) = {\operatorname{e}^{ - y(x)}}\left( {2x - 4} \right), \quad y(5) = 0$.
Es gilt
\begin{align*}
\int \operatorname{e}^y \d y &= \int (2x-4 \d x)\\
\operatorname{e}^y & = x^2-4x +C.
\end{align*}
Durch Einsetzen des Anfangswertes, erhalten wir die Konstante $C=-4$.
Die Lösung ist daher
$$
y = \ln(x^2-4x-4).
$$
Für die Gültigkeit muss gelten
$$
x^2-4x-4 > 0.
$$
Die Nullstellen der Funktion $x^2-4x-4$ sind $x=2 \pm 2\sqrt{2}$. Da die Funktion 
konvex ist, ist die Funktion positiv in dem Intervall
$$
-\infty < x < 2-2\sqrt{2} \quad \text{and} \quad 2+2\sqrt{2} < x < \infty.
$$
Da der Anfangswert bei $x=5$ liegt, ist der Gültigkeitsbereich das Intervall $2+2\sqrt{2} < x < \infty$.
\item $y'(x) = \dfrac{1}{x^2}, \quad y(x_0)=0$.
Es gilt
\begin{align*}
\d y &= \frac{\d x}{x^2}\\
\int \d y &= \int \frac{\d x}{x^2}\\
y &= -\frac{1}{x} + C.
\end{align*}
Durch Anwenden der Anfangwertbedingung erhalten wir 
$$C=\frac{1}{x_0},$$
woraus wir $x_0 \neq 0$ erhalten.
Die Lösung ist
$$
y = -\frac{1}{x} + \frac{1}{x_0}.
$$
Für den Gültigkeitsbereich muss gelten, dass $x\neq 0$, damit ist er gegeben als $0 < x < \infty$ falls $x_0 > 0$ und $-\infty < x < 0$ falls $x_0 < 0$.
\item $y'(x) = x^2, \quad y(0)=y_0$.
Diese einfache Gleichung hat die Lösung
$$
y = \frac{x^3}{3} + y_0,
$$
und der Gültigkeitsbereich ist der ganze $\mathbb R$.
\end{abc}
}


\ErgebnisC{sepvar01}
{
\begin{abc}
\item $y(x) = \dfrac{1}{9-3x^2}$ mit $-\sqrt{3} < x < \sqrt{3}$
\item $y(x) = 2\pm \sqrt{x^3+2x^2-4x+2}$ mit $x\geq \bar x \approx -3.36$ 
\item $y(x) = \ln(x^2-4x-4)$ mit $2+2\sqrt{2} < x < \infty$
\item $y(x) = -\frac{1}{x} + \frac{1}{x_0}$ mit $0 < x < \infty$ falls $x_0 > 0$ und $-\infty < x < 0$ falls $x_0 < 0$
\item $y(x) = \frac{x^3}{3} + y_0$ mit $x \in \mathbb{R}$
\end{abc}
}