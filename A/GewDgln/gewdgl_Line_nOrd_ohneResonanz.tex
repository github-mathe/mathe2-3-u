\Aufgabe[e]{Lineare Differentialgleichungen $n$-ter Ordnung ohne Resonanz}
{
Bestimmen Sie die allgemeinen L\"osungen der folgenden
Differentialgleichungen. Falls Anfangswerte gegeben sind, ermitteln Sie auch die L\"osung des Anfangswertproblems. 
\begin{abc}
\item $y^{\prime\prime}(x) - 2 y^\prime(x) - 3y(x) = 4 \EH{x}, \quad y(0) = 0, 
         \quad y'(0) = 6,$
%\item \textbf{R} $y^{\prime\prime}(x)+4y^\prime(x)+4y(x)=4 \EH{-2x}, \quad y(0)=1 ,\quad
%  y^\prime(0)=0$.
  \item $y^{\prime\prime}(x)+5y^\prime(x)+6y(t)=3\EH{3x}$, 
       % (x+2)(x+3) keine Resonanz
%  \item \textbf{R} $y^{\prime\prime}(x)-2y^\prime(x)+y(x)=4\EH x$, 
%	% (x-1)^2 Resonanz
  \item $y^{\prime\prime}(x)-y^\prime(x)-2y(x)= 4x\EH{x}$. 
	%(x+1)(x-2) keine Resonanz
\item  $y^{\prime\prime\prime}+y^{\prime\prime}-y^\prime-y=3\EH{-2x}$,

\end{abc}
}

\Loesung{
\begin{abc}

\item Die Nullstellen des charakteristischen Polynoms
$p(\lambda)=\lambda^2-2\lambda-3$ sind $\lambda_1=-1$ und $\lambda_2=3$. 
Eine Partikul\"arl\"osung der inhomogenen Gleichung berechnet man mit
dem Ansatz $y_p(x)=ae^x$, es folgt
$ -4 a e^x \stackrel{!}{=}4e^x$ und damit $a=-1$. Die allgemeine L\"osung
ist 
$$ y_{allg}(x) = - e^x + c_1 \EH{-x} + c_2 \EH{3x} \ \text{ mit } c_1,c_2 \in \R .$$
Aus den Anfangsbedingungen $y(0)=-1+c_1+c_2 \stackrel{!}{=} 0$
und $y^\prime(0) = -1 - c_1 + 3c_2 \stackrel{!}{=} 6$ folgt das lineare 
Gleichungssystem
$$ \begin{array}{rcrl}
   c_1 & + & c_2 & = 1, \\
   -c_1 & + & 3c_2 & = 7 
   \end{array} $$
mit L\"osung $c_2=2$ und $c_1=-1$. Damit ist 
$$ y_{AWP}(x) = -e^x-\EH{-x}+2\EH{3x}. $$

%\item Die Nullstelle von $p(\lambda)=\lambda^2+4\lambda+4$ ist
%$\lambda=-2$, dies ist eine doppelte Nullstelle. Als Ansatz f\"ur die
%Partikul\"arl\"osung muss man 
%$y_p(x)=ax^2 \EH{-2x}$ nehmen, denn man hat Resonanz der Ordnung 2. 
%Mit $y_p^\prime(x) = a \EH{-2x} \big( 2x-2x^2\big)$
%und $y_p^{\prime\prime}(x) = a \EH{-2x} \big(2-8x+4x^2\big)$
%folgt
%$2a \EH{-2x} \stackrel{!}{=} 4 \EH{-2x}$ und damit $a=2$. 
%Dies liefert die allgemeine L\"osung
%$$ y_{allg}(x) = \big( 2x^2 + c_1 x + c_2 \big) \EH{-2x}
%   \ \text{ mit } c_1,c_2 \in \R. $$
%Die Anfangsbedingungen  $y(0)=c_2 \stackrel{!}{=} 1$ und
%$y^\prime(0) = c_1-2c_2 \stackrel{!}{=} 0$ liefern $c_2=1$ und $c_1=2$ 
%und damit die L\"osung
%%$$ y_{AWP}(x) = \big( 2x^2 + 2x + 1 \big) \EH{-2x} . $$ 
\item Man berechnet zuerst die L\"osungen der homogenen Gleichung
$$ y^{\prime\prime}+5y^\prime+6y=0. $$
Das charakteristische Polynom
$p(\lambda)=\lambda^2+5\lambda+6$ hat die Nullstellen $\lambda_1=-2$ und 
$\lambda_2=-3$. Ein Fundamentalsystem ist 
$\{ \EH{-2x},\EH{-3x}\}$. Nun braucht man noch eine spezielle L\"osung
der inhomogenen Gleichung. Diese berechnet man mit dem Ansatz
$y_p(x)=a \EH{3x}$ mit $a \in \R$. 
Einsetzen in die inhomogene DGl liefert
$(9+15+6)a\EH{3x}=3\EH{3x}$, also $a=1/10$.
Die allgemeine L\"osung der
Gleichung ist
$$ y(x) = c_1 \EH{-2x}+c_2\EH{-3x}+\frac{1}{10} \EH{3x}
   \text{ mit } c_1,c_2 \in \R . $$
%\item Aus $p(\lambda)=\lambda^2-2\lambda+1=0$ folgt $\lambda_{1/2}=1$ (doppelte 
%Nullstelle). Damit hat man das Fundamentalsystem
%$\{ e^x, xe^x \}$.  F\"ur die partikul\"are L\"osung muss man nun den
%Ansatz $y_p(x)=a x^2 e^x$ machen, denn es liegt Resonanz der Ordnung 
%$2$ vor. Es ist $y_p^\prime(x)=a(2x+x^2)e^x$ und
%$y_p^{\prime\prime}(x)=a(2+4x+x^2)e^x$.
%Eingesetzt in die inhomogene DGL erhält man 
%$$ ae^x(2+4x+x^2-2(2x+x^2)+x^2 )= 4 e^x, $$
%$$2ae^x=4 e^x$$
%woraus $a=2$ folgt. Die allgemeine L\"osung der Gleichung ist
%$$ y(x) = \big(c_1 + c_2 x + 2x^2\big) e^x  \text{ mit } c_1,c_2 \in \R . $$
\item Das Polynom $p(\lambda)=\lambda^2-\lambda-2$ hat die Nullstellen
$\lambda_1=2$ und $\lambda_2=-1$, dies ergibt das Fundamentalsystem
$\{\EH{2x},\EH{-x}\}$. Der Ansatz f\"ur die Partikul\"arl\"osung ist
$y_p(x)=(ax+b) e^x$. Mit $y_p^\prime(x)=(ax+a+b)e^x$
und $y_p^{\prime\prime}=(ax+2a+b)e^x$ folgt
$$ \EH{x}(ax+2a+b-(ax+a+b)-2(ax+b))=4x\EH{x} $$
$$ -2ax+a-2b=4x $$
Koeffizientenvergleich liefert dann $a=-2$ und $2b=a$, $b=-1$. Damit hat man die allgemeine 
L\"osung 
$$ y(x) = c_1 \EH{2x}+c_2\EH{-x}-(2x+1)e^x  \text{ mit } c_1,c_2 \in \R. $$
\item Das charakteristische Polynom ist 
$p(\lambda) = \lambda^3+\lambda^2-\lambda-1$. Eine Nullstelle kann man raten, 
zum Beispiel $\lambda_1=1$. Polynomdivision oder Anwendung des
Horner--Schemas liefert dann
$$ p(\lambda) = (\lambda-1)\big(\lambda^2+2\lambda+1\big), $$
damit ist $\lambda_2=-1$ eine weitere, und zwar doppelte, Nullstelle.
Folglich hat die homogene Gleichung das Fundamentalsystem
$$ \big\{ e^x, \EH{-x}, x\EH{-x} \big\}. $$
Zur Berechnung einer Partikul\"arl\"osung benutzt man den Ansatz
$y_p(x) = a \EH{-2x}$. Einsetzen in die Differentialgleichung liefert
$$ a \EH{-2x} \big(-8+4+2-1 \big) \stackrel{!}{=} 3 \EH{-2x} $$
und damit $a=-1$. Die allgemeine L\"osung ist also
$$ y(x) = - \EH{-2x} + c_1 \EH{x} + c_2 \EH{-x} + c_3 x \EH{-x} 
   \ \text{ mit } c_1,c_2,c_3 \in \R. $$

\end{abc} 
}


\ErgebnisC{AufggewdglLinenOrdohneResonanz}
{
\begin{abc}
\item $y_{AWP}(x) = -e^x-\EH{-x}+2\EH{3x}$
\item $y(x) = c_1 \EH{-2x}+c_2\EH{-3x}+\frac{1}{10} \EH{3x}$
\item $y(x) = c_1 \EH{2x}+c_2\EH{-x}-(2x+1)e^x$
\item $y(x) = - \EH{-2x} + c_1 \EH{x} + c_2 \EH{-x} + c_3 x \EH{-x}$
\end{abc}
}
