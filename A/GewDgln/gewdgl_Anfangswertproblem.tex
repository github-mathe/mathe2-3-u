\Aufgabe[e]{Anfangswertproblem}
{
Gegeben ist die Dgl.
$$
y'(x) = \dfrac{1}{y\sqrt{x}}, \, x>0, \, y \neq 0
$$
mit der allgemeinen L\"osung
$$
y(x) = \pm \sqrt{4\sqrt{x}+C}, \, C\geq 0.
$$
\begin{abc}
\item Geben Sie, falls m\"oglich, die L\"osung der Dgl. an, die die Anfangswertbedingung $y(1)=3$ erf\"ullt.
\item Geben Sie, falls m\"oglich, die L\"osung der Dgl. an, die die Anfangswertbedingung $y(1)=-4$ erf\"ullt.
\item Geben Sie, falls m\"oglich, die L\"osung der Dgl. an, die die Anfangswertbedingung $y(-1)=3$ erf\"ullt.
\end{abc}
}

\Loesung{
\begin{abc}
\item Die Konstante $C$ muss so bestimmt werden, dass die allgemeine L\"osung $y(x) = \pm \sqrt{4\sqrt{x}+C}$ die Anfangsbedingung $y(1)=3$ erf\"ullt:
$$
3 = \pm \sqrt{4\sqrt{1}+C} \Rightarrow 3 = + \sqrt{4+C} \Rightarrow C=5.
$$
Die Anfangsbedingung legt also sowohl das Vorzeichen, als auch die Konstante fest. Die spezielle L\"osung des AWP ist also
$$
y(x) = \sqrt{4\sqrt{x}+5}, \, x\geq 0.
$$
\item Analog zu Aufgabenteil a):
$$
-4 = \pm \sqrt{4\sqrt{1}+C} \Rightarrow -4 = - \sqrt{4+C} \Rightarrow C=12.
$$
Es ergibt sich die spezielle L\"osung 
$$
y(x) = -\sqrt{4\sqrt{x}+12}, \, x\geq 0.
$$
\item Es gibt keine L\"osung, da die Dgl. nicht bei $x=-1$ definiert ist
\end{abc}
}


\ErgebnisC{gewdgl_Anfangswertproblem}{
\begin{abc}
\item $y(x) = \sqrt{4\sqrt{x}+5}, \, x\geq 0$
\item $y(x) = -\sqrt{4\sqrt{x}+12}, \, x\geq 0$
\item keine L\"osung, da Dgl. nicht bei $x=-1$ definiert ist
\end{abc}
}

