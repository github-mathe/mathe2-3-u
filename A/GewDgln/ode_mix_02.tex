\Aufgabe[e]{Differentialgleichungen erster Ordnung}
{
Klassifizieren Sie die folgenden Differentialgleichungen erster Ordnung:
\begin{multicols}{2}
\begin{enumerate}
\item $x^2y' = 2y + 1$. 
\item $y' = \cos(x)y$. 
\item $x^2 y' + y^2 = x y$.
\item $y'=\sin(y+1)$. 
\item $y'=(4x-y+1)^2$.
\item $y'+3y+2=\operatorname{e}^{2x}$. 
\end{enumerate}
\end{multicols}
%
\begin{abc}
\item Klassifizieren Sie diese als linear oder nicht linear.
In dem Fall einer linearen Differentialgleichung klassifizieren Sie die zusätzlich als
\begin{iii}
\item homogen oder inhomogen.
\item mit konstanten oder nicht-konstanten Koeffizienten.
\end{iii}
\item Klassifizieren Sie die Differentialgleichung als einen der folgenden Typen:
\begin{iii}
\item $y'= f(x)\cdot g(y)$ zu Lösen mit Trennung der Variablen.
\item $y'=g(y/x)$ zu Lösen mit der Substitution $u=y/x$.
\item $y'=f(ax+by+c)$ zu Lösen mit der Substitution $u=ax+by+c$.
\item $y'+p(x) \cdot y = q(x)$.
\end{iii}
\item Berechnen Sie die allgemeine Lösung aller Differentialgleichungen.
\end{abc}
}


\Loesung{
\begin{itemize}
\item[\textbf{1)}] 
\begin{iii}
\item $x^2y' = 2y + 1$. Linear, inhomogen, nicht-konstant Koeffizienten: Typ A (separabel) 
oder Typ D.
\item $y' = \cos(x)y$. Linear, homogen, nicht-konstant Koeffizienten, separabel (Typ A).
\item $x^2 y' + y^2 = x y(x)$. Nicht-linear, homogen, lösbar mit der Substitution von  Typ B.
\item $y'=\sin(y+1)$. Nicht-linear, separabel von Typ A.
\item $y'=(4x-y+1)^2$. Nicht-linear, Typ C.
\item $y'+3y+2=\operatorname{e}^{2x}$. Linear, inhomogen, Typ D mit konstanten Koeffizienten.
\end{iii}
\item[\textbf{2)}] 
\begin{iii}
\item $x^2y' = 2y + 1$. Linear, inhomogen, nicht-konstant Koeffizienten: Typ A (separabel)

\begin{align*}
\frac{\d y}{2y+1} &= \frac{\d x}{x^2},\\
\frac{1}{2} \ln(2y+1) &= -\frac{1}{x},\\
2y+1 &= \widetilde C\operatorname{e}^{-\frac{2}{x}},\\
y &= C e^{-\frac{2}{x}} -\frac{1}{2}.
\end{align*}

Diese Gleichung kann auch mit einem längeren Verfahren gelöst werden, wenn man sie als vom Typ D betrachtet: Zuerst wird die Lösung des homogenen Teils (H) bestimmt und dann die besondere Lösung (P) mit der Methode der unbestimmten Koeffizienten.
Wir beginnen mit (H):
Die Gleichung  
$$x^2y'=2y$$
ist separabel und sie hat die Lösung
$$
y_h(x) = C e^{\frac{-2}{x}}.
$$
Für den Teil (P) machen wir den Ansatz
$$
y_p(x) = C(x) e^{\frac{-2}{x}}.
$$
Dann leiten wir ab:
$$
y_p'(x) = C'(x) e^{\frac{-2}{x}} + C(x) \frac{2}{x^2} e^{\frac{-2}{x}}.
$$
Einsetzen in die Gleichung ergibt
\begin{align*}
C'(x) x^2 e^{\frac{-2}{x}} + 2C(x) e^{\frac{-2}{x}} &= 2 C(x) e^{\frac{-2}{x}} +1\\
C'(x) x^2 e^{\frac{-2}{x}} &= 1\\
C'(x) &= \frac{e^{\frac{2}{x}}}{x^2}\\
\int \d C &= \int \frac{e^{\frac{2}{x}}}{x^2}\d x.
\end{align*}
Wir integrieren das Integral auf der rechten Seite mit der Substitution $u=2/x$, woraus wir das Differential $\d u = -\frac{2}{x^2}\d x$ berechnen. Wir haben also
$$
\frac{\d x}{x^2} = -\frac{\d u}{2} 
$$
und
\begin{align*}
\int \d C &=  -\frac{1}{2} \int e^u \d u,\\
C &=  -\frac{1}{2} e^u + \widetilde C,\quad \text{ (z.B. } \widetilde C = 0)\\
C &=  -\frac{1}{2} e^\frac{2}{x}.\quad \text{ (nach Rücksubstitution)}
\end{align*}
Daher ist die partikuläre Gleichung
$$
y_p(x) = -\frac{1}{2}  e^\frac{2}{x}  e^\frac{-2}{x} =  -\frac{1}{2}
$$
und die allgemeine Lösung ist
$$
y(x) = y_h(x) + y_p(x) = C e^\frac{-2}{x} - \frac{1}{2}.
$$
\item $y' = \cos(x)y$. Linear, homogen, nicht-konstant Koeffizienten, separabel (Typ A).

Lösung:
\begin{align*}
\int \frac{1}{y} \d y &= \int \cos(x) \d x,\\
\ln(y) &= \sin(x) + \ln(C),\\
y &= C e^{\sin(x)}.
\end{align*}
\item $x^2 y' + y^2 = x y(x)$. Nicht-linear, homogen, lösbar mit der Substitution von  Typ B.

Die Gleichung kann umgeformt werden zu
$$
y'= \frac{y}{x} -\frac{y^2}{x^2}.
$$
Mit der Substitution $u=y/x$ erhalten wir 
\begin{align*}
y&=u x\\
y'&=u' x + u.
\end{align*}
Wir substituieren $y'$ und $y/x$ in die Differentialgleichung und erhalten
$$
u' x + u = u -u^2.
$$
Diese kann mit Seperation der Variablen gelöst werden.
%
\begin{align*}
u' &= -\frac{u^2}{x},\\
\int \frac{1}{u^2} \d u &= -\frac{1}{x} \d x,\\
-\frac{1}{u} &= \ln(x) + C,\\
u &= \frac{1}{\ln(x) + C},\\
y &= \frac{x}{\ln(x) + C}. \quad (\text{ nach Rücksubstitution } u=y/x)
\end{align*}
%
\item $y'=\sin(y+1)$. Nicht-linear, separabel von Typ A.

Wir lösen die Differentialgleichung mit Trennung der Variablen:
%
\begin{align*}
\int \frac{\d y}{\sin(y+1)} &= \int \d x,\\
\ln|\tan\left(\frac{y+1}{2}\right)| &= x + \ln(C),\\
\tan\left(\frac{y+1}{2}\right) &= C e^x,\\
\frac{y+1}{2} &= \arctan(Ce^x),\\
y &= 2\arctan(Ce^x)-1.
\end{align*}
%
\item $y'=(4x-y+1)^2$. Nicht-linear, Typ C.

Diese nichtlinesre Differentialgleichung kann mit Substitution gelöst werden. Wir setzen
$$
u = 4x-y+1
$$
und erhalten
$$
u'= 4-y' \quad \rightarrow \quad y'=4-u'.
$$
Substituieren $u=4x-y+1$ und $y'=4-u'$ in die Gleichung, führt zu der seperablen 
Gleichung
\begin{align*}
4-u' &= u^2,\\
u' & = 4-u^2,\\
\frac{\d u}{4-u^2} &= \d x,\\
\int \frac{\d u}{4-u^2} &= \int \d x,\\
\frac{1}{4} \ln|u+2| - \ln|u-2| &= x+ \ln(\widetilde C), \quad \text{ (mit Partialbruchzerlegung, siehe unten)}\\
\ln\frac{u+2}{u-2} &= 4x + \ln(\widetilde C^4),\\
u+2 &= Ce^{4x}(u-2),\quad (\text{mit } C=\widetilde C^4)\\
u & = -2Ce^{4x} + uCe^{4x} -2,\\
u(1-Ce^{4x}) &= -2\left(Ce^{4x}+1\right)\\
u &= -2 \frac{Ce^{4x}+1}{1-Ce^{4x}},\\
4x-y+1 &= -2 \frac{Ce^{4x}+1}{1-Ce^{4x}}, \quad \text{(Rücksubstitution)}\\
y &= 4x + \frac{3+Ce^{4x}}{1-Ce^{4x}}.
\end{align*}
%
Für die Partialbruchzerlegung im 5. Schritt oben ergibt sich
$$
\frac{-1}{u^2-4} = -\frac{1}{4} \frac{1}{x-2} + \frac{1}{4} \frac{1}{x+2}.
$$
%
\item $y'+3y+2=\operatorname{e}^{2x}$. Linear, inhomogen, Typ D mit konstanten Koeffizienten.

Diese Gleichung kann als Typ D mit der Methode der unbestimmten Koeffizienten oder als lineare inhomogene ODE mit konstanten Koeffizienten gelöst werden, wobei spezielle Ansätze für die Inhomogenitäten und das Superpositionsprinzip verwendet werden.

Beginnen wir mit der ersten Methode. Hier müssen wir das homogene Problem (H) lösen und dann eine partikuläre Lösung (P) finden.

Für das homogene Problem können wir die Variablentrennung verwenden oder im Falle konstanter Koeffizienten (nur in diesem Fall!) das charakteristische Polynom benutzen:
$$
\lambda + 3 = 0 \quad \rightarrow \quad \lambda = -3
$$
um den Lösungsteil zu bestimmen
$$
y_h(x) = C e^{-3x}.
$$
Um das Problem (P) zu lösen, machen wir den Ansatz
$$
y_p(x) = C(x) e^{-3x}
$$
und differentieren ihn
$$
y_p'(x) = C'(x) e^{-3x} - C(x) 3 e^{-3x}.
$$
Wir setzen $y_p$ und $y_p'$ in die gleichung ein und erhalten
\begin{align*}
C'(x)e^{-3x} -3C(x) e^{-3x} +3Ce^{-3x} +2 &= e^{2x}\\
C'(x)&= e^{5x}-2e^{3x}\\
C(x) &= \frac{1}{5}e^{5x} -\frac{2}{3}e^{3x}.
\end{align*}
Die partikuläre Lösung ist
$$
y_p(x) = \left( \frac{1}{5}e^{5x} -\frac{2}{3}e^{3x} \right) e^{-3x} = \frac{1}{5}e^{2x}  -\frac{2}{3}
$$
un die allgemeine Lösung ist
$$
y(x) = y_h(x) + y_p(x) = C e^{-3x} +  \frac{1}{5}e^{2x}  -\frac{2}{3}.
$$

Wie bereits erwähnt, können wir das Problem mit Hilfe spezieller Ansätze für die Inhomogenitäten und dem Superpositionsprinzip lösen.

Die rechte Seite der Gleichung ist $e^{2x} -2$, also finden wir zwei Lösungen für die beiden Terme getrennt. Zunächst für $e^{2x}$. Der Ansatz im exponentiellen Fall ist wieder exponentiell $y_{p1}=Ae^{\alpha x}$, wobei $\alpha$ der Exponent des rechten Terms ist:
$$
y_{p1} = A e^{2x}.
$$
Die Konstante $A$ wird durch Koeffizientenvergleich gefunden:
\begin{align*}
y_{p1}' +3y_{p1} &= e^{2x}, \quad \text{(Bemerkung: der Term -2 wird nicht betrachtet)}\\
2 A e^{2x} + 3 A e^{2x} &= e^{2x},\\
A &= \frac{1}{5}.
\end{align*}
Der erste Teil der partikulären Lösung ist

$$
y_{p1} = \frac{1}{5}e^{2x}.
$$
Der zweite Teil wird mit dem Polynom-Ansatz nullter Ordnung $y_{p2}=B$ berechnet, da der Term $-2$ eine Konstante ist. Setzt man die Ansatzfunktion in die Differentialgleichung ein, so erhält man
$$
3 B = -2 \quad B=-\frac{2}{3}.
$$
Wir haben also die partikuläre Lösung durch Superposition
$$
y_p(x) = y_{p1}(x) + y_{p2}(x) =  \frac{1}{5}e^{2x} - \frac{2}{3}
$$
un die allgemeine Lösung
$$
y(x) = y_h(x) + y_p(x) = Ce^{-3x} +  \frac{1}{5}e^{2x} - \frac{2}{3}.
$$
\end{iii}

\end{itemize}
}



