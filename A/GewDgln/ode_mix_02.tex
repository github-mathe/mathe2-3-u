\Aufgabe[e]{First order differential equations}
{
Classify the following ordinary differential equations (ODEs) of the first order:
\begin{multicols}{2}
\begin{enumerate}
\item $x^2y' = 2y + 1$. 
\item $y' = \cos(x)y$. 
\item $x^2 y' + y^2 = x y$.
\item $y'=\sin(y+1)$. 
\item $y'=(4x-y+1)^2$.
\item $y'+3y+2=\operatorname{e}^{2x}$. 
\end{enumerate}
\end{multicols}
%
\begin{abc}
\item Classify them as linear or nonlinear.
In case of linear ODEs classify them as 
\begin{iii}
\item homogeneous or inhomogeneous.
\item ODE with constant or non-constant coefficients.
\end{iii}
\item Classify them according to the type as in the lecture notes:
\begin{iii}
\item $y'= f(x)\cdot g(y)$ solved by separation of variables.
\item $y'=g(y/x)$ with substitution $u=y/x$ becomes of type A.
\item $y'=f(ax+by+c)$ with substitution $u=ax+by+c$ is of type A.
\item $y'+p(x) \cdot y = q(x)$.
\end{iii}
\item Compute the general solution of all equations.
\end{abc}
}


\Loesung{
\begin{itemize}
\item[\textbf{1)}] 
\begin{iii}
\item $x^2y' = 2y + 1$. Linear, inhomogeneous, non-constant coefficients: Type A (separable) or type D.
\item $y' = \cos(x)y$. Linear, homogeneous, non-constant coefficients, separable (type A).
\item $x^2 y' + y^2 = x y(x)$. Nonlinear, homogeneous, solvable with substitution of Type B.
\item $y'=\sin(y+1)$. Nonlinear, separable of Type A.
\item $y'=(4x-y+1)^2$. Nonlinear, type C.
\item $y'+3y+2=\operatorname{e}^{2x}$. Linear, inhomogeneous, Type D with constant coefficients.
\end{iii}
\item[\textbf{2)}] Compute the general solution of all equations.
\begin{iii}
\item $x^2y' = 2y + 1$. Linear, inhomogeneous, non-constant coefficients: type A.
This is of separable type:
\begin{align*}
\frac{\d y}{2y+1} &= \frac{\d x}{x^2},\\
\frac{1}{2} \ln(2y+1) &= -\frac{1}{x},\\
2y+1 &= \widetilde C\operatorname{e}^{-\frac{2}{x}},\\
y &= C e^{-\frac{2}{x}} -\frac{1}{2}.
\end{align*}

This equation can also be solved with a longer procedure, considering it of Type D: First determining the solution of the homogeneous part (H) and then the particular solution (P) using the method of the undetermined coeffcients.
Let's start with (H):

The equation 
$$x^2y'=2y$$
is separable and it has the solution
$$
y_h(x) = C e^{\frac{-2}{x}}.
$$
For the part (P) we make the ansatz
$$
y_p(x) = C(x) e^{\frac{-2}{x}}.
$$
Then we differentiate:
$$
y_p'(x) = C'(x) e^{\frac{-2}{x}} + C(x) \frac{2}{x^2} e^{\frac{-2}{x}}.
$$
Substituting into the equation yields
\begin{align*}
C'(x) x^2 e^{\frac{-2}{x}} + 2C(x) e^{\frac{-2}{x}} &= 2 C(x) e^{\frac{-2}{x}} +1\\
C'(x) x^2 e^{\frac{-2}{x}} &= 1\\
C'(x) &= \frac{e^{\frac{2}{x}}}{x^2}\\
\int \d C &= \int \frac{e^{\frac{2}{x}}}{x^2}\d x.
\end{align*}
We integrate the integral on the right with the substitution $u=2/x$ from which we compute the differential $\d u = -\frac{2}{x^2}\d x$. We have thus
$$
\frac{\d x}{x^2} = -\frac{\d u}{2} 
$$
and
\begin{align*}
\int \d C &=  -\frac{1}{2} \int e^u \d u,\\
C &=  -\frac{1}{2} e^u + \widetilde C,\quad \text{ (e.g. } \widetilde C = 0)\\
C &=  -\frac{1}{2} e^\frac{2}{x}.\quad \text{ (after back substitution)}
\end{align*}
The particular solution is therefore
$$
y_p(x) = -\frac{1}{2}  e^\frac{2}{x}  e^\frac{-2}{x} =  -\frac{1}{2}
$$
and the general solution is
$$
y(x) = y_h(x) + y_p(x) = C e^\frac{-2}{x} - \frac{1}{2}.
$$
\item $y' = \cos(x)y$. Linear, homogeneous, non-constant coefficients, separable (type A).

Solution:
\begin{align*}
\int \frac{1}{y} \d y &= \int \cos(x) \d x,\\
\ln(y) &= \sin(x) + \ln(C),\\
y &= C e^{\sin(x)}.
\end{align*}
\item $x^2 y' + y^2 = x y(x)$. Nonlinear, homogeneous, solvable with substitution of Type B.

The equation is reformulated as
$$
y'= \frac{y}{x} -\frac{y^2}{x^2}.
$$
With the substitution $u=y/x$ we have 
\begin{align*}
y&=u x\\
y'&=u' x + u.
\end{align*}
We substite $y'$ and $y/x$ in the ODE to get
$$
u' x + u = u -u^2.
$$
that can be solved by separation of variables
%
\begin{align*}
u' &= -\frac{u^2}{x},\\
\int \frac{1}{u^2} \d u &= -\frac{1}{x} \d x,\\
-\frac{1}{u} &= \ln(x) + C,\\
u &= \frac{1}{\ln(x) + C},\\
y &= \frac{x}{\ln(x) + C}. \quad (\text{after back substitution } u=y/x)
\end{align*}
%
\item $y'=\sin(y+1)$. Nonlinear, separable of Type A.

We solve it by separation of variables:
%
\begin{align*}
\int \frac{\d y}{\sin(y+1)} &= \int \d x,\\
\ln|\tan\left(\frac{y+1}{2}\right)| &= x + \ln(C),\\
\tan\left(\frac{y+1}{2}\right) &= C e^x,\\
\frac{y+1}{2} &= \arctan(Ce^x),\\
y &= 2\arctan(Ce^x)-1.
\end{align*}
%
\item $y'=(4x-y+1)^2$. Nonlinear, type C.

This nonlinear ODE is solved by substitution. We set
$$
u = 4x-y+1
$$
to get
$$
u'= 4-y' \quad \rightarrow \quad y'=4-u'.
$$
Substituting $u=4x-y+1$ and $y'=4-u'$ into the equation yields a separable equation
\begin{align*}
4-u' &= u^2,\\
u' & = 4-u^2,\\
\frac{\d u}{4-u^2} &= \d x,\\
\int \frac{\d u}{4-u^2} &= \int \d x,\\
\frac{1}{4} \ln|u+2| - \ln|u-2| &= x+ \ln(\widetilde C), \quad \text{ (with partial fraction decomposition, see below)}\\
\ln\frac{u+2}{u-2} &= 4x + \ln(\widetilde C^4),\\
u+2 &= Ce^{4x}(u-2),\quad (\text{with } C=\widetilde C^4)\\
u & = -2Ce^{4x} + uCe^{4x} -2,\\
u(1-Ce^{4x}) &= -2\left(Ce^{4x}+1\right)\\
u &= -2 \frac{Ce^{4x}+1}{1-Ce^{4x}},\\
4x-y+1 &= -2 \frac{Ce^{4x}+1}{1-Ce^{4x}}, \quad \text{(back substitution)}\\
y &= 4x + \frac{3+Ce^{4x}}{1-Ce^{4x}}.
\end{align*}
%
For the partial fraction decomposition in the 5th step above we have
$$
\frac{-1}{u^2-4} = -\frac{1}{4} \frac{1}{x-2} + \frac{1}{4} \frac{1}{x+2}.
$$
%
\item $y'+3y+2=\operatorname{e}^{2x}$. Linear, inhomogeneous, Type D with constant coefficients.

This equation can be solved as Type D using the method of undetermined coefficients or as a linear inhomogeneous ODE with constant coefficients, using special guesses for the inhomogeneities and the superposition principle.

Let's start with the first method. Her we have to solve the homogeneous problem (H) and then find a particular solution (P).

For the homogeneous problem we can use separation of variables or in case of constant coefficients (only in this case!) use the characteristic polynomial:
$$
\lambda + 3 = 0 \quad \rightarrow \quad \lambda = -3
$$
to determine the solution part
$$
y_h(x) = C e^{-3x}.
$$
To solve problem (P) we make the ansatz
$$
y_p(x) = C(x) e^{-3x}
$$
and differentiate it
$$
y_p'(x) = C'(x) e^{-3x} - C(x) 3 e^{-3x}.
$$
We insert $y_p$ and $y_p'$ into the ODE to get
\begin{align*}
C'(x)e^{-3x} -3C(x) e^{-3x} +3Ce^{-3x} +2 &= e^{2x}\\
C'(x)&= e^{5x}-2e^{3x}\\
C(x) &= \frac{1}{5}e^{5x} -\frac{2}{3}e^{3x}.
\end{align*}
The particular solution is
$$
y_p(x) = \left( \frac{1}{5}e^{5x} -\frac{2}{3}e^{3x} \right) e^{-3x} = \frac{1}{5}e^{2x}  -\frac{2}{3}
$$
and the general solution is
$$
y(x) = y_h(x) + y_p(x) = C e^{-3x} +  \frac{1}{5}e^{2x}  -\frac{2}{3}.
$$

As mentioned above, we can solve the problem using special guesses for the inhomogeneities and the superposition principle.

The right-hand side of the equation is $e^{2x} -2$, so we find two solutions for the two terms separately. First for $e^{2x}$. The ansatz in the exponential case is again exponential $y_{p1}=Ae^{\alpha x}$, where $\alpha$ is the exponent of the righ-hand term:
$$
y_{p1} = A e^{2x}.
$$
The constant $A$ is found by coefficients comparison:
\begin{align*}
y_{p1}' +3y_{p1} &= e^{2x}, \quad \text{(note: the term -2 is not considered)}\\
2 A e^{2x} + 3 A e^{2x} &= e^{2x},\\
A &= \frac{1}{5}.
\end{align*}
The first part of the particular solutions is
$$
y_{p1} = \frac{1}{5}e^{2x}.
$$
The second part is computed with the polynomial ansatz of the zero-th order $y_{p2}=B$ because the term $-2$ is a constant. Inserting the ansatz function into the ODE yields
$$
3 B = -2 \quad B=-\frac{2}{3}.
$$
We have thus the particular solution by superposition
$$
y_p(x) = y_{p1}(x) + y_{p2}(x) =  \frac{1}{5}e^{2x} - \frac{2}{3}
$$
and again the general solution
$$
y(x) = y_h(x) + y_p(x) = Ce^{-3x} +  \frac{1}{5}e^{2x} - \frac{2}{3}.
$$
\end{iii}

\end{itemize}
}



