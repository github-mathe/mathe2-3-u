\Aufgabe[e]{Zum L\"osungsbegriff von Dgl.}
{
\begin{abc}
\item Gegeben seien die beiden Differentialgleichungen
\begin{iii}
\item $y'(x) = y(x) \cos(x) +\sin(x)  \cos(x)$
\item $y''(x) + y'(x) = 0$
\end{iii}
Welcher der folgenden Funktionen ist L\"osung einer der Differentialgleichungen?
\begin{align*}
y_1(x) &= \cos(x), &\quad y_2(x) &=8,\\
y_3(x) &= e^x, &\quad y_4(x) &= \sin(x)-1,\\
y_5(x) &= e^{-\sin(x)}, &\quad y_6(x) &= y_4+y_5 = \sin(x)-1+e^{-\sin(x)}.
\end{align*}
\item F\"ur welche Werte der Konstanten $A$, $\omega$ und $\varphi_0$ ist
$$
u(t) = A \cdot \cos(\omega t+\varphi_0)
$$
L\"osung der Schwingungs-Differentialgleichung
$$
u''(t)+25u(t)=0.
$$
\end{abc}
}



\Loesung{
%\begin{abc}
\begin{enumerate}[label=\textbf{\alph*)}]
\item
\begin{itemize}
\item $y_1(x)=\cos(x), y'_1(x)=-\sin(x), y''_1(x)=-\cos(x), y'''_1(x)=\sin(x)$.\\
Einsetzen in die Dgl. i) ergibt:
$$
-\sin(x) \neq -\cos^2(x)+\sin(x)\cos(x),
$$
d.h. $y_1$ ist keine L\"osung von Dgl. i).\\
Einsetzen in Dgl. ii) ergibt:
$$
 \sin(x) -\sin(x) = 0 \, \text{f\"ur alle}\,  x \in \mathbb{R},
$$
d.h. $y_1$ ist auf ganz $\mathbb{R}$ L\"osung von Dgl. ii).

\item $y_2(x) = 8$ ist keine L\"osung von i) aber L\"osung von ii).
\item $y_3(x) = e^x$ ist werder L\"osung von i) noch von ii).
\item $y_4(x) =\sin(x)-1$ ist L\"osung von i) und von ii).
\item $y_5(x) =e^{-\sin(x)}$ ist werder L\"osung von i) noch von ii).
\item $y_6(x) = \sin(x)-1+e^{-\sin(x)}$ ist L\"osung von i) aber nicht von ii).
\end{itemize}
\item 
\begin{align*}
u(t) &= A \cos(\omega t +\varphi_0),\\
u'(t) &= -A \omega \sin(\omega t +\varphi_0), \\
u''(t) &= -A \omega^2 \cos(\omega t +\varphi_0).
\end{align*}
Einsetzen in die Dgl. ergibt:
$$
A \omega^2 \cos(\omega t +\varphi_0) = 25 A \cos(\omega t +\varphi_0).
$$
$$
\Rightarrow \omega^2 = 25, \, \text{d.h.} \, \omega = \pm 5 \, \text{und} A,\varphi_0 \in \mathbb{R}.
$$
\end{enumerate}
}


\ErgebnisC{gewdgl_Begriff_DGL}{
\begin{abc}
\item $y_4(x), y_6(x)$ L\"osung von Dgl. i), $y_1(x), y_2(x), y_4(x)$ L\"osung von Dgl. ii)
\item $\omega = \pm 5$, $A, \varphi \in \mathbb{R}$
\end{abc}
}

