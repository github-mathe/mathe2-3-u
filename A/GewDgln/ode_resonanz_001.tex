\Aufgabe[e]{Lineare Differentialgleichungen mit Resonanz}
{
Bestimmen Sie die allgemeine Lösung der folgenden linearen Differentialgleichungen
\begin{abc}
\item $y'' - y = 1$, 
\item $y''' + y'' = 1$, 
\item $y''+y'-2y = e^x$, 
\item $y''+y = \cos(x)$.
\end{abc}
}

\Loesung{
\begin{abc}
\item 
Wir lösen zunächst die homogene Gleichung
$$
y'' -y = 0.
$$
Das charakteristische Polynom ist
$$
\lambda^2 -1 = 0
$$
mit den Nullstellen $\lambda_1 = 1$ und $\lambda_2 = -1$.
Damit ist die homogene Lösung
$$
y_h = c_1 \operatorname{e}^x + c_2 \operatorname{e}^{-x}
$$
Für die partikuläre Lösung wählen wir den Ansatz
$$
y_p = A
$$
Damit ist 
$$
y_p'' = 0
$$
Eingesetzt in die Differentialgleichung erhalten wir
$$
0 - A = 1
$$
Also ist $y_p = -1$.
Die allgemeine Lösung ist dann 
\begin{align*}
y &= y_h + y_p\\
  &= c_1 \operatorname{e}^x + c_2 \operatorname{e}^{-x} -1
\end{align*}
\item
Wir lösen zunächst die homogene Gleichung
$$
y''' + y'' = 1.
$$
Das charakteristische Polynom ist
$$
\lambda^3 + \lambda^2 = \lambda^2(\lambda +1)= 0
$$
mit der doppelten Nullstelle $\lambda = 0$ und der einfachen Nullstelle $\lambda = -1$.
Damit ist die homogene Lösung
$$
y_h = (c_1+c_2x)+ c_3\operatorname{e}^{-x}
$$
Da $\lambda = 0$ eine doppelte Nullstelle ist und die rechte Seite eine Konstante, haben 
wir hier einen Fall mit Resonanz und wählen für die partikuläre Lösung den Ansatz
$$
y_p = Ax^2
$$
Wir bestimmen die Ableitungen
$$
y_p' = 2Ax \, , \, y_p'' = 2A \, , \, y_p''' = 0
$$
und setzen sie in die Differentialgleichung ein.
$$
0 + 2A = 1
$$
Damit ist $A = \frac{1}{2}$ und die partikuläre Lösung
$$
y_p = \frac{x^2}{2}.
$$
Die allgemeine Lösung ist
\begin{align*}
y &= y_h + y_p\\
  &= (c_1+c_2x)+ c_3\operatorname{e}^{-x} +\frac{x^2}{2}.
\end{align*}
\item
Wir lösen zunächst homogene Gleichung
$$
y''+y'-2y = 0.
$$
Das charakteristische Polynom ist
$$
\lambda^2 + \lambda -2 = 0
$$
mit den Nullstellen $\lambda_1 = -2$ und $\lambda_2 = 1$.
Damit ist die homogene Lösung
$$
y_h = c_1 \operatorname{e}^{-2x} + c_2 \operatorname{e}^x
$$
Wir haben hier wieder einen Fall mit Resonanz. Daher wählen wir für die partikuläre 
Lösung den Ansatz
$$
y_p = Ax \operatorname{e}^x
$$
Wir bestimmen die Ableitungen
\begin{align*}
y_p' &= A \operatorname{e}^x + Ax\operatorname{e}^x\\
y_p'' &= 2A \operatorname{e}^x + Ax\operatorname{e}^x
\end{align*}
In die Gleichung eingesetzt erhalten wir 
\begin{align*}
2A \operatorname{e}^x + Ax\operatorname{e}^x + A \operatorname{e}^x + Ax\operatorname{e}^x
 -2Ax \operatorname{e}^x  &= \operatorname{e}^x \\
 3A \operatorname{e}^x &= \operatorname{e}^x
\end{align*}
Wir erhalten dann $A = \frac{1}{3}$.
Die partikuläre Lösung ist dann also
$$
y_p = \frac{x \operatorname{e}^x}{3}
$$
Die allgemeine Lösung ist dann
\begin{align*}
y &= y_h + y_p \\
  &= c_1 \operatorname{e}^{-2x} + c_2 \operatorname{e}^x + \frac{x \operatorname{e}^x}{3}
\end{align*}
\item
Wir lösen zunächst die homogene Gleichung
$$
y'' + y = 0
$$
mit dem charakteristischen Polynom
$$
\lambda^2 + 1 = 0.
$$
Die Nullstellen sind $\lambda = \pm \operatorname{i}$
Damit ist die homogene Lösung
$$
y_h = c_1 \cos(x) + c_2\sin(x) 
$$
Als Ansatz für die partikuläre Gleichung wählen wir 
$$
y_p = x(A \cos(x) + B \sin(x))
$$
Wir erhalten die Ableitungen
\begin{align*}
y_p' &= (A+Bx) \cos(x) + (B-Ax)\sin(x)\\
y_p'' &= (2B-Ax)\cos(x) - (2A+Bx)\sin(x)
\end{align*}
Wir setzen die Ableitungen in die Gleichung ein 
$$
(2B-Ax)\cos(x) - (2A+Bx)\sin(x) + x(A \cos(x) + B \sin(x)) = \cos(x)
$$
Durch Koeffizientenvergleich erhalten wir $A = 0$ und $B = \frac{1}{2}$.
Die partikuläre Lösung ist damit 
$$
y_p = x \frac{1}{2} \sin(x)
$$
Die allgemeine Lösung ist
\begin{align*}
y &= y_h + y_p \\
  &=  c_1 \cos(x) + c_2\sin(x) + x \frac{1}{2} \sin(x)
\end{align*}

\end{abc}
}
