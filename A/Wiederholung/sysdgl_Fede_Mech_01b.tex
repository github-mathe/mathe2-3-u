\Aufgabe[e]{Gekoppelte Schwingung}
{
Zu untersuchen ist das unten skizzierte mechanische System aus zwei durch eine Feder
  (Federkonstante $D$) verbundene Massen $m_1$ und $m_2$. Die beiden Massen sind im Schwerefeld der
  Erde ($\vec g$)  an F\"aden der L\"ange $l$ so aufgeh\"angt, dass die Feder entspannt ist, wenn
  beide F\"aden senkrecht sind ($\varphi_1=\varphi_2=0$). F\"ur kleine Auslenkungen der Massen um
  die Winkel $\varphi_1$ bzw. $\varphi_2$ ergibt der Impulserhaltungssatz:
\begin{align*}
m_1l\ddot \varphi_1=& - m_1\varphi_1 g + Dl(\varphi_2-\varphi_1)\\
m_2l\ddot \varphi_2=& -m_2\varphi_2 g + Dl(\varphi_1-\varphi_2).
\end{align*}
\begin{center}
\begin{pspicture}(0,0)(5,4)
\psline[fillstyle=vlines](0,3.5)(0,4)(4,4)(4,3.5)(0,3.5)
\psline(.5,3.5)(1.5,.67)
\psline(3.5,3.5)(4,.54)
\psline[linecolor=gray](.5,3.5)(.5,2)
\psline[linecolor=gray](3.5,3.5)(3.5,2)
\psline
  (1.50000,  0.67000)
  (1.69231,  0.66000)
  (1.88462,  0.45000)
  (2.07692,   0.84000)
  (2.26923,   0.43000)
  (2.46154,   0.82000)
  (2.65385,   0.41000)
  (2.84615,   0.80000)
  (3.03846,   0.39000)
  (3.23077,   0.78000)
  (3.42308,   0.37000)
  (3.61538,   0.56000)
  (3.80769,   0.55000)
  (4.00000,   0.54000)
          
\pscircle[fillstyle=solid](1.5,.67){.3}
\pscircle[fillstyle=solid](4, .54){.4}
\psarc(.5,3.5){1}{270}{290}
\psarc(3.5,3.5){1}{270}{280}
\put(.8,2.9){$\varphi_1$}
\put(3.7,2.8){$\varphi_2$}
\put(1.3,.6){$m_1$}
\put(3.8,.5){$m_2$}
\put(1.3,1.8){$l$}
\put(2.8,.9){$D$}

\end{pspicture}
\end{center}
%\end{figure}
Durch den \"Ubergang von den Auslenkungen $\varphi_1,\,\varphi_2$ und deren Geschwindigkeiten $\dot\varphi_1,\, \dot\varphi_2$ zu $\vec y =
(\varphi_1,\, \dot\varphi_1,\, \varphi_2,\, \dot\varphi_2)^\top$ 
l\"asst sich das obige (Differential-)Gleichungssystem schreiben als 
$$
\dot{ \vec y}  = \vec A \vec y \text{ mit }\vec A 
= \begin{pmatrix} 
0&1&0&0\\
-\left(\frac gl+\frac D{m_1}\right)&0&\frac D{m_1}&0\\
0&0&0&1\\
\frac{D}{m_2}&0&-\left(\frac{g}l+\frac D{m_2}\right)&0\end{pmatrix}.
$$
\begin{abc}
\item Bestimmen Sie die Eigenfrequenzen des Systems. Die Eigenfrequenzen sind die Imagin\"arteile der komplexen Eigenwerte der Systemmatrix $\vec A$. 
\item Bestimmen Sie auch die Eigenvektoren zu den ermittelten Eigenwerten. 
\item Nehmen Sie nun an, dass der Realteil eines Eigenvektors die jeweilige \textit{Eigenschwingung} des Systems beschreibt: Die erste und dritte Komponente (Beachte $\vec y=(\varphi_1,\,\dot \varphi_1,\, \varphi_2,\, \dot \varphi_2)^\top$) geben jeweils den Maximalausschlag des jeweiligen Pendels an. \\
Beschreiben Sie anhand der Eigenvektoren die beiden Eigenschwingungen. 
\end{abc}

}
\Loesung{
\begin{abc}

\item Mit $\alpha=\frac gl$ und $\delta_j=\frac D{m_j}$ f\"ur $j=1,2$ haben wir das
charakteristische Polynom:
\begin{align*}
p(\lambda)=&\det\begin{pmatrix}
-\lambda         &         1  &            0  &       0\\
-\alpha-\delta_1 & -\lambda   &   \delta_1   &       0\\
          0      &         0  &  -\lambda     &       1\\
\delta_2        &         0  & -\alpha-\delta_2& -\lambda
\end{pmatrix}\\
=& -\lambda\det\begin{pmatrix}
 -\lambda   &   \delta_1   &       0\\
         0  &  -\lambda     &       1\\
        0  & -\alpha-\delta_2& -\lambda
\end{pmatrix} - 1\cdot \det \begin{pmatrix}
-\alpha-\delta_1 &   \delta_1   &       0\\
          0      &  -\lambda     &       1\\
\delta_2        & -\alpha-\delta_2& -\lambda
\end{pmatrix}\\
=& \lambda^2\det\begin{pmatrix}
  -\lambda     &       1\\
-\alpha-\delta_2& -\lambda
\end{pmatrix} +\\
&+\lambda  \det \begin{pmatrix}
-\alpha-\delta_1&       0\\
\delta_2        & -\lambda
\end{pmatrix}+1\cdot \det\begin{pmatrix}
-\alpha-\delta_1 &   \delta_1     \\
\delta_2        & -\alpha-\delta_2
\end{pmatrix}\\
=& \lambda^2(\lambda^2 +\alpha+\delta_2) + \lambda^2(\alpha+\delta_1)
+(\alpha+\delta_1)(\alpha+\delta_2)-\delta_1\delta_2\\
=& (\lambda^2)^2 + 2 \frac{\alpha+\delta_1 + \alpha+\delta_2}{2} \lambda^2 + \alpha^2 + \alpha(\delta_1+\delta_2).
\end{align*}
F\"ur seine Nullstellen (mit $p(\lambda)=0$) gilt:
\begin{align*}
&&\lambda^2=&-\frac{2\alpha+\delta_1+\delta_2}2 \pm \sqrt{\frac{(2\alpha+\delta_1+\delta_2)^2}4
- \alpha^2-\alpha(\delta_1+\delta_2)}\\
&&=& \frac{-2\alpha -\delta_1-\delta_2 \pm \sqrt{\delta_1^2+\delta_2^2+2\delta_1\delta_2
}}2=\frac{-2\alpha-(\delta_1+\delta_2)\pm (\delta_1+\delta_2)}2\\
\Rightarrow&&\lambda_{1,2}=&\pm \imag\sqrt\alpha,\, \lambda_{3,4}=\pm \imag \sqrt{\alpha+\delta_1+\delta_2}.
\end{align*}
Die Eigenfrequenzen des Systems sind damit 
$$\omega_1=\sqrt\alpha=\sqrt{\frac gl} \text{  und
} \omega_2=\sqrt{\alpha+\delta_1+\delta_2}=\sqrt{\frac gl + \frac D{m_1}+\frac D{m_2}}.$$
\item Ein Eigenvektor $\vec v_1$ zu $\lambda_1$ ist gegeben durch das Gleichungssystem 
$$(\vec A- \lambda_1\vec E)\vec v_1=0,$$
 also durch 
 $$\begin{array}{rrrr|r|l}
-\imag\sqrt \alpha         &         1  &            0  &       0  &   0  & \text{ 2. Zeile + $\imag\sqrt \alpha\cdot$ 1. Zeile            }\\
-\alpha-\delta_1 & -\imag\sqrt \alpha   &   \delta_1   &       0   &   0 & \text{1. Zeile             }\\
          0      &         0  &  -\imag\sqrt \alpha     &       1  &   0 & \text{4. Zeile +  $\imag\sqrt \alpha\cdot$ 3. Zeile }\\
\delta_2        &         0  & -\alpha-\delta_2& -\imag\sqrt \alpha &   0 & \text{3. Zeile     }\\\hline

-\delta_1   & 0       & \delta_1    & 0         & 0& \text{ +$\delta_1/\delta_2\cdot$ 3. Zeile}\\
-\imag\sqrt\alpha& 1  &        0    & 0         & 0& \text{              }\\
\delta_2    & 0       &-\delta_2    & 0         & 0& \text{              }\\
        0   & 0       &-\imag\sqrt\alpha& 1     & 0& \text{              }\\\hline

0           & 0       & 0           & 0         & 0& \text{              }\\
-\imag\sqrt\alpha& 1  &        0    & 0         & 0& \text{              }\\
\delta_2    & 0       &-\delta_2    & 0         & 0& \text{              }\\
        0   & 0       &-\imag\sqrt\alpha& 1     & 0& \text{              }\\
\end{array}$$

Daraus ergibt sich $\vec v_1 = (1,\, \imag\omega_1,\, 1,\, \imag\omega_1)^\top$. \\
Wegen $\lambda_2=\overline{\lambda_1}$ und weil die Matrix $\vec A$ reell ist, ist 
$$\vec v_2=\overline{\vec v_1}= \begin{pmatrix}1\\-\imag\omega_1\\1\\-\imag \omega_1\end{pmatrix}$$ ein Eigenvektor zu $\lambda_2$. \\
%Realteil und Imagin\"arteil von $\EH{i\omega_1 t}\vec v_1$ ergeben die ersten beiden
%Fundamentall\"osungen des Systems:
%$$\vec y_1(t) = \begin{pmatrix}\cos(\omega_1
%t)\\-\omega_1\sin(\omega_1t)\\\cos(\omega_1t)\\-\omega_1\sin(\omega_1t)\end{pmatrix}
%\text{ und } \vec
%y_2(t)=\begin{pmatrix} \sin(\omega_1t)\\\omega_1\cos(\omega_1t)\\\sin(\omega_1t)\\\omega_1\cos(\omega_1t)\end{pmatrix}.$$
F\"ur die zweite Eigenfrequenz $\omega_2$ ergibt sich 
$$\begin{array}{rrrr|r|l}
-\imag\omega_2  &         1  &            0  &       0  &   0  & \text{ 2. Zeile + $\imag\omega_2\cdot$ 1. Zeile            }\\
-\alpha-\delta_1 & -\imag\omega_2   &   \delta_1   &       0   &   0 & \text{1. Zeile             }\\
          0      &         0  &  -\imag\omega_2     &       1  &   0 & \text{4. Zeile +  $\imag\omega_2\cdot$ 3. Zeile }\\
\delta_2        &         0  & -\alpha-\delta_2& -\imag\omega_2 &   0 & \text{3. Zeile     }\\\hline

\omega_2^2-\alpha-\delta_1 & 0   &   \delta_1   &       0   &   0 & \text{                     }\\
-\imag\omega_2  &         1  &            0  &       0  &   0  & \text{                 }\\
\delta_2        &         0  & -\alpha-\delta_2+\omega_2^2&  0 &   0 & \text{             }\\
          0      &         0  &  -\imag\omega_2     &       1  &   0 & \text{                  }\\\hline

\delta_2 & 0   &   \delta_1   &       0   &   0 & \text{                     }\\
-\imag\omega_2  &         1  &            0  &       0  &   0  & \text{                 }\\
\delta_2        &         0  & \delta_1 &  0 &   0 & \text{             }\\
          0      &         0  &  -\imag\omega_2     &       1  &   0 & \text{                  }\\\hline


\end{array}$$
Daraus erh\"alt man den Eigenvektor $v_3= (1,\, \imag\omega_2,\, -\delta_2/\delta_1,\, -\imag\omega_2\delta_2/\delta_1)^\top$
zu $\lambda_3=\omega_2$ und den Eigenvektor 
$$\vec v_4=\overline{\vec v_3}=\begin{pmatrix}1\\-\imag \omega_2\\-\delta_2/\delta_1\\\imag\omega_2\delta_2/\delta_1\end{pmatrix}$$
zu $\lambda_4=\overline{\lambda_3}$. 

%und die beiden fehlenden Fundamentall\"osungen des Systems:
%$$\vec y_3(t) = \begin{pmatrix}
%\cos(\omega_2t)\\-\omega_2\sin(\omega_2t)\\-\frac{\delta_2}{\delta_1}\cos(\omega_2t)\\\frac{\delta_2}{\delta_1}\omega_2\sin(\omega_2t)\end{pmatrix}
%\text{ und } \vec
%y_4(t)=\begin{pmatrix} \sin(\omega_2t)\\\omega_2\cos(\omega_2t)\\-\frac{\delta_2}{\delta_1}\sin(\omega_2t)\\-\frac{\delta_2}{\delta_1}\omega_2\cos(\omega_2t)\end{pmatrix}.$$
\item Wir gehen wieder zu den urspr\"unglichen Unbekannten 
$\vec\varphi=(\varphi_1,\,\varphi_2)^\top$ \"uber und erhalten die folgenden Eigenschwingungen:
\begin{iii}
\item $\vec \varphi_1(t)=(\alpha\cos(\omega_1t)
+ \beta\sin(\omega_1t))\begin{pmatrix}1\\1\end{pmatrix}$. Dies beschreibt eine
gleichphasige Schwingung beider Pendel. \\
Da beide Amplituden (Komponenten des Vektors) positiv sind, schlagen beide Pendel gleichzeitig in dieselbe Richtung aus. \\
Die Feder bleibt immer entspannt. \\
\item
$\vec \varphi_2(t)=(\gamma\cos(\omega_2t)+\delta\sin(\omega_2t))\begin{pmatrix}1\\-\frac{\delta_2}{\delta_1}\end{pmatrix}.$
Dies beschreibt eine gegenphasige Schwingung beider Pendel. \\
Beide Amplituden haben ein unterschiedliches Vorzeichen, also schlagen die Pendel jeweils in unterschiedliche Richtungen aus. \\
Das Verh\"altnis beider Amplituden ist so groß, dass der Schwerpunkt beider Massen stets im selben Punkt liegt.  
\end{iii}
\end{abc}


}
\ErgebnisC{AufgsysdglFedeMech001}
{
Eigenwerte:
$\lambda_{1,2}=\pm \imag\sqrt\alpha,\, \lambda_{3,4}=\pm \imag \sqrt{\alpha+\delta_1+\delta_2}$
mit $\alpha=g/l$, $\delta_j=D/m_j$\\
Eigenfrequenzen: $\omega_1=\sqrt{\alpha}$, $\omega_2=\sqrt{\alpha+\delta_1+\delta_2}$\\
}
