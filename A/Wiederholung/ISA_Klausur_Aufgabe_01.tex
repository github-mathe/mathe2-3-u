\Aufgabe[e]{Normalschwingungen}{
Man betrachte das System von zwei gleichen Massenpunkten $m_1=m_2=m$, die durch eine elastische Feder mit einer Federkonstante $c$ gekoppelt sind, siehe Bild.

\begin{center}
\resizebox{0.25\textwidth}{!}{\input{../A/Wiederholung/two_masses.pdf_t}}
\end{center}
%

Von besonderer Bedeutung sind die sogenannten Normalschwingungen, bei denen beide Massen
harmonisch mit der gleichen Winkelfrequenz $\omega$ entlang der Systemachse
(x-Achse) schwingen. Die Koordinaten $x_1$ und $x_2$ beschreiben die momentane Lage der Massen. 
Sie sind periodische Funktionen der Zeit und erfüllen das folgende Differentialgleichungssystem

\renewcommand\theequation{\arabic{equation}}
\begin{align}
\label{eqsys}
\begin{split}
m \ddot x_1 &= -c (x_1-x_2),\\
m \ddot x_2 &= -c (x_2-x_1).
\end{split}
\end{align}

Die beiden Lösungskomponenten haben die Form $x_k = A_k \operatorname{e}^{i\omega\,t}$, wobei $A_k \in \mathbb C$ und $i$ die imaginäre Einheit ist. Es gilt dann
$$
\ddot {\vec x} = -\omega^2 \vec x
$$
%
und das Gleichungssystem \eqref{eqsys} kann so geschrieben werden
\renewcommand*{\arraystretch}{1.5}
\begin{align}
\label{ew}
\underbrace{
\begin{pmatrix}
-\frac{c}{m} & ~~\frac{c}{m}\\
\frac{c}{m} & -\frac{c}{m}
\end{pmatrix}
}_{\vec A}
\begin{pmatrix}
x_1\\x_2
\end{pmatrix}
= -\omega^2
\begin{pmatrix}
x_1\\x_2
\end{pmatrix}.
\end{align}
Dieses System definiert die Schwingungen der beiden Massen. Die Schwingungsdauer ist im Fall $\omega > 0$
$$
T = \frac{2\pi}{\omega}\quad [s].
$$
Ist $\omega = 0$, gibt es keine Schwingung und keine Schwingungsdauer.
%
Durch die Substitution $\lambda = -\omega^2$ wird \eqref{ew} zu einem Eigenwertproblem transformiert
\begin{equation}
\label{lambda}
\vec A \vec x = \lambda \vec x.
\end{equation}

Man betrachtet dieses Eigenwertproblem für $c=1\,\, [N/m]$ und $m=250\,\, [g]$.
\begin{itemize}
\item Man schreibe die Matrix des Systems \eqref{lambda} im internationalen Einheitensystem.
\item Man bestimme die Eigenwerte des Systems, $\lambda_1\,\, [s^{-2}]$ und $\lambda_2\,\, [s^{-2}]$.
\item Man bestimme die Eigenvektoren des Systems.
\item Man bestimme die Schwingungsdauer der beiden Eigenschwingungen $T_1 = \frac{2\pi}{\omega_1}\,\, [s]$ und $T_2 = \frac{2\pi}{\omega_2}\,\,[s]$.
\end{itemize}
}

\Loesung{
\begin{itemize}
\item
Systemmatrix:
\begin{align*}
A = \begin{pmatrix}
-4 & 4\\
4 & -4
\end{pmatrix}.
\end{align*}

\item
Eigenwerte:
\begin{align*}
\lambda_1 &= -8\\
\lambda_2 &= 0.
\end{align*}

\item
Eigenvektoren
\begin{align*}
\vec x_1 &= (1, -1)^T\\
\vec x_2 &= (1, 1)^T
\end{align*}

\item
Eigenfrequenzen und Schwingungsdauer
\begin{align*}
\omega_1 &= 2\sqrt{2}\\
\omega_2 &= 0.
\end{align*}

$T_1 = 2\pi/w_1 = \pi/\sqrt{2}$.

$T_2$ kann nicht definiert werden, da $w_2=0$.
\end{itemize}


}
