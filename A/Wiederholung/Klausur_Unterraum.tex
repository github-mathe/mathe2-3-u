\Aufgabe[e]{Unterr\"aume}{
Gegeben sei ein Vektorraum $\vec V \subset \mathbb{R}^n$ und eine Menge $\vec U \subset \mathbb{R}^n$
\begin{abc}
\item Welche Bedingungen muss $\vec U$ erf\"ullen, damit $\vec U$ ein Untervektorraum von $\vec V$ ist?
\item Was ist die Dimension eines Unterraumes?
\item Wie ist eine Basis definiert? Wie bestimmt man eine Basis?
\item Erkl\"aren Sie den Begriff lineare Unabh\"angigkeit.
\item Was ist eine Linearkombination? 
\end{abc}
}

\Loesung{
\begin{abc}
\item Es muss gelten $\vec U \subset \vec V$, das heißt die Menge $\vec U$ muss Teilmenge von $\vec V$ sein.\\
      Weiterhin muss $\vec U$ selbst ein Vektorraum sein. Das heißt $\vec U$ muss folgende Eigenschaften erf\"ullen.
      \begin{iii}
        \item $\vec 0 \in \vec U $
        \item $\vec u_1 + \vec u_2 \in \vec U \quad \text{ f\"ur } \vec u_1 , \vec u_2 \in \vec U$
        \item $\lambda \vec u \in \vec U \quad \text{ f\"ur } \vec u \in \vec U \text{ und } \lambda \in \mathbb{R}$
      \end{iii}
      
\item Die Dimension eines Unterraumes wird durch die Anzahl der Vektoren einer Basis von $\vec U$ bestimmt. 
      Dabei gilt dim($\vec U$)$\leq$ dim($\vec V$). Ist dim($\vec U$) = dim($\vec V$), so ist $\vec U$ = $\vec V$.

\item Ein System von Vektoren $\mathcal=\left\{ \vec v_1,...,\vec v_n\right\}$ eines Vektorraumen $\vec V$ hei$\ss$t Basis von $\vec V$ genau dann, wenn
      \begin{iii}
      \item $\left\{\vec v_1,..., \vec v_n\right\}$  linear unabh\"angig
      \item $\vec V$ = span$\lbrace \vec v_1,...,\vec v_n \rbrace$. \\
      \end{iii}
      Das bedeutet, dass jeder Vektor aus $\vec V$ eindeutig als Linearkombination der Basisvektoren
      geschrieben werden kann. Die Basis ist ein maximales linear unabh{\"a}ngiges System, d.h. es kann kein
      Vektor hinzugef{\"u}gt werden, ohne dass die Vektoren linear abh{\"a}ngig werden. Die Basis ist ebenfalls
      ein minimales linear unabh{\"a}ngiges System, d.h. es kann kein Vektor weggelassen werden, ohne dass der
      gesamte Vektorraum $\vec V$ mit Hilfe der Vektoren erzeugt werden kann. \\
      
      Um eine Basis zu bestimmen, werden die Vektoren z.B. des Unterraumes $\vec U$ zeilenweise in eine Matrix
      geschrieben. Es wird der Gau{\ss}-Algorithmus auf die Matrix angewendet, um die Zeilenstufenform zu
      erzeugen. Die Nichtnullzeilen der Zeilenstufenform bilden dann eine Basis von $\vec U$. Der Rang der
      Matrix entspricht der Dimension von $\vec U$.

\item Eine Menge von Vektoren sind linear abh{\"a}ngig, wenn sich mindestens einer der Vektoren als    
      Linearkombination der anderen darstellen l{\"a}sst. Zudem hei{\ss}en die Vektoren $\vec v_1,...,\vec v_n$ 
      linear abh{\"a}ngig, falls es $\lambda_1,...,\lambda_n \in \mathbb{R}$ gibt, die nicht alle gleich Null 
      sind, so dass $\lambda_1 \vec v_1 +...+ \lambda_n \vec v_n = \vec 0$ gilt.
% Zudem gilt, dass der Nullvektor als Linearkombination der Vektoren $\lambda_1 \vec v_1 +...+ \lambda \vec v_n = \vec 0$ mit $\lambda \in \mathbb{K}$ nur erzeugt werden 
%kann, wenn mindestens ein $\lambda$ ungleich Null ist.      


\item Eine Linearkombination ist eine Summe von Vektoren, wobei jeder Vektor mit einer Zahl $\lambda \in \mathbb{R}$ skaliert wird, z.B. \\
$\vec v = \lambda_1 \vec v_1 + \lambda_2 \vec v_2$.
\end{abc}
}
