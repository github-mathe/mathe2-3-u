\Aufgabe[e]{Lineare Gleichungssysteme}{
Infolge eines Lecks wurde der Pumpenraum eines Schiffes vollständig geflutet.
Bewerten Sie die resultierende Veränderung der Schwimmlage, in dem Sie die Änderung in Tiefgang $T$, 
Krängungswinkel $\phi$ und Trimmwinkel $\psi$ berechnen. 

Die Flutung des Pumpenraums bedeutet für das Schiff dabei eine zusätzliche Gewichtslast $\Delta F_g$, sowie ein zusätzliches äußeres 
Krängungsmoment $\Delta M_\phi$ und ein zusätzliches äußeres Trimmmoment $\Delta M_\psi$. 
Der Zusammenhang zwischen Änderung im Tiefgang $\Delta T$,
Änderung im Krängungswinkel $\Delta \phi$ und Änderung im Trimmwinkel $\Delta \psi$ in Folge dieser Lasten kann über die folgenden Gleichungen beschrieben werden

\begin{align*}
\rho g \: A_w \:  \Delta T \: + \: \rho g \: (A_w  y_w) \: \Delta \phi 
 \: &= \: \Delta F_g,  \\
\rho g \: (A_w y_w) \: \Delta T \: + \: D \: (\frac{I_{w,xx} \: \rho g}{D} + z_B - z_G) \: \Delta \phi,
\: &= \: \Delta M_\phi \\
D \: (\frac{I_{w,yy}\: \rho g}{D} + z_B - z_G) \: \Delta \psi \: &= \: \Delta M_\psi.
\end{align*}

\bigskip
{\mbox{} \hfill \textbf{\textit{\Large Bitte wenden!}}}
\newpage
Gehen Sie wie folgt vor:

\begin{abc}
\item Ersetzen Sie in den oben stehenden Gleichungen die Tiefgangsänderung $\Delta T$ durch die Variable $x_1$, die Änderung im Krängungswinkel 
$\Delta \phi$ durch die Variable $x_2$ und die Änderung im Trimmwinkel $\Delta \psi$ durch die Variable $x_3$.
\item Stellen Sie das dadurch definerte Lineare Gleichungssystem für die Variablen $x_1, x_2$ und $x_3$ in Matrix-Vektor-Schreibweise der Form

$$
\vec A \vec x \: = \: \vec b \qquad \text{mit } \vec x =
\begin{pmatrix}
x_1 \\ x_2 \\ x_3
\end{pmatrix} \: \: \text{und} \:\:  \vec b = \begin{pmatrix}
\Delta F_g \\ \Delta M_\phi \\ \Delta M_\psi
\end{pmatrix}
$$

dar.

\item Bringen Sie das Lineare Gleichungssystem vor dem Einsetzen der Werte in Zeilen-Stufenform.
\item Rechnen Sie alle Werte der Tab. \ref{tab:my_label} in das internationale Einheitensystem (m, N, s, kg usw.) um.
\item Setzen Sie, nach der Umrechnung in das internationale Einheitensystem, in die Zeilen-Stufenform die in Tab. \ref{tab:my_label} gegebenen Größen ein.
\item Lösen Sie das resultierende Lineare Gleichungssystem und bestimmen Sie die resultierende Änderung von Tiefgang $\Delta T$, Krängungswinkel
$\Delta \phi$ und Trimmwwinkel $\Delta \psi$.
\end{abc}


\begin{table}[H]
    \centering
    \begin{tabular}{|c|c|c|}
\hline \\        $\rho$ & Dichte von Seewasser & 1025 $\Big [\frac{\text{kg}}{\text{m}^3}\Big ]$    \\ \hline
         $g$ & Erdbeschleunigung & 9.81 $\Big [\frac{\text{m}}{\text{s}^2}\Big ]$ \\ \hline
         $A_w$ & Wasserlinienfläche & 4208 $[\text{m}^2]$ \\ \hline
             $y_w$ & Schwerpunkt der Wasserlinienfläche in seitlicher Richtung $y$ & -0.125 [m] \\ \hline
        $I_{w,xx}$ & Flächenträgheitsmoment der Wasserlinienfläche um Längenachse & 680 000 [m$^4$] \\ \hline
        $I_{w,yy}$ & Flächenträgheitsmoment der Wasserlinienfläche um Seitenachse & 10 500 000 [m$^4$] \\ \hline
        $z_B$ & Höhe des Verdrängungsschwerpunkt über Kiel & 4.87 [m] \\ \hline 
    $z_G$ & Höhe des Gewichtsschwerpunkt über Kiel & 85.12 [m] \\ \hline 
    $D$ & Verdrängung (= Gewicht) des Schiffes & 21898.9 [t] \\ \hline
        $\Delta F_g$ & Zusätzliche Gewichtslast durch Leck &  1250 [kN]\\ \hline
            $\Delta M_\phi$ & Zusätzliches Krängungsmoment durch Leck & 10000 [kNm] \\ \hline
                $\Delta M_\psi$ & Zusätzliches Trimmmoment durch Leck & 32000 [kNm] \\ \hline
    \end{tabular}
    \caption{Charakteristische Schiffsgrößen}
    \label{tab:my_label}
\end{table}
}

\Loesung{
\begin{abc}
\item Ersetzen Sie in den oben stehenden Gleichungen die Tiefgangsänderung $\Delta T$ durch die Variable $x_1$, die Änderung im Krängungswinkel 
$\Delta \phi$ durch die Variable $x_2$ und die Änderung im Trimmwinkel $\Delta \psi$ durch die Variable $x_3$.

\begin{align*}
\rho g \: A_w \:  x_1 \: + \: \rho g \: (A_w  y_w) \: x_2
 \: &= \: \Delta F_g  \\
\rho g \: (A_w y_w) \: x_1 \: + \: D \: (\frac{I_{w,xx} \: \rho g}{D} + z_B - z_G) \: 	x_2
\: &= \: \Delta M_\phi \\
D \: (\frac{I_{w,yy}\: \rho g}{D} + z_B - z_G) \: x_3 \: &= \: \Delta M_\psi  
\end{align*}

\item Darstellung in Matrix-Vektor Schreibweise:

$$ \begin{pmatrix}
\rho g \: A_w & \rho g \: (A_w  y_w)  & 0 \\
\rho g \: (A_w  y_w)  &  D \: (\frac{I_{w,xx} \: \rho g}{D} + z_B - z_G) & 0 \\
0 & 0 & D \: (\frac{I_{w,yy}\: \rho g}{D} + z_B - z_G) 
\end{pmatrix} \begin{pmatrix}
x_1 \\ x_2 \\ x_3 
\end{pmatrix} \: = \: \begin{pmatrix}
\Delta F_g \\ \Delta M_\phi \\ \Delta M_\psi
\end{pmatrix}
$$

\item Zeilen-Stufenform:

%\begin{tiny}
{\fontsize{9}{10} \selectfont
\begin{align*}
\begin{pmatrix}
\rho g \: A_w & \rho g \: (A_w  y_w)  & 0 \\
0  &  D \: (\frac{I_{w,xx} \: \rho g}{D} + z_B - z_G) - \frac{\rho g \: (A_w  y_w)}{\rho g \: A_w } \: \rho g \: (A_w  y_w) & 0 \\
0 & 0 & D \: (\frac{I_{w,yy}\: \rho g}{D} + z_B - z_G) 
\end{pmatrix} \begin{pmatrix}
x_1 \\ x_2 \\ x_3 
\end{pmatrix} \: \\
= \: \begin{pmatrix}
\Delta F_g \\ \Delta M_\phi - \frac{\rho g \: (A_w  y_w)}{\rho g \: A_w} \: \Delta F_g \\ \Delta M_\psi
\end{pmatrix}
\end{align*}
%\end{tiny}
}

Dies lässt sich vereinfachen zu 

%\begin{tiny}

{\fontsize{11}{10} \selectfont
\begin{align*}
\begin{pmatrix}
\rho g \: A_w & \rho g \: (A_w  y_w)  & 0 \\
0  &  D \: (\frac{I_{w,xx} \: \rho g}{D} + z_B - z_G) - \rho g \: (A_w  y_w^2) & 0 \\
0 & 0 & D \: (\frac{I_{w,yy}\: \rho g}{D} + z_B - z_G) 
\end{pmatrix} \begin{pmatrix}
x_1 \\ x_2 \\ x_3 
\end{pmatrix} \: \\
= \: \begin{pmatrix}
\Delta F_g \\ \Delta M_\phi - y_w \: \Delta F_g \\ \Delta M_\psi
\end{pmatrix}
\end{align*}
%\end{tiny}
}

und 


$$ \begin{pmatrix}
\rho g \: A_w & \rho g \: (A_w  y_w)  & 0 \\
0  &  1 & 0 \\
0 & 0 & 1
\end{pmatrix} \begin{pmatrix}
x_1 \\ x_2 \\ x_3 
\end{pmatrix} \: = \: \begin{pmatrix}
\Delta F_g \\ \frac{\Delta M_\phi - y_w \: \Delta F_g}{D \: (\frac{I_{w,xx} \: \rho g}{D} + z_B - z_G) - \rho g \: (A_w  y_w^2)} \\ \frac{\Delta M_\psi}{D \: (\frac{I_{w,yy}\: \rho g}{D} + z_B - z_G) } 
\end{pmatrix}
$$

\item Umrechnung in das internationale Einheitensystem

    \centering
    \begin{tabular}{|c|c|c|}
\hline \\        $\rho$ & Dichte von Seewasser & 1025 $\Big [\frac{\text{kg}}{\text{m}^3}\Big ]$    \\ \hline
         $g$ & Erdbeschleunigung & 9.81 $\Big [\frac{\text{m}}{\text{s}^2}\Big ]$ \\ \hline
         $A_w$ & Wasserlinienfläche & 4208 $[\text{m}^2]$ \\ \hline
             $y_w$ & Schwerpunkt der Wasserlinienfläche in seitlicher Richtung $y$ & -0.125 [m] \\ \hline
        $I_{w,xx}$ & Flächenträgheitsmoment der Wasserlinienfläche um Längenachse & 680 000 [m$^4$] \\ \hline
        $I_{w,yy}$ & Flächenträgheitsmoment der Wasserlinienfläche um Seitenachse & 10 500 000 [m$^4$] \\ \hline
        $z_B$ & Höhe des Verdrängungsschwerpunkt über Kiel & 4.87 [m] \\ \hline 
    $z_G$ & Höhe des Gewichtsschwerpunkt über Kiel & 85.12 [m] \\ \hline 
    $D$ & Verdrängung (= Gewicht) des Schiffes & 21 898 900 [kg] \\ \hline
        $\Delta F_g$ & Zusätzliche Gewichtslast durch Leck &  1 250 000 [N]\\ \hline
            $\Delta M_\phi$ & Zusätzliches Krängungsmoment durch Leck & 10 000 000 [Nm] \\ \hline
                $\Delta M_\psi$ & Zusätzliches Trimmmoment durch Leck & 32 000 000 [Nm] \\ \hline
    \end{tabular}

\item Nach Einsetzen der Werte aus Tab. \ref{tab:my_label}:

$$ \begin{pmatrix}
42 312 492 \text{[N/m]}  & -5 289 061.5 \text{[N]}  & 0 \\
0  & 1  & 0 \\
0 & 0 & 1  
\end{pmatrix} \begin{pmatrix}
x_1 \\ x_2 \\ x_3 
\end{pmatrix} \: = \: \begin{pmatrix}
1250 \: 1000 \text{[N]} \\ 0.002 \text{[rad]} \\ 0.00031 \text{[rad]}
\end{pmatrix}
$$

Als Lösung erhält man schließlich

$$
\begin{pmatrix}
\Delta T \\ \Delta \phi \\ \Delta \psi
\end{pmatrix} = \begin{pmatrix}
0.03 \text{m} \\
0.11^\circ \\
0.035^\circ
\end{pmatrix}
$$
\end{abc}
}

\ErgebnisC{LGS}
{
$$
\begin{pmatrix}
\Delta T \\ \Delta \phi \\ \Delta \psi
\end{pmatrix} = \begin{pmatrix}
0.03 \\
0.11 \\
0.035
\end{pmatrix}
\quad 
\begin{bmatrix}
\text{m}\\\text{radians}\\\text{radians}
\end{bmatrix}
$$
}
