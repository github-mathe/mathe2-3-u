\Aufgabe[f]{Vektorwertige Funktionen}{
Gegeben sei die vektorwertige Funktion $\vec f : \mathbb{R}^2 \to \mathbb{R}^2$ durch

\begin{align*}
\vec f(x_1 , x_2) = \begin{bmatrix} x_1 + 2x_2 \\ -x_1 + x_2\end{bmatrix}
\end{align*}

Zeigen Sie, dass die Funktion $\vec f$ bijektiv ist und bestimmen Sie die Inverse.
}

\Loesung{
Die Abbildung $\vec f : \mathbb{R}^2 \to \mathbb{R}^2$ kann äquivalent über die Matrix-Vektor Multiplikation $\vec A \vec x$ der Form

$$
\vec A \vec x = \begin{bmatrix}
1 & 2 \\ -1 & 1
\end{bmatrix} \begin{pmatrix}
x_1 \\ x_2
\end{pmatrix}
$$

dargestellt werden. \\ \\
%
%
%
Die inverse Abbildung ist entsprechend über die zu der Matrix $\vec A$ inverse Matrix $\vec A^{-1}$ gegeben, die sich für eine $2\times 2$-Matrix mittels

\begin{align}
\label{eq_inv}
\vec A^{-1} = \frac{1}{\text{det}(\vec A)} \begin{bmatrix}
a_{22} & -a_{21} \\ -a_{12} & a_{11}
\end{bmatrix}
\end{align}
%
%
%
berechnen lässt. \\ \\
%
%
Bevor die Inverse bestimmt wird, sollte erst geprüft werden, wann und unter welchen Umständen die Inverse existiert. Dazu gibt es die folgenden Interpretationen,
deren Bedeutungen zueinander jedoch äquivalent sind:

\begin{iii}
\item $\vec f(\vec x)$ ist bijektiv
\item $\vec A$ hat vollen Rang
\item $\vec A$ ist regulär
\item $\text{det}(\vec A) \neq 0$
\item In Gl. (\ref{eq_inv}) wird nicht durch 0 geteilt
\end{iii}

Da die Determinante ohnehin benötigt wird, berechnen wir diese zuerst mit

$$
\text{det}(\vec A) = a_{11} \cdot a_{22} - a_{21} \cdot a_{12} = 1 \cdot 1 - (-1) \cdot 2 = 3 \neq 0 \qquad \forall \vec x \in \mathbb{R} 
$$

Dementsprechend wurde gezeigt, dass die Inverse $\vec A^{-1}$ existiert und somit $\vec f(x)$ bijektiv ist. Die Inverse $\vec A^{-1}$ ergibt sich nach Gl. (\ref{eq_inv}) 
zu

$$
\vec A^{-1} =  \frac{1}{\text{det}(\vec A)} \begin{bmatrix}
a_{22} & -a_{21} \\ -a_{12} & a_{11}
\end{bmatrix} 
= \frac{1}{3} \begin{bmatrix}
1 & -2 \\ 1 & 1
\end{bmatrix}
$$

Wir prüfen dies, indem wir Inverse mit der ursprünglichen Abbildung multiplizieren.

$$
\vec A^{-1} \vec A = \frac{1}{3} \begin{bmatrix}
1 & -2 \\ 1 & 1
\end{bmatrix} 
\begin{bmatrix}
1 & 2 \\ -1 & 1
\end{bmatrix} =
\frac{1}{3} \begin{bmatrix}
3 & 0 \\ 0 & 3
\end{bmatrix} = \begin{bmatrix}
1 & 0 \\ 0 & 1
\end{bmatrix} = \vec I
$$

Dies bestätigt das berechnete Ergebnis.

Alternativ kann die Inverse mit dem Gauß-Algorithmus berechnet werden.
}

