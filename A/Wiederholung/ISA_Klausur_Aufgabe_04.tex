\Aufgabe[e]{Dimensionierung eines Behälters}{
Ein oben offener Behälter mit rechteckigem Querschnitt soll aus 9 $m^2$ Dünnblech hergestellt werden.
Man bestimme die Abmessungen des Behälters so, dass das Volumen maximiert wird.
}

\Loesung{
$A$ sei die Gesamtfläche des Metalls, das für die Herstellung des Behälters verwendet wird, und $x$ und $y$ seien die Länge und Breite und $z$ die Höhe. Dann
ist 
$$
A = \underbrace{2xz + 2yz}_{\text{4 Seitenflächen}} + \underbrace{xy}_{\text{Bodenfl\"ache}} = 9.
$$
Außerdem gilt
$$
V = xyz
$$
Daraus folgt, dass 
$$
V = xy \frac{9-xy}{2x+2y}
$$
Um das Volumen zu maximieren, finden wir die Maxima der Funktion $V$.
%
Dazu berechnen wir den Gradienten von $V$ und finden seine Nullstellen, die die kritischen Punkte der Funktion sind

\begin{align*}
V'_x(x,y) = -y^2\frac{x^2+2xy-9}{2(x+y)^2} = 0,\\
V'_y(x,y) = -x^2\frac{y^2+2xy-9}{2(x+y)^2} = 0.
\end{align*}
%
Der erste kritischer Punkt ist der Punkt $(0,0)^\top$, der aber ein leeres Volumen ergibt und ist somit zu eliminieren.

Die anderen kritischen Punkte sind die Punkte $\vec P = (x,y)^\top$, welche das nichtlineare System
\begin{align*}
\frac{x^2+2xy-9}{2(x+y)^2} = 0,\\
\frac{y^2+2xy-9}{2(x+y)^2} = 0
\end{align*}
erfüllen.

Aus der ersten Gleichung ist
$$
2xy = 9 - x^2.
$$
Setzt man diesen Term in die zweite Gleichung ein, erhält man
$$
y^2-x^2 = 0,
$$
was die folgenden zwei Lösungen hat
\begin{align*}
y &= x,\\
y &=-x.
\end{align*}
Die Lösung $y=-x$ ist nicht gültig, weil die Abmessungen des Behälters positiv sein müssen. Es bleibt die Lösung $y=x$ übrig, die, wenn sie in die erste Gleichung eingesetzt wird, ergibt
$$
x^2+2x^2-9 = 0,
$$
was die beiden Lösungen hat
\begin{align*}
x &= \sqrt{3},\\
x & = -\sqrt{3}
\end{align*}
und wieder muss die negative Lösung eliminiert werden. Damit bleibt $y= \sqrt{ 3} = x$ und aus der Beziehung.
$$
z = \frac{9-xy}{2x+2y}
$$
folgt
$$
z = \frac{\sqrt{3}}{2}.
$$
Das maximale Volumen ist dann
$$
V = \sqrt{3}\,\sqrt{3}\,\frac{\sqrt{3}}{2} = \frac32 \sqrt{3}.
$$










}

\ErgebnisC{AufgZugstab}
{
Der Behälter hat eine Länge von $\sqrt{2}$ $m$, eine Breite von $\sqrt{2}$ $m$ und eine Höhe von $\sqrt{2}$ $m$. Das maximale Volumen ist XXX.
}
