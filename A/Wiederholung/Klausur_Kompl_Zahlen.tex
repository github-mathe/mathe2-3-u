\Aufgabe[f]{Komplexe Zahlen}{
\begin{abc}
\item
    Finden Sie alle L\"osungen der Gleichung:
    $$
    x^3 + x^2 + x = 3^3 + 3^2 + 3
    $$
    Geben Sie alle komplexen L\"osungen sowohl in kartesischen Koordinaten als auch in 
    Polarkoordinaten an.
\item
    Nutzen Sie das Ergebnis von \textbf{a} und Substitution, um die folgende Gleichung zu lösen:
    $$
    (x^2+1)^3 + (x^2+1)^2 + (x^2+1) = 3^3 + 3^2 + 3
    $$
\end{abc}
}
\Loesung{
\begin{abc}
\item
Zun\"achst schreiben wir die Gleichung um zu:
$$
x^3 + x^2 + x - 39 = 0
$$.
Offensichtlich ist eine L\"osung der Gleichung $x=3$. Daher k\"onnen wir den Faktor $(x_1-3)$ ausklammern.
Polynomdivision liefert:
$$
\begin{array}{r@{\,}l r@{\,}l r@{\,}l r@{\,}l r  l}
                   & (\fbox{$x^3$} & & +x^2 & & +x & & -39 ) &
: (\fbox{$x$}-3) = \fbox{$x^2 + 4x +13$} = q(x)\\
                  -& (x^3 & & -3x^2) \\
\cline{1-5}
\phantom{\fbox{0}} &         & &(\fbox{$4x^2$} & &  +x)\\
                   &         &-& (4x^2 & & -12x)\\
\cline{3-7}
\phantom{\fbox{0}} &      & &      & & (\fbox{$13x$} & & -39)\\
                   &      & &       &-& (13x & & +39)\\
\cline{5-8}

\phantom{\fbox{0}} &      & &       & &  &  \fbox{$0$}
\end{array}
$$
Die beiden anderen L\"osungen erhalten wir durch anwenden der quadratischen Formel:
$$
x_2 = -2 + 3 \operatorname{i} = r\operatorname{e}^{\operatorname{i}\phi_2}
$$
und 
$$
x_3 = -2 - 3 \operatorname{i} = r\operatorname{e}^{\operatorname{i}\phi_3}.
$$
Die Polarkoordinaten berechnen sich durch $r = \sqrt{2^2 + 3^2} = \sqrt{13}$
und $\phi_2 = \arccos(-\frac{2}{\sqrt{13}})$ und $\phi_2 = - \arccos(-\frac{2}{\sqrt{13}})$.

\item
Mit der Substitution $z = x^2 +1$ erhalten wir die Gleichung
$$
z^3 + x^2 + x = 2^3 + 2^2 + 2.
$$
Aus Teil \textbf{a)} wissen wir $z_1 =3$, $z_2 = -2 + 3 \operatorname{i}$ und
$z_3 = -2 - 3 \operatorname{i}$. 

Die R\"ucksubstitution liefert ersten beiden L\"osungen aus $z_1 = 3$ 
$$
x_{11} = \sqrt{2} \quad \text{ und } \quad x_{12} = -\sqrt{2}
$$
Aus $z_2 = x^2 + 1 = -2 + 3 \operatorname{i}$ folgt 
$$
x^2 = -3 + 3\operatorname{i} = 3\sqrt{2} \operatorname{e}^{\operatorname{i}\frac{3\pi}{4}}
$$
Die L\"osungen lassen sich berechnen mit Hilfe der L\"osungsformel
$$
x_{2k} = \sqrt{3\sqrt{2}} \operatorname{e}^{\operatorname{i} \frac{2 \pi k + \frac{3\pi}{4}}{2} }, \quad k = 0,1
$$
Daraus ergeben sich
$$
x_{21} = \sqrt{3\sqrt{2}} \operatorname{e}^{\operatorname{i} \frac{3\pi}{8}} \text{ und }
x_{22} = \sqrt{3\sqrt{2}} \operatorname{e}^{\operatorname{i} \frac{11\pi}{8}}
$$
Analog folgt aus $z_3 = x^2 + 1 = -2 - 3 \operatorname{i}$ 
$$
x^2 = -3 - 3 \operatorname{i} = 3\sqrt{2} \operatorname{e}^{\operatorname{i}\frac{5\pi}{4}}
$$
Daraus ergeben sich die L\"osungen
$$
x_{31} = \sqrt{3\sqrt{2}} \operatorname{e}^{\operatorname{i} \frac{5\pi}{8}} \text{ und }
x_{32} = \sqrt{3\sqrt{2}} \operatorname{e}^{\operatorname{i} \frac{13\pi}{8}}
$$
\end{abc}
}
