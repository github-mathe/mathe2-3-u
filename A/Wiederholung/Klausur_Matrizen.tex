\Aufgabe[e]{Matrizen}{
Gegeben seien folgende Matrizen und Vektoren
\begin{center}
\begin{align*}
% 3x3 unit\"ar
&\vec A = 
    \frac{1}{\sqrt 2} \begin{pmatrix} 
    \operatorname{i} & 0 & -\operatorname{i} \\
    \operatorname{i} & 0 & \operatorname{i}\\
    0 & \sqrt{2} & 0\end{pmatrix}
     , \,
% 4x2
&\vec B = \begin{pmatrix}
    \operatorname{i} & 1\\
    -1 & -\operatorname{i}\\
    0 & 1\\
    2\operatorname{i} & 1
    \end{pmatrix}\, , \,
% 2x2 orthogonal
&\vec C = \begin{pmatrix}
    \frac{\sqrt{2}}{2} & \frac{\sqrt{2}}{2}\\
    -\frac{\sqrt{2}}{2} & \frac{\sqrt{2}}{2}
    \end{pmatrix}\, , \\
% 4x4 symmetrisch
&\vec D = \begin{pmatrix}
    1 & 2 & 3 & 4\\
    2 & 1 & 5 & 6\\
    3 & 5 & 1 & 7\\
    4 & 6 & 7 & 1
    \end{pmatrix}\, , \,
% 3x2
&\vec E = \begin{pmatrix}
    1 & 1\\
    1 & 0\\
    0 & 1
    \end{pmatrix} \, , \,
%
&\vec F = \begin{pmatrix}
    0 & \operatorname{i}\\
    -\operatorname{i} & 0
    \end{pmatrix}\\
% v_1 , v_2 
&\vec v_1 = \begin{pmatrix}
    \operatorname{i} \\ 1 - 2\operatorname{i} \\ 3
    \end{pmatrix}\, , \,
&\vec v_2 = \begin{pmatrix}
    1 -\operatorname{i} \\ -2 \\ -2 + \operatorname{i}
    \end{pmatrix}
\end{align*}
\end{center}
\begin{abc}
\item Welche Matrixprodukte und Matrix-Vektorprodukte sind definiert? (Zur Übung können Sie alle 
      Produkte berechnen.)
\item Erkl\"aren Sie die Begriffe unit\"ar, hermitesch, symmetrisch, orthogonal anhand der gegebenen Matrizen.
\item Bestimmen Sie die Transponierte, Adjungierte und komplex Konjungierte von $\vec B$
\item Bestimmen Sie die Skalarprodukte $\langle \vec v_1,\vec v_2 \rangle$ und $\langle \vec v_2,\vec v_1 \rangle$.
\item Bestimmen Sie das Matrixprodukt $\vec E \vec C$.
\end{abc}

}

\Loesung{
\begin{abc}
\item Zwei Matrizen $\vec M_1\in \mathbb{K}^{n\times m}$ und $\vec M_2 \in \mathbb{K}^{k\times l}$ k\"onnen miteinander
      multipliziert werden, wenn $m = k$. Das bedeutet die Anzahl der Spalten von $\vec M_1$ entspricht der Anzahl der Zeilen
      von $\vec M_2$. Das bedeutet die m\"oglichen Matrix-Matrix-Produkte sind: 
      $$
      \vec A \vec E, \vec B \vec C, \vec D \vec B, \vec E \vec C, \vec B \vec F, \vec C \vec F, \vec E \vec F, \vec F \vec C
      $$
      sowie die Matrix-Vektor-Produkte 
      $$
      \vec A \vec v_1 \text { und } \vec A \vec v_2
      $$
\item Die Matrix $\vec A$ ist unit\"ar, da $\vec A^* \vec A = \vec E$. Das heißt, die Adjungierte von $\vec A$ ist die Inverse von $\vec A$.\\
      Die Matrix $\vec C$ ist orthogonal, da $\vec C^T \vec C = \vec E$. Das heißt, die Transponierte von $\vec C$ ist die Inverse von $\vec C$.\\
      Die Matrix $\vec D$ ist symmetrisch, da $\vec D^T = \vec D$. Das heißt, die Transponierte von $\vec D$ ist die Matrix $\vec D$ selbst.\\
      Die Matrix $\vec F$ ist hermitesch, da $\vec F^* = \vec F$. Das heißt, die Adjungierte von $\vec F$ ist die Matrix $\vec F$ selbst.
\item Die Transponierte von $\vec B$ ist
      $$
      \vec B^T = \begin{pmatrix}
                 \operatorname{i} & -1 & 0 & 2\operatorname{i}\\
                 1 & -\operatorname{i} & 1 & 1
                 \end{pmatrix}.
      $$
      Die Adjungierte von $\vec B$ ist
      $$
      \vec B^* = \begin{pmatrix}
                 -\operatorname{i} & -1 & 0 & -2\operatorname{i}\\
                 1 & \operatorname{i} & 1 & 1
                 \end{pmatrix}.
      $$
      Die komplex Konjungierte von $\vec B$ ist
      $$
      \vec{\overline B} = \begin{pmatrix}
               -\operatorname{i} & 1\\
               -1 & \operatorname{i}\\
               0 & 1\\
               -2\operatorname{i} & 1
               \end{pmatrix}.
      $$
\item Bei dem Skalarprodukt mit komplexen Vektoren ist zu beachten, dass der erste Vektor komplex konjugiert werden muss
      \begin{align*}
      \langle \vec v_1,\vec v_2 \rangle &= \overline{\vec v}_1^T \vec v_2\\
        &= \begin{pmatrix} - \operatorname{i} ,& 1+2\operatorname{i}, &3\end{pmatrix} \begin{pmatrix}1-\operatorname{i} \\-2\\-2+\operatorname{i} \end{pmatrix} \\
        &= -9-2\operatorname{i}
      \end{align*}
      Vertauscht man die beiden Vektoren, ist das Ergebnis das komplex konjugierte.
      \begin{align*}
      \langle \vec v_2,\vec v_1 \rangle = \overline{\langle \vec v_1,\vec v_2 \rangle} = -9 + 2\operatorname{i}
      \end{align*}
\item 
      \begin{align*}
      \vec E \vec C = 
      \begin{pmatrix}
        1 & 1\\
        1 & 0\\
        0 & 1
      \end{pmatrix}
      \begin{pmatrix}
        \frac{\sqrt{2}}{2} & \frac{\sqrt{2}}{2}\\
        -\frac{\sqrt{2}}{2} & \frac{\sqrt{2}}{2}
      \end{pmatrix}
      = \frac{\sqrt{2}}{2}\begin{pmatrix}
      0 & 2\\
      1 & 1\\
      -1 & 1
      \end{pmatrix}
      \end{align*}
\end{abc}
}
