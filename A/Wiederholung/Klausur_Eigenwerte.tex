\Aufgabe[e]{Eigenwerte}{
Gegeben sei die Matrix

\begin{align*}
\boldsymbol{A}=\begin{pmatrix}
~~\dfrac23 & ~\dfrac23 & -\dfrac23\\[1em]
-\dfrac23 & ~\dfrac73 & -\dfrac43\\[1em]
~~0 & ~0 & ~~1
\end{pmatrix}.
\end{align*}

mit dem charakteristischen Polynom

\begin{align*}
p(\lambda) = (1-\lambda) (\lambda^2-3\lambda+2).
\end{align*}

Gegeben seien weiterhin die Eigenvektoren 
		\begin{align*}
			\boldsymbol{v}_1 &= (-2, 0, 1)^\top,\\
			\boldsymbol{v}_2 &= (2, 1, 0)^\top, \\
            \boldsymbol{v}_3 &= (1, 2, 0)^\top.
		\end{align*}
\begin{abc}
\item Bestimmen Sie alle Eigenwerte. 
\item Zeigen Sie, dass die gegebenen Vektoren Eigenvektoren von $\boldsymbol A$ sind und geben Sie die zugehörigen Eigenwerte an.
\item Bestimmen Sie die algebraische und geometrische Vielfachheit aller Eigenwerte. 
      Was sagen die jeweiligen Vielfachheiten \"uber die Eigenwerte und Eigenvektoren aus?
\end{abc}
}

\Loesung{
\begin{abc}
\item Das charakteristische Polynom l\"asst sich schreiben als:
$$
p(\lambda) = (1-\lambda)^2 (2 - \lambda).
$$
Dies kann durch l\"osen der quadratischen Gleichung $\lambda^2-3\lambda+2 = 0$ bestimmt werden.\\
Damit erhalten wir die Eigenwerte $\lambda_1 = 1$ und $\lambda_2 = 2$.
\item Ein Eigenvektor muss die Gleichung 
$$
\boldsymbol{A} \boldsymbol{v} = \lambda \boldsymbol{v}
$$
erf\"ullen.
Nachrechnen ergibt f\"ur $\boldsymbol{v}_1$
\begin{align*}
\begin{pmatrix}
~~\dfrac23 & ~\dfrac23 & -\dfrac23\\[1em]
-\dfrac23 & ~\dfrac73 & -\dfrac43\\[1em]
~~0 & ~0 & ~~1
\end{pmatrix}
\begin{pmatrix}
-2\\0\\1
\end{pmatrix} = \begin{pmatrix}
-2\\0\\1
\end{pmatrix},
\end{align*}
f\"ur $\boldsymbol{v}_2$

\begin{align*}
\begin{pmatrix}
~~\dfrac23 & ~\dfrac23 & -\dfrac23\\[1em]
-\dfrac23 & ~\dfrac73 & -\dfrac43\\[1em]
~~0 & ~0 & ~~1
\end{pmatrix}
\begin{pmatrix}
2\\0\\1
\end{pmatrix} = 
\begin{pmatrix}
2\\0\\1
\end{pmatrix}
\end{align*}
und f\"ur $\boldsymbol{v}_3$
\begin{align*}
\begin{pmatrix}
~~\dfrac23 & ~\dfrac23 & -\dfrac23\\[1em]
-\dfrac23 & ~\dfrac73 & -\dfrac43\\[1em]
~~0 & ~0 & ~~1
\end{pmatrix}
\begin{pmatrix}
1\\2\\0
\end{pmatrix} = \begin{pmatrix}
2\\4\\0 
\end{pmatrix}
= 2 \begin{pmatrix}
1\\2\\0
\end{pmatrix}.
\end{align*}
Damit ist gezeigt, dass $\boldsymbol{v}_1$ und $\boldsymbol{v}_2$ die Eigenvektoren 
zu dem Eigenwert $\lambda_1 = 1$ sind und $\boldsymbol{v}_3$ der Eigenvektor zu dem 
Eigenwert $\lambda_2 = 2$ ist.
\item
Die algebraische Vielfachheit eines Eigenwertes l\"asst sich aus dem charakteristischen Polynom
ablesen. Die algebraische Vielfachheit ist die Vielfachheit der Nullstelle. Das heißt, 
$$
\alpha(1) = 2 \,\, \text{ und } \,\, \alpha(2) = 1.
$$
Die geometrische Vielfachheit ist gegeben durch die Anzahl der Eigenvektoren zu dem jeweiligen Eigenwert
beziehungsweise durch die Dimension des jeweiligen Eigenraumes.
Das heißt,
$$
\gamma(1) = 2 \,\, \text{ und } \gamma(2) = 1.
$$
\textbf{Beachte: } Die geometrische Vielfachheit ist immer kleiner gleich der algebraischen Vielfachheit.
\end{abc}
}
