\Aufgabe[e]{Laplace transform}
{
Answer the following points giving and example if necessary:
\begin{iii}
\item How is the Laplace transform defined?
\item Why the variable $s$ has to be positive?
\item What is the Heaviside function $h_{t_0}(t)$?
\item Explain with an example what is the damping of a function $f(t)$.
\item Is it true that $\mathcal{L}\left\{f(t)+g(t)\right\} = F(s)+G(s)$?
\item Is it true that $\mathcal{L}\left\{f(t)\,g(t)\right\} = F(s)\,G(s)$?
\item Recapitulate how to solve an initial value problem using the Laplace transform.
\item Write the properties of the Dirac delta function $\delta(t)$ (also called $\delta$-distribution).
\end{iii}
}


\Loesung{
\begin{iii}
\item How is the Laplace transform defined?

$\mathcal{L}\left\{f(t)\right\} = \int_0^\infty \operatorname{e}^{-st} f(t) \d t$.

\item Why the variable $s$ has to be positive?

Because the upper limit of the integral in the definition of the Laplace transform is $\infty$ and the exponential function $\operatorname{e}^{-st}$ is bounded for $t\rightarrow \infty$ only if $s>0$.
\item What is the Heaviside function $h_{t_0}(t)$?

It is the step function that is 0 for $t<t_0$ and 1 for $t\geq t_0$.
\item Explain with an example what is the damping of a function $f(t)$.

Show the graph of $\operatorname{e}^{-t} f(t)$ with $f(t)=\dfrac{t}{10}$ or $f(t)=\sin(t)$.

\begin{tikzpicture}
    \begin{axis}[
     axis lines=middle,clip=false,
            xmin=-1,xmax=10, ymin=-0.5,ymax=0.5,
            xticklabel style={black},
            xlabel=$x$,
            ylabel=$y$]
    \addplot[domain=0:10,samples=200,red]{exp(-x)*x}
				node[draw, above,pos=1.3,font=\footnotesize]{$f(x)=\operatorname{e}^{-t}\, \frac{t}{10}$};
    \addplot[domain=0.5:10,samples=200,blue]{exp(-x)}
				node[left,pos=0,font=\footnotesize]{$f(x)=\operatorname{e}^{-t}$};
    \addplot[domain=0:5,samples=200,cyan]{x/10}
				node[right,pos=1.,font=\footnotesize]{$f(x)=\dfrac{t}{10}$};
    \end{axis}
  \end{tikzpicture}

\begin{tikzpicture}
    \begin{axis}[
     axis lines=middle,clip=false,
            xmin=-1,xmax=6.28, ymin=-0.5,ymax=2,
            xticklabel style={black},
            xlabel=$x$,
            ylabel=$y$]
    \addplot[domain=0:6.28,samples=200,red]{exp(-x)*sin(deg(x))}
				node[right, draw,pos=1.1,font=\footnotesize]{$f(x)=\operatorname{e}^{-t}\, \sin(t)$};
    \addplot[domain=-0.5:6.28,samples=200,blue]{exp(-x)}
				node[left,pos=0,font=\footnotesize]{$f(x)=\operatorname{e}^{-t}$};
    \addplot[domain=0:6.28,samples=200,cyan]{sin(deg(x))}
				node[above,pos=0.3,font=\footnotesize]{$f(x)=\sin(x)$};
    \end{axis}
  \end{tikzpicture}

\item Is it true that $\mathcal{L}\left\{f(t)+g(t)\right\} = F(s)+G(s)$? YES.

\item Is it true that $\mathcal{L}\left\{f(t)\,g(t)\right\} = F(s)\,G(s)$? NO

\item Recapitulate how to solve an initial value problem using the Laplace transform.

\renewcommand{\labelenumi}{\arabic{enumi}}
 \setcounter{enumi}{0}
\begin{enumerate}
\item Transform right and left hand side of the ODE: Apply $\mathcal{L}\{\cdot\}$.
\item Substitute the initial values and solve for $Y(s)$.
\item Inverse transform (use the table) left and right hand side of the equation to get $y(t)=\mathcal{L}^{-1}(Y(s))$.
\end{enumerate}

\item Write the properties of the Dirac delta function $\delta(t)$ (also called $\delta$-distribution).

\begin{align*}
&\int_{-\infty}^\infty \delta(t) \d t = 1, \\
&\int_{-\infty}^\infty \delta(t-a) \d t = 1, \quad \text{ (arbitrary } a)\\
&\int_{-\infty}^\infty \delta(t-t_0)\, f(t) \d t = f(t_0)\\
\end{align*}

\end{iii}

}



