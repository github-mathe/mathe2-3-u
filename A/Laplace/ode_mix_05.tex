\Aufgabe[e]{Laplace-Transformation}
{
Beantworten Sie die folgenden Fragen und geben Sie gegebenenfalls ein Beispiel.
% Answer the following points giving and example if necessary:
\begin{abc}
\item Wie ist die Laplace-Transformation definiert?
\item Warum muss die Variable $s$ positiv sein?
\item Was ist die Heaviside-Funktion $h_{t_0}(t)$?
\item Erklären Sie anhand eines Beispiels, was die Dämpfung einer Funktion $f(t)$.
\item Ist die folgende Aussage wahr? $$\mathcal{L}\left\{f(t)+g(t)\right\} = F(s)+G(s)$$
\item Ist die folgende Aussage wahr?
$$\mathcal{L}\left\{f(t)\,g(t)\right\} = F(s)\,G(s)$$
\item Wiederholen Sie, wie man ein Anfangswertproblem mithilfe der Laplace-Transformation
lösen kann.
\item Schreiben Sie die Eigenschaften der Dirac-Delta-Funktion $\delta(t)$ (auch $\delta$-Distribution) auf.
\end{abc}
}


\Loesung{
\begin{abc}
\item Wie ist die Laplace-Transformation definiert?

$\mathcal{L}\left\{f(t)\right\} = \int_0^\infty \operatorname{e}^{-st} f(t) \, \mathrm{d} t$.

\item Warum muss die Variable $s$ positiv sein?

Weil die obere Grenze des Integrals der Definition der Laplace-Transformation $\infty$ ist
und die Exponentialfunktion $\operatorname{e}^{-st}$ nur beschränkt ist für 
$t\rightarrow \infty$, wenn $s>0$.


\item Was ist die Heaviside-Funktion $h_{t_0}(t)$?

Die Heaviside-Funktion ist die Stufenfunktion, die 0 ist für $t<t_0$ und 1 für $t\geq t_0$.

\item Erklären Sie anhand eines Beispiels, was die Dämpfung einer Funktion $f(t)$.

Zeigen Sie den Graphen von $\operatorname{e}^{-t} f(t)$ mit $f(t)=\dfrac{t}{10}$ oder $f(t)=\sin(t)$.

\begin{tikzpicture}
    \begin{axis}[
     axis lines=middle,clip=false,
            xmin=-1,xmax=10, ymin=-0.5,ymax=0.5,
            xticklabel style={black},
            xlabel=$x$,
            ylabel=$y$]
    \addplot[domain=0:10,samples=200,red]{exp(-x)*x}
				node[draw, above,pos=1.3,font=\footnotesize]{$f(x)=\operatorname{e}^{-t}\, \frac{t}{10}$};
    \addplot[domain=0.5:10,samples=200,blue]{exp(-x)}
				node[left,pos=0,font=\footnotesize]{$f(x)=\operatorname{e}^{-t}$};
    \addplot[domain=0:5,samples=200,cyan]{x/10}
				node[right,pos=1.,font=\footnotesize]{$f(x)=\dfrac{t}{10}$};
    \end{axis}
  \end{tikzpicture}

\begin{tikzpicture}
    \begin{axis}[
     axis lines=middle,clip=false,
            xmin=-1,xmax=6.28, ymin=-0.5,ymax=2,
            xticklabel style={black},
            xlabel=$x$,
            ylabel=$y$]
    \addplot[domain=0:6.28,samples=200,red]{exp(-x)*sin(deg(x))}
				node[right, draw,pos=1.1,font=\footnotesize]{$f(x)=\operatorname{e}^{-t}\, \sin(t)$};
    \addplot[domain=-0.5:6.28,samples=200,blue]{exp(-x)}
				node[left,pos=0,font=\footnotesize]{$f(x)=\operatorname{e}^{-t}$};
    \addplot[domain=0:6.28,samples=200,cyan]{sin(deg(x))}
				node[above,pos=0.3,font=\footnotesize]{$f(x)=\sin(x)$};
    \end{axis}
  \end{tikzpicture}

\item Ist die folgende Aussage wahr? $$\mathcal{L}\left\{f(t)+g(t)\right\} = F(s)+G(s)$$ Ja.

\item  Ist die folgende Aussage wahr?
$$\mathcal{L}\left\{f(t)\,g(t)\right\} = F(s)\,G(s)$$ Nein.

\item Wiederholen Sie, wie man ein Anfangswertproblem mithilfe der Laplace-Transformation
lösen kann.

\begin{iii}
\item Transformieren Sie die rechte und die linke Seite der Dgl.: Wende $\mathcal{L}\{\cdot\}$ an.
\item Setzen Sie die Anfangsbedingungen ein und lösen Sie nach $Y(s)$.
\item Rücktransformieren Sie die linke und die rechte Seite der Gleichung (mithilfe der Tabelle), um die Lösung $y(t)=\mathcal{L}^{-1}(Y(s))$ zu erhalten.
\end{iii}

\item Schreiben Sie die Eigenschaften der Dirac-Delta-Funktion $\delta(t)$ (auch $\delta$-Distribution) auf.

\begin{align*}
&\int_{-\infty}^\infty \delta(t) \, \mathrm{d} t = 1, \\
&\int_{-\infty}^\infty \delta(t-a) \, \mathrm{d} t = 1, \quad \text{ (beliebig } a)\\
&\int_{-\infty}^\infty \delta(t-t_0)\, f(t) \, \mathrm{d} t = f(t_0)\\
\end{align*}

\end{abc}

}




\ErgebnisC{odemix05}{
\begin{abc}
\item $\mathcal{L}\left\{f(t)\right\} = \int_0^\infty \operatorname{e}^{-st} f(t) \, \mathrm{d} t$
\item $\operatorname{e}^{-st}$ nur beschränkt ist für 
$t\rightarrow \infty$, wenn $s>0$.
\item  $h_{t_0}$: Stufenfunktion, die 0 für $t<t_0$ und 1 für $t\geq t_0$ ist.
\item /
\item Ja
\item Nein
\item /
\item $\int_{-\infty}^\infty \delta(t) \, \mathrm{d} t = 1, \, \int_{-\infty}^\infty \delta(t-a) \, \mathrm{d} t = 1, \, \int_{-\infty}^\infty \delta(t-t_0)\, f(t) \, \mathrm{d} t = f(t_0)$
\end{abc}
}
