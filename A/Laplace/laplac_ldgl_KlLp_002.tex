\Aufgabe[e]{Lineare Differentialgleichung}
{
Gegeben sei das Anfangswertproblem für \ $u(t)$
$$	u'' +\,u = \sin(t)\cdot\Big(1-h(t-\pi)\Big)\ ,\quad u(0)=u'(0)=0$$
wobei \ $h(t)$ \ die Heaviside--Funktion ist.

\begin{abc}
\item Bestimmen Sie zun\"achst die L\"osung im Bereich $0 \leq t \leq \pi$ mit Hilfe des Exponentialansatzes und dann die L\"osung im Bereich $t \geq \pi$.

\item Bestimmen Sie die L\"osung des AWPs mit Hilfe der Laplace-Transformation
\end{abc}
}

\Loesung{
\begin{abc}
\item Im Bereich $0 \leq t \leq \pi$ gilt f\"ur die homogene lin. Dgl. $u''(t)+u(t)=0$. Das charakteristische Polynom
$$
\lambda^2+1=0
$$
besitzt die Nullstellen $\lambda_{1,2} = \pm i$. Daraus folgt die allgemeine homogene L\"osung
$$
u_{\text{h}}(t) = a \sin(t) + b \cos(t), \, a,b \in \mathbb{R}.
$$
F\"ur die partikul\"are L\"osung w\"ahlen wir den Ansatz
$$
u_{\text{p}}(t) = A t \sin(t) + B t \cos(t).
$$
Bilden der Ableitungen
\begin{align*}
u_{\text{p}}'(t)  &= (A-Bt)\sin(t) +(At+B)\cos(t) \\
u_{\text{p}}''(t) &= (2A-Bt)\cos(t) -(At+2B)\sin(t)
\end{align*}
und einsetzen in die Dgl. $u''(t)+u(t)=\sin(t)$ liefert
$$
2A\cos(t) -2B\sin(t) = \sin(t).
$$
Damit ergibt sich $A=0$ und $B= -1/2$. Damit folgt
$$
u_{\text{p}}(t) = - \frac{1}{2}t \cos(t)
$$
und damit die allgemeine L\"osung 
$$
u(t) = a \sin(t) + b \cos(t) - \frac{1}{2}t \cos(t), \, a,b \in \mathbb{R}.
$$
Mit den Anfangswertbedingungen
\begin{align*}
u(0) &= 0 = a \sin(0) +b \cos(0) \Rightarrow b = 0\\
u'(0) &= a \cos(0) -b\sin(0)-\frac{1}{2} \cos(0) +\frac{1}{2}\cdot 0 \cdot \sin(0) \Rightarrow a = \frac{1}{2}
\end{align*}
ergibt sich die L\"osung des AWPs
$$
u_{\text{AWP}}(t) = \frac{1}{2} \sin(t) -\frac{1}{2} t\cos(t), \, 0\leq t \leq \pi.
$$

\noindent
Am Endes des Bereiches bei $t=\pi$ gilt:
$$
u_{\text{AWP}}(\pi) = \frac{\pi}{2} \, \text{und} \, u_{\text{AWP}}'(\pi) = 0.
$$
Dies sind die Anfangswerte f\"ur den zweiten Bereich. Im zweiten Bereich ist die lin. Dgl. homogen mit der schon bekannten allg. L\"osung
$$
u(t)= a \sin(t) +b \cos(t) \Rightarrow  u_{\text{AWP}}(t) = - \frac{\pi}{2} \cos(t), \, t \geq \pi.
$$

\item 
F\"ur den Verschiebungssatz 
$$
\mathcal{L}\left( f(t-a) h(t-a) \right) = F(s) e^{-as}, a>0,
$$
muss die Funktion erst umgeschrieben werden:
$$
\sin(t) = - \sin(t-\pi).
$$

\noindent
Damit ergibt sich Laplace--Transformation des AWPs ergibt
	\[
	u''(t)+u(t)=\sin(t)+h(t-\pi)\sin(t-\pi) \Rightarrow s^2\,U(s) +U(s)= \frac{1}{s^2+1} +\frac{1}{s^2+1} e^{-\pi s} .
\]
Die Laplace--Transformierte der Lösung ergibt sich damit zu
	\[
	U(s) =  \frac{1}{(s^2+1)^2} \left( 1+ e^{-\pi s} \right).
\]
Hier braucht keine Partialbruchzerlegung mehr durchgef\"uhrt werden.

Die Rücktransformation ergibt die gesuchte Lösung
\begin{align*}
u(t) 
  &= \frac{\sin(t) - t\cos(t)}{2}
     + \left( \frac{\sin(t - \pi) - (t - \pi)\cos(t - \pi)}{2} \right) h(t - \pi) \\
  &= \frac{\sin(t) - t\cos(t)}{2}
     + \left( \frac{-\sin(t) + (t - \pi)\cos(t)}{2} \right) h(t - \pi) \\
  &= 
     \begin{cases}
        \dfrac{\sin(t) - t\cos(t)}{2} & \text{f\"ur} \, 0 \leq t \leq \pi \\
        -\dfrac{\pi}{2} \cos(t) & \text{f\"ur} \,  t> \pi.
     \end{cases}
\end{align*}

\end{abc}
}

\ErgebnisC{AufglaplacldglKlLp002}{
\text{a),b)} $u_{\text{AWP}}(t) = \frac{1}{2} \sin(t) -\frac{1}{2} t\cos(t), \, 0\leq t \leq \pi$ und $u_{\text{AWP}}(t) = - \frac{\pi}{2} \cos(t), \, t \geq \pi$.
}
