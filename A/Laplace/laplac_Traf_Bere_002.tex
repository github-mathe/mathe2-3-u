\Aufgabe[e]{Laplace--Transformierte}
{
Bestimmen Sie unter Verwendung von \ ${\cal{L}}\big\{\sin(t)\big\} = \dfrac{1}{s^2+1}$ \ und geeigneten Rechenregeln folgende Ausdr\"ucke
	\[
	\textbf{a)} \ {\cal{L}}\left\{\frac{\sin(t)}{t}\right\}\ ,\quad
	\textbf{b)} \ {\cal{L}}\left\{\int\limits_0^t\frac{\sin(\tau)}{\tau}\ \text d\tau\right\}\ ,\quad
	\textbf{c)} \ \int\limits_{0}^{\infty}\frac{\sin(t)}{t}\ \text dt\ ,\quad
	\textbf{d)} \ {\cal{L}}\left\{\EH{-t}\,\frac{\sin(t)}{t}\right\}\ .
\]



\textbf{e)} \ Bestimmen Sie \textbf{mit Hilfe des Faltungssatzes} der Laplace-Transformation die Urbildfunktion \  $f(t)$ \ zur Bildfunktion
  \[
  F(s) := \dfrac{s}{(s+1)(s^2+1)}\ .
\]
}

\Loesung{
\textbf{a)} \ Mit der Transformationsformel \ \ ${\cal{L}}\left\{\dfrac{f(t)}{t}\right\} = \int\limits_{s}^\infty  F(\sigma)\ \text d\sigma$ \ \ erhält man
	\[
	{\cal{L}}\left\{\frac{\sin(t)}{t}\right\} = \int\limits_{s}^\infty  \dfrac{1}{\sigma^2+1}\ \text d\sigma = \Big[\arctan(\sigma)\Big]_s^\infty = \dfrac \pi2-\arctan(s)\ .
\]

\textbf{b)} \ Mit der Transformationsformel \ \ ${\cal{L}}\left\{\int\limits_{0}^{t} f(\tau) \text d\tau\right\} = \dfrac{F(s)}{s}$ \ \ erhält man 
	\[
	{\cal{L}}\left\{\int\limits_{0}^{t}\frac{\sin(\tau)}{\tau}\ \text d\tau\right\} = \frac 1s\cdot\left(\dfrac \pi2-\arctan(s) \right)
\]


\textbf{c)}

\textbf{1. Lösungsweg}\newline
Mit Hilfe des Anfangs- und Endwertsatzes ergibt sich 
\begin{align*}
\int\limits_{0}^\infty\frac{\sin(t)}{t}\ \text dt
	   =& \lim_{t\to\infty} \int\limits_{0}^t\frac{\sin\tau}{\tau}\d \tau = \lim_{s\to 0}\left(
	   s\cdot {\cal{L}}\left\{\int\limits_{0}^t\frac{\sin(\tau)}{\tau}\d \tau\right\}\right) \\
=& \lim_{s\to 0}\left(\dfrac \pi2-\arctan(s) \right) = \dfrac \pi2\ .
\end{align*}

\textbf{2. Lösungsweg}\newline
Nach Definition der Laplace-Transformierten gilt
$${\cal{L}}\left\{\dfrac{\sin(t)}{t}\right\}=\int\limits_{0}^\infty\EH{-st}\cdot\frac{\sin(t)}{t}\ \text dt=U(s)$$

Aus der Teilaufgabe \textbf{a)} folgt $U(s)=\dfrac \pi2-\arctan(s)$.\newline
Somit erhält man
$$\int\limits_{0}^\infty\frac{\sin(t)}{t}\ \text dt=\int\limits_{0}^\infty\EH{-0\cdot t}\cdot\frac{\sin(t)}{t}\ \text dt=U(0)=\dfrac \pi2$$





\textbf{d)} \ Mit der Transformationsformel \ \ ${\cal{L}}\left\{\EH{-at}\,f(t)\right\} = F(s+a)$ \ \ erhält man
	\[
	{\cal{L}}\left\{\EH{-t}\,\frac{\sin(t)}{t}\right\} = \dfrac \pi2-\arctan(s+1)\ .
\]


\textbf{e)} \ Es gilt
$$
F(s) = F_1(s) \cdot F_2(s) \quad \mbox{mit} \quad F_1(s) = \dfrac{1}{s+1}\,, \quad F_2(s)= \dfrac{s}{s^2+1}\,.
$$
Mit den Rücktransformierten 
$$
f_1(t) = \EH{-t} \quad \mbox{und} \quad f_2(t) = \cos( t)
$$
folgt
\begin{align*}
f(t) & = f_1(t) \ast f_2(t) = \int_0^t \EH{-(t-\tau)}\cos (\tau) \text d \tau = \EH{-t} \int_0^t \EH{\tau}\cos (\tau) \text d \tau\,.
\end{align*}
Mit 
\begin{align*}
I=&\int_0^t \EH{\tau}\cos \tau \text d \tau  = \EH{\tau}\sin \tau\Big|_0^t -  \int_0^t \EH{\tau}\sin \tau \text d \tau\\[1ex]
& =  \EH{t}\sin t + \EH{\tau}\cos \tau |_0^t - \int_0^t \EH{\tau}\cos \tau \text d \tau\\[1ex]
& =  \EH{t}\sin t + \EH{t}\cos t - 1  - I
\end{align*}
folgt
$$2I=\EH{t}\sin t + \EH{t}\cos t - 1$$
und
$$
I = \dfrac{1}{2}\,\EH{t}\sin t + \dfrac{1}{2}\,\EH{t}\cos t - \dfrac{1}{2}
$$


Alternativ kann man im Komplexen rechnen:
\begin{align*}
I=&\int_0^t \EH{\tau}\cos \tau \text d \tau  =\int_0^t \Real(\EH{\tau}\cdot\EH{i\tau})  \text d \tau \\[1ex]
&=\Real\left(\int_0^t (\EH{(1+i)\tau}\text d \tau\right) =\left[\Real\left(\dfrac{\EH{(1+i)\tau}}{1+i}\right)\right]_0^t=\left[\EH{\tau}\Real\left(\dfrac{(\cos\tau+i\sin\tau)(1-i)}{2}\right)\right]_0^t\\[1ex]
&=\left[\EH{\tau}\cdot\dfrac{(\cos\tau+\sin\tau)}{2}\right]_0^t=\EH{t}\cdot\dfrac{(\cos t+\sin t)}{2}-\dfrac{1}{2}\\
\end{align*}

Somit ergibt sich
$$
\ f(t) = \EH{-t} \cdot I=\dfrac{1}{2} (\sin t + \cos t) - \dfrac{1}{2}\,\EH{-t}\ .\ 
$$

}

\ErgebnisC{AufglaplacTrafBere002}
{
\textbf{ a)} $\frac \pi 2 - \arctan s$, 
\textbf{ b)} $\frac {\pi/2-\arctan s} s$, 
\textbf{ c)} $\frac \pi 2$,
\textbf{ d)} $\frac \pi 2 - \arctan (s+1)$,
\textbf{ e)} $\frac {\sin t + \cos t}2 - \frac{\EH{-t}}2$
}
