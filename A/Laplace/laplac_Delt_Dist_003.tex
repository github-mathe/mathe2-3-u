\Aufgabe[e]{AWP und $\delta$--Distribution}
{
Ein mechanisches Pendel werde durch das folgende Anfangswertproblem beschrieben
	\[
	u''(t) + 2\,u'(t)+5\,u(t) = f(t)\ ,\quad u(0)=2\ ,\ \ u'(0)=-2 \ .
\]
$u''(t)$ steht nach dem zweiten Newtonschen Gesetz f\"ur die Beschleunigung einer Masse. Der Term $5u(t)$ modelliert ein repulsives Potential (Federkraft) und der Term $2u'(t)$ die D\"ampfung des Systems. Das Pendel befindet sich zum Zeitpunkt $t=0$ am Ort $u(0)=2$ und hat die Geschwindigkeit $u'(0)=-2$.   
\begin{itemize}
\item[a)]  Bestimmen Sie mit Hilfe der Laplace--Transformation die L\"osung des AWPs f\"ur $f(t) = 0$, $t>0$. (Es wirken keine \"außeren Kr\"afte.)
\item[b)]  Bestimmen Sie den Zeitpunkt $t_0$ des ersten Nulldurchgangs, d.h.\ $u(t_0)=0$, der L\"osung aus Teil a)\,.
\item[c)]  Zum Zeitpunkt $t_0$ aus Teil b) wird ein $\delta$--Impuls $f(t) 
=\alpha\cdot\delta(t-t_0)$ so auf das System ausge\"ubt, dass das System anschließend in Ruhe ist. \\
Dies modelliert ein starres Hindernis, auf welches das Pendel (nicht elastisch) aufprallt, so dass die Bewegung sofort endet. \\
Wie groß muss die Impulsst\"arke $\alpha$ sein?
\end{itemize}
}
\Loesung{
\textbf{Zu a)}  Die Laplace--Transformation des AWPs ergibt mit \ $\mathcal L\big(u(t)\big)=\mathcal U(s)$\,:
	\[
	s^2\,\mathcal U-2\,s+2+2\cdot\big[s\,\mathcal U-2\big]+ 5\,\mathcal U = 0 \,\Rightarrow\, \mathcal U(s) = \frac{2(s+1)}{(s+1)^2+2^2} \ .
\]
Die R\"ucktransformation ergibt die L\"osung des AWPs
\[
\boxed{\ u_{\text{AWP}}(t) = 2\,\EH{-t}\cdot \cos(2t)\ }\ .
\]
  
\bigskip
\textbf{Zu b)} Aus \ $2\,t_0=\pi/2$ \ erh\"alt man \ $\boxed{\ t_0=\pi/4\ }$\,.

\bigskip
\textbf{Zu c)} Das inhomogene lineare AWP lautet
\[
	u''(t) + 2\,u'(t)+5\,u(t) = \alpha\cdot\delta(t-\pi/4) \ .
\]
Die Laplace--Transformation ergibt
	\[
	s^2\,\mathcal U-2\,s+2+2\cdot\big[s\,\mathcal U-2\big]+ 5\,\mathcal U = \alpha\,\EH{-s\cdot\pi/4} 
\] 
und nach \ $\mathcal U(s)$ \ aufgel\"ost:
 \[
	\mathcal U(s) = \frac{2(s+1)}{(s+1)^2+2^2}+\EH{-s\cdot\pi/4}\cdot\frac{\alpha}{(s+1)^2+2^2}\ .
\]
Die R\"ucktransformation ergibt
	\[
\begin{array}{rcl}
	u_{\text{AWP}}(t) & = & 2\,\EH{-t}\cdot \cos(2t) + \dfrac{\alpha}{2}\cdot\EH{-(t-\pi/4)}\cdot\sin\big(2(t-\frac\pi 4)\big)\cdot h(t-\frac \pi 4) \\
	\\
	& = & \EH{-t}\cdot\Big\{ 2\,\cos(2t) + \dfrac{\alpha}{2}\cdot\EH{\pi/4}\cdot\Big[\sin(2t)\,\cos(\frac \pi 2)\\[3ex]
            & & \qquad \quad -\cos(2t)\,\sin(\frac \pi 2)\Big]\cdot h(t-\frac \pi 4)\Big\} \\
	\\
	& = & \EH{-t}\,\cos(2t)\cdot\Big\{2-\dfrac \alpha 2\,\EH{\pi/4}\cdot h(t-\frac \pi 4)\Big\}
\end{array}
\]
Damit die L\"osung f\"ur \ $t\ge t_0$ \ verschwindet, muß \ $2=\dfrac \alpha 2\,\EH{\pi/4}$ \ sein, der $\delta$--Implus also die St\"arke
\[
 \boxed{\ \alpha = 4\,\EH{-\pi/4}\ }
\]
haben.
}

\ErgebnisC{AufglaplacDeltDist003}
{
\textbf{a)} $u_{\text{AWP}}(t) = 2\,\EH{-t}\cdot \cos(2t)$, \textbf{b)} $t_0=\pi/4$, 
\textbf{c)} $\alpha = 4\,\EH{-\pi/4}$.
}


