\Aufgabe[e]{Balkenbiegung}
{
Ein homogener Balken ($E,J$\ konstant) der Länge \ $L=3$ \ möge an beiden Enden gelenkig gelagert sein. Bei 2/3 der Länge greife eine
punktförmige Last \ $F$ \ an. Berechnen Sie die Lage des tiefsten Punktes des Balkens, wobei sein
Eigengewicht vernachlässigt werden darf.\\
Das Materialgesetz des Balkens wird als
\[
EJ\cdot w''''(x)=-F\cdot \delta (x-l) \qquad (\text{mit }l=\frac 23 L)
\]
angenommen. \\
\textbf{Hinweise}: $EJ$ bezeichnet die Biegesteifigkeit des Balkens. Zur Vereinfachung k\"onnen Sie annehmen $EJ=1$. Ebenson k\"onnen Sie $F=1$ setzen. \\
Gehen Sie in den folgenden Schritten vor: 
\begin{abc}
\item Ermitteln Sie die L\"osung $w_H(x)$ der homogenen Differentialgleichung. 
\item Bestimmen Sie  eine spezielle Lösung $w_P(x)$ (bzw. $W_P(s)$) der inhomogenen Differentialgleichung, indem Sie die
Laplace-Transformation nutzen, wobei Sie von
homogenen Anfangswerten ausgehen k\"onnen. 
\item Bestimmen Sie die Integrationskonstanten der allgemeinen L\"osung der inhomogenen Gleichung
$w(x)=w_H(x)+w_P(x)$ aus den Randbedingungen 
$$w(0)=w(L)=0\qquad \text{ und }\qquad w''(0)=w''(L)=0.$$
\item Berechnen Sie den Extremwert der so erhaltenen Funktion. 
\end{abc}
}
\Loesung{
Wir vereinfachen die Differentialgleichung zu 
$$w^{(4)}(x)=-6\alpha \delta(x-l)$$
mit der neuen Konstanten $\alpha=\frac{F}{6EJ}$. 
\begin{abc}
\item Die homogene Gleichung $w_H^{(4)}(x)=0$ kann durch einfache Integration gel\"ost werden: 
$$w_H(x)=A+Bx+Cx^2+Dx^3.$$
\item 
Für \ \ $w_P(0)=w_P'(0)=w_P''(0)=w_P'''(0)=0$ \ \ lautet die Laplace--Transformation der inhomogenen
linearen Differentialgleichung \ \ $w_P^{(4)}(x)=-6\alpha\cdot \delta (x-l)\;:$
\[
s^4\;W_P(s)=-6\alpha\cdot \,\text{e}^{-l\,s} \,\Rightarrow\, W_P(s)=-6\alpha\cdot \frac{\text{e}^{-l\,s}}{s^4}\;. 
\]

Die Rücktransformation ergibt 
\[
w_{P}(x)=-6\alpha\cdot \frac{(x-l)^3}6\cdot h(x-l)=-\alpha(x-l)^3\cdot h(x-l)\;. 
\]
Dabei ist $h(x)=\left\{\begin{array}{rr}0&\text{ f\"ur }x<0\\ 1&\text{ f\"ur }x\geq
0\end{array}\right.$ die Heaviside-Funktion. 
\item 
Damit lautet die allgemeine Lösung: 
\[
w(x)=w_H(x)+w_P(x)=A+B\,x+C\,x^2+D\,x^3-\alpha \,(x-l)^3\cdot h(x-l). 
\]

Die Randbedingungen ergeben: 
\[
\begin{array}{rcllll}
w(0) = 0 & : &  & A=0 &  &  \\ 
&  &  &  &  &  \\ 
w''(0)= 0 & : &  & 2\,C=0 &  &  \\ 
&  &  &  &  &  \\ 
w(L) = 0 & : &  & B\cdot L + D\cdot L^3 - \alpha (L-l)^3  = 0 \\
&  &  &  &  &  \\ 
w''(L) = 0 & : &  & 6\cdot D \cdot L -6\,\alpha (L-l)= 0 \\
\end{array}
\]
In den letzen beiden Zeilen wurde $A=C=0$ ber\"ucksichtigt. \\
Aus der letzten erh\"alt man 
$$D=\frac{\alpha(L-l)}{L}=\frac{\alpha }{3}$$
und damit aus der dritten:
$$B=\frac{\alpha(L-l)^3-DL^3}{L}=\frac{\alpha\frac{L^3}{27}- \frac{\alpha}3L^3}L = \frac{-8\alpha L^2}{27}.$$
Insgesamt haben wir so als L\"osung der Randwertaufgabe: 
\begin{align*}
w(x)=&-\frac{8\alpha}{27}L^2x + \frac{\alpha}3 x^3- \alpha(x-l)^3\cdot h(x-l)\\
=& -\frac 83 \alpha x + \frac \alpha 3 x^3 - \alpha(x-2)^3\cdot h(x-2).
\end{align*}
\item Das Minimum dieser Funktion liegt entweder in einem station\"aren Punkt ($w'(x)=0$) oder an
den R\"andern des Definitionsbereichs ($x=0$, $x=3$) oder an der Sprungstelle der
Funktionsdefinition ($x=2$). Dort ist die Funktion zwar zweimal differenzierbar, aber auf die
Berechnung der Ableitung wird hier verzichtet. Die station\"aren Punkte in den Teilintervallen
$[0,2]$ und $[2,3]$ ergeben sich zu: 
\begin{iii}
\item $0< x<2 $: 
\begin{align*}
&&0=& w'(x)= -\frac 8 3 \alpha +\alpha x^2&
\Rightarrow && x=& \sqrt{\frac 83} 
\end{align*}
Die negative Wurzel entf\"allt wegen der Bedingung $0<x$. 
\item $2<x<3$:
\begin{align*} 
&&0=& w'(x)= -\frac 8 3 \alpha +\alpha x^2 -3\alpha (x-2)^2\\
\Rightarrow && 0=& (1-3)x^2+12x-12-\frac 83 \\
\Rightarrow && 0=& x^2-6x+\frac{22}3=\left( x- 3\right)^2- 9 + \frac{22}3\\
\Rightarrow&&  x=&3\pm\sqrt{\frac 53}
\end{align*}
Beide L\"osungen liegen außerhalb des betrachteten Definitionsintervalls $(2<x<3)$
\end{iii}
Kandidaten f\"ur das Minimum sind also $x_1=0,\, x_2=\sqrt{8/3},\, x_3=2,\, x_4=3$ 
mit 
$$w(0)=0,\, w(x_2)=-\frac 23 x_2^3\alpha< w(2)=-\frac 83\alpha,\, w(3)=0.$$
Damit liegt das Minimum bei $x_2=\sqrt{\frac 83}$. 
\end{abc}
\begin{center}
\begin{pspicture}(-1,-3)(4,.4)
\psgrid[gridcolor=lightgray,linecolor=lightgray](-1,-3)(4,0)
\psplot[plotpoints=100, plotstyle=curve, linecolor=black]
{0}{2}
{
-8 27 div 9 mul x mul x 3 exp 3 div add 
}
\psplot[plotpoints=100, plotstyle=curve]
{2}{3}
{
-8 27 div 9 mul x mul x 3 exp 3 div add x -2 add 3 exp neg add
}
\psline[linewidth=2pt]{->}(2,-1)(2,-2.666)
\end{pspicture}
\end{center}
}
\ErgebnisC{AufglaplacBalkBieg001}
{
$w_P(x)=-\frac{F}{6EJ}(x-l)^3\cdot h(x-l)$, $x_{\text{min}}=\sqrt{\frac 8{27}}L$
}
