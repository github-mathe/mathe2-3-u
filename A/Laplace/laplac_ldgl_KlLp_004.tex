% \Aufgabe[e]{Lineare Differentialgleichung}
% {
% Gegeben sei das Anfangswertproblem für \ $u(t)$
% $$	u'' + 4\,u' +3\,u = 12\cdot\Big(1-h(t-2)\Big)\ ,\quad u(0)=u'(0)=0$$
% wobei \ $h(t)$ \ die Heaviside--Funktion ist.


% \textbf{a)} \ Bestimmen Sie die Lösung mit Hilfe der Laplace--Transformation.

% \textbf{b)} \ Geben Sie die Lösung in den Bereichen  $0\le t < 2$  und  $ 2\le t$ ohne Verwendung der Heaviside--Funktion an und fassen Sie die Terme sinnvoll zusammen.

% }
\Aufgabe[e]{Linear ODE}
{
Gegeben sie das folgende Anfangswertproblem für $y(t)$ durch
$$	y'' + 4\,y = t\ ,\quad y(0)= 1, y'(0)=2.$$

\begin{abc}
\item Berechnen Sie die Lösung mit dem Exponentialansatz.

\item Berechnen Sie die Lösung mit Hilfe der Laplace-Transformation.
\end{abc}
}

\Loesung{
\textbf{a)} 
Die allgemeine Lösung ist gegeben durch die Lösung des homogenenen Systems und der 
partikulären Lösung
% The solution is given by the solution of the homogeneous system plus a particular solution
$$y(t) = y_h(t) + y_p(t).$$
Für das homogene System betrachten wir die Nullstellen des charakteristischen Polynoms
% For the homogeneous system we consider the zeros of the characteristic polynomial
$$\lambda^2+4=0,$$
welche die komplex Konjugierten sind
% which are the complex conjugates
$$\lambda = \pm 2\operatorname{i}.$$
Daher ist die Lösung des homogenen Systems
% Therefore, the solution of the homogeneous system is
$$y_h(t) = C_1 \cos{2t} + C_2 \sin{2t}.$$
Die beiden Konstanten werden mit den Anfangswerten bestimmt.
% The two constants will be determined with the initial conditions.
Für die partikuläre Lösung machen wir den Ansatz
% For the particular solution we make the ansatz
$$y_p(t) = A_0 + A_1 t,$$
weil die rechte Seite der Differentialgleichung ein lineares Polynom ist.
% because the right hand side of the differential equation is a linear polynomial.
Die beiden Konstanten $A_0$ und $A_1$ werden durch einsetzen in deiGleichung bestimmt.
% The two constants $A_0$ and $A_1$ are determined inserting the solution in the equation.
Da $y'=A_1$ und $y''=0$, erhalten wir
$$4(A_0+A_1t) = t$$
und mit Koeffizientenvergleich gilt
% and the coefficients comparison gives 
$$A_0=0, \qquad A_1 = \frac{1}{4}.$$
Insgesamt ergibt sich die allgemeine Lösung
$$y(t) = \frac{1}{4}t + C_1 \cos{2t} + C_2 \sin{2t}.$$
Aus den Anfangswerten erhalten wir
$$y(0) = C_1 = 1$$
und 
$$y'(0)= \frac{1}{4} + 2C_2 \stackrel{!}{=} 2 \rightarrow C_2 = \frac{7}{8}.$$
Die spezielle Lösung ist dann
% The solution is
$$y(t) = \frac{1}{4}t+\cos{2t} + \frac{7}{8}\sin{2t}.$$
\noindent
\textbf{b)}
Mit der Laplace-Transformation gilt
% With the Laplace transform it is
$$s^2Y(s) -s-2+4Y(s) = \frac{1}{s^2}$$
$$(s^2+4)Y(s) = \frac{1}{s^2}+s+2$$
$$Y(s) = \frac{s^3+2s^2+1}{s^2(s^2+4)}$$
Durchführen einer Partialbruchzerlegung:
$$\frac{s^3+2s^2+1}{s^2(s^2+4)}= \frac{A}{s}+ \frac{B}{s^2} + \frac{Cs+D}{s^2+4}$$
Multiplikation beider Seiten mit $s^2(s^2+4)$ ergibt
$$s^3+2s^2+1= As(s^2+4)+ B(s^2+4) + (Cs+D)s^2.$$
Der Wert $s=0$ führt zu $B=\dfrac{1}{4}$.
$$s^3+2s^2+1= As^3+4As+ \frac{1}{4}s^2+1 + Cs^3+Ds^2$$
Mit Koeffizientenvergleich erhalten wir
\begin{align*}
A&=0\\
C&=1\\
D+\frac{1}{4}&=2 \quad \rightarrow \quad D=\frac{7}{4}
\end{align*}
$$Y(s) = \frac{1}{4}\cL^{-1}\left\{ \frac{1}{s^2} \right\} + \cL^{-1}\left\{ \frac{s}{s^2+2^2} \right\} + \frac{7}{4}\frac{1}{2}\cL^{-1}\left\{ \frac{2}{s^2+2^2} \right\}$$
$$y(t) = \frac{1}{4}t + \cos{2t} + \frac{7}{8}\sin{2t}.$$

}

%\newcounter{AufglaplacldglKlLp001}
%\setcounter{AufglaplacldglKlLp001}{\theAufg}
%\Ergebnis{\subsubsection*{Ergebnisse zu Aufgabe \arabic{Blatt}.\arabic{AufglaplacldglKlLp001}:}
%
%}


\ErgebnisC{laplacldglKIlp004}{
\begin{abc}
\item $y(t)= \frac{1}{4}t+\cos{2t} + \frac{7}{8}\sin{2t}$
\item $y(t)= \frac{1}{4}t + \cos{2t} + \frac{7}{8}\sin{2t}$
\end{abc}
}
