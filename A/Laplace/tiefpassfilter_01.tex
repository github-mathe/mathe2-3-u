\Aufgabe[e]{Tiefpassfilter}
{\label{aufgabe:Tiefpassfilter}
Ein elektrischer Tiefpassfilter erster Ordnung besteht aus einem Widerstand $R$ und einem Kondensator $C$, die in Reihe geschaltet sind. Die Ausgangsspannung $u_\text{out}(t)$ wird über dem Kondensator abgegriffen. Die Eingangsspannung ist eine Funktion der Zeit.

\begin{center}
\begin{circuitikz}
    \draw
    (0,0) to[sV, l=$u_\text{in}(t)$] (0,3)
          to[R, l=$R$] (3,3)
          to[C, l=$C$] (3,0) -- (0,0)
    (3,3) -- (4,3)
          to[open, v^=$u_\text{out}(t)$] (4,0) -- (3,0);
\end{circuitikz}
\end{center}

Die Beziehung zwischen Eingangs- und Ausgangsspannung ergibt sich durch folgende Differentialgleichung:
\[
RC\, \frac{d u_\text{out}(t)}{dt} + u_\text{out}(t) = u_\text{in}(t)
\]

Die \textbf{Übertragungsfunktion} eines Systems beschreibt den Zusammenhang zwischen Ein- und Ausgang im Laplace-Bildbereich und ist definiert als
\[
H(s) = \frac{U_\text{out}(s)}{U_\text{in}(s)}.
\]

\begin{abc}
    \item Bestimme die Übertragungsfunktion $H(s)$ dieses Systems mithilfe der Laplace-Transformation unter der Annahme von Null-Anfangsbedingungen.

    \item Untersuche das Frequenzverhalten des Filters, indem Sie den Betrag der Übertragungsfunktion für $s = \mathrm{j}\omega$ berechnest. Bestimme, für welche Werte von $\omega$ das Eingangssignal nahezu ungedämpft übertragen wird und bei welchen Frequenzen es stark abgeschwächt wird.
\end{abc}
}

\Loesung{
\begin{abc}
    \item Laplace-Transformation der Differentialgleichung unter Null-Anfangsbedingungen ergibt:
    \[
    RC\, s\, U_\text{out}(s) + U_\text{out}(s) = U_\text{in}(s)
    \Rightarrow U_\text{out}(s)\, (1 + sRC) = U_\text{in}(s)
    \]
    Also ist die Übertragungsfunktion:
    \[
    H(s) = \frac{U_\text{out}(s)}{U_\text{in}(s)} = \frac{1}{1 + sRC}
    \]

    \item Setze $s = \mathrm{j} \omega$, um das Frequenzverhalten zu analysieren:
    \[
    H(\mathrm{j}\omega) = \frac{1}{1 + \mathrm{j} \omega RC}, \quad |H(\mathrm{j}\omega)| = \frac{1}{\sqrt{1 + (\omega RC)^2}}
    \]
    Interpretation:  
    Für kleine $\omega$: $|H| \approx 1 \rightarrow$ niedrige Frequenzen werden nahezu ungehindert durchgelassen.  
    Für große $\omega$: $|H| \to 0 \rightarrow$ hohe Frequenzen werden stark abgeschwächt.
\end{abc}
}

\ErgebnisC{Tiefpassfilter}{
\[
H(s) = \frac{1}{1 + sRC}, \quad
|H(\mathrm{j} \omega)| = \frac{1}{\sqrt{1 + (\omega RC)^2}}
\]
}
