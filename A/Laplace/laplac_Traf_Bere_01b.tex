\Aufgabe[e]{Laplace-Transformierte}
{
%\begin{abc}
%\item 
Berechnen Sie die Laplace-Transformierte 
$$\mathcal{L}\{f(t)\}=F(s)=\int\limits_{t=0}^\infty \EH{-st}\cdot f(t)\, \mathrm{d}  t$$ der folgenden Funktionen: 
\begin{iii}
\begin{multicols}{2}
\item $f(t) = 1$,
\item $f(t) = t$,
\item $f(t) = t^2$,
\item $f(t) = t^3$,
\item $f(t) = e^{-at}$,
\item $f(t) = e^{-at}\cdot t$,
\item ${\mathcal L}\{g'(t)\}$, f\"ur eine allgemeine (gegebene) Funktion $g(t)$,
\item ${\mathcal L}\{g''(t)\}$, f\"ur eine allgemeine (gegebene) Funktion $g(t)$.
\end{multicols}
\end{iii}
\textbf{Hinweis}: F\"ur die letzten beiden Aufgaben kann die Laplacetransformierte $G(s)=\sL\{g(t)\}$ der Funktion $g(t)$ als bekannt vorausgesetzt werden. 

%\item Berechnen Sie die Laplace-Transformierte von $f(t)=\sqrt{t}$. \\
%\end{abc}

%\textbf{Hinweise}:
%\begin{itemize}
%\item Substituieren Sie $u=\sqrt t$. 
%\item Spalten Sie $u^2$ in $u\cdot u$ auf und integrieren Sie partiell. 
%\item Das Quadrat des verbleibenden Integrals k\"onnen Sie l\"osen, indem Sie Polarkoordinaten einführen.
%\end{itemize}
}

\Loesung{
\begin{abc}
\item Diese Laplace-Transformationen müssen durch Anwendung der Definition und Bestimmung der Integrale berechnet werden. Es wird in der Regel partiell integriert. 
\begin{iii}
\item \begin{align*}
\mathcal{L}\{1\} =& \int\limits_{t=0}^\infty \EH{-st}1 \, \mathrm{d}  t
= \left[\frac{\EH{-st}}{-s}\right]_{t=0}^\infty\\
=&  \dfrac{1}{s}
\end{align*}
\item \begin{align*}
\mathcal{L}\{t\} =& \int\limits_{t=0}^\infty \EH{-st}t \, \mathrm{d} t
= \left[t\frac{\EH{-st}}{-s}\right]_{t=0}^\infty+\frac 1 s\int\limits_{t=0}^\infty \EH{-st}\, \mathrm{d} t\\
=& 0 - \left.\frac{\EH{-st}}{-s^2}\right|_{t=0}^\infty
= \dfrac{1}{s^2}
\end{align*}
\item \begin{align*}
\mathcal{L}\{t^2\}  =& \int\limits_{t=0}^\infty t^2\EH{-st} \, \mathrm{d} t
= \left[t^2\frac{\EH{-st}}{-s}\right]_{t=0}^\infty+\frac 1 s \int\limits_{t=0}^\infty 2t\EH{-st}\, \mathrm{d} t\\
=& 0 + \left[\frac{2t\EH{-st}}{-s^2}\right]_{t=0}^\infty + \frac{2}{s^2}\int\limits_{t=0}^\infty \EH{-st}\, \mathrm{d} t\\
=&  0 +\left.\frac{2\EH{-st}}{-s^3}\right|_{t=0}^\infty= \dfrac{2}{s^3}
\end{align*}

\item \begin{align*}
\mathcal{L}\{t^3\}  =& \int\limits_{t=0}^\infty t^3\EH{-st} \, \mathrm{d} t
= \left[t^3\frac{\EH{-st}}{-s}\right]_{t=0}^\infty+\frac 1 s \int\limits_{t=0}^\infty 3t^2\EH{-st}\, \mathrm{d} t\\
=& 0 + \left[\frac{3t^2\EH{-st}}{-s^2}\right]_{t=0}^\infty + \frac{3}{s^2}\int\limits_{t=0}^\infty 2t\EH{-st}\, \mathrm{d} t\\
=&  0 +\left.\frac{6t\EH{-st}}{-s^3}\right|_{t=0}^\infty+\int\limits_{t=0}^\infty\frac{6\EH{-st}}{s^3}\, \mathrm{d} t\\
=& 0 + \left. \frac{6\EH{-st}}{s^4} \right|_{t=0}^\infty = \frac 6{s^4}
\end{align*}

\item \begin{align*}
\mathcal{L}\{\EH{-at}\} =& \int\limits_{t=0}^\infty \EH{-st}\EH{-at} \, \mathrm{d} t
= \left[\frac{\EH{-(s+a)t}}{-(s+a)}\right]_{t=0}^\infty\\
=&  \dfrac{1}{s+a}
\end{align*}

\item \begin{align*}
\mathcal{L}\{t\EH{-at}\} =& \int\limits_{t=0}^\infty t\EH{-at}\EH{-st} \, \mathrm{d} t
= \left[\frac{t\EH{-(s+a)t}}{-(s+a)}\right]_{t=0}^\infty+\frac 1{s+a}\int\limits_{t=0}^\infty \EH{-(s+a)t} \, \mathrm{d} t \\
=& 0 + \left.\frac{\EH{-(s+a)t}}{-(s+a)^2}\right|_{t=0}^\infty
=  \dfrac{1}{(s+a)^2}
\end{align*}

\item \begin{align*}
\mathcal{L}\{g'(t)\} =& \int\limits_{t=0}^\infty \EH{-st}g'(t) \, \mathrm{d} t\\
=& \left[\EH{-st}g(t)\right]_{t=0}^\infty - \int\limits_{t=0}^\infty(-s\EH{-st}g(t)) \, \mathrm{d} t\\
=& 0-g(0)+s\mathcal{L}\{g(t)\}
= sG(s)-g(0)
\end{align*}

\item \begin{align*}
\mathcal{L}\{g''(t)\} =& \int\limits_{t=0}^\infty \EH{-st}g''(t) \, \mathrm{d} t\\
=&\left[\EH{-st}g'(t)\right]_{t=0}^\infty - \int\limits_{t=0}^\infty(-s\EH{-st}g'(t)) \, \mathrm{d} t\\
=& 0-g'(0)+s\mathcal{L}\{g'(t)\}\\
=& s(sG(s)-g(0))-g'(0) = s^2G(s)-g'(0)-sg(0)
\end{align*}

\end{iii}

%\item Mit der Substitution \ $u=\sqrt{t} \ \Rightarrow \ \text dt=2\,u\,\text du$ \ erhält man
%	\[
%	F(s) = \int\limits_{0}^{\infty} \EH{-s\,u^2}\cdot u\cdot 2\,u\ \text du = -\dfrac{1}{s}\cdot\int\limits_{0}^\infty  u\cdot(-2s u)\,\EH{-s\,u^2}\ \text du\ .
%\]
%
%Partielle Integration ergibt
%	\begin{align*}
%	F(s) & = -\dfrac{1}{s}\left(u\EH{-s\,u^2}\Big|_0^\infty - \int\limits_{0}^\infty  
%\EH{-s\,u^2}\ \text du\right)\\[1ex]
%& =-\dfrac{1}{s}\left(0-0 - \int\limits_{0}^\infty  
%\EH{-s\,u^2}\ \text du\right)\\[1ex]
%&  = \dfrac{1}{s}\cdot\int\limits_{0}^\infty  \EH{-s\,u^2}\ \text du  \ .
%\end{align*}
%
%Durch Quadrieren der Gleichung  erh\"alt man
%  \[
%  \big(F(s)\big)^2 \ = \ \dfrac{1}{s}\,\int\limits_{0}^\infty \EH{-s\,x^2}\ \text dx 
%                   \cdot \dfrac{1}{s}\,\int\limits_{0}^\infty \EH{-s\,y^2}\ \text dy \ = \ 
%                   \dfrac{1}{s^2}\,\int\limits_{0}^\infty  \int\limits_{0}^\infty  \EH{-s\,(x^2+y^2)}\ \text dx\,\text dy 
%\]
%und durch Übergang zu Polarkoordinaten
%  \[
%  \big(F(s)\big)^2 \ = \ \dfrac{1}{s^2}\,\int\limits_{0}^\infty  \int\limits_{0}^{\pi/2} \EH{-s\,r^2}\ r\,\text d\phi\,\text dr \ = \
%         \dfrac{1}{s^2}\cdot\dfrac{\pi}{2}\cdot \left[\frac{-1}{2\,s}\,\EH{-s\,r^2}\right]_0^\infty \ = \ \dfrac{\pi}{4\,s^3} \ .
%\]
%Damit ist 
%	\[
%	 F(s) = {\cal L}\left(\sqrt{t}\right) = \sqrt{\dfrac{\pi}{4\,s^3}} .
%\]
\end{abc}
}



\ErgebnisC{AufglaplacTrafBere01b}
{
%\textbf{a)}
\begin{multicols}{3}
\begin{iii}
\item $\dfrac{1}{s}$,
\item $\dfrac{1}{s^2}$,
\item $\dfrac{2}{s^3}$,
\item $\dfrac{3!}{t^4}$,
\item $\frac{1}{s+a}$,
\item $\dfrac{1}{(s+a)^2}$,
\item $ sG(s) - g(0)$,
\item $ s^2G(s) - sg(0) - g'(0)$.
\end{iii}
\end{multicols}
%\textbf{b)}$F(s)=\sqrt{\frac{\pi}{4s^3}}\,.$
}
