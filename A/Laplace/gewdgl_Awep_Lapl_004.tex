\Aufgabe[]{}{
Gegeben sei das Anfangswertproblem
%$$u''(t)-2u'(t)+u(t)=t+\cos(t)\cdot h(t-\pi)$$
$$u''(t)-2u'(t)+u(t)=\cos(t)\cdot h(t-\pi)$$
mit $u(0)=0$ und $u'(0)=0$. Dabei ist $h(t)$ die Heaviside-Funktion. 
\begin{abc}
\item Zeigen Sie, dass die L\"osung des Anfangswertproblems im Bildbereich der Laplace-Transformation die folgende Gestalt hat:
%$$U(s)=\left( \frac 1{s^2} - \frac{s}{1+s^2}\EH{-s\pi}\right) \frac 1{(s-1)^2}$$
$$U(s)=- \frac{s\, \EH{-s\pi}}{(1+s^2)(s-1)^2}$$
\item Bestimmen Sie die L\"osung der Differentialgleichung $u(t)$ im Urbildbereich.
\end{abc}
}
\Loesung{
\begin{abc}
\item Die Laplace-Transformierte des Anfangswertproblems ist
%$$s^2U(s)-2sU(s)+U(s)=\frac 1{s^2}+\sL\{\cos(t)\cdot h(t-\pi)\}.$$
$$s^2U(s)-2sU(s)+U(s)=\sL\{\cos(t)\cdot h(t-\pi)\}.$$
F\"ur die Transformation des letzten Terms wird der Verschiebungssatz angewendet:
\begin{align*}
\sL\{\cos(t)\cdot h(t-\pi)\}=&\sL\{\cos(t-\pi+\pi)\cdot h(t-\pi)\}=\sL\{\cos(t+\pi)\}\EH{-s\pi}\\
=& \sL\{-\cos(t)\}\EH{-s\pi}=\frac{-s}{1+s^2}\EH{-s\pi}.
\end{align*}
Damit ist die transformierte Differentialgleichung:
%$$U(s)(s^2-2s+1)=\frac 1{s^2}-\frac{s}{1+s^2}\EH{-s\pi}.$$
$$U(s)(s^2-2s+1)=-\frac{s}{1+s^2}\EH{-s\pi}.$$
Die L\"osung im Bildbereich ist
%$$U(s)=\frac 1{(s-1)^2}\left( \frac 1{s^2}-\frac{s}{1+s^2}\EH{-s\pi}\right).$$
$$U(s)=-\frac{s}{(1+s^2)(s-1)^2}\EH{-s\pi}.$$
\item Die R\"ucktransformation ergibt
%\begin{align*}
%u(t)=&\sL^{-1}\left\{\frac 1{(s-1)^2s^2}\right\} - \sL^{-1}\left\{\frac{s}{(s-1)^2(1+s^2)}\EH{-s\pi}\right\}\\
%\end{align*}
\begin{align*}
u(t)=& - \sL^{-1}\left\{\frac{s}{(s-1)^2(1+s^2)}\EH{-s\pi}\right\}
\end{align*}
%Die Partialbruchzerlegung der beiden Terme ergibt
%\begin{align*}
%&&\frac 1{(s-1)^2s^2}=& \frac{A  }{s-1} + \frac{ B }{(s-1)^2} + \frac{C  }{s}+\frac{D  }{s^2}\\
%&&=& \frac{(A+C)s^3+(-A+B-2C+D)s^2+(C-2D)s+D}{(s-1)^2s^2}\\
%\Rightarrow&&A+C=&0,\, -A+B-2C+D=0,\, C-2D=0,\, D=1\\
%\Rightarrow&&D=&1,\, C=2,\, A=-2,\, B=A+2C-D=1
%\end{align*}
%und
Die Partialbruchzerlegung des Bruches ergibt
\begin{align*}
&&&\frac{s}{(s-1)^2(1+s^2)}\\
&&=& \frac{E}{s-1}+\frac{F}{(s-1)^2}+\frac{G+Hs}{1+s^2}\\
&&=& \frac{(E+H)s^3+(-E+F+G-2H)s^2+(E-2G+H)s-E+F+G}{(s-1)^2(1+s^2)}\\
\Rightarrow&&E+H=&0,\, -E+F+G-2H=0,\, E-2G+H=1,\, -E+F+G=0\\
\Rightarrow&&H=&0,\, E=0,\, G=-\frac 12,\, F=\frac 12
\end{align*}
Damit ist dann
\begin{align*}
u(t)=&%\sL^{-1}\left\{ \frac{-2}{s-1}+\frac 1{(s-1)^2}+\frac 2{s}+\frac 1{s^2}\right\} + 
-\frac 12 \sL^{-1}\left\{ \left( \frac 1 {(s-1)^2}-\frac 1{1+s^2}\right)\EH{-\pi s}\right\}\\
=& %-2\EH t + t\EH t+2 +t + 
-\frac 12 \left[t\EH t-\sin(t)\right]_{t\leftarrow t-\pi}h(t-\pi)\qquad\text{ (Verschiebungssatz)}\\
=&% (t-2)\EH t + 2 + t +
 -\frac 12\left[ (t-\pi)\EH{t-\pi}-\sin(t-\pi)\right] h(t-\pi)\\
=& %(t-2)\EH t + 2+t + 
-\frac 12\left[ (t-\pi)\EH{t-\pi}+\sin(t)\right]h(t-\pi)\\
=&\left\{\begin{array}{ll}
0&\text{ f\"ur } t\leq \pi\\
-\frac{1}2\left[(t-\pi)\EH{t-\pi}+\sin(t)\right]&\text{ f\"ur } t>\pi.
\end{array}\right.
\end{align*}

\end{abc}
}


\ErgebnisC{gewdglAwepLapl004}
{
\begin{abc}
\item /
\item $u(t) = -\frac 12\left[ (t-\pi)\EH{t-\pi}+\sin(t)\right]h(t-\pi)$
\end{abc}
}

