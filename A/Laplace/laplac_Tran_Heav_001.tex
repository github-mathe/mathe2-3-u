\Aufgabe[e]{Heaviside-Funktion}
{
Gesucht ist die Laplace--Transformierte von
\begin{iii}
\item   $g(t) := h(t-2) \cdot (t-2)^2\, $
\item	$f(t) := h(t-2)\cdot t^2 \ , $
\end{iii}
wobei \ $h$ \ die Heaviside--Funktion ist.

\begin{abc}
\item Mit Hilfe der Integraldarstellung der Definition.

\item Mit Hilfe des Verschiebungssatzes und der Tabelle der Laplace--Transformierten.
\end{abc}
}

\Loesung{
\begin{iii}
\item 
\textbf{a)} Mit partieller Integration ergibt sich: \\
	\[
	F(s) = \int\limits_{2}^{\infty} \EH{-s\,t}\,(t-2)^2\ \text dt = \left[-\left(\frac{(t-2)^2}{s}+\frac{2\,(t-2)}{s^2}+\frac{2}{s^3}\right)\cdot\EH{-s\,t}\right]_2^\infty = \frac{2}{s^3}\cdot\EH{-2\,s}\ .
\]
\noindent
\textbf{b)} \ Es wird der Verschiebungssatz angewendet
	\[
	{\cal{L}}\Big( f(t-a)\cdot h(t-a)\Big)= F(s)\cdot\EH{-a\,s}\ ,\ \ a>0\ ,
	\]
Damit erhält man die Laplace--Transformierte
	\[
	{\cal{L}}\Big( (t-2)^2\cdot h(t-2)\Big) = {\cal{L}}\Big( t^2 \Big) \cdot\EH{-a\,s} = \frac{2}{s^3}\cdot\EH{-2\,s}
\]


\item 
\textbf{a)} Mit partieller Integration ergibt sich: \\
	\[
	F(s) = \int\limits_{2}^{\infty} \EH{-s\,t}\,t^2\ \text dt = \left[-\left(\frac{t^2}{s}+\frac{2\,t}{s^2}+\frac{2}{s^3}\right)\cdot\EH{-s\,t}\right]_2^\infty = \left(\frac{4}{s}+\frac{4}{s^2}+\frac{2}{s^3}\right)\cdot\EH{-2\,s}\ .
\]
\noindent
\textbf{b)} \ Für den Verschiebungssatz
	\[
	{\cal{L}}\Big( f(t-a)\cdot h(t-a)\Big) = F(s)\cdot\EH{-a\,s}\ ,\ \ a>0\ ,
\]
 muss die Funktion erst umgeschrieben werden:
	\[
	t^2 = (t-2)^2+4(t-2)+4 \ .
\]
Damit erhält man die Laplace--Transformierte
	\[
	{\cal{L}}\Big(\big( (t-2)^2+4(t-2)+4\big)\cdot h(t-2)\Big) = \left(\frac{2}{s^3}+\frac{4}{s^2}+\frac{4}{s}\right)\cdot\EH{-2\,s}
\]
\end{iii}
}

\ErgebnisC{laplacTranHeav001}{
$F(s)= \left(\frac{2}{s^3}+\frac{4}{s^2}+\frac{4}{s}\right)\cdot\EH{-2\,s}$
}
