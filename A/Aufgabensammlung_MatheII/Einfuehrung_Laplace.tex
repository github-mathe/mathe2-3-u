$f:[0,\infty) \rightarrow \mathbb{R}$ hei\ss t Laplace-transformierbar, wenn das Integral 
$$
F(s) := \mathcal{L}{f(t)} := \int_0^{\infty} \e^{-st} f(t) \mathrm{d}t \ \ \ f\"ur s \in \mathbb{R}
$$
existiert. Dabei ist $F(s)$ die Bildfunktion (auch Laplace-Transformierte genannt) zur Urbildfunktion $f(t)$. Hilfreich bei der Laplace-Transformation bzw. bei der R\"ucktransformation sind sogenannte Korrespondenztabellen, die zur Urbildfunktion die dazugeh\"orige Bildfunktion angeben. Diese Tabellen befinden sich in der Formelsammlung.