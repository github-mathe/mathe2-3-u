\Aufgabe[e]{Kurvendiskussion}{\label{KurvDisk002}
\begin{abc}
\item  Gegeben sei die Funktion
$$	f(x) = \dfrac{x^2+3\,x}{1-x}.$$
\begin{iii}
	\item Geben Sie den maximalen Definitionsbereich der Funktion \ $f$ \ an.
	
	\item Bestimmen Sie die Nullstellen der Funktion.
	
	\item Bestimmen Sie die kritischen Punkte für die Extrema der Funktion und deren Funktionswerte.
	
	\item Bestimmen Sie die (nicht vertikale) Asymptote der Funktion, d.\,h. diejenige Gerade \ $g(x) = a+b\,x$\, für die 
$$	\lim\limits_{x\rightarrow\pm\infty} \big(f(x)-g(x)\big)= 0$$
ist.

  \item Skizzieren Sie die Funktion.
\end{iii}
\item Gegeben sei die Funktion
$$	g\,:\,\R\rightarrow\R\,:\,x\rightarrow \big(x^2-4\big)^2\cdot\EH{-x} .$$
Bestimmen Sie alle relativen Minima und Maxima der Funktion \ $g$ \ \textbf{ohne} die zweite Ableitung zu berechnen.
\end{abc}
}

\Loesung{
\begin{abc}
\item
\begin{iii}
	\item Der maximale Definitionsbereich ist \ $\mathcal D = \R\backslash\left\{1\right\}$\,.
	
	\item Die Nullstellen der Funktion sind \ $x_{\text N_1} = -3 \text{ und } x_{\text N_2} = 0$\,.
	
	\item Die kritischen Punkte sind die Nullstellen der ersten Ableitung. Aus
$$	f'(x) = \frac{(2x+3)(1-x)+(x^2+3x)}{(1-x)^2}= \dfrac{-x^2+2\,x+3}{(1-x)^2}=0  $$
  folgt
$$
  x_{\text K_1}= -1\text{ mit } f(-1)=-1 \text{ und } x_{\text K_2}= 3\text{ mit } f(3)=-9\ .
$$
  	
	\item Aus  $f(x) = -x-4+\dfrac{4}{1-x}$  folgt, dass  $g(x)=-x-4$  die Asymptote ist.
\item \quad\\
\end{iii}
\end{abc}
\begin{center}	
\psset{unit=0.5cm}
\begin{pspicture}(-5,-16)(7,6)

\psgrid[griddots=8,subgriddiv=0](-5,-16)(7,6)
\psline[linestyle=dashed](-5,1)(7,-11)
\psline[linestyle=dashed](1,-16)(1,6)

\pscircle*(-3,0){0.1}
\pscircle*(-1,-1){0.1}
\pscircle*(-0,0){0.1}
\pscircle*(3,-9){0.1}
\psplot[plotpoints=100, plotstyle=curve]
{-5}{.63}
{
x x mul 3 x mul add 1 x neg add div
}
\psplot[plotpoints=100, plotstyle=curve]
{1.37}{7}
{
x x mul 3 x mul add 1 x neg add div
}

\end{pspicture}

\end{center}

Bei \ $(-1,-1)$ \ handelt es sich also um ein (lokales) Minimum, bei \ $(3,-9)$ \ um ein Maximum.
\begin{abc}\setcounter{enumi}{1}
\item Da die Funktion nirgends negativ ist, sind die Nullstellen automatisch lokale Minima: \ $x_{\text {min}_{1,2}} = \pm 2$\,. 

Die Nullstellen der ersten Ableitung sind die kritischen Punkte. Aus
$$
	g'(x) = 4x\,(x^2-4)\,\EH{-x}-(x^2-4)^2\,\EH{-x} = (-x^2+4x+4)(x^2-4)\,\EH{-x}=0  
$$
folgt
$$
	x_{1,2} =\pm 2 \text{ und } x_{3,4} = 2\pm\sqrt 8.
$$
Die ersten beiden kritischen Punkte sind die schon bekannten Nullstellen und die beiden anderen sind
Maxima, da ein einfacher kritischer Punkt zwischen zwei Minima nur ein Maximum sein kann und die
Funktion für \ $x\rightarrow\infty$ \ gegen \ 0 \ geht.
\end{abc}
}

\ErgebnisC{AufganalysKurvDisk002}
{
\textbf{a)}\textbf{ii)} $0,\, -3$, 
\textbf{iii)} -1,\, 3, 
\textbf{iv)} $g(x)=-x-4$
}
