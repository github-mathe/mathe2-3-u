\Aufgabe[e]{Kurvendiskussion}{\label{KurvDisk002}

Gegeben sei die Funktion
$$	f(x) = \dfrac{x^2+3\,x}{1-x}.$$
\begin{iii}
	\item Geben Sie den maximalen Definitionsbereich der Funktion \ $f$ \ an.
	
	\item Bestimmen Sie die Nullstellen der Funktion.
	
	\item Bestimmen Sie die kritischen Punkte der Funktion und deren Funktionswerte. Klassifizieren Sie alle kritischen Punkte als Minimum, Maximum oder Wendepunkt.
	
	\item Untersuchen Sie das Monotonieverhalten der Funktion. 
	
	\item Bestimmen Sie alle Asymptoten der Funktion.

	\item Bestimmen Sie den Wertebereich der Funktion.

	\item Skizzieren Sie die Funktion.
\end{iii}
}

\Loesung{
\begin{abc}
\item
\begin{iii}
	\item Der maximale Definitionsbereich ist \ $\mathcal D = \R\backslash\left\{1\right\}$\,.
	
	\item Die Nullstellen der Funktion sind \ $x_{N_1} = -3 \text{ und } x_{N_2} = 0$ \,.
	
	\item Die kritischen Punkte sind die Nullstellen der ersten Ableitung. Aus
$$	f'(x) = \frac{(2x+3)(1-x)+(x^2+3x)}{(1-x)^2} = \dfrac{-x^2+2\,x+3}{(1-x)^2}=0  $$
  folgt
$$
  x_{K_1}= -1\text{ mit } f(-1)=-1 \text{ und } x_{K_2}= 3\text{ mit } f(3)=-9\ .
$$
  	Die zweite Ableitung ist:
  	$$
		f''(x) = \frac{8}{(1-x)^3}.
  	$$
  	Für $x_{K_1}= -1$ ist $f''(-1)=1 > 0$. Es handelt sich also um ein Minimum.
  	Für $x_{K_2}= 3$ ist $f''(3)=-1 < 0$. Es handelt sich also um ein Maximum.
  	
	\item
	Aus den stationären Punkten un der Definitionslücke ergeben sich die Monotonieintervalle.
	In $(-\infty, -1)$ und $(3,\infty)$ ist die Funktion monoton fallend. In $(-1,1)$ und 
	$(1,3)$ ist die Funktion monoton steigend. 
	\item Aus  $f(x) = -x-4+\dfrac{4}{1-x}$  folgt, dass  $g(x)=-x-4$  die Asymptote ist.
	Wegen $\lim_{x \to 1^-} f(x) = \infty$ und $\lim_{x \to 1^+} f(x) = -\infty$ gibt es 
	eine senkrechte Asymptote bei $x=1$.
	\item Der Wertebereich ist $\mathcal{W} = (-\infty,-9] \cup [-1,\infty)$. 
	\item \quad\\
\end{iii}
\end{abc}
\begin{center}	
\psset{unit=0.5cm}
\begin{pspicture}(-5,-16)(7,6)

\psgrid[griddots=8,subgriddiv=0](-5,-16)(7,6)
\psline[linestyle=dashed](-5,1)(7,-11)
\psline[linestyle=dashed](1,-16)(1,6)

\pscircle*(-3,0){0.1}
\pscircle*(-1,-1){0.1}
\pscircle*(-0,0){0.1}
\pscircle*(3,-9){0.1}
\psplot[plotpoints=100, plotstyle=curve]
{-5}{.63}
{
x x mul 3 x mul add 1 x neg add div
}
\psplot[plotpoints=100, plotstyle=curve]
{1.37}{7}
{
x x mul 3 x mul add 1 x neg add div
}

\end{pspicture}

\end{center}

Bei \ $(-1,-1)$ \ handelt es sich also um ein (lokales) Minimum, bei \ $(3,-9)$ \ um ein Maximum.
}

\ErgebnisC{AufganalysKurvDisk002}
{
\textbf{a)}\textbf{ii)} $0,\, -3$, 
\textbf{iii)} -1,\, 3, 
\textbf{iv)} $g(x)=-x-4$
}
