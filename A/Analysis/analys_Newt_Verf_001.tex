\Aufgabe[e]{Newton-Verfahren}{
\begin{abc}
\item Gegeben seien die Funktionen 
$$f(x)=\frac x3\text{  und }g(x)=\sin(x^2).$$
\begin{iii}
\item Skizzieren Sie die Funktionen und bestimmen Sie N\"aherungen f\"ur die Schnittstellen der beiden Funktionsgraphen. 
\item Bestimmen Sie die kleinste positive Schnittstelle mit dem Newton-Verfahren auf f\"unf
Nachkommastellen genau. Hierf{\"u}r ist die Verwendung eines Taschenrechners erlaubt.
\end{iii}
\item F\"uhren Sie das Verfahren ebenso f\"ur die Funktionen 
$$f(x)=x^3\,\text{ und }\, g(x)=\cos(2\pi x)$$
und die betragskleinste Schnittstelle durch. 
\end{abc}
}

\Loesung{
{ a)}\\
\begin{center}
\psset{xunit=2cm, yunit=2cm, runit=1cm}
\begin{pspicture}(-3,-1)(3,1)
\psgrid[subgriddiv=5,griddots=1,gridlabels=.3](-3,-1)(3,1)
\psplot[plotpoints=200, plotstyle=curve]
{-3}{3}
{x .3333 mul}

\psplot[plotpoints=200, plotstyle=curve]
{-3}{3}
{x x mul 57.296 mul sin}

\put(-1.4,-.8){$f(x)$}
\put(.8,1){$g(x)$}
\end{pspicture}
\end{center}

\begin{iii}
\item Aus der Skizze kann man die ungef\"ahren Schnittpunkte
$$(-2.3,-0.8),\, (-2,-0.7),\, (0,0), \, (0.3,0.1),\, (1.6,0.5), \, (2.7,0.9),\, (2.9,0.9)$$
ablesen. 
\item Die kleinste positive Schnittstelle $z$ liegt im Intervall  $[0.3,\, 0.4]$. Sie ist Nullstelle
der Funktion $F(x)=f(x)-g(x)$ mit 
$$F'(x)=\frac 13 - 2x \cos(x^2).$$
Die Iterationsvorschrift f\"ur das Newton-Verfahren lautet: 
$$x_{i+1}=x_i-\frac{F(x_{i})}{F'(x_i)}.$$
Mit $x_0=0.35$ liefert sie
\end{iii}
\begin{center}
\begin{tabular}{r|r|r|r|r}
 \multicolumn{1}{c|}{$n$}&  \multicolumn{1}{c|}{$x_n$}&  \multicolumn{1}{c|}{$F(x_n)$}&  \multicolumn{1}{c|}{$F'(x_n)$}&  \multicolumn{1}{c}{$F(x_n)/F'(x_n)$}\\\hline
   0&    0.3500000&  -0.0055272&  -0.3614210&   0.0152929\\
   1&    0.3347071&  -0.0002256&  -0.3318845&   0.0006798\\
   2&    0.3340273&  -0.0000004&  -0.3305673&   0.0000014\\
   3&    0.3340259&   0.0000000&  -0.3305647&  -0.0000000\\
\end{tabular}
\end{center}
Bereits nach drei Schritten findet keine Korrektur der ersten sechs Nachkommastellen mehr statt, der
gesuchte Schnittpunkt liegt also bei 
$z\approx0.33403.$\\

{ b)}\\
\begin{center}
\psset{xunit=2cm, yunit=2cm, runit=1cm}
\begin{pspicture}(-1,-1)(1,1)
\psgrid[subgriddiv=5,griddots=1,gridlabels=.3](-1,-1)(1,1)
\psplot[plotpoints=200, plotstyle=curve]
{-1}{1}
{x x mul x mul}

\psplot[plotpoints=200, plotstyle=curve]
{-1}{1}
{360 x mul cos}
\put(-1.7,1.7){$g(x)$}
\put(.7,.8){$f(x)$}
\end{pspicture}
\end{center}

\begin{iii}
\item Aus der Skizze kann man die ungef\"ahren Schnittpunkte
$$(-0.6,-0.4),\, (-0.2,0.0),\, (0.2,0.0), \, (0.9,0.6),\, (1.0,1.0)$$
ablesen. 
\item Die betraglich kleinste Schnittstelle $z$ liegt im Intervall  $[0.2,\, 0.3]$. Sie ist Nullstelle
der Funktion $G(x)=f(x)-g(x)$ mit 
$$G'(x)=3x^2+2\pi \sin(2\pi x).$$
Die Iterationsvorschrift f\"ur das Newton-Verfahren lautet: 
$$x_{i+1}=x_i-\frac{G(x_{i})}{G'(x_i)}.$$
Mit $x_0=0.25$ liefert sie
\end{iii}
\begin{center}
\begin{tabular}{r|r|r|r|r}
 \multicolumn{1}{c|}{$n$}&  \multicolumn{1}{c|}{$x_n$}&  \multicolumn{1}{c|}{$G(x_n)$}&  \multicolumn{1}{c|}{$G'(x_n)$}&  \multicolumn{1}{c}{$G(x_n)/G'(x_n)$}\\\hline
   0&   0.250000& 0.015625 &  6.470685&   0.002415\\
   1&   0.247585& 0.000004 &  6.466358&   0.000001\\
   2&   0.247585& 0.000000 &  6.466356&   0.000000\\
   3&   0.247585& &&
\end{tabular}
\end{center}


F\"ur diesen Fall hat das Verfahren bereits nach zwei Schritten die gew\"unschte Genauigkeit
erreicht und das Ergebnis ist 
$z\approx 0.24759.$\\

}

\ErgebnisC{AufganalysNewtVerf001}
{
Die gesuchten Schnittpunkte liegen bei { a)} $z\approx 0.33403$ und { b)} $z\approx 0.24759$
}


