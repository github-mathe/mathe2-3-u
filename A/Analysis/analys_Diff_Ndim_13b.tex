\Aufgabe[e]{alte Klausuraufgabe}
{
Gegeben sei die Funktion 
$$g: \R^2\rightarrow \R, \quad \text{ mit } g(x,y)=(y-x^2)(y-2x^2). $$
\begin{abc}
\item Zeigen Sie, dass die Funktion $g(x,y)$ auf allen Geraden durch den Ursprung $(0,0)$ lokale Minima hat.
\item Hat die Funktion $g(x,y)$ im Ursprung ein lokales Minimum?\\
\textbf{Hinweis}: Untersuchen Sie die Funktion l\"angs der Kurve $\vec q(t)=(t,3t^2/2)^\top$.

%\item Berechnen Sie das Taylor-Polynom zweiten Grades der Funktion 
%$g(x,y)$ im Punkt $(1,1)$. 
\end{abc}
}
\Loesung{
\begin{abc}
\item Eine Gerade durch den Ursprung kann durch $\vec k(t)=(at,bt)^\top,\quad t\in\R$ parametrisiert werden. 
Der Funktionsverlauf l\"angs dieser Geraden ist dann 
$$\varphi(t)=g(\vec k(t))=(bt-a^2t^2)(bt-2a^2x^2)=2a^4t^4-3a^2bt^3+b^2t^2$$
mit $\varphi'(t)=8a^4t^3-9a^2bt^2+2b^2t$ und $\varphi''(t)=24a^4t^2-18a^2b^2t+2b^2$. \\
Wegen $\varphi'(0)=0$ und $\varphi''(0)=2b^2>0$ liegt f\"ur $b\neq 0$ ein Minimum der Funktion im Ursprung vor. \\
Falls $b=0$ und $a\neq 0$ ist, hat man 
$\varphi(t)=2a^4x^4$. Auch diese Funktion hat im Ursprung ihr Minimum. 
\item N\"ahert man sich dem Ursprung auf der Kurve $\vec q(t)=(t,3t^2/2)^\top,\qquad t\in\R$, hat man 
$$\psi(t)=g(\vec q(t))=\left(\frac{3t^2}2-t^2\right)\left( \frac{3t^2}2 - 2t^2\right)=\frac 12 \cdot \left( -\frac 12\right)t^4=-\frac{t^4}4<0 \text{ f\"ur }t>0.$$
In jeder Umgebung von $(0,0)^\top$ hat man also auch Funktionswerte $g(x,y)<0$. Im Ursprung kann also kein Minimum vorliegen. 
%\item Wir ben\"otigen die ersten beiden Ableitungen von $g(x,y)$: 
%\begin{align*}
%g(x,y)=& (y-x^2)(y-2x^2)&\Rightarrow&&g(1,1)=&0\\
%\nabla g(x,y)=& \begin{pmatrix}-2x(y-2x^2)-4x(y-x^2)\\ (y-2x^2)+(y-x^2)\end{pmatrix} &\Rightarrow&& \nabla g(1,1)=& \begin{pmatrix} 2\\-1\end{pmatrix}\\
%\vec H_g(x,y)=& \begin{pmatrix}-2y+12x^2-4y+12x^2 & -6x\\
%-6x& 2\end{pmatrix} &\Rightarrow&& \vec H_g(1,1)=&\begin{pmatrix} 18&-6\\-6&2\end{pmatrix}
%\end{align*}
%Das Taylorpolynom ist damit 
%\begin{align*}
%T_2(x,y)=&g(1,1)+\nabla g(1,1)\cdot \begin{pmatrix}x-1\\y-1\end{pmatrix} + \frac 12 (x-1,y-1)\vec H_g(1,1)\begin{pmatrix}x-1\\y-1\end{pmatrix} \\
%=& 2(x-1)-(y-1) + 9(x-1)^2+(y-1)^2-6(x-1)(y-1).
%\end{align*}
\end{abc}
}

