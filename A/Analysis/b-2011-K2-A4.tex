\Aufgabe[e]{~}{
\begin{itemize}
\item[a)]  Bestimmen Sie drei verschiedene (reelle) Nullstellen der Ableitung der Funktion 
\[
f\,:\,\mathbb{R\longrightarrow R}\,,\, f(x)=\int_{0}^{x^{3}}\text{e}^{t^{4}}\cdot (t^{2}-t-2)\;{d}t\;. 
\]
\textbf{Hinweis}: Das Integral \textbf{nicht} berechnen!
\item[b)] Gegeben seien die Funktionen (Das Integral \textbf{nicht} berechnen!) 
$$
F(x):=\int_0^x\frac{\sinh t^2}{t^2}\,\text{e}^{t^2}\,dt,\qquad G(x):=\text{e}^{-2x^2},\quad x\in \R \,.
$$
Berechnen Sie den Grenzwert $\displaystyle\lim_{x\to\infty}G(x)\cdot F(x)$. \\
\textbf{Hinweis}: Regel von L'Hospital.
\item[c)] Bestimmen Sie die \textbf{reelle} Partialbruchzerlegung von $$\ f(x)=\dfrac{1}{(x-1)(x^2+x+1)}\,.$$
\end{itemize}
}

\Loesung{
\textbf{L\"osung}\\[1ex]
\textbf{Zu a)} Mit dem Hauptsatz der Differential-- und Integralrechnung und der Kettenregel folgt: 
\[
\begin{array}{lll}
\displaystyle \left( \int_{0}^{\;x^{3}}\text{e}^{t^{4}}\cdot (t^{2}-t-2)\;\text{d}t\right)' & = & 
\left( \text{e}^{(x^{3})^{4}}\cdot \left( (x^{3})^{2}-(x^{3})-2\right)
\right) \cdot 3\,x^{2} \\ 
&  &  \\ 
& = & 3\,\text{e}^{x^{12}}\cdot x^{2}\cdot \left( x^{6}-x^{3}-2\right)
\end{array}
\]
Der erste Term wird nie Null, d.h. die Nullstellen sind die Nullstellen der beiden letzten Terme: 
\[
x=0\;\text{(doppelt)}\;,\;\;x=-1\;,\;\;x=\sqrt[3]{2}\;. 
\]


\bigskip
\textbf{Zu b)} Wir betrachten als Erstes 
\[f(t)=\frac{\sinh t^2}{t^2}\,\text{e}^{t^2}= \frac{\left(\text{e}^{t^2}-\text{e}^{-t^2}\right)\cdot \text{e}^{t^2}}{2t^2} = \frac{\text{e}^{t^2} -1}{2t^2}\]
\[\lim_{x\to0} f(t) \overset{\text{l'Hosp.}}{=} \lim_{x\to0} \frac{4t\text{e}^{2t^2}}{4t} = 1\]
\[\lim_{x\to\infty} f(t) \overset{\text{l'Hosp.}}{=} \lim_{x\to\infty} \text{e}^{2t^2} = + \infty\]
Die Funktion \(f(t)\) ist positiv f\"ur \(t\in(0,\infty)\). Dadurch entspricht \(F(x)\) der Fl\"ache zwischen dem Funktionsgraphen von \(f(t)\), der \(t\)-Achse und den Geraden \(t=0\) und \(t=x\).
Da f\"ur \(t\to\infty\) die Funktion \(f(t)\) uneingeschr\"ankt w\"achst, gilt auch f\"ur die Fl\"ache \(\underset{{x\to\infty}}\lim F(x) = \infty\).
\begin{center}
\begin{pspicture}(-2,0)(3,4)
%\psgrid(-2,0)(3,4)
\psline[fillstyle=solid, fillcolor=lightgray, linecolor=lightgray, linewidth=0.0](0,0)(2.2,0)(2.2,1.3064)(0,.1)(0,0)
\psplot[plotpoints=100,plotstyle=curve, fillstyle=solid, fillcolor=white, linecolor=white]
{-.01}{2.2}{
2.7183 x x mul exp 2.7183 x x mul neg exp neg add 20 div x x mul div
}
\psplot[plotpoints=100,plotstyle=curve]
{-.01}{2.5}{
2.7183 x x mul exp 2.7183 x x mul neg exp neg add 20 div x x mul div
}

\psline{->}(-2,0)(3,0)
\psline{->}(0,0)(0,4)
\psline(2.2,-.1)(2.2,.1)
\put(2.7, .1){$t$}
\put(2.2,-.3){$x$}
\put(.1,3.7){$f(t)$}
\put(2.6,0.6){$F(x)$}
\psline(2,.3)(2.6,.7)
\end{pspicture}
\end{center}
Mit dem Grenzwert
\[\lim_{x\to\infty}G(x) = \text{e}^{-2x^2} = \lim_{x\to\infty} \frac{1}{\text{e}^{2x^2}} = 0\]
ist der Ausdruck \(\underset{x\to\infty}\lim G(x)\cdot F(x)= 0 \cdot \infty\) unbestimmt. Wir behandeln ihn mit der Regel von l'Hospital:
\begin{align*}
\lim_{x\to\infty}G(x)\cdot F(x) & =\lim_{x\to\infty}\frac{F(x)}{\frac1{G(x)}}
=\lim_{x\to\infty}\frac{F'(x)}{4x\text{e}^{2x^2}}\\[2ex]
& =\lim_{x\to\infty}\frac{\sinh x^2}{4x^3\text{e}^{x^2}}
= \lim_{x\to\infty}\frac{\frac{1}{2}\,(\text{e}^{x^2}-\text{e}^{-x^2})}{4x^3\text{e}^{x^2}}\\[2ex]
& =\lim_{x\to\infty}\frac{1-\text{e}^{-2x^2}}{8x^3}=0\,.
\end{align*}


\bigskip
\textbf{Zu c)} Der Faktor $x^2+x+1$ hat keine reellen Nullstellen und kann deshalb nicht in Linearfaktoren zerlegt werden (zumindest nicht in $\R$). Deshalb verwendet man den Ansatz
$$ 
f(x)=\dfrac{1}{(x-1)(x^2+x+1)} = \dfrac{A}{x-1} + \dfrac{Bx+C}{x^2+x+1}. 
$$
Durch Multiplikation mit dem Hauptnenner erh\"alt man
\[ 1 = A(x^2+x+1)+(Bx+C)(x-1)\]

Einsetzen der reellen Nullstelle des Hauptnenners in die Gleichung liefert 
f\"ur \(x=1\) die Gleichung \(1 = 3\cdot A \Rightarrow A = \frac13\) und Koeffizientenvergleich f\"ur \(x^2\) liefert \(0 = A+B \Rightarrow B = -\frac13\). Nun w\"ahlen wir noch \(x = 0\) und erhalten \( 1=A-C  \Rightarrow C =A-1=-\frac23\,.\)\\

Damit ist
$f(x)= \dfrac{1}{3(x-1)}-\dfrac{x+2}{3(x^2+x+1)}$. 

}


% \ErgebnisC{b-2013-K-A1}{
% 
% }



