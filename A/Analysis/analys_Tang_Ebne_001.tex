\Aufgabe[e]{\"Aquipotentialfl\"ache und Tangentialebene}{
Gegeben seien die Funktion $f(x,y,z)=y^2-xz$ und der Punkt $\vec p_0=(1,\, -2,\, 3)^\top$. 
\begin{abc}
\item Bestimmen Sie die \"Aquipotentialfl\"ache der Funktion $f$ durch den Punkt $\vec p_0$:
$$\vec F= \{\vec x\in\R^3|\quad f(x,y,z)=f(\vec p_0)\}.$$
Skizzieren Sie die Schnitte der Fl\"ache $\vec F$ mit zur $xy$-Ebene parallelen Ebenen, d. h. f\"ur
konstante $z$-Werte. (z. B. $z=0\pm,\, 1\pm,\, 2\pm,\, 3\pm$ und $z\to\infty$)
\item Bestimmen Sie $\nabla f(\vec p_0)$. 
%\item Bestimmen Sie die Hessesche Normalform der Tangentialebene $\vec E$ an $\vec F$ im Punkt $\vec p_0$. 
\item Bestimmen Sie die Punkt-Normalen-Form der Tangentialebene $\vec E$ an $\vec F$ im Punkt $\vec p_0$. 
\item Bestimmen Sie den Abstand der Tangentialebene $\vec E$ zum kritischen Punkt der Funktion $f$. 
\end{abc}
}

\Loesung{
\begin{abc}
\item Mit $f(\vec p_0)=1$ ergibt sich die \"Aquipotentialfl\"ache zu 
$$\vec F = \left\{ \vec x\in\R^3\left| \quad y^2-xz=1\right.\right\}.$$
F\"ur konstante $z$-Werte ergeben sich f\"ur die Schnittkurven Parabeln ($z\neq 0$): 
$$x=\frac{y^2-1}z,$$
w\"ahrend sich f\"ur $z=0$ die Geraden $y=\pm 1$ ergeben:\\
\end{abc}

\begin{center}
\psset{xunit=1cm, yunit=1cm, runit=1cm}
\begin{pspicture}(-3,-3)(3,3)
\psgrid[subgriddiv=1,griddots=10,gridlabels=.3](-3,-3)(3,3)
\psline{-}(-3,-1)(3,-1)
\psline{-}(-3,1)(3,1)
\psrotateright{
\psplot[plotpoints=200, plotstyle=curve]
{-2}{2}
{x x mul -1 add}

\psplot[plotpoints=200, plotstyle=curve]
{-2.65}{2.65}
{x x mul -1 add .5 mul}

\psplot[plotpoints=200, plotstyle=curve]
{-3.16}{3.16}
{x x mul -1 add .333 mul}
}
\psrotateleft{
\psplot[plotpoints=200, plotstyle=curve]
{-2}{2}
{x x mul -1 add}

\psplot[plotpoints=200, plotstyle=curve]
{-2.65}{2.65}
{x x mul -1 add .5 mul}

\psplot[plotpoints=200, plotstyle=curve]
{-3.16}{3.16}
{x x mul -1 add .333 mul}
}

\psline[linecolor=black](0,-3)(0,3)

\put(2.1,.7){$z=0$}
\put(2.1,-.9){$z=0$}
\put(2.1,1.5){$z=1$}
\put(1,2.6){$z=3$}
\put(2,2.3){$z=2$}
\put(-1.2,-2.5){$z\rightarrow\infty$}
\end{pspicture}
\end{center}
\begin{abc}\setcounter{enumi}{1}
\item Es ist 
$$\nabla f(x,y,z)=\left( -z,2y,-x\right)^\top\text{ und damit }\nabla f(\vec p_0)=(-3,\, -4,\,
-1)^\top.$$
%\item Die Ebene $\vec E$ hat den Normalenvektor $\nabla f(\vec p_0)$, dessen Normierung
%$$\vec n_0=\frac 1{\sqrt{26}} \begin{pmatrix}-3\\-4\\-1\end{pmatrix} $$
%ergibt und enth\"alt den Punkt $\vec
%p_0$. Ihre Hessesche Normalform ist also 
%\begin{align*}
%\vec E = &\{\vec x\in \R^3|\, \skalar{\vec x-\vec p_0,\, \vec n_0}=0\}\\
%=&\left\{\vec
%x\in \R^3\left|\, \skalar{\begin{pmatrix}x-1\\y+2\\z-3\end{pmatrix},\frac 1{\sqrt{26}} \begin{pmatrix}-3\\ -4\\
%-1\end{pmatrix}}=0\right.\right\}.
%\end{align*}
\item Die Ebene $\vec E$ hat den Normalenvektor $\nabla f(\vec p_0)$ und enth\"alt den Punkt $\vec
p_0$. Ihre Punkt-Normalen-Form ist also 
\begin{align*}
\vec E = &\{\vec x\in \R^3|\, \skalar{\vec n, \vec x-\vec p_0}=0\}\\
=&\left\{\vec
x\in \R^3\left|\, \skalar{\begin{pmatrix}-3\\ -4\\-1\end{pmatrix}, \begin{pmatrix}x-1\\y+2\\z-3\end{pmatrix}}=0\right.\right\}.
\end{align*}
%\item Ein kritischer Punkt erf\"ullt die Bedingung $\nabla f(\vec x)=0$. Damit ist $\vec x = \vec 0$ der
%einzige kritische Punkt. Der Abstand ergibt sich aus dem Skalarprodukt der Hesseschen Normalform: 
%$$d=\left|\skalar{\vec 0-\vec p_0,\vec n_0}\right|=\frac 1{\sqrt{26}} |-2|=\sqrt{\frac 2{13}}.$$
\item Ein kritischer Punkt erf\"ullt die Bedingung $\nabla f(\vec x)=0$. Damit ist $\vec x = \vec 0$ der einzige kritische Punkt. Der Abstand ergibt sich aus dem folgenden Skalarprodukt, wobei der normierte Normalenvektor $n_0$ verwendet wird: 
$$d=\left|\skalar{\vec 0-\vec p_0,\vec n_0}\right|=\frac 1{\sqrt{26}} |-2|=\sqrt{\frac 2{13}}.$$

\end{abc}

}

\ErgebnisC{AufganalysTangEbne001}
{
{ a)} $\vec F=\{(x,y,z)^\top\in\R^3|\, y^2-xz=1\}$,\, { b)} $(-3,\, -4,\, -1)^\top$

}


