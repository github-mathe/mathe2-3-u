\Aufgabe[e]{Alte Klausuraufgabe: lokale Extrema in 2 Dimensionen}{
Gegeben sei die folgende multivariate Funktion:
$$f(x,y) = (x^2 + 2y^2) \operatorname{e}^{-(x^2+y^2)}.$$
Bestimmen Sie alle stationären Punkte und klassifizieren Sie diese als Maximum, Minimum oder
Sattelpunkt.
}

\Loesung{
Der Gradient von $f$ ist
\begin{align*}
\nabla f (x,y) &= 
\begin{pmatrix}
2x\operatorname{e}^{-(x^2+y^2)} + (x^2 + 2y^2)\operatorname{e}^{-(x^2+y^2)} (-2x) \\
4y\operatorname{e}^{-(x^2+y^2)} + (x^2 + 2y^2)\operatorname{e}^{-(x^2+y^2)}(-2y)
\end{pmatrix}\\
&= \operatorname{e}^{-(x^2+y^2)}
\begin{pmatrix}
2x - 2x(x^2 + 2y^2)\\
4y - 2y(x^2 + 2y^2)
\end{pmatrix}
\end{align*}

$\nabla f(x,y) \overset{!}{=} 0$ führt zu dem System
\begin{align*}
0 &= 2x(1 - x^2 -2y^2)\\
0 &= 2y(2 - x^2 - 2y^2)
\end{align*}
Aus der ersten Gleichung erhalten wir:
\begin{align*}
x = 0 \quad \text{ oder } \quad & 0 = 1 - x^2 - 2y^2 \\
                 \Leftrightarrow & 2y^2 = 1 - x^2  
\end{align*}
Aus der zweiten Gleichung erhalten wir
\begin{align*}
y = 0 \quad \text{ oder } \quad & 0 = 2 - x^2 -2y^2\\
                    \Leftrightarrow & 2y^2 = 2 - x^2
\end{align*}
Daher müssen wir die folgenden drei Fälle betrachten:
\begin{iii}
\item
\begin{align*}
x = 0, \quad y = 0
\end{align*}
\item
\begin{align*}
x = 0, \quad 2y^2 &= 2 - x^2 \\ 
             2y^2 &= 2\\
              y^2 &= 1\\
                  &\Rightarrow x = 0, \quad y = -1\\
                  &            x = 0, \quad y = 1
\end{align*}
\item
\begin{align*}
y=0, \quad 2y^2 &= 1 - x^2\\
            0   &= 1- x^2\\
            x^2 &= 1\\
                &\Rightarrow x = -1, \quad y = 0\\
                &            x = 1, \quad  y = 0
\end{align*}
\end{iii}
Die Funktion hat die folgenden 5 stationären Punkte:
\begin{align*}
\begin{pmatrix} 0 \\ 0 \end{pmatrix},
\begin{pmatrix} 0 \\ -1 \end{pmatrix},
\begin{pmatrix} 0 \\ 1 \end{pmatrix},
\begin{pmatrix} -1 \\ 0 \end{pmatrix},
\begin{pmatrix} 1 \\ 0 \end{pmatrix}
\end{align*}

Um die stationären Punkte zu klassifizieren, benötigen wir die Hesse-Matrix
$$
H(x,y) = 2 \operatorname{e}^{-x^2-y^2} 
\begin{pmatrix}
1 - 5x^2 - 2y^2 + 2x^4 + 4x^2y^2 & - 6xy + 2x^3y + 4xy^3\\
 - 6 xy + 2x^3y + 4xy^3 & 2 - x^2 - 10y^2 + 2x^2y^2 + 4y^4
\end{pmatrix}
$$

Die Hesse-Matrix in dem kritischen Punkt $\begin{pmatrix} 0,0 \end{pmatrix}^T$ ist
positiv definit
$$
H(0,0) = 
2 \begin{pmatrix}
1 & 0\\
0 & 2
\end{pmatrix}.
$$
Daher ist $\begin{pmatrix} 0,0 \end{pmatrix}^T$ ein Minimum.

Die Hesse-Matrix in den kritischen Punkten $\begin{pmatrix} 0,-1 \end{pmatrix}^T$ und 
$\begin{pmatrix} 0,1 \end{pmatrix}^T$ ist negativ definit
$$
H(0,-1) =
\frac{2}{\operatorname{e}}
\begin{pmatrix}
-1 & 0\\
0 & -4
\end{pmatrix}
$$
$$
H(0,1) =
\frac{2}{\operatorname{e}}
\begin{pmatrix}
-1 & 0\\
0 & -4
\end{pmatrix}
$$
Daher sind $\begin{pmatrix} 0, -1 \end{pmatrix}^T$ und $\begin{pmatrix} 0, 1 \end{pmatrix}^T$ Maxima.

Die Hesse-Matrix in den kritischen Punkten $\begin{pmatrix} -1, 0 \end{pmatrix}^T$ und 
$\begin{pmatrix} 1, 0 \end{pmatrix}^T$ ist indefinit
$$
H(-1,0) =
\frac{2}{\operatorname{e}}
\begin{pmatrix}
-2 & 0\\
0 & 1
\end{pmatrix}
$$

$$
H(1,0) =
\frac{2}{\operatorname{e}}
\begin{pmatrix}
-2 & 0\\
0 & 1
\end{pmatrix}
$$
Daher sind $\begin{pmatrix} -1,0 \end{pmatrix}^T$ und $\begin{pmatrix} -1,0 \end{pmatrix}^T$ 
Sattelpunkte.

}

\ErgebnisC{analysExtrNdim003}{
$
\begin{pmatrix} 0 \\ 0 \end{pmatrix},
\begin{pmatrix} 0 \\ -1 \end{pmatrix},
\begin{pmatrix} 0 \\ 1 \end{pmatrix},
\begin{pmatrix} -1 \\ 0 \end{pmatrix},
\begin{pmatrix} 1 \\ 0 \end{pmatrix}
$
}
