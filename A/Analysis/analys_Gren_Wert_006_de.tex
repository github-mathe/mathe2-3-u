\Aufgabe[e]{Grenzwert Analyse - Definition}{
\begin{abc}
\item Notieren Sie die Definition des Grenzwertes und zeigen Sie, dass die Folge $a_n = \displaystyle\frac{1}{n}$ gegen den Grenzwert
$a=0$ konvergiert. \\
(Dies ist gleichbedeutend mit dem Nachweis, dass $\forall k \in \N$ eine Zahl $N\in \mathbb R$ existiert, so dass für alle $n\in \N$ mit $n>N$
gilt:  $|a_n-a|< 10^{-k}$).
%
\item Berechnen Sie den Grenzwert $a = \lim\limits_{n\rightarrow \infty}a_n$ der untenstehenden Folgen und dokumentieren Sie die Rechenregel, die Sie zur 
Berechnung des Grenzwertes verwendet haben (Produktregel, Einschließungssatz, Produkt beschränkter Folgen, Produkt von Nullfolgen etc).
%
%\setlength{\columnsep}{0.2cm}
%\begin{multicols}{2}
%\begin{iii}
%\item $a_n=\dfrac{n^2+5n}{3n^2+1}$.
%\item $a_n=\log_{10}(10n^2-2n)-\log_{10}(n^2+1)$.
%\item $a_n=\dfrac{(n+1)!}{n!-(n+1)!}$.
%\item $a_n=\left(1+\dfrac{1}{n}\right)^{3n}$.
%\item $a_n=\dfrac{\cos{n}}{n}$.
%\item $a_n=\sqrt{n+1}-\sqrt{n}$.
%\item $a_n=\dfrac{2^n}{n!}$.
%\end{iii}
%\end{multicols}
	$$
	\begin{array}{lllll}
	\textbf{i)} & a_n = \dfrac{n^2+5n}{3n^2+1} & &
	\textbf{ii)} &a_n = \log_{10}(10n^2-2n)-\log_{10}(n^2+1)\\
	& & & &\\
	\textbf{iii)} & a_n=\dfrac{(n+1)!}{n!-(n+1)!} & &
	\textbf{iv)} & a_n=\left(1+\dfrac{1}{n}\right)^{3n}\\
	& & & &\\
	\textbf{v)} & a_n=\dfrac{\cos{n}}{n} & &
	\textbf{vi)} & a_n=\sqrt{n+1}-\sqrt{n} \\
	& & & &\\
	\textbf{vii)} & a_n=\dfrac{2^n}{n!} 
	\end{array}
        $$
\end{abc}
}

\Loesung{
\begin{abc}
\item
Es ist zu zeigen: $\forall k \in \N : \exists N \in \mathbb R :$
\begin{align*}
|n^{-1} - 0| < 10^{-k}, \, \forall n>N.
\end{align*}
Wählen Sie $N =10^k$, um die Eigenschaft zu zeigen.
\item
Im vorigen Aufgabenteil wurde gezeigt, dass
\begin{align*}
\lim\limits_{n\rightarrow\infty} \frac{1}{n} = 0.
\end{align*}

Mit dem Ergebnis aus {\textbf a)} gilt 
\begin{align*}
\lim\limits_{n\rightarrow\infty} \frac{1}{n^\alpha} = 0, \, \text{ for all } \alpha \in \mathbb R, \, \alpha > 0.
\end{align*}

Des Weiteren wird die Stetigkeit der Funktionen vorausgesetzt und ausgenutzt. 

\begin{iii}
\item
Teilen von Zähler und Nenner durch $n^2$ liefert
\begin{align*}
\dfrac{1 + \frac{5}{n}}{3+\frac{1}{n^2}}
\end{align*}
Der Grenzwert des Zählers ist 1, der Grenzwert des Nenners ist 3.
Somit liefert die Quotientenregel für Grenzwerte das Ergebnis $a=\frac{1}{3}$.
\item
\begin{align*}
\log_{10}(10n^2-2n)-\log_{10}(n^2+1) &= \log_{10} {\frac{10n^2-2n}{n^2+1}}\\
&= \log_{10} {\frac{10-\frac{2}{n^2}}{1+\frac{1}{n^2}}}.
\end{align*}
%Using the product and addition rule you can show the limit of the sequence $b_n = \frac{10n^2-2n}{n^2+1}$
Der Grenzwert des Arguments $b$ des Logarithmus-Terms liefert 
\begin{align*}
b = \lim\limits_{n\rightarrow \infty} {\frac{10-\frac{2}{n^2}}{1+\frac{1}{n^2}}}=10.
\end{align*}
Aufgrund der Stetigkeit der Logarithmus-Funktionen innerhalb ihres Definitionsbereichs
auf $x>0$, berechnet sich der Grenzwert zu
\begin{align*}
a = \lim\limits_{n\rightarrow \infty}\log_{10} b_n = \log_{10} b = 1.
\end{align*}
\item
%
Umformung des Bruchausdrucks liefert
\begin{align*}
\frac{(n+1)!}{n!-(n+1)!} = \frac{(n+1)n!}{n!-(n+1)n!} = \frac{(n+1)}{-n} = -(1+\frac{1}{n}).
\end{align*}
Unter Verwendung der Summenregel ergibt sich der Grenzwert
\begin{align*}
a = \lim\limits_{n\rightarrow \infty}\frac{(n+1)!}{n!-(n+1)!} = \lim\limits_{n\rightarrow \infty}-(1+\frac{1}{n}) = -1.
\end{align*}
\item
Es gilt
\begin{align*}
\left(1+\dfrac{1}{n}\right)^{3n} = \left(\left(1+\dfrac{1}{n}\right)^n\right)^3
\end{align*}
Der Grenzwert der Folge
$b_n = \left(1+\dfrac{1}{n}\right)^n$ ist genau die Eulersche Zahl $\operatorname{e}$.
\begin{align*}
b = \lim\limits_{n\rightarrow \infty}\left(1+\dfrac{1}{n}\right)^n = \operatorname{e}.
\end{align*}
Die Produktregel liefert dann
\begin{align*}
\lim\limits_{n\rightarrow \infty} b_n^3 = \lim\limits_{n\rightarrow \infty} b_n \lim\limits_{n\rightarrow \infty} b_n \lim\limits_{n\rightarrow \infty} b_n = b^3 = \operatorname{e}^3.
\end{align*}
Alternativ kann auch die Stetigkeit der Funktion $x^3$ ausgenutzt werden, um das Ergebnis zu erhalten.
\item
Die Folge
\begin{align*}
a_n = \frac{\cos{n}}{n}
\end{align*}
kann als Produkt $a_n = b_n c_n$ einer beschränkten Teilfolgen
$b_n = \cos{n} \neq 0$ 
und einer Nullfolge
$c_n = \frac{1}{n}$ aufgefasst werden. 
Entsprechend ist deren Produkt ebenfalls eine Nullfolge und der Grenzwert ist $a=0$.
\item
Es gilt 
\begin{align*}
\sqrt{n+1}-\sqrt{n} &= \frac{\left(\sqrt{n+1}-\sqrt{n}\right) \left(\sqrt{n+1}+\sqrt{n}\right)}{\sqrt{n+1}+\sqrt{n}}\\
&=\frac{1}{\sqrt{n+1}+\sqrt{n}} = \frac{\frac{1}{\sqrt{n}}}{\sqrt{1+\frac{1}{n}}+\sqrt{\frac{1}{n}}}.
\end{align*}
Unter Verwendung der Produkt und Additionsregel erhalten wir den Grenzwert $a=0$.
\item
Die Folge $a_n = \dfrac{2^n}{n!}$ kann für $n>3$ nach oben und unten beschränkt werden
\begin{align*}
0 \leq \frac{2^n}{n!} = \frac{\overbrace{2\cdot 2\cdot 2\dots 2}^{n}}{1\cdot 2\cdot 3\dots n} \leq \frac{2}{1}\cdot \frac{2}{2}\cdot \underbrace{\frac{2}{3}\cdot\frac{2}{4}\dots\frac{2}{n-1}}_{\leq 1} \cdot \frac{2}{n} \leq \frac{4}{n}.
\end{align*}
Da die Folge von oben und unten durch zwei Nullfolge beschränkt ist, erhält man mit dem Einschließungssatz
den Grenzwert $a=0$. \\

Alternativ kann das Quotientenkriterium verwendet werden:
\begin{align*}
\lim\limits_{n\rightarrow \infty} \frac{a_{n+1}}{a_n} &= \lim\limits_{n\rightarrow \infty} \frac{2^{n+1}}{(n+1)!} \frac{n!}{2^n} = \lim\limits_{n\rightarrow \infty} \frac{2}{n+1} = 0 < 1.
\end{align*}
Da der Grenzwert kleiner als 1 ist, konvergiert die Folge $a_n$.
\end{iii}
\end{abc}
}

\ErgebnisC{analysGrenWert006de}
{
\textbf{b)} \begin{iii}
\item $a = \frac{1}{3}$
\item $a = 1$ 
\item $a = -1$
\item $a = \operatorname{e}^3$
\item $a = 0$
\item $a = 0$
\item $a = 0$
\end{iii}
}
