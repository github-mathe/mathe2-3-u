\Aufgabe[e]{Richtungsableitungen} {
\begin{abc}
\item
Gegeben seien die skalarwertigen Funktionen
$$f(x,y,z)=y^2-xz\qquad \text{ und } \qquad g(x,y,z)=x^2\sin(y)+\cos(z).$$
Berechnen Sie die Richtungsableitung beider Funktionen in Richtung $\vec h:=(-2,3,4)^\top$.
% \item
% Gegeben sei die folgende vektorwertige Funktion $$\vec f(x,y) = (\operatorname{e}^{xy}, x^2+y^3)^\top.$$ 
% \begin{iii}
% \item Berechnen Sie die Jacobi-Matrix von $\vec f$ in dem Punkt $\vec P_1=(1,2)^\top$.
% \item Berechnen Sie die Richtungsableitung von $\vec f$ in Richtung $\vec v=(1,0)^\top$ in dem Punkt $\vec P_2=(1,1)^\top$.
% \end{iii}
\end{abc}
}
% \Aufgabe[e]{Directional derivatives} {
% Given are the functions
% $$f(x,y,z)=y^2-xz\qquad \text{ and } \qquad g(x,y,z)=x^2\sin(y)+\cos(z).$$
% Compute the directional derivative of both functions in direction $\vec h:=(-2,3,4)^T$.
% }


\Loesung{
\begin{abc}
\item
Zun\"achst berechnen wir die Gradienten der beiden Funktionen:
\begin{align*}
\nabla f(x,y,z)=& (-z,\, 2y,\, -x)^\top\\
\nabla g(x,y,z)=& (2x\sin(y),\, x^2\cos(y),\, -\sin(z))^\top.
\end{align*}
Desweiteren ben\"otigen wir den Normalenvektor in Richtung $\vec h$:
$$\vec{\hat h}=\frac 1{\sqrt{4+9+16}}(-2,\, 3,\, 4)^\top=\frac 1{\sqrt{29}}(-2,\, 3,\, 4)^\top.$$
Damit ergeben sich dann die Richtungsableitungen:
\begin{align*}
\frac{\partial f}{\partial \vec{\hat h}}(x,y,z)=& \skalar{\vec{\hat h},\, \nabla f(x,y,z)}=\frac
1{\sqrt{29}} \left( 2z+6y-4x\right)\\
\frac{\partial g}{\partial \vec{\hat h}}(x,y,z)=& \skalar{\vec{\hat h},\, \nabla g(x,y,z)}
=\frac 1{\sqrt{29}} \left( -4x\sin(y)+3x^2\cos(y)-4\sin(z)\right).
\end{align*}
% \item
% \begin{iii}
% \item Die Jacobi-Matrix ist
% $$
% \vec J(\vec x) = 
% \begin{pmatrix} 
% f_{1,x} & f_{1,y}\\
% f_{2,x} & f_{2,y}
% \end{pmatrix}
% = \begin{pmatrix} 
% y\operatorname{e}^{xy} & x\operatorname{e}^{xy}\\
% 2x & 3y^2
% \end{pmatrix}
% $$
% 
% Die Jacobi-Matrix ausgewertet in dem Punkt $\vec P_1=(1,2)^\top$ ist
% $$
% \vec J(\vec P_1) 
% = \begin{pmatrix}
% 2\operatorname{e}^{2} & \operatorname{e}^{2}\\
% 2 & 12
% \end{pmatrix}
% $$
% \item Um die Richtungsableitung zu bestimmen, ben\"otigen wir einen Einheitsvektor. Da der gegebene Vektor $\vec v$ 
% bereits Einheitsl\"ange hat, kann die Richtungsableitung berechnet werden durch
% $$
% \frac{\partial \vec f}{\partial \vec h}(\vec x) = \vec J(x) \vec v = (y\operatorname{e}^{xy}, 2x)^\top
% $$
% 
% Die Richtungsableitung ausgewertet in dem Punkt $\vec P_2$ ist
% $$
% \frac{\partial \vec f}{\partial \vec h}(\vec P_2) = (\operatorname{e}, 2)^\top.
% $$
% \end{iii}
\end{abc}
}
% \Loesung{
% First, we compute the gradient of both functions:
% \begin{align*}
% \nabla f(x,y,z)=& (-z,\, 2y,\, -x)^T\\
% \nabla g(x,y,z)=& (2x\sin(y),\, x^2\cos(y),\, -\sin(z))^T.
% \end{align*}
% Further, we need the normal vector in direction $\vec h$:
% $$\vec{\hat h}=\frac 1{\sqrt{4+9+16}}(-2,\, 3,\, 4)^T=\frac 1{\sqrt{29}}(-2,\, 3,\, 4)^T.$$
% This gives the directional derivatives:
% \begin{align*}
% \frac{\partial f}{\partial \vec{\hat h}}(x,y,z)=& \skalar{\vec{\hat h},\, \nabla f(x,y,z)}=\frac
% 1{\sqrt{29}} \left( 2z+6y-4x\right)\\
% \frac{\partial g}{\partial \vec{\hat h}}(x,y,z)=& \skalar{\vec{\hat h},\, \nabla g(x,y,z)}
% =\frac 1{\sqrt{29}} \left( -4x\sin(y)+3x^2\cos(y)-4\sin(z)\right).
% \end{align*}
% }

\ErgebnisC{AufganalysRichAblt003}
{
$\frac{-4x+6y+2z}{\sqrt{29}}$,\qquad $\frac{-4x\sin y + 3 x^2\cos y - 4\sin z}{\sqrt{29}}$
}
