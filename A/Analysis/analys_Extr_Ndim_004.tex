\Aufgabe[e]{Station\"are Punkte} {
 Gegeben sei die Funktion
$$
f(x,y) =  ax^2 + 2xy + ay^2-ax-ay.
$$
Bestimmen Sie diejenigen $a\in \R$, f\"ur welche die Funktion 
$f$ relative Maxima, relative Minima und Sattelpunkte besitzt und geben Sie 
diese an. 
}

\Loesung{
In station\"aren Punkten  muss der Gradient der Funktion Null werden: 
\begin{align*}
&\vec 0= \nabla f(x,y)=\begin{pmatrix} 2ax+2y-a\\2x+2ay-a\end{pmatrix}\\
&\begin{array}{r|rr|r|l}
\mathrm{I}  & 2a     &    2 &     a& \mathrm{I'} = \mathrm{II}\\
\mathrm{II} & 2      &   2a &     a& \mathrm{II'} = \mathrm{I}\\\hline
\mathrm{I}' & 2      &   2a &     a&  \\
\mathrm{II'} & 2a     &    2 &     a&\mathrm{II''}= \mathrm{II'} -a\cdot \mathrm{I'}\\\hline
\mathrm{I'}& 2     &    2a &     a  & \mathrm{I''}= \mathrm{I'}: 2\\
\mathrm{II''}& 0     &    2-2a^2 &     a  & \mathrm{II'''} = \mathrm{II''} : (2-2a^2), \text{ f\"ur }a^2\neq 1\\\hline
\mathrm{I''}& 1     &    a & \frac a 2  & \\
\mathrm{II'''}& 0 &    1 & \frac{a(1-a)}{2(1-a^2)}\\
\end{array}
\end{align*}
Dann ist 
\begin{align*}
y &= \frac{a(1-a)}{2(1+a)(1-a)} = \frac{a}{2(1+a)}\\
x &= \frac{a}{2} - a \cdot \frac{a}{2(1+a)} = \frac{a(1+a-a)}{2(1+a)} = \frac{a}{2(1+a)}\,.
\end{align*}

F\"ur $a\neq \pm 1$ liegt der einzige kritische Punkt der Funktion also bei 
$$\vec x_0 = \begin{pmatrix} \frac{ a}{2(1+a)}\\ \frac{a}{2(1+a)}\end{pmatrix}. 
$$
F\"ur $a=+1$ hat das obige Gleichungssystem die L\"osung $(t,\frac 12 -t)$ ($t\in\R$). \\
F\"ur $a=-1$ hat das Gleichungssystem keine L\"osung und damit $f$ keinen station\"aren Punkt. \\
Zur Charakterisierung der station\"aren Punkte ben\"otigen wir die Hesse-Matrix: 
$$\vec H_f(x,y)=\begin{pmatrix} 2a& 2\\ 2 & 2a\end{pmatrix}.$$
Ihre Determinante ist \(\det (\vec H_f) = 4 (a^2-1)\).
Wir unterscheiden zun\"achst die drei F\"alle: 
\begin{iii}
\item $-1<a<1$: F\"ur die Determinante gilt \(\det(\vec H_f) < 0\). Da die Determinante aus dem Produkt der Eigenwerte berechnet wird, folgt daraus, dass die Eigenwerte verschiedene Vorzeichen haben. $\vec H_f$ ist indefinit und $\vec x_0$ ist ein
Sattelpunkt. 
\item $a<-1$: F\"ur die Determinante gilt \(\det(\vec H_f) > 0\). Daraus folgt, dass die Eigenwerte gleiches Vorzeichen haben. Da die Spur \(\operatorname{Sp}(\vec H_f) = 4 a\)  in diesem Fall negativ ist, sind die beiden Eigenwerte negativ. Also ist $\vec H_f$  negativ definit und die Funktion $f$ hat ein Maximum in $\vec x_0$. 
\item $1<a$: Sowohl die Determinante, als auch die Spur sind positiv. Deswegen sind beide Eigenwerte positiv und bei $\vec x_0$ ist ein Minimum. 
\end{iii}
F\"ur $a=1$ gilt \(\det(\vec H_f)=0\), also ist mindestens ein Eigenwert gleich Null. In diesem Fall gilt \[f(x,y) = x^2+2xy+y^2-x-y = (x+y)^2 - (x+y) = \left(x+y-\frac12\right)^2 - \frac14\,.\]
In allen station\"are Punkten $(t,\, \frac 12 -t)^\top$ ist der quadratische Term \((x+y-\frac12)^2\) gleich Null, aber in allen anderen Punkten ist er positiv. Deshalb liegt in diesen station\"aren Punkten ein Minimum vor. Diese Punkte
befinden sich auf einer Geraden, auf der $f$ konstant ist. \\
Insgesamt besitzt die Funktion $f$ also 
\begin{itemize}
\item f\"ur $a<-1$ ein Maximum in $\vec x_0=\frac a{2(1+a)}(1,1)^\top$. 
\item f\"ur $a=-1$ keinen station\"aren Punkt. 
\item f\"ur $-1<a<1$ einen Sattelpunkt in $\vec x_0=\frac a{2(1+a)}(1,1)^\top$. 
\item f\"ur $a=1$ die Minimalstellen $(t,\, \frac 12 - t)^\top$, $t\in\R$. 
\item f\"ur $1<a$ ein Minimum in $\vec x_0=\frac a{2(1+a)}(1,1)^\top$. 
\end{itemize}
}

\ErgebnisC{AufganalysExtrNdim004}
{
{
F\"ur $a\neq \pm 1$ ist der station\"are Punkt $\frac a{2(1+a)}(1,1)^\top$. F\"ur $a=1$ gibt es
mehrere station\"are Punkte. 
}
}

