\Aufgabe[e]{Regel von L'Hospital}{
Berechnen Sie mit Hilfe der Regel von L'Hospital die Grenzwerte
\begin{align*}
A=\underset{x\rightarrow 0}\lim \,  \left(\frac{\tan x}{x}\right),\quad&&
B=\underset{x\rightarrow 0}\lim \, \left(\frac{1-\cosh x}{x^3}\right),\quad&&
C=\underset{x\rightarrow 1}\lim \, \left(\frac{\ln x}{\sin(\pi x)}\right)
\end{align*}
}

\Loesung{
Es gilt $\underset{x\to 0}\lim \tan x = \underset{x\to0}\lim x =0$, also darf der
Satz von L'Hospital angewendet werden: 
\begin{align*}
A=& \underset{x\to0}\lim \frac{\tan x}x = \underset{x\to 0}\lim \frac {\frac 1{\cos^2
x}}{1} = 1. 
\end{align*}
F\"ur $B$ gehen Z\"ahler und Nenner gegen Null ($\underset{x\rightarrow 0}\lim \cosh x = 1$), also kann auch hier der Satz von L'Hospital
angewendet werden: 
\begin{align*}
B=&\underset{x\to 0}\lim \frac {1-\cosh x}{x^3} = \underset{x\to 0}\lim \frac{-\sinh x}{3x^2}\text{
(erneut ist }\underset{x\rightarrow 0}\lim \sinh x = \underset{x\to 0}\lim x^2=0)\\
=&\underset{x\to 0}\lim \frac{-\cosh x}{6x}= \begin{cases} -\infty \quad &\text{, f\"ur } x > 0 \text{ (rechtsseitiger Grenzwert)}\\ \infty \quad &\text{, f\"ur } x < 0 \text{ (linksseitiger Grenzwert)}\end{cases}.
\end{align*}
F\"ur $C$ gilt wiederum $\underset{x\to 1}\lim \ln x = \underset{x\to 1}\lim\sin(\pi x)=0$, also
kann auch hier der Satz von L'Hospital angewendet werden: 
\begin{align*}
C=& \underset{x\rightarrow 1}\lim \frac{\ln x }{\sin(\pi x)} = \underset{x\to 1}\lim \frac{\frac
1x}{\pi \cos (\pi x)}\\
=& \underset{x\to 1}\lim \frac{1}{\pi x\cos (\pi x)}=-\frac 1{\pi}.
\end{align*}

}

\ErgebnisC{AufganalysReglLhos002}
{
$A=1$, $B=\pm \infty$, $C=-1/\pi$
}



