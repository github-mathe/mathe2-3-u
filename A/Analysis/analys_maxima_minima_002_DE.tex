\Aufgabe[e]{Minimaler Abstand}{
Gegeben seien die folgenden Funktionen:
\begin{multicols}{2}
\begin{iii}
\item $f_1(x) = x^2+1$, mit $D(f_1) = \R$,
\item $f_2(x) = \ln(x)$, mit $D(f_2) = \R^+$.
\end{iii}
\end{multicols}
\begin{abc}
\item Skizzieren Sie die Funktionen $f_1$ und $f_2$. 
\item Bestimmen Sie den minimalen vertikalen Abstand beider Funktionsgraphen:
$$d=\text{min }\left\{ |f_1(x)-f_2(x)|;\, x\in\R^+\right\}.$$
\end{abc}
}

\Loesung{
\begin{abc}
\item 

\begin{minipage}{\linewidth}
\centering
\begin{tikzpicture}
\begin{axis}[
axis lines=middle,clip=false,
xmin=-3,
xmax=3,
ymin=-6,
ymax=6,
xticklabel style={black},
xlabel=$x$,
ylabel=$y$]

\addplot[domain=-2:2,samples=200,blue]{x^2+1}
node[right,pos=1.,font=\footnotesize]{$f_1(x)=x^2+1$};
\addplot[domain=0:3,samples=200,red]{ln(x)}
node[right,pos=1.,font=\footnotesize]{$f_2(x)=\ln(x)$};
\end{axis}
\end{tikzpicture}

\end{minipage}

\item 
Der gesuchte Abstand entspricht dem Minimum der Funktion
$$
h(x) = |f_1(x) - f_2(x)|.
$$
Der Definitionsbereich von $h$ ergibt sich als
$$
D(h) = D(f_1) \cap D(f_2) = \R^+.
$$
Da $f_1(x)$ für alle $x \in \R^+$ oberhalb von $f_2(x)$ liegt, vereinfacht sich $h(x)$ zu
$$
h(x) = f_1(x) - f_2(x),
$$
ohne den Betrag betrachten zu müssen.

Die Grenzen des Definitionsbereichs zeigen, dass:
\[
\lim_{x \to \infty} h(x) = \lim_{x \to \infty} \left( x^2 + 1 - \ln(x) \right).
\]
Durch Faktorisieren von $x^2 + 1$:
\[
\lim_{x \to \infty} h(x) = \lim_{x \to \infty} \left( x^2 + 1 \right) \left( 1 - \frac{\ln(x)}{x^2 + 1} \right).
\]

Es gilt
\[
\lim_{x \to \infty} \frac{\ln(x)}{x^2 + 1} = 0.
\]
Da der Bruch die Form $\frac{\infty}{\infty}$ hat, kann die Regel von L’Hôpital angewendet werden:
\[
\lim_{x \to \infty} \frac{\ln(x)}{x^2 + 1} = \lim_{x \to \infty} \frac{\frac{1}{x}}{2x}.
\]
Der neue Bruch ergibt:
\[
\lim_{x \to \infty} \frac{1}{2x^2} = 0.
\]
Daher gilt:
\[
\lim_{x \to \infty} h(x) = \infty.
\]
Weiterhin gilt:
\[
\lim_{x \to 0^+} h(x) = \lim_{x \to 0^+} \left( x^2 + 1 - \ln(x) \right) = \infty.
\]
Damit wird innerhalb des Definitionsbereichs ein Minimum angenommen.\\

Die lokalen Extrema von $h(x)$ liegen in den Nullstellen der ersten Ableitung:
\[
h'(x) = 2x - \frac{1}{x}.
\]
Setzen $h'(x) = 0$ ergibt:
\[
2x - \frac{1}{x} = 0 \quad \Rightarrow \quad 2x^2 = 1 \quad \Rightarrow \quad x_{1/2} = \pm\frac{1}{\sqrt{2}}.
\]
Da $x \in \R^+$, bleibt nur $x_1 = \frac{1}{\sqrt{2}}$.\\

Die zweite Ableitung von $h(x)$ ist:
\[
h''(x) = 2 + \frac{1}{x^2}.
\]
Da $h''(x) > 0$ für alle $x \in \R^+$, ist $h(x)$ an der Stelle $x_1 = \frac{1}{\sqrt{2}}$ streng konvex, und es handelt sich um ein lokales Minimum.\\

An dieser Stelle ist:
\[
d = \frac{3}{2} - \ln\left( \frac{\sqrt{2}}{2} \right).
\]

Der Graph von $h(x)$ ist:

\begin{minipage}{\linewidth}
\centering

\begin{tikzpicture}
\begin{axis}[
axis lines=middle,clip=false,
xmin=-3,
xmax=3,
ymin=-1,
ymax=6,
xticklabel style={black},
xlabel=$x$,
ylabel=$y$]

\addplot[domain=0:2.5,samples=200,blue]{x^2+1-ln(x)}
node[right,pos=1.,font=\footnotesize]{$h(x)=|x^2+1-\ln(x)|$};

\end{axis}
\end{tikzpicture}
\end{minipage}

\end{abc}

}


\ErgebnisC{MaxMin001}
{

$d = \frac{3}{2} - \ln\left( \frac{\sqrt{2}}{2} \right)$
}
