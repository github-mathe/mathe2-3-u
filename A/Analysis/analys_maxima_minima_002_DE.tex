\Aufgabe[e]{Minimaler Abstand}{
Gegeben seien die folgenden Funktionen%
\begin{multicols}{2}
\begin{iii}
\item $f_1(x) = x^2+1$,
\item $f_2(x) = \ln(x)$.
\end{iii}
\end{multicols}
\begin{abc}
\item Skizzieren Sie die Funktionen $f_1$ und $f_2$. 
\item Bestimmen Sie den minimalen vertikalen Abstand beider Funktionsgraphen
$$d=\text{min }\left\{ |f_1(x)-f_2(x)|;\, x\in\R^+\right\}.$$
\end{abc}
}


\Loesung{
\begin{abc}
\item 

\begin{minipage}{\linewidth}
\centering
\begin{tikzpicture}
\begin{axis}[
axis lines=middle,clip=false,
xmin=-3,
xmax=3,
ymin=-6,
ymax=6,
xticklabel style={black},
xlabel=$x$,
ylabel=$y$]

\addplot[domain=-2:2,samples=200,blue]{x^2+1}
node[right,pos=1.,font=\footnotesize]{$f_1(x)=x^2+1$};
\addplot[domain=0:3,samples=200,red]{ln(x)}
node[right,pos=1.,font=\footnotesize]{$f_2(x)=\ln(x)$};
\end{axis}
\end{tikzpicture}

\end{minipage}

\item Der gesuchte Abstand ist das Minimum der Funktion 
 $g(x) = |h(x)| \text{ mit } h(x)=f_1(x) - f_2(x)$.\\
 Es gilt $f_1(x) > f_2(x)$ für alle $x \in D$. Daher ist die Differenz
 immer positiv. 
 
 
Dieses Minimum liegt in einem lokalen Extremum von $h(x)$, da an den Grenzen des Definitionsbereiches $D$ gilt: 
\begin{align*}
\underset{x\to\infty}\lim h(x)=& \underset{x\to\infty}\lim \left( x^2+1-\ln(x)\right) \\
=&\ln\left(  \underset{x\to\infty}\lim \EH{x^2+1-\ln(x)}\right)\\
=&\ln \left( \underset{x\to\infty}\lim \frac{\EH{x^2+1}}{\EH{\ln x}}\right)\qquad\text{ L'Hospital wg.} \EH{x^2+1}\to\infty,\, \EH{\ln x}=x\to \infty.\\
=&\ln \left( \underset{x\to\infty}\lim \frac{2x\EH{x^2+1}}{1}\right)=\infty\\
\underset{x\to 0}\lim h(x)=& \underset{x\to 0}\lim \left(x^2+1-\ln x\right)=\infty
\end{align*}
Das heißt innerhalb des Definitionsbereiches wird ein Minimum angenommen.

Die lokalen Extrema von $h(x)$ liegen in den Nullstellen der ersten Ableitung: 
\begin{align*}
&&h'(x)=2x-\frac 1x=&0\\
\Leftrightarrow&&2x^2=&1\\
\Leftrightarrow&& x_{1/2}=&\pm\frac 1{\sqrt 2}
\end{align*}
von diesen beiden Werten liegt nur $x_1=\frac 1{\sqrt 2}$ im Definitionsbereich $D$. Dort ist 
$$d=g(x_1)=h(x_1)=  \frac 12 + 1 - \ln \left(\frac 1{\sqrt 2}\right) = \frac 32 - \ln\left(\frac{\sqrt 2}2\right).$$

Der Graph von $h(x)=g(x)$ ist 

\begin{minipage}{\linewidth}
\centering

\begin{tikzpicture}
\begin{axis}[
axis lines=middle,clip=false,
xmin=-3,
xmax=3,
ymin=-1,
ymax=6,
xticklabel style={black},
xlabel=$x$,
ylabel=$y$]

\addplot[domain=0:2.5,samples=200,blue]{x^2+1-ln(x)}
node[right,pos=1.,font=\footnotesize]{$g(x)=|x^2+1-\ln(x)|$};

\end{axis}
\end{tikzpicture}
\end{minipage}

\end{abc}

}
