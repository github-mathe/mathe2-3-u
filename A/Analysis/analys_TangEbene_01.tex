%% % % Norms
\Aufgabe[e]{Flächen, Tangentialebenen und Normalenvektoren}{
Gegeben sei die Funktion
        \[f(x,y,z) = x^2 + y^2 - z.\]
\hspace{-0.3cm}
\begin{abc}
\item Skizzieren Sie die Äquipotentialfläche
        \begin{align*}
                F = \{(x,y,z)\in \mathbb R^ 3: f(x,y,z) &= 0\}.\\
        \end{align*}
\item Bestimmen Sie den Gradienten $\vec g = \vec \nabla f$ und werten Sie ihn im Punkt $\vec P=(1,1,2)^T$ aus. Welche Eigenschaft hat die Steigung $\vec g$ in $\vec P$ in Bezug auf die Äquipotentialfläche $F$?
\item Skizzieren Sie den Gradienten $\vec g$ am Punkt $\vec P$ im Unterraum, der durch den Punkt $\vec P$ geht. 
\item Schreiben Sie die Gleichung der Tangentialebene an die Fläche $F$, die durch den Punkt $\vec P$ verläuft.
\end{abc}
}
\ifthenelse{\boolean{mitLoes}}{\ruleBig \cleardoublepage}{}
