\Aufgabe[e]{Bewegung}{
}
Ein Elektron bewegt sich mit der Geschwindigkeit
$$
v(t) = e^{-\cosh(3t^2+2t+1)},
$$
wobei $t$ in Sekunden gegeben ist.
\begin{abc}
\item Wie hoch ist seine Beschleunigung nach $t=10$ Sekunden?
\item Kommt das Elektron jemals zur Ruhe? Man begründe die Antwort.
\end{abc}

\Loesung{
Die Beschleunigung ist 
$$
a(t) = -2(3t+1) \sinh(3t^2 + 2t + 1) e^{-\cosh(3t^2+2t+1)}
$$
Nach 10 Sekunden hat das Elektron dann eine Beschleunigung von 
$$
a(10) = 10^{10^{138,75}}
$$

}

\ErgebnisC{analysAblt01}
{
}
