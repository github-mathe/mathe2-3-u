\Aufgabe[e]{Funktionenlimes}{
Berechnen Sie die folgenden Grenzwerte. Wenn der Grenzwert nicht existiert, begründen Sie Ihre Antwort.
\begin{abc}
\item
\begin{multicols}{2}
\begin{iii}
    \item $ \lim_{x \to 2} (3x^2 - 4x + 5) $

    \item $ \lim_{x \to \infty} \frac{2x^2 + 3}{5x^2 + 7} $

    \item $ \lim_{x \to 1} \frac{1}{x - 1} $

    \item $ \lim_{x \to \infty} \sin(x) $

    \item $ \lim_{x \to 0^+} x \ln(x) $

    \item $ \lim_{x \to \infty} e^{-x} $

    \item $ \lim_{x \to 0} \frac{\sin(x)}{x} $

    \item $ \lim_{x \to 0} \sin\left(\frac{1}{x}\right) $

    \item $ \lim_{x \to 0} \frac{|x|}{x} $
\end{iii}
\end{multicols}
\item
Gegeben die stückweise definierte Funktion:
\[
f(x) = \begin{cases} 
x^2, & \text{falls } x < 0, \\
3x + 1, & \text{falls } x \geq 0.
\end{cases}
\]
Berechnen Sie $ \lim_{x \to 0} f(x) $.
\end{abc}
}

\Loesung{

\begin{abc}
\item
\begin{iii}
    \item $ \lim_{x \to 2} (3x^2 - 4x + 5) = 3(2)^2 - 4(2) + 5 = 9 $

    \item
    \[
    \lim_{x \to \infty} \frac{2x^2 + 3}{5x^2 + 7}
    = \lim_{x \to \infty} \frac{x^2(2 + \frac{3}{x^2})}{x^2(5 + \frac{7}{x^2})}
    = \lim_{x \to \infty} \frac{2 + \frac{3}{x^2}}{5 + \frac{7}{x^2}}.
    \]
    Da $ \frac{3}{x^2} \to 0 $ und $ \frac{7}{x^2} \to 0 $ für $ x \to \infty $, folgt:
    \[
    \lim_{x \to \infty} \frac{2 + \frac{3}{x^2}}{5 + \frac{7}{x^2}} = \frac{2}{5}.
    \]

    \item Der Grenzwert existiert nicht, da $ \frac{1}{x - 1} \to \infty $ für $ x \to 1^+ $ und $ \frac{1}{x - 1} \to -\infty $ für $ x \to 1^- $.

    \item Der Grenzwert existiert nicht, da $ \sin(x) $ zwischen $-1$ und $1$ oszilliert, während $x \to \infty$.

    \item $ \lim_{x \to 0^+} x \ln(x) = \lim_{x \to 0^+} \frac{\ln(x)}{1/x} = \lim_{x \to 0^+} \frac{1/x}{-1/x^2} = \lim_{x \to 0^+} -x = 0 $ (L’Hôpital-Regel angewendet).

    \item $ \lim_{x \to \infty} e^{-x} = 0 $ (Exponentielles Abklingen).

    \item
    \[
    \lim_{x \to 0} \frac{\sin(x)}{x} = \lim_{x \to 0} \frac{\cos(x)}{1} = \cos(0) = 1.
    \]
    (L’Hôpital-Regel angewendet).

    \item Der Grenzwert existiert nicht, da $ \sin\left(\frac{1}{x}\right) $ unendlich oft zwischen $-1$ und $1$ oszilliert, während $x \to 0$.

    \item Der Grenzwert existiert nicht, da die linksseitige Grenze $ \lim_{x \to 0^-} \frac{|x|}{x} = -1 $ und die rechtsseitige Grenze $ \lim_{x \to 0^+} \frac{|x|}{x} = 1 $ unterschiedlich sind.
\end{iii}
\item
Gegeben die Funktion:
\[
f(x) = \begin{cases} 
x^2, & \text{falls } x < 0, \\
3x + 1, & \text{falls } x \geq 0.
\end{cases}
\]
Da $ \lim_{x \to 0^-} f(x) = 0^2 = 0 $ und $ \lim_{x \to 0^+} f(x) = 3(0) + 1 = 1 $, ist der Grenzwert $ \lim_{x \to 0} f(x) $ nicht definiert.
\end{abc}
}

\ErgebnisC{analysFunkLims003}
{
\begin{abc}
\item
\begin{iii}
\begin{multicols}{2}
    \item $ 9 $
    \item $ \frac{2}{5} $
    \item Existiert nicht
    \item Existiert nicht
    \item $ 0 $
    \item $ 0 $
    \item $ 1 $
    \item Existiert nicht
    \item Existiert nicht
\end{multicols}
\end{iii}

\item Für $ f(x) $ existiert der Grenzwert bei $ x \to 0 $ nicht.
\end{abc}
}
