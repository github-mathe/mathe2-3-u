\Aufgabe[e]{Kurvendiskussion}{\label{KurvDisk008}
 Gegeben sei die reelle Funktion 
	\[
	f(x)=\frac{x^2+7\,x+10}{x+1}\ .
	\]
	\begin{enumerate}
		\item[i)] Geben Sie den maximalen Definitionsbereich der Funktion an.
		
		\item[ii)] Bestimmen Sie die Nullstellen der Funktion.
		
		\item[iii)] Bestimmen Sie die kritischen Punkte für die Extrema der Funktion und deren Funktionswerte.
		
		\item[iv)] Bestimmen Sie die (nicht vertikale) Asymptote der Funktion, d.\,h.\ diejenige Gerade \ $g(x)=a+bx$, für die
		\[
		\lim\limits_{x\rightarrow\pm\infty} \left(f(x)-g(x) \right) =0
		\]
		gilt.
		
		\item[v)] Skizzieren Sie die Funktion.
	\end{enumerate}
%  Gegeben sei die Funktion
% 	\[
% 	g\ :\ \R\rightarrow\R\ : \ x\mapsto x^8\cdot\operatorname{e}^x\, .
% 	\]
% 	Bestimmen Sie alle relativen Minima und Maxima der Funktion $g$.

}

\Loesung{
\textbf{Zu a)}  
\begin{enumerate}
	\item[i)] Die Funktion ist nur an der Nullstelle des Nenners, $x_{\text{Pol}}=-1$ , nicht definiert und hat dort eine einfache Polstelle.
	
	\item[ii)] Die Nullstellen sind die Nullstellen des Zählers: \ $x_{N1}=-5 \quad \text{und} \quad x_{N2}=-2$ .
	
	\item[iii)] Die kritischen Punkte für die Extrema sind die Nullstellen (des Zählers) der ersten Ableitung:
	\[
	f'(x)=\frac{x^2+2x-3}{(x+1)^2} \;\; \Longrightarrow \;\;  x_{E1}=-3 \;\, \text{un d} \;\; x_{E2}=1 
	\]
	mit $f(-3)=1$ und $f(1)=9$.
	
	\item[iv)] Durch Polynomdivision erhält man die Asymptote:
	\[
	f(x) = \frac{x^2+7\,x+10}{x+1} = x + 6 + \frac{4}{x+1} \;\; \Longrightarrow \;\;  g(x)=x+6\,.
	\]
	
	\item[v)] Skizze:
	
	\psset{unit=0.6cm,linewidth=0.8pt}
	\begin{pspicture}(-10,-4.5)(6,14.25)
	\psgrid[griddots=8,subgriddiv=0](-6,-4)(4,14)
	\psline[linewidth=1.2pt]{->}(-6.25,0)(4.5,0)
	\psline[linewidth=1.2pt]{->}(0,-4.25)(0,14.5)
	\psline[linestyle=dashed]{-}(-6,0)(4,10)
	\psline[linestyle=dashed]{-}(-1,-4)(-1,14)
	\pscircle*(-5,0){0.1}
	\pscircle*(-2,0){0.1}
	\pscircle*(-3,1){0.1}
	\pscircle*(1,9){0.1}
	\bezier{200}( -6.000, -0.800)( -5.735, -0.577)( -5.497, -0.386)
	\bezier{200}( -5.497, -0.386)( -5.230, -0.172)( -4.993,  0.005)
	\bezier{200}( -4.993,  0.005)( -4.724,  0.206)( -4.490,  0.364)
	\bezier{200}( -4.490,  0.364)( -4.218,  0.546)( -3.986,  0.674)
	\bezier{200}( -3.986,  0.674)( -3.711,  0.826)( -3.483,  0.906)
	\bezier{200}( -3.483,  0.906)( -3.203,  1.004)( -2.979,  1.000)
	\bezier{200}( -2.979,  1.000)( -2.691,  0.994)( -2.476,  0.814)
	\bezier{200}( -2.476,  0.814)( -2.172,  0.560)( -1.972, -0.086)
	\bezier{200}( -1.972, -0.086)( -1.633, -1.183)( -1.469, -4.000)
	
	\bezier{200}( -0.531, 14.000)( -0.367, 11.183)( -0.028, 10.086)
	\bezier{200}( -0.028, 10.086)(  0.172,  9.440)(  0.476,  9.186)
	\bezier{200}(  0.476,  9.186)(  0.691,  9.006)(  0.979,  9.000)
	\bezier{200}(  0.979,  9.000)(  1.203,  8.996)(  1.483,  9.094)
	\bezier{200}(  1.483,  9.094)(  1.711,  9.174)(  1.986,  9.326)
	\bezier{200}(  1.986,  9.326)(  2.218,  9.454)(  2.490,  9.636)
	\bezier{200}(  2.490,  9.636)(  2.724,  9.794)(  2.993,  9.995)
	\bezier{200}(  2.993,  9.995)(  3.230, 10.172)(  3.497, 10.386)
	\bezier{200}(  3.497, 10.386)(  3.735, 10.577)(  4.000, 10.800)
	\end{pspicture} 
	
	Aus der Skizze ersieht man, dass bei \ $x=-3$ \ ein Maximum vorliegt und bei \ $x=1$ \ ein Minimum.
\end{enumerate}

\medskip
\textbf{Zu b)} Da die Funktion beliebig oft differenzierbar ist, können die Extrema nur an Nullstellen der ersten Ableitung liegen:
\[
g'(x) = x^{7}\cdot(8+x)\cdot\operatorname{e}^x = 0 \quad \Longrightarrow \quad  x_1 = -8  \; \text{und} \;\; x_{2,...,8} = 0\,.
\]
Da $\lim\limits_{x\rightarrow-\infty} g(x)=0$ und $g(0)=0$ ist und $x=-8$ der einzige kritische Punkt im Innern dieses Intervalls ist, liegt wegen $g(-8) > 0$ an der Stelle $x=-8$ ein Maximum vor. 

An der Stelle \ $x=0$ gilt \ $g(x)= 0$. Da \ $x=0$ \ die einzige Nullstelle ist, muß dies ein Minimum sein.
}

%\ErgebnisC{AufganalysKurvDisk008}
%{
%}
