\Aufgabe[e]{Integration in Kugelkoordinaten} {
Gegeben sei das Rotationsellipsoid
$$E=\left\{(x,y,z)\in\R^3\bigl|\, x^2+4y^2+z^2\leq 9\bigr.\right\}.$$
Man berechne das Integral $\int\limits_B (x^2+y+z^2)\d(x,y,z)$ unter Verwendung von 
\begin{abc}
\item Zylinderkoordinaten
\item an das Ellipsoid angepasste Kugelkoordinaten: 
$$\begin{pmatrix}x\\y\\z\end{pmatrix} = \begin{pmatrix}
r\cos\varphi\cos\psi\\
\frac r2 \sin\varphi \cos\psi\\
r\sin\psi
\end{pmatrix}.$$
\end{abc}
\textbf{Hinweis}(zu \textbf{a)}): Verwenden Sie um die $y-$Achse rotationssymmetrische Zylinderkoordinaten 
$$\vec x(r,\varphi,y)=\begin{pmatrix}r\cos(\varphi)\\y\\r\sin(\varphi)\end{pmatrix}.$$
}
\Loesung{
\begin{abc}
\item Wir nutzen Zylinderkoordinaten der Form 
$$\begin{pmatrix}x\\y\\z\end{pmatrix}=\begin{pmatrix}r\cos\varphi\\y\\r\sin\varphi\end{pmatrix}, \, \d
(x,y,z)=r\d(r,\varphi,y).$$
Damit hat man mit den Integrationsgrenzen f\"ur $y$
\begin{align*}
y_\pm =& \pm \frac 12 \sqrt{9-r^2} 
\end{align*}
das Integral 
\begin{align*}
I:=&\int\limits_B
(x^2+y+z^2)\d(x,y,z)=\int\limits_{\varphi=0}^{2\pi}\int\limits_{r=0}^{3} \int\limits_{y=y_-}^{y_+}
(r^2\cos^2\varphi + y+r^2\sin^2\varphi)r\d y\d r \d \varphi\\
=&\int\limits_{\varphi=0}^{2\pi}\int\limits_{r=0}^{3} \left(
r^3(y_+-y_-)+r \frac{y_+^2-y_-^2}2\right)\d r\d\varphi\\
=&\int\limits_{\varphi=0}^{2\pi}\int\limits_{r=0}^{3} 
r^3\sqrt{9-r^2}\d r\d\varphi\\
=&2\pi \left( \left.-r^2\frac{(9-r^2)^{3/2}}3\right|_{r=0}^3 + \int\limits_0^3
2r\frac{(9-r^2)^{3/2}}3\d r \right)\\
=&2\pi \cdot \frac 23\cdot \int\limits_0^3 r(9-r^2)^{3/2}\d r
= 2\pi \cdot \frac 23\cdot \left.\frac{-(9-r^2)^{5/2}}5\right|_0^3=\frac{4\cdot 3^4\pi}{5}=\frac{324\pi}{5}.
\end{align*}
\item In den angepassten Kugelkoordinaten hat man 
\begin{align*}
\d(x,y,z) =& \left|\det\begin{pmatrix}
\cos\varphi\cos\psi & -r\sin\varphi\cos\psi & -r \cos\varphi\sin\psi\\
\frac 12 \sin\varphi\cos\psi& \frac r2 \cos\varphi\cos\psi & -\frac r2 \sin\varphi\sin\psi\\
\sin\psi& 0 & r\cos\psi\end{pmatrix}\right| \\
=& \frac 12\left|\sin\psi r^2(\sin\psi\cos\psi) +
r^2\cos\psi \cdot\cos^2\psi\right|\\
=& \frac {r^2}2 \cos\psi>0 \text{ (wegen $-\pi/2\leq \psi\leq \pi/2$)}.
\end{align*}
Damit ergibt sich 
\begin{align*}
I=& \int\limits_{\varphi=0}^{2\pi}\int\limits_{\psi=-\pi/2}^{\pi/2}\int\limits_{r=0}^3\left( r^2\cos^2\varphi\cos^2\psi
+ \frac r2 \sin\varphi\cos\psi + r^2\sin^2\psi\right)\frac{r^2}2\cos\psi\d r \d\psi\d\varphi\\
=& \frac 12\int\limits_{\varphi=0}^{2\pi}\int\limits_{\psi=-\pi/2}^{\pi/2}\left( \frac{3^5}5\cos^2\varphi\cos^2\psi
+ \frac {3^4}8\sin\varphi\cos\psi + \frac  {3^5}5\sin^2\psi\right)\cos\psi \d\psi\d\varphi\\
=& \frac 12\int\limits_{\psi=-\pi/2}^{\pi/2}\left( \frac{3^5}5\cdot \pi \cos^2\psi
+0+ \frac  {3^5}5\sin^2\psi\cdot 2\pi \right)\cos\psi \d\psi\\
=& \frac{3^5\pi}{10}\int\limits_{-\pi/2}^{\pi/2}(1+\sin^2\psi)\cos\psi\d\psi
= \frac{3^5\pi}{10}\left[\sin\psi+\frac 13\sin^3\psi\right]_{-\pi/2}^{\pi/2}=\frac{4\cdot 3^4\pi}{5}\frac{324\pi}{5}.\\
\end{align*}

\end{abc}
}

\ErgebnisC{AufganalysIntgKuko001}{
$\frac{324\pi}{5}$
}
