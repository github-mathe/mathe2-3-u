\Aufgabe[f]{Transformationsformel} {
\begin{abc}
\item Berechnen Sie 
 $	I := \int\limits_{\vec D}  (x^2\,y^4+x)  \d (x,y) \text{ mit }\vec  D := [-2,2]\times[1,3].$
in den Koordinaten 
$$\begin{pmatrix}u\\v\end{pmatrix}  = \begin{pmatrix}x+y\\x-y\end{pmatrix}.$$
\item Berechnen Sie f\"ur den Bereich $B$, der von der Kurve $r=\varphi,\, 0\leq \varphi\leq \pi$
(in Polarkoordinaten) und der x-Achse eingeschlossen wird, das Integral 
$$J:=\int\limits_{B}\frac{x}{\sqrt{x^2+y^2}}\d B.$$
F\"uhren Sie Ihre Rechnung in \textbf{Polarkoordinaten} durch.
\end{abc}
}


\Loesung{
\textbf{ a)} Zun\"achst skizzieren wir den Integrationsbereich $\vec D$ in den beiden
Koordinatensystemen: 
\begin{center}
\psset{xunit=1cm, yunit=1cm, runit=1cm}
\begin{pspicture}(-3,0)(3,4)
\psgrid[subgriddiv=1,griddots=15,gridlabels=.3](-3,0)(3,4)
\psline[fillstyle=solid, fillcolor=gray, linewidth=1pt, linecolor=black]
(-2,1)(2,1)(2,3)(-2,3)(-2,1)
\put(0.4,1.4){$\vec D$}
\put(2.7,.2){$x$}
\put(-2.9,3.5){$y$}
\end{pspicture}
\end{center}

\begin{center}
\psset{xunit=1cm, yunit=1cm, runit=1cm}
\begin{pspicture}(-2,-6)(6,2)
\psgrid[subgriddiv=1,griddots=10,gridlabels=.3](-2,-6)(6,2)
\psline[fillstyle=solid, fillcolor=gray, linewidth=1pt, linecolor=black]
(-1,-3)(3,1)(5,-1)(1,-5)(-1,-3)
\put(2,-2){$\tilde{\vec D}$}
\put(5.7,-5.8){$u$}
\put(-1.9,1.5){$v$}
\psline[linestyle=dashed](-1,-6)(-1,-3)
\psline[linestyle=dashed](1,-6)(1,-1)
\psline[linestyle=dashed](3,-6)(3,1)
\psline[linestyle=dashed](5,-6)(5,-1)
\end{pspicture}
\end{center}
Wir parametrisieren $\tilde {\vec D}$ als Normalbereich bez\"uglich $u$: 
\begin{align*}
\tilde{\vec D} =& \{(u,v)^\top|\, -4-u\leq v\leq -2+u,\, -1\leq u\leq 1\}\cup \\
&\cup\{(u,v)^\top|\,-6+u\leq v\leq -2+u,\, 1\leq u\leq 3\}\cup\\
&\cup \{(u,v)^\top|\, -6+u\leq v\leq 4-u,\, 3\leq u\leq 5\}.
\end{align*}
F\"ur die Anwendung der Transformationsformel ben\"otigen wir außerdem die Determinante der
Jacobi-Matrix der Transformation $(u,v)^\top$: 
$$\det \vec J_{\vec u} = \det \begin{pmatrix}1 & 1 \\ 1 & -1\end{pmatrix} = -2.$$
Das Integral berechnet sich damit zu: 
\begin{align*}
I=&\int\limits_{D} (x^2y^4+x)\d(x,y) = \int\limits_{\tilde D} \left( x(u,v)^2
y(u,v)^4+x(u,v)\right)\frac{\d (u,v)}{|-2|}=I_1+I_2\end{align*}
mit 
$$I_1=\int\limits_{\tilde D} x(u,v)^2y(u,v)^4\frac{\d(u,v)}2\text{ und }I_2  = \int\limits_{\tilde
D} x(u,v)\frac{\d (u,v)}{2}.$$
Das einfachere Integral von beiden ist $I_2$: 
\begin{align*}
I_2=& \int\limits_{u=-1}^1\int\limits_{v=-4-u}^{-2+u} \frac {u+v}4 \d v  \d u
+ \int\limits_{u=1}^3\int\limits_{v=-6+u}^{-2+u} \frac{u+v}4 \d v \d u
+ \int\limits_{u=3}^5\int\limits_{v=-6+u}^{4-u}\frac{u+v}4 \d v \d u\\
=& \frac 18\left( \int\limits_{-1}^1 \left[(u+v)^2\right]_{v=-4-u}^{-2+u}\d u
+ \int\limits_{1}^3\left[(u+v)^2\right]_{v=-6+u}^{-2+u} \d u
+ \int\limits_{3}^5\left[(u+v)^2\right]_{v=-6+u}^{4-u}\right)\\
=& \frac 18 \left( \int\limits_{-1}^3(-2+2u)^2\d u - \int\limits_{-1}^1(-4)^2\d u
- \int\limits_{1}^5(-6+2u)^2\d u + \int\limits_{3}^54^2\d u\right)\\
=& \frac 1{8}\left( \frac 43\left.(-1+u)^3\right|_{-1}^3 - 32 - \frac
43 \left.(-3+u)^3\right|_{1}^5+32\right)=\frac{16-16  }{6}=0
\end{align*}
Also ist $I=I_1$: 
\begin{align*}
I=& \frac 12 \int\limits_{u=-1}^1\int\limits_{v=-4-u}^{-2+u} \left( \frac{u+v}2\right)^2 \left( \frac{u-v}2\right)^4
 \d v \d u +\\
&+ \frac 12\int\limits_{u=1}^3\int\limits_{v=-6+u}^{-2+u} \left( \frac{u+v}2\right)^2 \left( \frac{u-v}2\right)^4
 \d v \d u+\\
&+ \frac 12 \int\limits_{u=3}^5\int\limits_{v=-6+u}^{4-u} \left( \frac{u+v}2\right)^2 \left( \frac{u-v}2\right)^4
 \d v \d u\\
=& \frac 1{128}\Bigl(\int\limits_{u=-1}^1\Bigl[-(u+v)^2\frac{(u-v)^5}5
- \frac{2(u+v)(u-v)^6}{5\cdot 6} - \frac{2(u-v)^7}{5\cdot 6\cdot 7}\Bigr]_{v=-4-u}^{-2+u}\d u +\\
&\qquad\qquad +\int\limits_{u=1}^3\Bigl[\hdots\Bigr]_{v=-6+u}^{-2+u}\d u
+ \int\limits_{u=3}^5\Bigl[\hdots\Bigr]_{v=-6+u}^{4-u}\d u\Bigr)\\
=&\frac{1}{128}\Bigl( \int\limits_{-1}^3\Bigl[-(-2+2u)^2\frac{2^5}5-\frac{2(-2+2u)2^6}{30}-\frac{2\cdot
 2^7}{210}\Bigr]\d u+ \\
& \qquad - \int\limits_{-1}^1\Bigl[ -(-4)^2\frac{(4+2u)^5}5 - \frac{2(-4)(4+2u)^6}{30}-\frac{2(4+2u)^7}{210}\Bigr]\d u+\\
&\qquad -\int\limits_{1}^5\Bigl[ -(-6+2u)^2\frac{6^5}5
- \frac{2(-6+2u)6^6}{30}-\frac{2\cdot 6^7}{210}\Bigr]\d u+\\
&\qquad + \int\limits_{3}^5\Bigl[ -4^2\frac{(2u-4)^5}5-\frac{2\cdot 4 \cdot
 (2u-4)^6}{30}-\frac{2\cdot (2u-4)^7}{210}\Bigr]\d u\Bigr)
\end{align*}
und weiter
\begin{align*}
I=& \int\limits_{-1}^3\Bigl[-\frac{(-1+u)^2}5-\frac{(-1+u)}{15}-\frac{1}{105}\Bigr]\d u + \\
& - \int\limits_{-1}^1\Bigl[ -4\frac{(2+u)^5}5 + \frac{2(2+u)^6}{15}-\frac{(2+u)^7}{105}\Bigr]\d u+\\
& -\int\limits_{1}^5\Bigl[ -(-3+u)^2\frac{3^5}5- \frac{(-3+u)3^6}{15}-\frac{3^7}{105}\Bigr]\d u+\\
& + \int\limits_{3}^5\Bigl[ -\frac{4(u-2)^5}5-\frac{2 \cdot (u-2)^6}{15}-\frac{(u-2)^7}{105}\Bigr]\d
u\\
=&\frac 15 \Bigl( \left[-\frac{(-1+u)^3}3 - \frac{(-1+u)^2}{6} - \frac{-1+u}{21}\right]_{-1}^3 + \\
&\qquad -\left[ -\frac{2(2+u)^6}{3}  +\frac{2(2+u)^7}{21} - \frac{(2+u)^8}{21\cdot 8}\right]_{-1}^1
+ \\
&\qquad + \left[ \frac{3^5(-3+u)^3}{3} + \frac{3^5(-3+u)^2}{2} + \frac{3^6(-3+u)}{7}\right]_{1}^5+\\
&\qquad + \left[-\frac{2(u-2)^6}{3} - \frac{2(u-2)^7}{21} - \frac{(u-2)^8}{8\cdot
21}\right]_3^5\Bigr)\\
=& \frac 15 \Bigl( \left[-\frac{16}3-\frac{4}{21} \right] - \left[-\frac{2\cdot 3^6-2}3
+ \frac{2\cdot 3^7-2}{21}-\frac{3^8-1}{21\cdot 8}\right] + \\
&\qquad + \left[3^4\cdot 2^4+\frac{3^6\cdot 4}{7}\right] + \left[ -\frac{2(3^6-1)}3
- \frac{2(3^7-1)}{21}-\frac{3^8-1}{8\cdot 21}\right]\Bigr)\\
=& \frac 15 \Bigl( \frac{-16-2+2}3+\frac{-4+2+2}{21}+3^5\left( 2-2\right)+\frac{-2\cdot 3^6+4\cdot
3^6-2\cdot 3^6}{7}+\\
&\qquad + \frac{3^8-1-3^8+1}{8\cdot 21}+2^4\cdot 3^4\Bigr)\\
=& \frac 15 \Bigl( -\frac{16}3 +2^4\cdot 3^4\Bigr) = \frac{16}5\cdot \frac {3^5-1}3 = \frac{32\cdot 121}{15}=\frac{3872}{15}
\end{align*}
\textbf{ b)}  Der angegebene Bereich l\"asst sich besser in Polarkoordinaten parametrisieren als in
kartesischen: 
$$\tilde B = \{(r,\varphi)\in\R^2|\, 0\leq r\leq \varphi,\, 0\leq \varphi\leq \pi\}.$$
Die Determinante der Jakobimatrix der Koordinatentransformation 
$$\begin{pmatrix}x(r,\varphi)\\y(r,\varphi)\end{pmatrix}=\begin{pmatrix}r\cos\varphi\\r\sin\varphi\end{pmatrix}$$ 
ist
$$\det \begin{pmatrix}\cos\varphi&-r\sin\varphi\\ \sin\varphi&r\cos\varphi\end{pmatrix}=
r\cos^2\varphi + r\sin^2\varphi=r.$$
Damit berechnet sich das Integral zu:
\begin{align*}
J=& \int\limits_{\tilde B} \frac{r\cos \varphi}{\sqrt{r^2\cos^2\varphi+ r^2\sin^2\varphi}}|r|\d
(r,\varphi)
= \int\limits_{\varphi=0}^\pi \int\limits_{r=0}^\varphi r\cos\varphi \d r
d\varphi=\int\limits_{\varphi=0}^\pi \frac 12 \varphi^2\cos\varphi\d\varphi\\
=& \left.\frac {\varphi^2}2 \sin\varphi\right|_0^\pi
- \int\limits_0^\pi \varphi \sin\varphi \d\varphi = 0-\left.\varphi(-\cos\varphi)\right|_{0}^\pi
- \int\limits_0^\pi \cos\varphi\d\varphi = -\pi +0=-\pi. 
\end{align*}
}

\ErgebnisC{analysIntgTraf001}
{
\textbf{a)} $\frac{3872}{15}$,\, \textbf{b)} $-\pi$
}

