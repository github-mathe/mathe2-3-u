\Aufgabe[e]{Integration}{
\begin{abc}
\item Berechnen Sie folgende Integrale mittels partieller Integration: 
\begin{align*}
&\int x\cdot \sin x \,\d x,&\quad&\int \sin^2(x) \,\d x, &\quad
%  \int\frac{x^2+1}{\sqrt x}\,\d x
&\int x^2\EH{1-x} \,\d x\\
&\int \frac{x}{\cos^2 x}\,\d x,&&\int\limits_{0}^{\pi/4} \frac x{\cos^2 x}\,\d x
\end{align*}
\item Berechnen Sie folgende Integrale mittels einer geeigneten Substitution: 
\begin{align*}
&\int\limits_1^2 \frac{3x^2+7}{x^3+7x-2}\,\d x,&\quad&\int\limits_\pi^{3\pi/2}x^2\cos(x^3+2)\,\d x,&\quad&\int\limits_1^2 \frac
1x \EH{1+\ln x}\,\d x\\
&\int\limits_{1/4}^1 \EH{\sqrt x}\,\d x,&&\int\cosh^2 x \sinh x \,\d x
% ,&&\int\limits_{-1}^32x|x|\,\d x
\end{align*}
\end{abc}
}


\Loesung{
\begin{abc}
\item \begin{iii}
\item \begin{align*}
\int \underbrace{x}_{u}\underbrace{\sin x}_{v'} \,\d x =& \underbrace{x}_{u} \underbrace{(-\cos
x)}_{v}- \int \underbrace{1}_{u'} \cdot \underbrace{(-\cos x)}_{v}\\
=&-x \cos x + \int \cos x = -x\cos x + \sin x + C
\end{align*}
\item \begin{align*}
&&\int\underbrace{\sin x}_u \underbrace{\sin x}_{v'}  \,\d x=&\underbrace{\sin
x}_{u}\underbrace{(-\cos x)}_{v} - \int \underbrace{\cos x}_{u'}\underbrace{(-\cos x)}_{v}\,\d x\\
&&=&-\sin x\cos x +\int\underbrace{\cos x\cos x}_{=1-\sin^2 x}\,\d x\\
&&=& -\sin x \cos x +\int 1 \,\d x - \int \sin^2 x\,\d x\\
\Rightarrow &&2\int\sin^2 x \,\d x =& -\sin x \cos x + x + 2C\\
\Rightarrow &&\int\sin^2 x \,\d x =& \frac {x-\sin x \cos x}{2} + C
\end{align*}
Alternativ kann man die Beziehung \(\sin^2(x) = \frac 12 (1-\cos(2x))\) (siehe Formelsammlung) nutzen. Damit bekommt man:
\begin{align*}
 \int \sin^2(x) \,\d x = \frac 12 \int (1-\cos(2x)) \,\d x = \frac 12\left(x-\frac 12 \sin(2x)\right) + C
\end{align*}
Dies l\"asst sich mithilfe von \(\sin(x) \cos(x) = \frac 12 \sin(x)\) in die andere Darstellung der L\"osung umwandeln.


% \item \begin{align*}
% \int\underbrace{(x^2+1)}_u\underbrace{\frac 1{\sqrt x}}_{v'}\,\d x
% =& \underbrace{(x^2+1)}_u\underbrace{2\sqrt x}_{v}-\int\underbrace{2x}_{u'=u_2}\cdot \underbrace{2\sqrt
% x}_{v=v_2'}\,\d x\\
% =& 2(x^2+1)\sqrt x -\underbrace{2x}_{u_2}\cdot \underbrace{\frac 43
% x^{3/2}}_{v_2}+\int\underbrace{2}_{u_2'}\cdot \underbrace{\frac 43 x^{3/2}}_{v_2}\,\d x\\
% =&2\sqrt x\left(x^2+1-\frac 43 x^2\right) + \frac{16}{15}x^{5/2} + C
% =\frac 2{5}\sqrt x\left( x^2+5\right) + C
% \end{align*}

\item \begin{align*}
\int \underbrace{x^2}_{u}\underbrace{\EH{1-x}}_{v'}\,\d
x=& \underbrace{x^2}_u\underbrace{(-\EH{1-x})}_{v}-\int\underbrace{2x}_{u'=u_2}\underbrace{(-\EH{1-x})}_{v=v_2'}\,\d
x\\
=&
-x^2\EH{1-x}-\underbrace{2x}_{u_2}\underbrace{\EH{1-x}}_{v_2}+\int\underbrace{2}_{u_2'}\underbrace{\EH{1-x}}_{v_2}\,\d
x\\
=& -(x^2+2x)\EH{1-x}-2\EH{1-x} + C =-(x^2+2x+2)\EH{1-x} + C
\end{align*}
\item \begin{align*}
\int \underbrace{x}_u\underbrace{\frac 1{\cos^2 x}}_{v'}\,\d x = & \underbrace{x}_u\underbrace{\tan
x}_{v}-\int\underbrace{1}_{u'}\underbrace{\tan x }_v\,\d x\\
=&x\tan x + \ln |\cos x| + C
\end{align*}
\item $\bigl[x\tan x+\ln |\cos x|\bigr]_{x=0}^{\pi/4}=\frac \pi 4 + \ln \frac 1{\sqrt 2} -
0=\frac \pi 4 -\frac {\ln 2}{2}$
\end{iii}
\item\begin{iii}
\item Mit $y=x^3+7x-2$ und $\,\d y = (3x^2+7)\,\d x$ hat man: 
\begin{align*}
\int\limits_1^2\frac{3x^2+7}{x^3+7x-2}\,\d x=&\int\limits_{y(1)}^{y(2)}\frac 1 y \,\d y= \Bigl[ \ln
|y|\Bigr]_6^{20}=\ln \frac {20}6=\ln \frac {10}3
\end{align*}
\item Hier w\"ahlt man $y=x^3+2$ und  erh\"alt daraus $\,\d y = 3x^2 \, \d x$ und setzt ein: 
\begin{align*}
\frac 13 \int\limits_{\pi}^{3\pi/2} 3x^2\cos(x^3+2)\,\d x=& \frac
13 \int\limits_{y(\pi)}^{y(3\pi/2)}\cos(y)\,\d y\\
=&\frac 13\left( \sin\left( \left( \frac
{3\pi}2\right)^3+2\right)-\sin (\pi^3+2)\right)
\end{align*}
\item Mit $y=1+\ln x$ und $\,\d y = \frac{\,\d x}x$ hat man
\begin{align*}
\int\limits_1^2 \frac 1x \EH{1+\ln x} \,\d x =& \int\limits_{y(1)}^{y(2)} \EH{y}\,\d y=\EH{1+\ln
2}-\EH{1}=\EH{}
\end{align*}
\item Wir w\"ahlen $y=\sqrt x$. Damit folgt (f\"ur $x>0$):
$$x=y^2\,\Rightarrow \, \,\d x = 2y\,\d y$$
und schlie\ss{}lich
\begin{align*}
\int\limits_{1/4}^1\EH{\sqrt x}\,\d x=& \int\limits_{y(1/4)}^{y(1)} \EH y \cdot 2y \,\d
y= 2\int\limits_{1/2}^1 y\EH y\,\d y\\
=& \bigl[2y\EH y\bigr]_{1/2}^1 -2 \int\limits_{1/2}^1\EH y\,\d y\text{ (partielle Integration)}\\
=&2\EH{ }-\sqrt{\EH{ }}-2(\EH{ }-\sqrt{\EH { }})=\sqrt{\EH{ }}
\end{align*}
\item Mit $y=\cosh x$ und $\,\d y = \sinh x \,\d x$ hat man
\begin{align*}
\int\cosh^2 x \sinh x \,\d x=& \int y^2\,\d y = \frac{y^3}3 + C = \frac{\cosh^3 x}3 + C
\end{align*}
% \item Wir stellen den Betrag dar als $|x|=\sqrt{x^2}$ und substituieren $y=x^2$. Die Ableitung ist
% $$\frac{\,\d y}{\,\d x} = 2x,$$
% sie wechselt ihr Vorzeichen bei $x=0$. Also muss das Integral aufgeteilt werden, um die Substitution
% doch zu verwenden: 
% $$\int\limits_{-1}^3 2x|x|\,\d x=\int\limits_{-1}^02x|x|\,\d x + \int\limits_0^3 2x|x|\,\d x.$$
% Auf jedem Teilintervall ist $y(x)$ monoton. Nun kann man substituieren: 
% \begin{align*}
% \int\limits_{-1}^3 2x|x|\,\d x =& \int\limits_{y(-1)}^{y(0)}\sqrt{y}\,\d y
% + \int\limits_{y(0)}^{y(3)}\sqrt y  \,\d y\\
% =& \left[\frac 23 y^{3/2}\right]_1^0 +\left[ \frac 23 y^{3/2}\right]_0^9=\frac 23 \left(
% 0-1+27-0\right)=\frac{52}3
% \end{align*}
\end{iii}
\end{abc}
}

\ErgebnisC{AufganalysInteGral001}
{
{ a)} $-x\cos x + \sin x + C$, $\frac {x-\sin x \cos x}{2} + C$, 
$-(x^2+2x+2)\EH{1-x} + C$, $x\tan x + \ln |\cos x| + C$, $\frac{\pi}4-\frac{\ln 2}2$\\
{ b)} $\ln \frac{10}3$, $\frac 13\left( \sin \frac{27\pi^3+16}8-\sin (\pi^3+2)\right)$, $\EH{1+\ln
2}-\EH{ }$, $\sqrt{\text{e}}$, $\frac 13 \cosh^3 x + C$
}


