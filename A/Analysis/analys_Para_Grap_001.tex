\Aufgabe[e]{Tangenten}{
\begin{abc}
\item Bestimmen Sie die Geradengleichungen (in der Form $ax+by=c$) der Tangenten an den Nullstellen
der Funktionen 
$$\varphi_1(x)=x^3-2x^2-5x+6, \qquad \varphi_2(x)=x^2 \text{ und } \varphi_3(x)=\left\{\begin{array}{lll}
\sqrt x& \quad & \text{ f\"ur } x\geq 0\\
-\sqrt{-x}& \quad & \text{ f\"ur } x<0\end{array}\right..
$$

\item Geben Sie alle Punkte ($x-$Werte) an, in denen die Tangenten von $f(x)=x^2$ und $g(x)=x^3$ parallel sind. 

\item Stellen Sie den Verlauf der beiden vektorwertigen Funktionen 
$$\vec v(t)=\begin{pmatrix} t\\ \varphi_3(t)\end{pmatrix} \qquad\text{ mit $\varphi_3(t)$ aus
Aufgabenteil \textbf{a)}}$$
und 
$$\vec w(s)=\begin{pmatrix}s\cdot|s|\\s\end{pmatrix}$$
im selben Graphen dar. \\
Berechnen Sie, wo m\"oglich, die Ableitungen beider Funktionen $\vec v(t)$ und $\vec
w(s)$. \\

\textbf{Hinweis}: Die Funktion $\vec w(s)$ l\"asst sich analog zu $\varphi_3(t)$ mit Hilfe einer
Fallunterscheidung auch ohne die Betragsfunktion darstellen. 

\end{abc}
}

\Loesung{
\begin{abc}
\item \begin{iii}
\item Eine Nullstelle der Funktion $\varphi_1(x)$ ist $x_1=1$. Mittels Horner-Schema erh\"alt man
das Restpolynom: 
$$\begin{array}{r|r|r|r|r}
   &   1     &  -2 &  -5  &  6  \\\hline
1  &\setminus&   1 &  -1  & -6  \\\hline
   &   1     &  -1 &  -6  &  0  \\\end{array}$$
F\"ur weitere Nullstellen muss also gelten 
$$1x^2-1x-6=0\qquad\Rightarrow\qquad x_{2/3}=\left\{\begin{array}{r}-2\\3\end{array}\right..$$
Die Ableitung der Funktion liefert die Steigung der Tangenten. An den Nullstellen $x_1,\, x_2$ und
$x_3$ hat man  
$$\varphi_1'(x)=3x^2-4x-5\,\Rightarrow\, \varphi_1'(1)=-6,\, \varphi_1'(-2)=15,\, \varphi_1'(3)=10.$$
Damit sind die Tangentengeraden gegeben durch:
\begin{align*}
x_1=1:&\qquad& y&=-6x+6&\,\Rightarrow\, &&6x+y&=6\\
x_2=-2:&\qquad& y&=15x+30&\,\Rightarrow\, &&15x-y&=-30\\
x_3=3:&\qquad& y&=10x-30&\,\Rightarrow\, &&10x-y&=30\\
\end{align*}
\item $\varphi_2(x)$ hat nur die Nullstelle $x_0=0$. Die Ableitung $\varphi_2'(x)=2x$ hat dort den
Wert Null, so dass die Tangente durch 
$$y=0$$
gegeben ist. 
\item Die Funktion $\varphi_3(x)$ hat nur die Nullstelle $\varphi_0=0$. Dort ist $\varphi_3$ nicht
differenzierbar: 
\begin{align*}
\frac{\varphi_3(x)-\varphi(0)}{x-0}&=\left\{\begin{array}{lll}
\frac{\sqrt{x}-0}x,&\quad&x>0\\
\frac{-\sqrt{-x}-0}x,&&x<0\end{array}\right.\\
&=\left\{\begin{array}{lll}
\frac{1}{\sqrt x},&\quad&x>0\\
\frac{-\sqrt{-x}-0}{-(\sqrt{-x})^2},&&x<0\end{array}\right.\\
&=\frac 1{\sqrt{|x|}}\underset{x\to 0}{\longrightarrow}\infty. 
\end{align*}
Die Steigung der Funktion ist dort also unendlich und sie hat die senkrechte Tangente: 
$$x=0.$$
\end{iii}
\item Die beiden Funktionen haben im Punkt $x$ parallele Tangenten, wenn ihre Ableitungen dort denselben Wert annehmen: 
$$f'(x)=g'(x)\,\Leftrightarrow\, 2x=3x^2$$
Dies ist f\"ur $x=0$ oder f\"ur $x=\frac 23$ der Fall.
\item Die Funktionsgraphen beider Funktionen stimmen \"uberein. 
\end{abc}
\begin{center}
\begin{pspicture}(-4,-2)(4,2)
\psgrid[subgriddiv=5,griddots=1,gridlabels=.3](-4,-2)(4,2)
\psparametricplot[plotstyle=curve, plotpoints=100]
{0}{4}
{t
t sqrt}
\psparametricplot[plotstyle=curve, plotpoints=100]
{-4}{0}
{t
t neg sqrt neg}

\psparametricplot[plotpoints=100, plotstyle=curve]
{-2}{2}
{t t mul sqrt t mul
t}
\end{pspicture}
\end{center}
In Aufgabenteil \textbf{a)} wurde bereits gezeigt, dass $\varphi_3(t)$ und damit auch $\vec v(t)$ an der Stelle $t=0$ nicht differenzierbar ist. In allen anderen Punkten hat man: 
$$\vec v'(t)=\begin{pmatrix}\frac{\d t}{\d t}\\ \varphi_3'(t)\end{pmatrix}=\begin{pmatrix}1\\\frac{1}{2\sqrt{|t|}}\end{pmatrix}.$$
Durch die Umparametrisierung der Kurve hat man eine differenzierbare Funktion $\vec w(s)$ mit
\begin{align*}
\vec w'(s)&=\begin{pmatrix}2s\\1\end{pmatrix}&\text{ f\"ur } s>0\\
\vec w'(s)&=\begin{pmatrix}-2s\\1\end{pmatrix}&\text{ f\"ur } s<0
\end{align*}
F\"ur $s=0$ ergibt sich f\"ur die erste Komponente $w_1(s)$: 
\begin{align*}
\frac{w_1(s)-w_1(0)}{s-0}&=\left\{\begin{array}{lll}\frac{s^2-0}{s}&\quad&\text{ f\"ur }s>0\\
\frac{-s^2-0}{s}&\quad&\text{ f\"ur }s<0\end{array}\right.\\
&=\left\{\begin{array}{lll}{s}&\quad&\text{ f\"ur }s>0\\
{-s}&\quad&\text{ f\"ur }s<0\end{array}\right.\\
&\underset{s\to 0 }\longrightarrow 0
\end{align*}
Damit hat man 
$$\vec w'(0)=\begin{pmatrix}0\\1\end{pmatrix}$$
und insgesamt
$$\vec w'(s)=\begin{pmatrix}2|s|\\1\end{pmatrix}.$$

}

\ErgebnisC{AufganalysParaGrap001}
{
\textbf{a)} Nullstellen: $\varphi_1(1)=\varphi_1(-2)=\varphi(3)=0$, $\varphi_2(0)=0$, $\varphi_3(0)=0$
}
