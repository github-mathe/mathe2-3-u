\Aufgabe[e]{Alte Klausuraufgabe}{
\begin{abc}
\item Berechnen Sie das Integral $\iint_D\frac{x^2}{y^2}\mathrm{d} x\mathrm{d} y$, wobei 
$D$ den von den Geraden $x=2$, $y=x$ und der Hyperbel $xy=1$ begrenzten Bereich 
des $\R^2$ bezeichne.
 
%\item Berechnen Sie das Integral 
%$$
%I = \int_{B} x^2 \d B\,.
%$$
%Dabei ist $B$ jener Bereich, der von den beiden Zylindern $x^2+z^2=1$ und $x^2 
%+ y^2 = 1$ \ eingeschlossen wird. 

\item Gegeben sei der K\"orper
$$ 
K = \big\{ (x,y,z) \in \R^3 \, | \, x^2+y^2-z^2 \le 1, \ -1 \le z \le 2 \big\} 
\,. 
$$
Skizzieren Sie den K\"orper und berechnen Sie dessen Volumen.

\end{abc}

}

\Loesung{
\begin{abc}
\item Der Integrationsbereich hat die folgende Gestalt: \\
\end{abc}
\begin{center}
\begin{pspicture}(-1,-1)(3,3)
\psline{->}(0,-1)(0,3)
\psline{->}(-1,0)(3,0)
\put(.1,2.6){$y$}
\put(2.6,.1){$x$}

\psplot[plotpoints=100, plotstyle=curve, fillstyle=solid, fillcolor=gray]
{1}{2}
{
1 x div
}

\psline[fillstyle=solid, fillcolor=gray, linecolor=gray](1,1)(2,.5)(2,2)(1,1)

\psplot[plotpoints=100, plotstyle=curve]
{.35}{3}
{
1 x div
}
\put(.5,2.5){$xy=1$}
\psline(2,-1)(2,3)
\put(2.1,1.6){$x=2$}
\psline(-1,-1)(3,3)
\put(2.6,2.2){$x=y$}

\end{pspicture}
\end{center}

$D$ ist Normalbereich bez\"uglich $x$,
$$
D =\left\{\begin{pmatrix}x\\y\end{pmatrix}\in\R^2: \frac{1}{x}\leq y \leq x\text{,}\quad 1\leq x\leq 
2\right\}\,.
$$
Es gilt
\begin{align*}
\iint_D\frac{x^2}{y^2}\mathrm{d} x\mathrm{d} y &= \int_1^2\left(\int_{1/x}^x 
\frac{x^2}{y^2}\mathrm{d}y\right)\mathrm{d} x\\[1ex]
 &= \int_1^2 x^2 \cdot \left[\frac{-1}{y}\right]_{y = 1/x}^{y=x} \mathrm{d} x\\[1ex]
 &= \int_1^2 \left(-x+x^3\right)\mathrm{d} x = \frac{9}{4}\,.
\end{align*}
\begin{abc}\setcounter{enumi}{1}
%\item Der Bereich $B$ ist ein Normalbereich mit
%$$
%-\sqrt{1-x^2}\leq z \leq \sqrt{1-x^2}\,, \qquad -\sqrt{1-x^2}\leq y \leq 
%\sqrt{1-x^2}\,, \qquad -1 \leq x \leq 1\,.
%$$
%Damit folgt
%\begin{align*}
%I & = \int_{x=-1}^{1}  \int_{y=-\sqrt{1-x^2}}^{y=\sqrt{1-x^2}} 
%\int_{z=-\sqrt{1-x^2}}^{z=\sqrt{1-x^2}} x^2 \d z \d y\d x\\[3ex] 
%& =  2 \int_{x=-1}^{1}  \int_{y=-\sqrt{1-x^2}}^{y=\sqrt{1-x^2}} x^2 
%\sqrt{1-x^2} 
%\d y\d x \\[3ex]
%& = 4 \int_{x=-1}^{1} x^2 \sqrt{1-x^2} \sqrt{1-x^2}\d x = 4 \int_{x=-1}^{1} 
%(x^2-x^4) \d x\\[3ex]
%& = 8 \left(\dfrac{1}{3}- \dfrac{1}{5}\right) = \boxed{\dfrac{16}{15}\,.}
%\end{align*}

\item Die Ungleichung in der Definition des Integrationsgebietes
 l\"asst sich schreiben als 
$$ x^2+y^2 \le 1 + z^2. $$
Skizze: 

\end{abc}

\begin{center}
\begin{pspicture}(-3,-2)(3,3)
\psline{->}(-3,0)(3,0)
\put(2.6,.1){$x$}
\psline{->}(0,-2)(0,3)
\put(.1,2.7){$z$}

\psline{->}(-3,-.8)(3,.8)
\put(2.75,.9){$y$}

\psparametricplot[plotpoints=100, plotstyle=curve]
{-1}{2}
{
1 t t mul add sqrt
t
}
\psparametricplot[plotpoints=100, plotstyle=curve]
{-1}{2}
{
1 t t mul add sqrt neg
t
}

\psellipse(0,-1)(1.44,.4)
\psellipse(0,2)(2.24,.6)
\psellipse(0,0)(1,.27)
\end{pspicture}
\end{center}
Das Volumen berechnet man in Zylinderkoordinaten, 
$$ x = r \cos \varphi, \quad \ y = r \sin \varphi, \quad z = z . $$
Die zugeh\"orige Funktionaldeterminante ist $ \left| 
\dfrac{\partial(x,y,z)}{\partial(r,\varphi,z)}\right| = r$. Das Volumen ist:
\begin{align*}
  \int_{z=-1}^2 \int_{\varphi = 0}^{2\pi} \int_{r=0}^{\sqrt{1+z^2}}
	r \, \mathrm{d}r \mathrm{d}\varphi \d z
  & =  \int_{z=-1}^2 2\pi \frac{1+z^2}{2} \mathrm{d} z\\[2ex]
  & = \pi \left( 3 + \left[ \frac{z^3}{3} \right]_{z=-1}^2 \right) \\[2ex]
  & =  \pi \left( 3 + \frac{8}{3} + \frac{1}{3} \right) =  6 \pi\,.
\end{align*}
}

\ErgebnisC{analysIntgNdim020}
{
a) $9/4$, b) %$16/15$, c) 
$6\pi$
}
