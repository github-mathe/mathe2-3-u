\Aufgabe[e]{Fehlersuche}{
Behauptung:
	\[
	\int\limits_{-2}^{2} x^2\;\text dx = 0\;
\]
Beweis: \ Mit der Substitution \ \ $t = x^2 \,\Rightarrow\, \text dt = 2x\,\text dx$ \ \ gilt:
	\[
	\int\limits_{-2}^{2} x^2\;\text dx = \dfrac 12 \int\limits_{-2}^{2} x\cdot 2x\;\text dx = \dfrac 12 \int\limits_{4}^{4} \sqrt{t}\;\text dt = 0\;.
\]
Wo steckt der Fehler?
}

\Loesung{
Der Fehler liegt bei der \glqq naiven\grqq\ Ersetzung von \ $x$ \ durch \ $\sqrt{t}$\;. Die Umkehrung von \ $t=x^2$ \ ist:
	\[
	\begin{array}{lcl}
	x = -\sqrt t & \text{für} & x<0 \\
	x = +\sqrt t & \text{für} & x\geq 0 
	\end{array} \;.
\]
Damit gilt
	\[
	\begin{array}{rcl}
	\int\limits_{-2}^{2} x^2\;\text dx & = & \dfrac 12\int\limits_{-2}^{0} x\cdot 2x\;\text dx + \dfrac 12\int\limits_{0}^{2} x\cdot 2x\;\text dx \\
	& & \\
	& = & \dfrac 12\int\limits_{4}^{0} -\sqrt{t}\;\text dt + \dfrac 12\int\limits_{0}^{4} +\sqrt{t}\;\text dt \\
	& & \\
	& = &  \int\limits_{0}^{4} \sqrt{t}\;\text dt \;=\; \left[\dfrac 23\,t^{3/2}\right]_0^4 \;=\; \dfrac 23\cdot 8 \;=\; \dfrac{16}{3}\ ,
	\end{array}
\]
was das richtige Ergebnis ist.
}

% \ErgebnisC{wz-2013-1-1}
% {
% 
% }
