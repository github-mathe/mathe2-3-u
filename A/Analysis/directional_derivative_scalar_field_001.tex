\Aufgabe[e]{Tangentialebene und Richtungsableitung}{
Gegeben sei die multivariate Funktion 
$$f (x, y, z) = 3x^3y - 2xy^2 - z.$$ 

\begin{abc}
\item Bestimmen Sie die Gleichung der Tangentialebene zu der Äquipotentialfläche
$$f(x,y,z) = 0$$
in dem Punkt auf der Oberfläche mit den Koordinaten $x=2$ und $y=1$.
\item Berechnen Sie die Richtungsableitung von $f$ in die Richtung 
      des Vektors \\$\vec v = (1, 2, 1)^T$ in dem Punkt $\vec P = (1, 1, 1)^T$.
\end{abc}

}
%
\Loesung{
\begin{abc}
\item
Wir schreiben 
$$z=F(x,y) = 3x^3y-2xy^2.$$
Der Ausdruck der Tangentialebene ist
\begin{align*}
z &= z_0 + F_x(x_0,y_0) (x-x_0) + F_y(x_0,y_0)(y-y_0),\\
z &= F(x_0,y_0) + F_x(x_0,y_0) (x-x_0) + F_y(x_0,y_0)(y-y_0),
\end{align*}

wobei $F_x(x,y) = 9x^2y-2y^2$ und $F_y(x,y)=3x^3-4xy$.

Die Gleichung der Ebene ist dann:
%
\begin{align*}
z &= 20 + 34 (x-2) + 16(y-1),\\
z &= 34x+16y-64.
\end{align*}

\item Für die Richtungsableitung benötigen wir den Gradienten von $f$:
$$
\nabla f(x,y,z) = \begin{pmatrix}
9x^2y-2y^2\\
3x^3-4xy\\
-1
\end{pmatrix},
$$

Wir werten den Gradienten in dem Punkt $(1,1,1)$ aus
$$
\nabla f(1,1,1) = \begin{pmatrix}
7\\
-1\\
-1
\end{pmatrix},
$$
\end{abc}
Damit ist die Richtungsableitung 
\begin{align*}
\frac{\partial f}{\partial \vec v} &= \langle \nabla f, \frac{\vec v}{\|\vec v\|} \rangle, \\
&= (7,-1,-1)^T \cdot \frac{(1,2,1)^T}{\|\sqrt{6}\|} = \frac{4}{\sqrt{6}}.
\end{align*}


}

