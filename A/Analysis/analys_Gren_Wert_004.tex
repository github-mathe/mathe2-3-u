\Aufgabe[e]{Grenzwertdefinition}{
Bestimmen Sie zu den unten angegebenen Folgen $(a_n)$ mit dem Grenzwert $a$ und die angegebenen
Werte f\"ur $k$ jeweils ein $N$ so, dass f\"ur alle $n\in \N$ mit $n>
N$ gilt 
$$|a_n-a|< 10^{-k}.$$
\begin{abc}
\item $a_n=\dfrac{1}{\sqrt n}$, $a=0$, $k=2$
\item $a_n=\dfrac{3n+1}{ n+1}$, $a=3$, $k=4$
\item $a_n=\dfrac{(-1)^n}{n!}+1$, $a=1$, $k=3$
\end{abc}
}

\Loesung{
\begin{abc}
\item Es soll gelten $|a_n-a|=\frac 1{\sqrt n}\overset !<10^{-2}$. Dies l\"asst sich umstellen zu: 
$$n>\left(\frac 1{10^{-2}}\right)^2=10^4=10000=N.$$
\item Hier ergibt sich 
\begin{align*}
&&|a_n-a|&<10^{-4}\\
\Leftrightarrow&& 10^{-4}&>\left|\frac{3n+1}{n+1}-3\right|=\frac{|-2|}{n+1}\\
\Leftrightarrow&& \frac{n+1}2&>10^4\\
\Leftrightarrow&& n&> 20000-1=19999=N. 
\end{align*}
\item F\"ur diese Folge ist $|a_n-a|=\left|\frac{(-1)^n}{n!}\right|=\frac 1{n!}$ Mit $k=3$ soll also
gelten: 
$$\frac 1{n!}<10^{-3}\quad\Leftrightarrow\quad n!>1000.$$
Das ist beispielsweise erf\"ullt f\"ur $n>1000=N$. Ein kleinstm\"ogliches $N$ kann man durch die
Untersuchung von $n!$ ermitteln. Es ist $6!=720<1000$ und $7!=7\cdot 6!=5040>1000$. 
Die Bedingung ist also bereits f\"ur $n>6$ erf\"ullt. 
\end{abc}
}

\ErgebnisC{analysGrenWert004}
{
\textbf{a)} $N=10000$, \textbf{b)} $N=19999$, \textbf{c)} $N=6$ 
}
