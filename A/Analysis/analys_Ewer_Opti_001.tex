\Aufgabe[e]{Optimierungsaufgabe}{
Ein Elektrizit\"atswerk (E) liegt an einem 400 Meter breiten geradlinig verlaufenden Fluss. Es soll
eine Leitung zu einem 1800 Meter flussabw\"arts auf der anderen
Flussseite gelegenen Haus (H) verlegt werden. \\
Ein Meter Leitung zu Wasser kosten das dreifache eines Meters Leitung zu Land. Welche Leitungsf\"uhrung verursacht die geringsten Kosten?
\begin{center}
\psset{xunit=1cm, yunit=1cm, runit=1cm}
\begin{pspicture}(-1,-1)(10,3)
%\psgrid[subgriddiv=1,griddots=10,gridlabels=0](-1,-6)(5,6)
\psline[fillstyle=solid, fillcolor=lightgray, linewidth=0pt, linecolor=black](-1,0)(10,0)(10,2)(-1,2)(-1,0)
%\psline(-1,0)(10,0)
%\psline(2,0)(2,0)
\psline{<->}(0,0)(0,2)
\psline[linewidth=2pt, linestyle=dashed, linecolor=gray](0,2)(2,2)(5,0)(9,0)
\psdot(0,2)
\put(.0,2.1){E}
\put(.1,1){400m}
\psdot(9,0)
\put(8.9,-.4){H}
\psline{<->}(0,-.5)(9,-.5)
\put(4,-.4){1800m}
\psline[linewidth=2pt, linestyle=dashed](0,2)(3,0)(9,0)
\end{pspicture}
\end{center}
}

\Loesung{
Der g\"unstigse Weg der Leitung wird aus zwei geradlinigen Leiungsabschnitten bestehen, von denen
einer l\"angs des Flusses verl\"auft und einer den Fluss \"uberquert. (Tats\"achlich wird in diesem
Modell auch eine Leitung aus drei Abschnitten, die nach einem geradlinigen Leitungsst\"uck auf der Flussseite des
E-Werkes den Fluss \"uberquert und danach ein weiteres St\"uck auf der Uferseite des Hauses
verl\"auft, dieselben Kosten verursachen.) 
Die Kosten der zweigeteilten Leitungstrasse mit $x$ Metern Strecke zu Land betragen: 
$$K(x)=x+3\cdot\sqrt{400^2+(1800-x)^2}.$$
Diese Funktion muss minimiert werden, es soll also gelten: 
\begin{align*}
&&0&=K'(x)
= 1+\frac{-3(1800-x)}{\sqrt{400^2+(1800-x)^2}}\\
\Leftrightarrow&&9(1800-x)^2&=400^2+(1800-x)^2\\
\Leftrightarrow&&x=&1800- \frac{400}{\sqrt 8}
 = 200(9-\frac{1}{\sqrt 2}) \approx 1660
\end{align*}
Der g\"unstigste Weg verl\"auft also zu etwa 1660 Metern an Land und zu 420 Metern zu Wasser. 

}

\ErgebnisC{analysisEwerOpti001}
{
Die Gesamtl\"ange der Stromleitung betr\"agt ca. 2080 Meter. 
}
