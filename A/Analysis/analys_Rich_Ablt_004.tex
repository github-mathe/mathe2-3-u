\Aufgabe[e]{Richtungsableitungen} {
Gegeben sei die folgende vektorwertige Funktion $$\vec f(x,y) = (\operatorname{e}^{xy}, x^2+y^3)^\top.$$ 
\begin{iii}
\item Berechnen Sie die Jacobi-Matrix von $\vec f$ in dem Punkt $\vec P_1=(1,2)^\top$.
\item Berechnen Sie die Richtungsableitung von $\vec f$ in Richtung $\vec v=(1,0)^\top$ in dem Punkt $\vec P_2=(1,1)^\top$.
\end{iii}

}


\Loesung{
\begin{iii}
\item Die Jacobi-Matrix ist
$$
\vec J(\vec x) = 
\begin{pmatrix} 
f_{1,x} & f_{1,y}\\
f_{2,x} & f_{2,y}
\end{pmatrix}
= \begin{pmatrix} 
y\operatorname{e}^{xy} & x\operatorname{e}^{xy}\\
2x & 3y^2
\end{pmatrix}
$$

Die Jacobi-Matrix ausgewertet in dem Punkt $\vec P_1=(1,2)^\top$ ist
$$
\vec J(\vec P_1) 
= \begin{pmatrix}
2\operatorname{e}^{2} & \operatorname{e}^{2}\\
2 & 12
\end{pmatrix}
$$
\item Um die Richtungsableitung zu bestimmen, ben\"otigen wir einen Einheitsvektor. Da der gegebene Vektor $\vec v$ 
bereits Einheitsl\"ange hat, kann die Richtungsableitung berechnet werden durch
$$
\frac{\partial \vec f}{\partial \vec h}(\vec x) = \vec J(x) \vec v = (y\operatorname{e}^{xy}, 2x)^\top
$$

Die Richtungsableitung ausgewertet in dem Punkt $\vec P_2$ ist
$$
\frac{\partial \vec f}{\partial \vec h}(\vec P_2) = (\operatorname{e}, 2)^\top.
$$
\end{iii}
}


\ErgebnisC{AufganalysRichAblt003}
{
$\frac{-4x+6y+2z}{\sqrt{29}}$,\qquad $\frac{-4x\sin y + 3 x^2\cos y - 4\sin z}{\sqrt{29}}$
}
