\Aufgabe[e]{Taylor-Entwicklung}{
Gegeben sei die Funktion
$$
f(x) = \sin(x)\ln(x).
$$
\begin{abc}
%\item Determine the Taylor polynomial of the second order $T_2(x)$ of $f(x)$ about the point $x=1$.
%\item Determine the difference between the Taylor polynomial $T_2(x)$ and the function $f(x)$ at the point $x=0$, i.e. determine $d(0)$ where
\item Geben Sie den maximalen Definitionsbereich von $f(x)$ an.
\item Kann die Funktion $f(x)$ im Punkt $x=0$ stetig fortgesetzt werden? Wenn ja, geben Sie die stetige Fortsetzung an.
\item Bestimmen Sie das Taylor-Polynom zweiter Ordnung $T_2(x)$ von $f(x)$ um den Punkt $x=1$.
\item Bestimmen Sie die Differenz zwischen dem Taylor-Polynom $T_2(x)$ und der Funktion $f(x)$ im Punkt $x=0$, d.h. bestimmen Sie $d(0)$, wobei
$$d(x):= |T_2(x) - f(x)|.$$
%Note that the function $f(x)$ has to be extended by continuity at $x=0$.
Man beachte, dass die Funktion $f(x)$ an der Stelle $x=0$ stetig fortgesetzt werden muss.
\end{abc}
}

\Loesung{
\begin{abc}
\item Die Funktion $f(x)$ ist auf $\mathbb{R}\setminus\{0\}$ definiert.
\item 
Wir müssen den Grenzwert $\lim_{x\to 0}f(x)$ berechnen, da die Funktion $\sin(x)\ln(x)$ in 0 nicht definiert ist. Wenn der Grenzwert existiert, erweitern wir die Funktion um den Wert des Grenzwertes. 
\begin{align*}
\lim_{x\to 0} \sin(x) \ln(x) &= \lim_{x\to 0} \frac{\ln(x)}{\frac{1}{\sin{x}}} \\
&= \lim_{x\to 0} -\frac{1}{x} \frac{\sin^2(x)}{\cos(x)} \\
&= \lim_{x\to 0} -\frac{\sin(x)}{x}\lim_{x\to 0}\sin(x) \lim_{x\to 0} \frac{1}{\cos(x)} \\
&= \lim_{x\to 0} -\frac{\cos(x)}{1}\lim_{x\to 0}\sin(x) \lim_{x\to 0} \frac{1}{\cos(x)} 
&= -1 \cdot 0 \cdot 1 = 0.
\end{align*}
Wir erweitern die Funktion bei $x=0$ durch Stetigkeit mit dem Grenzwert $f(0)=0$. \\
  \begin{align*}
  \quad f(x) = \begin{cases} \sin(x)\ln(x) & \text{f\"ur} \, x > 0, \\
                       0 & \text {f\"ur}\, x=0,\end{cases}\\
\end{align*}
\item
Die Ableitungen von $f(x)$ sind: 
$$f'(x)=\cos(x)\ln(x) + \frac{\sin(x)}{x},$$
$$f''(x) =  -\sin(x)\ln(x) + 2\frac{\cos(x)}{x} - \frac{\sin(x)}{x^2} .$$
Das Taylor-Polynom ist
\begin{align*}
T_2(x;1) &= f(1) + f'(1) (x-1) +\frac12 f''(x) (x-1)^2\\
&= \sin(1) (x-1) + \left(\cos(1) - \frac{\sin(1)}{2}\right)(x-1)^2
\end{align*}
\item Die Differenz ist
$$d(x):= |T_2(x) - f(x)| = |\sin(x) \ln(x) - \sin(1) (x-1) - \left( \cos(1) -\frac{\sin(1)}{2}\right) (x-1)^2|.$$
%We have to compute the limit $\lim_{x\to 0}d(x)$ because the function $\sin(x)\ln(x)$ is not defined in 0. If the limit exists we extend the function with the value of the limit.
%We extend the function at $x=0$ by continuity with the limit value $f(0)=0$.
%The difference is
Die Differenz ausgewertet im Punkt $x=0$ ist
$$
d(0) = \left|\sin(1) + \frac{\sin(1)}{2} - \cos(1)\right| = \frac{3\sin(1)}{2} - \cos(1).
$$
\end{abc}


}

\ErgebnisC{analysTayl01}
{
Die Differenz ist
$$
d(0) = \frac{3\sin(1)}{2} - \cos(1).
$$

}
