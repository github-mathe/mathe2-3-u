\Aufgabe[e]{Station\"are Punkte}{
\begin{abc}
\item Berechnen Sie jeweils den Gradienten $ \nabla f(x, y) $.
\item Bestimmen Sie jeweils die kritischen Punkte von $ f $ ohne sie zu klassifizieren, indem Sie $ \nabla f(x, y) =0$ setzen.
%\item Klassifiziere jeden kritischen Punkt als lokales Minimum, lokales Maximum oder Sattelpunkt.
\end{abc}

\begin{itemize}
\item[i)] $ f(x, y) = x^2 + xy + y^2 - 3x - 3y $
\item[ii)] $ f(x, y) = x^3 - 3xy + y^3 $
\item[iii)] $ f(x, y) = x^4 + y^4 - 4xy $
\item[iv)] $ f(x, y) = x^4 - 2x^2 + y^4 - 2y^2 $
\end{itemize}
}

\Loesung{

\begin{itemize}
\item[i)]
\begin{enumerate}
\item \textbf{Berechnung des Gradienten:}

Der Gradient von $ f $ ist:
\[
\nabla f(x, y) = \left( \frac{\partial f}{\partial x}, \frac{\partial f}{\partial y} \right)
\]

Berechnung der partiellen Ableitungen:
\[
\frac{\partial f}{\partial x} = 2x + y - 3
\]
\[
\frac{\partial f}{\partial y} = x + 2y - 3
\]

Daher:
\[
\nabla f(x, y) = (2x + y - 3, x + 2y - 3)
\]

\item \textbf{Ermittlung der kritischen Punkte:}

Kritische Punkte treten dort auf, wo der Gradient Null ist:
\[
2x + y - 3 = 0
\]
\[
x + 2y - 3 = 0
\]

Lösung dieses linearen Gleichungssystems:
\[
\begin{cases}
2x + y = 3 \\
x + 2y = 3
\end{cases}
\]

Der kritische Punkt ist also $ (1, 1) $.
%
%\item \textbf{Klassifizierung des kritischen Punkts:}
%
%Um den kritischen Punkt zu klassifizieren, untersuchen wir die zweiten partiellen Ableitungen:
%\[
%f_{xx} = \frac{\partial^2 f}{\partial x^2} = 2
%\]
%\[
%f_{yy} = \frac{\partial^2 f}{\partial y^2} = 2
%\]
%\[
%f_{xy} = \frac{\partial^2 f}{\partial x \partial y} = 1
%\]
%
%Die Hessische Determinante $ D $ ist:
%\[
%D = f_{xx} f_{yy} - (f_{xy})^2 = (2)(2) - (1)^2 = 4 - 1 = 3
%\]
%
%Da $ D > 0$ und $ f_{xx} > 0$ ist, hat die Funktion ein lokales Minimum bei $ (1, 1)$.
\end{enumerate}

\item[ii)]
\begin{enumerate}
\item \textbf{Berechnung des Gradienten:}

Der Gradient von $ f $ ist:
\[
\nabla f(x, y) = \left( \frac{\partial f}{\partial x}, \frac{\partial f}{\partial y} \right)
\]

Berechnung der partiellen Ableitungen:
\[
\frac{\partial f}{\partial x} = 3x^2 - 3y
\]
\[
\frac{\partial f}{\partial y} = -3x + 3y^2
\]

Daher:
\[
\nabla f(x, y) = (3x^2 - 3y, -3x + 3y^2)
\]

\item \textbf{Ermittlung der kritischen Punkte:}

Kritische Punkte treten dort auf, wo der Gradient Null ist:
\[
3x^2 - 3y = 0
\]
\[
-3x + 3y^2 = 0
\]

Vereinfachung dieser Gleichungen:
\[
x^2 = y \quad \text{(1)}
\]
\[
x = y^2 \quad \text{(2)}
\]

Einsetzen von Gleichung (2) in Gleichung (1):
\[
(y^2)^2 = y \implies y^4 = y \implies y(y^3 - 1) = 0
\]

Auflösen nach $ y $:
\[
y = 0 \quad \text{oder} \quad y^3 = 1 \implies y = 1
\]

Für $ y = 0 $:
\[
x = 0
\]

Für $ y = 1 $:
\[
x = (1)^2 = 1
\]

Daher sind die kritischen Punkte $ (0, 0) $ und $ (1, 1) $.

%\item \textbf{Klassifizierung der kritischen Punkte:}
%
%Um die kritischen Punkte zu klassifizieren, untersuchen wir die zweiten partiellen Ableitungen:
%\[
%f_{xx} = \frac{\partial^2 f}{\partial x^2} = 6x
%\]
%\[
%f_{yy} = \frac{\partial^2 f}{\partial y^2} = 6y
%\]
%\[
%f_{xy} = \frac{\partial^2 f}{\partial x \partial y} = -3
%\]
%
%Die Hessische Determinante $ D$ lautet:
%\[
%D = f_{xx} f_{yy} - (f_{xy})^2 = (6x)(6y) - (-3)^2 = 36xy - 9
%\]
%
%Auswertung von $ D$ an jedem kritischen Punkt:
%
%- Bei $ (0, 0) $:
%\[
%D = 36(0)(0) - 9 = -9
%\]
%Da $ D < 0 $, ist $ (0, 0) $ ein Sattelpunkt.
%
%- Bei $ (1, 1) $:
%\[
%D = 36(1)(1) - 9 = 27
%\]
%Da $ D > 0$ und $f_{xx}(1, 1) = 6 > 0$ ist, ist $ (1, 1)$ ein lokales Minimum.
\end{enumerate}
\item[iii)]
\begin{enumerate}
\item \textbf{Berechnung des Gradienten:}

Der Gradient von $ f $ ist:
\[
\nabla f(x, y) = \left( \frac{\partial f}{\partial x}, \frac{\partial f}{\partial y} \right)
\]

Berechnung der partiellen Ableitungen:
\[
\frac{\partial f}{\partial x} = 4x^3 - 4y
\]
\[
\frac{\partial f}{\partial y} = 4y^3 - 4x
\]

Daher:
\[
\nabla f(x, y) = (4x^3 - 4y, 4y^3 - 4x)
\]

\item \textbf{Ermittlung der kritischen Punkte:}

Kritische Punkte treten dort auf, wo der Gradient Null ist:
\[
4x^3 - 4y = 0
\]
\[
4y^3 - 4x = 0
\]

Vereinfachung dieser Gleichungen:
\[
x^3 = y \quad \text{(1)}
\]
\[
y^3 = x \quad \text{(2)}
\]

Einsetzen von Gleichung (2) in Gleichung (1):
\[
(y^3)^3 = y \implies y^9 = y \implies y(y^8 - 1) = 0
\]

Auflösen nach $ y $:
\[
y = 0 \quad \text{oder} \quad y^8 = 1 \implies y = \pm 1
\]

Für $ y = 0 $:
\[
x^3 = 0 \implies x = 0
\]

Für $ y = 1 $:
\[
x^3 = 1 \implies x = 1
\]

Für $ y = -1 $:
\[
x^3 = -1 \implies x = -1
\]

Daher sind die kritischen Punkte $ (0, 0) $, $ (1, 1) $ und $ (-1, -1) $.

%\item \textbf{Klassifizierung der kritischen Punkte:}
%
%Um die kritischen Punkte zu klassifizieren, untersuchen wir die zweiten partiellen Ableitungen:
%\[
%f_{xx} = \frac{\partial^2 f}{\partial x^2} = 12x^2
%\]
%\[
%f_{yy} = \frac{\partial^2 f}{\partial y^2} = 12y^2
%\]
%\[
%f_{xy} = \frac{\partial^2 f}{\partial x \partial y} = -4
%\]
%
%Die Hessische Determinante $ D $ ist:
%\[
%D = f_{xx} f_{yy} - (f_{xy})^2 = (12x^2)(12y^2) - (-4)^2 = 144x^2y^2 - 16
%\]
%
%Auswertung von $ D$ an jedem kritischen Punkt:
%
%- Bei $ (0, 0) $:
%\[
%D = 144(0)^2(0)^2 - 16 = -16
%\]
%Da $ D < 0 $, ist $ (0, 0) $ ein Sattelpunkt.
%
%- Bei $ (1, 1) $:
%\[
%D = 144(1)^2(1)^2 - 16 = 128
%\]
%Da $ D > 0$ und $ f_{xx}(1, 1) = 12 > 0$ ist, ist $ (1, 1)$ ein lokales Minimum.
%
%- Bei $ (-1, -1)$:
%\[
%D = 144(-1)^2(-1)^2 - 16 = 128
%\]
%Da $ D > 0$ und $f_{xx}(-1, -1) = 12 > 0$ ist, ist $(-1, -1)$ ein lokales Minimum.
\end{enumerate}

\item[iv)]
\begin{enumerate}
\item \textbf{Berechnung des Gradienten:}

Der Gradient von $ f $ ist:
\[
\nabla f(x, y) = \left( \frac{\partial f}{\partial x}, \frac{\partial f}{\partial y} \right)
\]

Berechnung der partiellen Ableitungen:
\[
\frac{\partial f}{\partial x} = 4x^3 - 4x
\]
\[
\frac{\partial f}{\partial y} = 4y^3 - 4y
\]

Daher:
\[
\nabla f(x, y) = (4x(x^2 - 1), 4y(y^2 - 1))
\]

\item \textbf{Ermittlung der kritischen Punkte:}

Kritische Punkte treten dort auf, wo der Gradient Null ist:
\[
4x(x^2 - 1) = 0
\]
\[
4y(y^2 - 1) = 0
\]

Lösung dieser Gleichungen:
\[
x(x^2 - 1) = 0 \implies x = 0, \, x = \pm 1
\]
\[
y(y^2 - 1) = 0 \implies y = 0, \, y = \pm 1
\]

Kombiniert man diese Ergebnisse, ergeben sich folgende kritische Punkte:
\[
(0, 0), \, (1, 0), \, (-1, 0), \, (0, 1), \, (0, -1), \, (1, 1), \, (1, -1), \, (-1, 1), \, (-1, -1)
\]

%\item \textbf{Klassifizierung der kritischen Punkte:}
%
%Zur Klassifizierung der kritischen Punkte untersuchen wir die zweiten partiellen Ableitungen:
%\[
%f_{xx} = \frac{\partial^2 f}{\partial x^2} = 12x^2 - 4
%\]
%\[
%f_{yy} = \frac{\partial^2 f}{\partial y^2} = 12y^2 - 4
%\]
%\[
%f_{xy} = \frac{\partial^2 f}{\partial x \partial y} = 0
%\]
%
%Die Hessische Determinante $ D $ ist:
%\[
%D = f_{xx} f_{yy} - (f_{xy})^2 = (12x^2 - 4)(12y^2 - 4)
%\]
%
%Auswertung von $ D $ an jedem kritischen Punkt:
%
%- Bei $ (0, 0) $:
%\[
%f_{xx} = -4, \quad f_{yy} = -4, \quad D = (-4)(-4) - 0 = 16
%\]
%Da $ D > 0 $ und $ f_{xx} < 0 $, ist $ (0, 0) $ ein lokales Maximum.
%
%- Bei $ (1, 0) $ und $ (-1, 0) $:
%\[
%f_{xx} = 8, \quad f_{yy} = -4, \quad D = (8)(-4) - 0 = -32
%\]
%Da $ D < 0 $, sind diese Punkte Sattelpunkte.
%
%- Bei $ (0, 1) $ und $ (0, -1) $:
%\[
%f_{xx} = -4, \quad f_{yy} = 8, \quad D = (-4)(8) - 0 = -32
%\]
%Da $ D < 0 $, sind diese Punkte Sattelpunkte.
%
%- Bei $ (1, 1) $, $ (1, -1) $, $ (-1, 1) $ und $ (-1, -1) $:
%\[
%f_{xx} = 8, \quad f_{yy} = 8, \quad D = (8)(8) - 0 = 64
%\]
%Da $ D > 0 $ und $ f_{xx} > 0 $, sind diese Punkte lokale Minima.
\end{enumerate}
\end{itemize}
}

\ErgebnisC{analysAblt01}
{
\begin{itemize}
\item[i)] (1,1)
\item[ii)] (0,0), (1,1)
\item[iii)] (-1,-1), (0,0), (1,1)
\item[iv)]  (-1,-1), (1,-1), (0,-1), (0,1), (-1,0), (1,0), (0,0), (1,1), (-1,1)
\end{itemize}
}
