\Aufgabe[e]{Masse einer Halbkugel}
{
Berechnen Sie die Masse einer Halbkugel mit Mittelpunkt $(0,0,0)^\top$, Radius $a>0$ sowie $z\geq 0$ und der Massendichte
$$\rho(x,y,z)=\frac{z}{\sqrt{x^2+y^2}}.$$
\textbf{Hinweis}: Die Masse $M$ eines K\"orpers $K$ mit Massendichte $\rho(\vec x)$ ergibt sich aus
$$M=\int\limits_{K}\rho(\vec x)\d \vec x.$$
}

\Loesung{
Es bietet sich die Rechnung in Kugelkoordinaten an. Wegen der Bedingung $z\geq 0$ wird $\theta$ auf
das Intervall $[0,\, \pi/2]$ eingeschr\"ankt. 
\begin{align*}
M=& \int\limits_{\theta=0}^{\pi/2}\int\limits_{\varphi=0}^{2\pi}\int\limits_{r=0}^a\rho\cdot
r^2\sin\theta \d r \d\varphi\d\theta
=
\int\limits_{\theta=0}^{\pi/2}\int\limits_{\varphi=0}^{2\pi}\int\limits_{r=0}^a\frac{r\cos\theta}{r\sin\theta}r^2\sin\theta
\d r \d\varphi\d\theta\\
=& \frac {a^3}3\cdot 2\pi \int\limits_{\theta=0}^{\pi/2}\cos\theta\d\theta = \frac{2\pi a^3}3.
\end{align*}
}
\ErgebnisC{halbkugel}
{$M=2\pi a^3/3$}
