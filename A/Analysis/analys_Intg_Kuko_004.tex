\Aufgabe[e]{Masse einer Halbkugel}
{
Berechnen Sie die Masse einer Halbkugel mit Mittelpunkt $(0,0,0)^\top$, Radius $a>0$ sowie $z\geq 0$ und der Massendichte
\begin{abc}
\item mithilfe von Kugelkoordinaten,
\item mithilfe von Zylinderkoordinaten.
\end{abc}
$$\rho(x,y,z)=\frac{z}{\sqrt{x^2+y^2}}.$$
\textbf{Hinweis}: Die Masse $M$ eines K\"orpers $K$ mit Massendichte $\rho(\vec x)$ ergibt sich aus
$$M=\int\limits_{K}\rho(\vec x)\mathrm{d} \vec x.$$
}

\Loesung{
\begin{abc}
\item
Es bietet sich die Rechnung in Kugelkoordinaten an. Wegen der Bedingung $z\geq 0$ wird $\theta$ auf
das Intervall $[0,\, \pi/2]$ eingeschr\"ankt. 
\begin{align*}
M=& \int\limits_{\theta=0}^{\pi/2}\int\limits_{\varphi=0}^{2\pi}\int\limits_{r=0}^a\rho\cdot
r^2\sin\theta \mathrm{d}  r \mathrm{d} \varphi \mathrm{d} \theta
=
\int\limits_{\theta=0}^{\pi/2}\int\limits_{\varphi=0}^{2\pi}\int\limits_{r=0}^a\frac{r\cos\theta}{r\sin\theta}r^2\sin\theta
\mathrm{d}  r \mathrm{d} \varphi\mathrm{d} \theta\\
=& \frac {a^3}3\cdot 2\pi \int\limits_{\theta=0}^{\pi/2}\cos\theta\mathrm{d} \theta = \frac{2\pi a^3}3.
\end{align*}
\item
Aus der Beziehung $a^2 = r^2 + z^2$ erhalten wir $0 < r < \sqrt{a^-z^2}$.
In Zylinderkoordinaten erhalten wir das Integral
\begin{align*}
M =& \int_{\varphi = 0}^{2\pi} \int_{z=0}^a \int_{r=0}^{\sqrt{a^2-z^2}} \frac{z}{r}r\mathrm{d}  r \mathrm{d}  z \mathrm{d}  \varphi \\
  =& \int_{\varphi = 0}^{2\pi} \int_{z=0}^a \int_{r=0}^{\sqrt{a^2-z^2}} z\mathrm{d}  r \mathrm{d} z \mathrm{d}  \varphi \\
  =& \int_{\varphi = 0}^{2\pi} \int_{z=0}^a z\sqrt{a^2-z^2} \mathrm{d}  z \mathrm{d}  \varphi
\end{align*}
Mit der Substitution $u = a^2-z^2$ erhalten wir
\begin{align*}
M =& \int_{\varphi = 0}^{2\pi} \int_{u=a^2}^0 -\frac{1}{2} \sqrt{u} \mathrm{d}  u\mathrm{d}  \varphi \\
  =& \int_{\varphi = 0}^{2\pi} \frac{1}{3}a^3 \mathrm{d}  \varphi \\
  =& \frac{2\pi a^3}{3}.
\end{align*}
\end{abc}
}
\ErgebnisC{halbkugel}
{$M=2\pi a^3/3$}
