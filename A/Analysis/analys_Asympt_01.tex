\Aufgabe[e]{Asymptoten}{
Man bestimme die (waagerechten bzw.\ senkrechten bzw.\ schrägen) Asymptoten der folgenden Funktionen:
\begin{multicols}{2}
\begin{abc}
\item $f(x) = \frac{x}{4-x^2}$
\item $g(x) = \operatorname{e}^{-x^2}$
\item $h(x) = \frac{x^2-3x}{2x-2}$
\item $l(x)=x^2e^{-x}$
\end{abc}
\end{multicols}
}

\Loesung{
\begin{abc}
\item
Um die senkrechten Asymptoten zu finden, bestimmen wir zunächst die Nullstellen 
des Nenners. Die Nullstellen liegen bei $x_1 = -2$ und $x_2 = 2$.

An diesen untersuchen wir das Verhalten der Funktion:
$$
\lim_{x\to -2^-} \frac{x}{4-x^2} = -\infty \, , \lim_{x\to -2^+} \frac{x}{4-x^2} = \infty
$$
und
$$
\lim_{x\to 2^-} \frac{x}{4-x^2} = \infty \, , \lim_{x\to 2^+} \frac{x}{4-x^2} = -\infty
$$
Das bedeutet, es gibt eine senkrechte Asymptote bei $ x = -2$ und $x = 2$.

Um die waagerechten Asymptoten zu finden, untersuchen wir das Verhalten im 
Unendlichen. 
$$
\lim_{x \to -\infty} \frac{x}{4-x^2} = 0
$$
und
$$
\lim_{x \to \infty} \frac{x}{4-x^2} = 0.
$$
Das hei\ss t, es gibt eine waagerechte Asymptote bei $y = 0.$

\item
Die Funktion $g(x) = \operatorname{e}^{-x^2}$ hat keine Definitionsl\"ucke. 
Es gibt also keine senkrechten Asymptoten.

F\"ur die waagerechten Asymptoten untersuchen wir das Verhalten im Unendlichen.
Es gilt: 
$$
\lim_{x \to -\infty} \operatorname{e}^{-x^2} = 0
$$
und 
$$
\lim_{x \to \infty} \operatorname{e}^{-x^2} = 0.
$$
Es gibt also eine waagerechte Asymptote bei $y = 0$.

\item
Die Funktion $h(x) = \frac{x^2-3x}{2x-2}$ hat eine Definitionsl\"ucke bei $x = 1$.
Es gilt:
$$
\lim_{x \to 1^-} \frac{x^2-3x}{2x-2} = \infty
$$
und 
$$
\lim_{x \to 1^+} \frac{x^2-3x}{2x-2} = -\infty.
$$
Da der Z\"ahler von $h(x)$ genau einen Polynomgrad h\"oher ist als der des Nenners, 
gibt es eine schr\"age Asymptote.

Durch Polynomdivision erhalten wir:
$$
(x^2 -3x):(2x -2) = (\frac{1}{2}x - 1) + \frac{-2}{2x-2}.
$$
Es gibt also eine schr\"age Asymptote bei $y = \frac{1}{2}x - 1$.
\item
Die Funktion $l(x)=x^2e^{-x}$ hat keine Definitionsl\"ucke. Daher hat sie keine 
senkrechte Asymptote. 
Wir untersuchen das Verhalten im Unendlichen mit Hilfe der Regel von L'Hospital,
um die waagerechten Asymptoten zu finden:
$$
\lim_{x \to -\infty} x^2e^{-x} = \lim_{x \to -\infty} -2xe^{-x}
=\lim_{x \to -\infty} 2e^{-x} = \infty.
$$
und 
$$
\lim_{x \to \infty} x^2e^{-x} = \lim_{x \to \infty} -2xe^{-x}
=\lim_{x \to \infty} 2e^{-x} = 0.
$$
Es gibt also eine waagerechte Asymptote bei $y = 0$.
\end{abc}

}

\ErgebnisC{analysAsymp01}
{
\textbf{a)} waagerechte Asymptote bei $y=0$,
\textbf{b)} waagerechte Asymptote bei $y=0$,
\textbf{c)} schr\"age Asymptote bei $y=\frac{1}{2}x-1$,
\textbf{d)} waagerechte Asymptote bei $y=0$.
}
