\Aufgabe[e]{Ableitung der Umkehrfunktion}{
\begin{abc}
\item Leiten Sie die Formel für die Ableitung der Umkehrfunktion her.
\item Leiten Sie eine Formel für die zweiten Ableitung der Umkehrfunktion her.
\item Gegeben sei die Funktion
$$ f(x) = \operatorname{e}^{x} + 2x . $$
\begin{iii}
\item Zeigen Sie, dass $f:\R \to \R$ umkehrbar ist.
\item Bestimmen Sie die Ableitungen $g^\prime(1)$ und $g^{\prime\prime}(1)$ der Umkehrfunktion 
$$g=f^{-1}.$$ 
\end{iii}
\end{abc}
}


\Loesung{
% \textbf{Zu a)} Es gilt:
% \begin{itemize}
% \item[i)] \qquad 
% $f_1'(x) = 3\cdot 4 x^3(\ln x)^2 + 3 x^4\cdot 2\ln x \cdot \dfrac{1}{x}=6x^3\ln x(2\ln x+1)\,,$
% \item[ii)] \qquad
% $\begin{array}[t]{r@{\;}c@{\;}l}
% f_2(x) & = & x^{x+\ln x} = \exp\left((x+\ln x)\cdot \ln x\right)\,,\\[3ex]
% f_2'(x) & = & \exp\left((x+\ln x)\cdot \ln x\right) \cdot \left[\left(1+\dfrac{1}{x}\right)\ln x + (x + \ln x)\cdot \dfrac{1}{x}\right]\\[2ex]
%  & = & \left(1+\ln x + \dfrac{2}{x}\ln x\right)\cdot x^{x+\ln x}\,,
% \end{array}
% $
% \item[iii)] \qquad $\begin{array}[t]{r@{\;}c@{\;}l}
% f_3'(x) & = & \dfrac{1}{\sqrt{1-x^2}} + \dfrac{2\sqrt{1-x^2}+2x\cdot \dfrac{1}{2}\cdot \dfrac{-2x}{\sqrt{1-x^2}}}{\sqrt{1-4x^2(1-x^2)}} - \dfrac{3}{\sqrt{1-x^2}}\\[3ex]
% & = & - \dfrac{2}{\sqrt{1-x^2}} + \dfrac{2\sqrt{1-x^2}- \dfrac{2x^2}{\sqrt{1-x^2}}}{1-2x^2}\\[3ex] 
% & = & - \dfrac{2}{\sqrt{1-x^2}} + \dfrac{2-4x^2}{(1-2x^2)\sqrt{1-x^2}} = 0\,,
% \end{array}
% $\\[1ex]
% d.h.\ $f_3$ ist konstant. 
% \end{itemize}
\begin{abc}
\item 
Für die Umkehrfunktion von $f$ gilt:
$$
x = f(f^{-1}(x))
$$
Wir leiten beide Seiten dieser Identität einmal ab.
Damit erhalten wir mit der Kettenregel
$$
1 = f'(f^{-1}(x))\cdot(f^{-1}(x))'
$$
Wir stellen nach der Ableitung der Umkehrfunktion um und erhalten.
$$
(f^{-1})'(x) = \frac{1}{f'(f^{-1}(x))}
$$
\item 
Um die Formel für die zweite Ableitung der Umkehrfunktion herzuleiten, 
leiten wir die Identität
$$
x = f(f^{-1}(x)) 
$$
zweimal ab. Wir erhalten
$$
0 = f''(f^{-1}(x))\cdot(f^{-1}(x))'\cdot (f^{-1}(x))' + f'(f^{-1}(x))\cdot(f^{-1}(x))''  
$$
Wir stellen nach der zweiten Ableitung der Umkehrfunktion um. Damit erhalten wir:
$$
(f^{-1})''(x) = - f''(f^{-1}(x) \cdot \frac{((f^{-1}(x))')^2}{f'(f^{-1}(x))}.
$$
Wir setzen die Formel für die erste Ableitung der Umkehrfunktion ein. Damit erhalten 
wir:
$$
(f^{-1})''(x) = -f''(f^{-1}(x))\cdot\frac{1}{(f'(f^{-1}(x))^3} 
$$
Alternativ kann man die Formel auch herleiten, indem man die Formel für die erste
Ableitung der Umkehrfunktion nochmals ableitet.
\item
\begin{iii}
\item Die Funktion $f:\R \to \R$ ist streng monoton steigend, denn $x \mapsto 2x$ und $x \mapsto \operatorname{e}^x$
 sind beide streng monoton steigend. \\
Außerdem ist $\lim\limits_{x \to -\infty} f(x) = - \infty$ und
 $\lim\limits_{x \to +\infty} f(x) = +\infty$.\\
Also werden von der stetigen Funktion $f(x)$ alle reellen Werte angenommen. Wegen der Monotonie
 wird jeder Wert nur genau ein Mal angenommen und die Funktion $f(x)$ ist umkehrbar. 


\item Es ist $f(0)=\operatorname{e}^0+2 \cdot 0=1$, also hat man $x_0=0$ und $y_0=f(x_0)=1$. Mit den Formeln f\"ur die Ableitung der Umkehrfunktion aus der Vorlesung und den Ableitungen
$f^\prime(x)=\operatorname{e}^x+2$ sowie $f^{\prime\prime}(x)=\operatorname{e}^x$ folgt 
$$ g^\prime(1)=\dfrac{1}{f^\prime(0)}=\dfrac{1}{3} 
   \text{ und } 
   g^{\prime\prime}(1)=-\dfrac{f^{\prime\prime}(0)}{(f^\prime(0))^3}
	= -\dfrac{1}{27}\,. $$
	
\end{iii}
\end{abc}
}

\ErgebnisC{analysUmkeFunk004}
{
{\textbf{a)}} $(f^{-1})'(x) = \frac{1}{f'(f^{-1}(x))}$, {\textbf{b)}}  $(f^{-1})''(x) = -f''(f^{-1}(x))\cdot\frac{1}{(f'(f^{-1}(x))^3}$, {\textbf{c)}} {\textbf{ii)}} $g^{\prime\prime}(1)	= -\dfrac{1}{27}\,. $
}
