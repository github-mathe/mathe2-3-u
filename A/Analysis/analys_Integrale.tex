\Aufgabe[e]{ }{
Berechnen Sie die Integrale
\begin{iii}
\item $I_1={\displaystyle \int } \dfrac{\sin(\sqrt{x})}{\sqrt{x}}\mathrm{d} x$
\item $I_2={\displaystyle \int \limits_0^2} (x^2-2) e^{-2x}\mathrm{d} x$ 
\item $I_3={\displaystyle \int \limits_1^3} x^3 \ln(x^2)\mathrm{d} x$
\item $I_4={\displaystyle \int \limits_0^1} \dfrac{e^x}{1+e^x} \mathrm{d} x$
\item $I_5={\displaystyle \int} \dfrac{5x^2+7x+3}{x^3-3x+2} \mathrm{d} x$
\end{iii}
}

\Loesung{
 \begin{iii}
\item Wir substituieren $u=\sqrt{x}$, $\mathrm{d} u = \dfrac{\mathrm{d} x}{2\sqrt{x}}$ 
\begin{align*}
I_1=& \int \dfrac{\sin(\sqrt{x})}{\sqrt{x}}\mathrm{d} x\\
=& \int 2 \sin(u) \mathrm{d} u\\
=& -2 \cos(u)+C \\
=& -2\cos(\sqrt{x}) +C
\end{align*}

\item Wir erhalten durch zweifache partielle Integration 
\begin{align*}
I_2=& \int \limits_0^2 (x^2-2) e^{-2x}\mathrm{d} x\\
=& \left[ -\frac{1}{2}(x^2-2)e^{-2x}\right]_0^2 + \frac{1}{2} \int_0^2 e^{-2x} 2x \mathrm{d} x\\
=& \left[ -\frac{1}{2}(x^2-2)e^{-2x} - \frac{1}{2} x e^{-2x}\right]_0^2 +\frac{1}{4} \int_0^2 2 e^{-2x}\mathrm{d} x\\
=& \left[ -\frac{1}{2}(x^2-2)e^{-2x} - \frac{1}{2} e^{-2x} - \frac{1}{4} e^{-2x}\right]_0^2 \\
=& -e^{-4} -e^{-4} -\frac{1}{4} e^{-4} -1 +\frac{1}{4} \\
=& -\frac{9}{4}e^{-4}-\frac{3}{4}
\end{align*}

\item Wir erhalten durch partielle Integration
\begin{align*}
I_3 =& \int_0^3 x^3 \ln(x^2)\mathrm{d} x\\ 
=& 2\int_0^3  x^3 \ln(x) \mathrm{d} x\\ 
=& 2 \left\lbrace \left[ \frac{1}{4}x^4 \ln(x)\right]_1^3 - \frac{1}{4} \int_1^3 x^4 \dfrac{1}{x} \mathrm{d} x\right\rbrace \\
=& 2 \left\lbrace \left[ \frac{1}{4}x^4 \ln(x)\right]_1^3 - \frac{1}{4} \int_1^3 x^3 \mathrm{d} x\right\rbrace \\
=& 2 \left\lbrace \left[ \frac{1}{4}x^4 \ln(x) -\frac{1}{16}x^4  \right]_1^3 \right\rbrace \\
=& \left[ \frac{1}{2} x^4 \ln(x) - \frac{1}{8} x^4 \right] _1^3 \\
=& \frac{81}{2} \ln(3) - \frac{80}{8}
\end{align*}

\item  Wir substituieren $u=1+e^x$, $\mathrm{d} u = e^x \mathrm{d} x$
\begin{align*}
I_4 &= \int_0^1 \dfrac{e^x}{1+e^x}\mathrm{d} x \\
&=\int_{u(0)}^{u(1)} \dfrac{1}{u} \mathrm{d} u \\
&= \left[ \ln|u|\right]_{u(0)}^{u(1)} \\
&=  \left[ \ln|1+e^x|\right]_0^1 \\
&= \ln(1+e) - \ln(2) \\
&= \ln \left( \dfrac{1+e}{2}\right) 
\end{align*}

\item Wir f\"uhren eine Partialbruchzerlegung durch. Zun\"achst werden dazu die Nullstellen des Nenners bestimmt. Die erste Nullstelle wird geraten $x_0 =1$. Anschließend wird eine Polynomdivision durchgef\"uhrt:
$(x^3-3x+2):(x-1)=x^2+x-2$. Daraus ergeben sich die weiteren Nullstellen $x_1 = 1$ und $x_2 =-2$. Damit ergibt sich
\begin{align*}
 \dfrac{5x^2+7x+3}{x^3-3x+2} &=  \dfrac{5x^2+7x+3}{(x-1)^2(x+2)}  \\
 &= \dfrac{A}{x-1} + \dfrac{B}{(x-1)^2} + \dfrac{C}{x+2} \\
\end{align*}
Es wird ein Koeffizientenvergleich durchgef\"uhrt:
\begin{align*}
5x^2+7x+3 &= A(x-1)(x+2) +B(x+2)+C(x-1)^2 \\
15 &= 3B \Rightarrow B=5 \\
9 &= 9C \Rightarrow C =1 \\
3 &= -2A+2B+C \Rightarrow A =4
\end{align*}
Daraus ergibt sich das Integral
\begin{align*}
I_5 &= \int \dfrac{4}{x-1} + \dfrac{5}{(x-1)^2} + \dfrac{C}{x+2}  \mathrm{d} x \\
&= 4 \ln|x-1| - \dfrac{5}{x-1} + \ln|x+2| + C
\end{align*}
\end{iii}
}

\ErgebnisC{AnalysIntegrale}
{
{{\textbf{i)}} $I_1 =-2\cos(\sqrt{x}) +C$, {\textbf{ii)}} $ I_2=-\frac{9}{4}e^{-4}-\frac{3}{4}$, {\textbf{iii)}} $I_3=\frac{81}{2} \ln(3) - \frac{80}{8}$, {\textbf{iv)}} $I_4=\ln \left( \dfrac{1+e}{2}\right)$, {\textbf{v)}} $I_5= 4 \ln|x-1| - \dfrac{5}{x-1} + \ln|x+2| + C$
}
}


