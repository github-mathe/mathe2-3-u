\Aufgabe[e]{Fl\"acheninhalt, Rotationsk\"orper} {
\begin{abc}
\item Skizzieren Sie den Bereich $B\subset\C$ mit 
$$B=\left\{ |z+\bar z| + |z-\bar z|\leq 4\right\}\cap \left\{ 0\leq \Real (z)\leq 1\right\}$$
in der Gauß'schen Ebene. 
\item Interpretieren Sie $B$ als Teilmenge des $\R^2$ und berechnen Sie den Fl\"acheninhalt. 
\item Berechnen Sie ebenso den Schwerpunkt $\vec x_s$ des Bereichs $B$. 
\item Berechnen Sie das Volumen des Rotationsk\"orpers, der durch Rotation von $B$ um die $y$-Achse
entsteht. 
\end{abc}
}
\Loesung{
{ a)} Eine Zahl $z=x+\imag y\in B$ muss zum einen in dem Streifen $0\leq x\leq 1$ enthalten sein,
zum anderen muss die folgende Ungleichung erf\"ullt sein: 
\begin{align*}
4\geq&|z+\bar z| + |z-\bar z|=2 |x| + 2 |y|.
\end{align*}
Beide Bereiche sowie deren Schnittmenge $B$ sind im folgenden skizziert: 
\begin{center}
\psset{xunit=1.3cm, yunit=1.3cm, runit=1cm}
\begin{pspicture}(-5,-5)(5,5)
\psline[linecolor=gray, fillcolor=lightgray, fillstyle=solid, linewidth=0]
(0,-5)(2,-5)(2,5)(0,5)(0,-5)
\psline(0,-5)(0,5)
\psline(1,-5)(1,5)
\psline[fillstyle=solid, fillcolor=gray]
(-4,0)(0,-4)(4,0)(0,4)(-4,0)
\psline[fillstyle=solid, fillcolor=darkgray]
(0,-4)(2,-2)(2,2)(0,4)(0,-4)
\psgrid[subgriddiv=1,griddots=20,gridlabels=0](-5,-5)(5,5)
\psdot(2,0)
\put(.1,2.8){$\imag$}
\psdot(0,2)
\put(2.8,.1){$1$}
\psdot(0,0)
\put(.1,.1){$0$}
\put(.1,6.0){$0\leq \Real\, z\leq 1$}
\put(-4.05,.3){$|z+\bar z|+|z-\bar z|\leq 4$}
\put(.3,1){$B$}
\end{pspicture}
\end{center}
\begin{abc}
\setcounter{enumi}{1}
\item Die Integrationsgrenzen f\"ur $B\subset \R^2$ sind 
$$ B=\{(x,y)^\top\in\R^2|\, x-2\leq y\leq 2-x\text{ und } 0\leq x\leq 1\}.$$
Damit ist der Fl\"acheninhalt
$$A=\int\limits_B \mathrm{d} y \mathrm{d} x = \int\limits_0^1\int\limits_{x-2}^{2-x}\mathrm{d} y \mathrm{d}
x =\int \limits_0^1(4-2x)\mathrm{d} x = 4-1=3.$$
\item Der Schwerpunkt $(x_s,\, y_s)^\top$ ergibt sich zu 
\begin{align*}
x_s=&\frac 1 A \cdot \int\limits_B x \mathrm{d} y \mathrm{d}x
 = \frac 13 \int\limits_0^1 x\int\limits_{x-2}^{2-x}\mathrm{d} y \mathrm{d} x\\
=& \frac 13 \int\limits_0^1 (4x-2x^2)\mathrm{d} x = \frac 13 \left[2x^2 - \frac 23 x^3\right]_0^1\\
=& \frac {4}{9}\\
y_s=& \frac 1A \cdot \int\limits_B y \mathrm{d} y \mathrm{d} x
 = \frac 13 \int\limits_0^1 \int\limits_{x-2}^{2-x}y\mathrm{d} y \mathrm{d} x\\
=& \frac 13 \int\limits_0^1 \left[\frac {y^2}2\right]_{x-2}^{2-x} \mathrm{d} x
= \frac 1{6} \int\limits_0^10\mathrm{d} x=0
\end{align*}
Der Schwerpunkt liegt also bei $\vec x = (4/9,\, 0)^\top$. 
\item Das Volumen des Rotationsk\"orpers um die $y$-Achse ergibt sich (in Zylinderkoordinaten) zu: 
\begin{align*}
V=& \int\limits_{\varphi=0}^{2\pi}\int\limits_{r=0}^1\int\limits_{z=r-2}^{2-r}1 r\mathrm{d} z\mathrm{d} r \mathrm{d} \varphi\\
=& 2\pi\int\limits_{r=0}^1[(2-r)-(r-2)]r\mathrm{d} r = 2\pi\int\limits_{r=0}^1(4r-2r^2)\mathrm{d}r\\
=& 2\pi \left[\frac{4r^2}2-\frac{2r^3}3\right]_{r=0}^1 = \frac{8\pi}3.
\end{align*}
\end{abc}

}

\ErgebnisC{analysIntgNdim003}{
\textbf{a)} $B$ ist ein Trapez, dessen Grundseite in der $y$-Achse enthalten ist. \\
\textbf{d)} $\frac {8\pi}3$
}
