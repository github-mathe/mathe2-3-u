\Aufgabe[e]{Bereichsintegrale}{
Gegeben sei das Doppelintegral
	\[
	I := \int\limits_{-1}^{1}\ \int\limits_{y-1}^{1-y} \dfrac{y+x^3}{4} \ \ \text dx\ \text dy\ .
\]
\begin{itemize}
	\item[ a)] Skizzieren Sie den von den Grenzen beschriebenen Teilbereich der $x$-$y$--Ebene.
	
	\item[ b)] Berechnen Sie das Integral. 
\end{itemize}
}
\Loesung{
\textbf{a)}
Der Bereich ist durch die Geraden $x=y-1\Rightarrow y=x+1$ und $x=1-y\Rightarrow y=x+1$, sowie durch die Geraden $y=-1$ und formal auch $y=1$, eingeschlossen.
Das ist ein Dreieck:
\begin{center}
\begin{pspicture}(-3,-2)(3,2)
\psgrid(-3,-2)(3,2)
\psline[linewidth=2pt](-2,-1)(2,-1)(0,1)(-2,-1)
\end{pspicture}
\end{center}

\textbf{b)}
	\[
	I = \int\limits_{-1}^{1}\ \int\limits_{y-1}^{1-y} \dfrac{y}{4} \ \ \text dx\ \text dy + \int\limits_{-1}^{1}\ \int\limits_{y-1}^{1-y} \dfrac{x^3}{4} \ \ \text dx\ \text dy\ .
\]
Der zweite Summand ist Null, da eine in \ $x$ \ ungerade Funktion über einen zur $y$--Achse symmetrischen Bereich integriert wird (aber auch ohne Beachtung der Symmetrie ist die Berechnung einfach!). 

Damit ist
\begin{align*}
	I =& \int\limits_{-1}^{1}\ \dfrac{y}{4} \ \int\limits_{y-1}^{1-y} 1 \ \ \text dx\ \text dy=\int\limits_{-1}^{1}\ \dfrac{y}{4}\cdot \Big[x\Big]_{x=y-1}^{1-y}  \ \ \text dx \\ 
	  =& \int\limits_{-1}^{1}\ \dfrac{y}{4} \cdot \Big((1-y)-(y-1)\Big) \ \text dy \\ 
	  =& \int\limits_{-1}^{1}\ \dfrac{y-y^2}{2} \ \text dy\ = \ -\dfrac{1}{3}\ .
\end{align*}

}

\ErgebnisC{AufganalysIntgNdim013}
{
$I=-1/3$.
}
