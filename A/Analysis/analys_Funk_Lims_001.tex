\Aufgabe[e]{Funktionenlimes}{
\begin{abc}
\item Gegeben sei die Funktion
$$
f(x):=\frac{x^3+|x+1|+{\mathrm{sign}}\,(x+1)}{{\mathrm{sign}}\,x},\;x\in D(f):=\R\,.
$$
Bestimmen Sie $\lim\limits_{x\to 0+}f(x)$, $\lim\limits_{x\to 0-}f(x)$, $\lim\limits_{x\to (-1)+}f(x)$ und $\lim\limits_{x\to (-1)-}f(x)$.  
\item Gegeben sei die Funktion
$$
f(x) = \dfrac{\sinh x}{\cosh(ax)}\,, \quad a\in \R\,.
$$
Bestimmen Sie $\lim\limits_{x\rightarrow \pm \infty}f(x)$. \\
\textbf{Hinweise}: \\
- Mit der zun\"achst als bekannt vorausgesetzten Exponentialfunktion $\EH{x}$ gilt 
$$\sinh x:= \frac{e^x-e^{-x}}2\qquad\text{ und }\qquad \cosh x:=\frac{e^x + e^{-x}}2.$$
- Die Signum-Funktion liefert das Vorzeichen des Argumentes: 
$$\sign(z)=\left\{\begin{array}{ll}
+1&, z\geq 0\\
-1&, z<0
\end{array}\right.$$
\end{abc}
}

\Loesung{
\begin{abc}
\item \begin{iii}
\item F\"ur $x>0$ gilt $\sign(x)=\sign(1+x)=1$ sowie $|x+1|=x+1$, damit gilt
$$\underset{x\to 0+}\lim f(x)= \underset{x\to 0+}\lim (x^3+x+1+1)=2.$$
\item F\"ur $-1<x<0$ gilt $\sign(x)=-1$, $\sign(1+x)=1$ sowie $|x+1|=x+1$, damit gilt
$$\underset{x\to 0-}\lim f(x)= \underset{x\to 0-}\lim \frac{x^3+x+1+1}{-1}=-2.$$
\item F\"ur $-1<x<0$ gilt $\sign(x)=-1$, $\sign(1+x)=1$ sowie $|x+1|=x+1$, damit gilt
$$\underset{x\to (-1)+}\lim f(x)= \underset{x\to (-1)+}\lim \frac{x^3+x+1+1}{-1}=0.$$
\item F\"ur $x<-1$ gilt $\sign(x)=\sign(1+x)=-1$ sowie $|x+1|=-(x+1)$, damit gilt
$$\underset{x\to (-1)-}\lim f(x)= \underset{x\to (-1)-}\lim \frac{x^3-x-1-1}{-1}=2.$$
\end{iii}
\item Die Funktion $f(x)$ ist eine ungerade Funktion: 
$$f(-x)=\frac{\sinh(-x)}{\cosh(ax)} = \frac{\EH{-x}-\EH{-(-x)}}{\EH{-ax}+\EH{-(-ax)}}
= -\frac{\EH{x}-\EH{-x}}{\EH{ax}+\EH{-ax}}=-f(x)$$
Daher ist $\underset{x\to-\infty}\lim f(x)=-\underset{x\to+\infty}\lim f(x)$ und es gen\"ugt einen der beiden
Grenzwerte zu berechnen. \\
Es wird vorausgesetzt, dass $\EH x$ stetig ist und dass gilt $\underset{x\to\infty}\lim \EH{x}=\infty$.
%\begin{align*}
%\underset{x\to\infty}\lim f(x)=& \underset{x\to\infty}\lim \frac{\EH x-\EH{-x}}{\EH{ax}+\EH{-ax}}
%= \underset{z\to\infty}\lim \frac{z-\frac 1z}{z^a+\frac 1{z^a}}\\
%=& \underset{z\to\infty}\lim \left( \frac{z}{z^a}\cdot \frac{1-\frac 1{z^2}}{1+\frac{1}{z^{2a}}}\right)
%\end{align*}
%Der Grenzwert des ersten Bruches ist 
%$$\underset{z\to\infty}\lim \frac z{z^a}=\underset{z\to\infty}\lim z^{1-a}=\left\{\begin{array}{ll}
%\infty,& a<1\\
%1,&a=1\\
%0,&a>1\end{array}\right..
%$$
%F\"ur den zweiten Bruch hat man 
%$$\underset{z\to\infty}\lim \frac{1-\frac 1{z^2}}{1+\frac{1}{z^{2a}}}=\left\{\begin{array}{ll}
%0,& a<0\\
%\frac 12,& a=0\\
%1,& a>0\end{array}\right..$$
%Insgesamt ergibt sich daraus: 
%$$\underset{x\to\infty}\lim f(x)=\left\{\begin{array}{ll}
%\infty,& 0\leq a<1\\
%1,& a=1\\
%0,& a>1
%\end{array}\right.
%$$
%F\"ur $a<0$ hat man 
%\begin{align*}
%\underset{x\to\infty}\lim
%f(x)&=
%\underset{z\to \infty}\lim \left( \frac{z}{z^{a}z^{-2a}}\cdot \frac{1-\frac 1{z^2}}{z^{2a}+1}\right)\\
%&=\underset{z\to \infty}\lim \left( \frac{z}{z^{-a}}\cdot \frac{1-\frac 1{z^2}}{z^{2a}+1}\right)
%=\left\{\begin{array}{ll}
%0,&a<-1\\
%1,&a=-1\\
%\infty,&-1<a<0\end{array}\right..
%\end{align*}

\begin{align*}
\underset{x\to\infty}\lim f(x)=& \underset{x\to\infty}\lim \frac{\EH x-\EH{-x}}{\EH{ax}+\EH{-ax}}
= \underset{x\to\infty}\lim \frac{\EH x-\frac{1}{\EH{x}}}{\EH{ax}+ \frac{1}{\EH{ax}}}\\
=& \underset{x\to\infty}\lim \frac{\frac{1}{\EH x}(\EH {2x}-1)}{\frac{1}{\EH {ax}}(\EH{2ax}+ 1)} 
= \underset{x\to\infty}\lim \frac{\EH {ax} \EH {2x}}{\EH x \EH {2ax}} \frac{1-\frac{1}{\EH {2x}}}{1+\frac{1}{\EH{2ax}}}\\
=& \underset{x\to\infty}\lim \frac{\EH {x}}{\EH {ax}} \frac{1-\frac{1}{\EH {2x}}}{1+\frac{1}{\EH{2ax}}} 
\end{align*}
Der Grenzwert des ersten Bruches ist 
$$\underset{x\to\infty}\lim \frac {\EH x}{\EH {ax}}=\underset{x\to\infty}\lim \EH {x(1-a)}=\left\{\begin{array}{ll}
\infty,& a<1\\
1,&a=1\\
0,&a>1\end{array}\right..
$$
F\"ur den zweiten Bruch hat man 
$$\underset{x\to\infty}\lim  \frac{1-\frac{1}{\EH {2x}}}{1+\frac{1}{\EH{2ax}}}=\left\{\begin{array}{ll}
0,& a<0\\
\frac 1 2,& a=0\\
1,& a>0\end{array}\right..$$
Insgesamt ergibt sich daraus f\"ur $a>0$: 
$$\underset{x\to\infty}\lim f(x)=\left\{\begin{array}{ll}
\infty,& 0\leq a<1\\
1,& a=1\\
0,& a>1
\end{array}\right.
$$
F\"ur $a<0$ hat man 
\begin{align*}
\underset{x\to\infty}\lim f(x) 
&= \underset{x\to\infty}\lim \frac{\EH {x}}{\EH {ax}} \frac{1-\frac{1}{\EH {2x}}}{1+\frac{1}{\EH{2ax}}} 
= \underset{x\to\infty}\lim \frac{\EH {x}}{\EH {ax}} \frac{1-\frac{1}{\EH {2x}}}{\frac{\EH {2ax}}{\EH {2ax}}+\frac{1}{\EH{2ax}}} \\
&= \underset{x\to\infty}\lim \frac{\EH {x}}{\EH {ax} \EH {-2ax}} \frac{1-\frac{1}{\EH {2x}}}{\EH {2ax}+1}
= \underset{x\to\infty}\lim \frac{\EH {x}}{\EH {-ax} } \frac{1-\frac{1}{\EH {2x}}}{\EH {2ax}+1}\\
\end{align*}



Der Grenzwert des ersten Bruches ist 
$$\underset{x\to\infty}\lim \frac {\EH x}{\EH {-ax}}=\underset{x\to\infty}\lim \EH {x(1+a)}=\left\{\begin{array}{ll}
0,& a<-1\\
1,&a=-1\\
\infty,&-1<a<0\end{array}\right..
$$
F\"ur den zweiten Bruch hat man 
$$\underset{x\to\infty}\lim \frac{1-\frac{1}{\EH {2x}}}{\EH {2ax}+1}=1 \, \text{f\"ur $a<0$}.$$
Insgesamt ergibt sich daraus: 
$$\underset{x\to\infty}\lim f(x)=\left\{\begin{array}{ll}
0,&a<-1\\
1,&a=-1\\
\infty,&-1<a<0\end{array}\right..
$$




\end{abc}
}

\ErgebnisC{analysFunkLims001}
{
\textbf{a)} $\underset{x\to 0\pm}\lim f(x)=\pm 2$, $\underset{x\to{-1}\pm}\lim f(x)=1\pm (-1)$\\
\textbf{b)} $\underset{x\to \infty}\lim f(x)= \left\{\begin{array}{ll}0,&|a|>1\\1,&|a|=1\\\infty,&|a|<1\end{array}\right.$
}
