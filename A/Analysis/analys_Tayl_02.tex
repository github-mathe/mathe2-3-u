\Aufgabe[e]{Taylor-Entwicklung}{
Gegeben sei die Funktion $$f(x)=\ln(x).$$
\begin{abc}
%\item Determine the Taylor polynomial of order two, $T_2(x)$, of $f(x)$ at the expansion point $x=1$.
%\item Determine the reminder term $R_2(x;1)$ and estimate $$\max_{x\in[1,2]}|R(x;1)|.$$
\item Bestimmen Sie das Taylor-Polynom der Ordnung zwei, $T_2(x)$, von $f(x)$ an der Stelle $x=1$.
\item Bestimmen Sie das Restglied $R_2(x;1)$ und schätzen Sie $$\max_{x\in[1,2]}|R_2(x;1)|.$$
\end{abc}
}

\Loesung{
\begin{abc}
\item 
Die Ableitungen der Funktion sind
\begin{align*}
f'(x) = \frac{1}{x}, \quad f''(x) = -\frac{1}{x^2}, \quad f'''(x) = \frac{2}{x^3}.
\end{align*}
Das Taylor-Polynom ist
$$
f(x) = f(1) + f'(1)(x-1) +\frac{1}{2} f''(1)(x-1)^2+R_2(x;1),
$$
und das Taylor-Polynom der zweiten Ordnung an der Stelle $x=1$ ist
$$
T_2(x) = 0 + 1 (x-1) + \frac{1}{2}(-1)(x-1)^2 = -\frac{1}{2}x^2 + 2x -\frac{3}{2}
$$
\item Das Restglied ist
$$
R_2(x;1) = f'''(\xi)\frac{(x-1)^3}{3!}, \quad \xi \in [1,2].
$$
Es ist
$$
R_2(x;1) = \frac{2}{\xi^3}\frac{(x-1)^3}{3!}, \quad \xi \in [1,2].
$$
%An upper bound for the funtion $R(x;1)$ for $\xi$ and $x$ in the interval $[1,2]$ is found by minimizing the denominator and maximizing the numerator. The minimim of the denominator is at $\xi=1$ and the maximum of the numerator is at $x=2$. This because the cubic function is monotonically increasing since its derivative is always positive. We have thus the estimate
Eine obere Schranke für die Funktion $R(x;1)$ für $\xi$ und $x$ im Intervall $[1,2]$ wird durch Minimieren des Nenners und Maximieren des Zählers gefunden. Das Minimum des Nenners liegt bei $\xi=1$ und das Maximum des Zählers ist bei $x=2$. Dies liegt daran, dass die kubische Funktion monoton steigend ist, da ihre Ableitung immer positiv ist. Wir haben also die Schätzung
$$
R_2(x;1) \leq \dfrac{2}{1^3}\dfrac{(2-1)^3}{3!} = \frac{1}{3}.
$$
%In the following the function $R_2(x;1)$ is plotted for the two values of $\xi=1$ and $\xi=2$:
Im Folgenden wird die Funktion $R_2(x;1)$ für die beiden Werte $\xi=1$ und $\xi=2$ skizziert:

\vspace{0.2cm}
%
\begin{tikzpicture}
    \begin{axis}[
     axis lines=middle,clip=false,
            xmin=0,xmax=2, ymin=-0.5,ymax=0.5,
            xticklabel style={black},
            xlabel=$x$,
            ylabel=$y$]
    \addplot[domain=0:2,samples=200,red]{2/1^3*(x-1)^3/6}
                                %node[right,pos=0.,font=\footnotesize]{$R_2(x;1)=\dfrac{2}{\xi^3}\dfrac{(x-1)^3}{3!}\quad \text{ for } \xi=1$};
                                node[pos=0.5,font=\footnotesize,label={[label distance=20pt]below:{$R_2(x;1)=\dfrac{2}{\xi^3}\dfrac{(x-1)^3}{3!}\quad \text{ f\"ur } \xi=1$}}]{};
    \end{axis}
  \end{tikzpicture}

\begin{tikzpicture}
    \begin{axis}[
     axis lines=middle,clip=false,
            xmin=0,xmax=2, ymin=-0.5,ymax=0.5,
            xticklabel style={black},
            xlabel=$x$,
            ylabel=$y$]
    \addplot[domain=0:2,samples=200,blue]{2/2^3*(x-1)^3/6}
                                node[pos=0.5,font=\footnotesize,label={[label distance=20pt]below:{$R_2(x;1)=\dfrac{2}{\xi^3}\dfrac{(x-1)^3}{3!}\quad \text{ f\"ur } \xi=2$}}]{};
    \end{axis}
  \end{tikzpicture}
%
\end{abc}
}

\ErgebnisC{analysTayl02}
{
Eine Abschätzung des Restglieds ist
$$
R_2(x;1) \leq \frac{1}{3}.
$$
}
