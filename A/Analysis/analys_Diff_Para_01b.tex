\Aufgabe[e]{Differentiation}{
\begin{abc}
\item Geben Sie Konstanten $b,\, c,\, d\in\R$ an, damit die Funktion 
$$
g(x)=\left\{\begin{array}{lll}
b x^3+cx^2+d& \quad & \text{ f\"ur } x\leq 1\\
\ln x &&\text{ f\"ur } x>1\end{array}\right.
$$
auf $\R$ zweimal differenzierbar ist. \\
Ist $g(x)$ f\"ur diese Wahl auch dreimal differenzierbar?
\item \"Uberpr\"ufen Sie, dass die Funktion 
$$  f:\R \to \R, \quad f(x) = \begin{cases}
     x \sin(1/x) & \text{ f\"ur } x \ne 0 , \\ 0 & \text{ f\"ur } x=0 
  \end{cases} $$
in $x=0$ stetig ist, aber \textit{nicht } differenzierbar. \\
\textbf{Hinweis: } Betrachten Sie zur Untersuchung der Differenzierbarkeit die Folge $(x_n)_{n\in \N}$ mit $x_n=\frac{1}{n\pi+\pi/2}$.
\item Die Funktion 
$$ f:\R \to \R, \quad f(x) = \begin{cases}
    x^2 \sin(1/x) & \text{ f\"ur } x \ne 0 , \\
       0 & \text{ f\"ur } x=0 \end{cases} $$
ist in $x=0$ differenzierbar. Ist die Ableitung dort stetig? Hinweis: Betrachten Sie die Folge $(x_n)_{n\in \N}$ mit $x_n=\frac{1}{2n\pi}$.
\end{abc}

}

\Loesung{
\begin{abc}
\item Außerhalb des kritischen Punktes $x=1$ hat man 
$$g'(x)=\left\{\begin{array}{lll}3bx^2+2cx&\quad & \text{ f\"ur } x< 1\\
\frac 1x&&\text{ f\"ur }x>1\end{array}\right.$$
sowie
$$g''(x)=\left\{\begin{array}{lll}6bx+2c&\quad & \text{ f\"ur } x< 1\\
-\frac 1{x^2}&&\text{ f\"ur }x>1\end{array}\right.$$
Da die Funktion zweimal differenzierbar sein soll, m\"ussen $g(x)$ und $g'(x)$ stetig sein, im Punkt
$x=1$ muss also gelten: 
$$g(1)=b+c+d=\ln 1=0 \text{ und } g'(1)=3b+2c=\frac 11=1.$$
Aus der zweiten Relation erh\"alt man $c=\frac {1-3b}2$. \\
Die linksseitige zweite Ableitung im Punkt $x=1$ ist 
\begin{align*}
g''_-(1)&=\underset{\tiny\begin{array}{l}x\to 1\\x<1\end{array}}\lim \frac{g'(x)-g'(1)}{x-1}
= \underset{\tiny\begin{array}{l}x\to 1\\x<1\end{array}}\lim \frac{3bx^2+2cx-1}{x-1}\\
&= \underset{\tiny\begin{array}{l}x\to 1\\x<1\end{array}}\lim \frac{3bx^2+(1-3b)x-1}{x-1}
= \underset{\tiny\begin{array}{l}x\to 1\\x<1\end{array}}\lim \frac{3b(x^2-x)+x-1}{x-1}\\
& = \underset{\tiny\begin{array}{l}x\to 1\\x<1\end{array}}\lim (3bx+1)=3b+1
\end{align*}
und die rechtsseitige zweite Ableitung: 
\begin{align*}
g''_+(1)&=\underset{\tiny\begin{array}{l}x\to 1\\x>1\end{array}}\lim \frac{g'(x)-g'(1)}{x-1}
= \underset{\tiny\begin{array}{l}x\to 1\\x>1\end{array}}\lim \frac{\frac 1x-1}{x-1}\\
&= \underset{\tiny\begin{array}{l}x\to 1\\x>1\end{array}}\lim \frac{1-x}{x(x-1)}
 = \underset{\tiny\begin{array}{l}x\to 1\\x>1\end{array}}\lim \left(-\frac 1x\right)=-1
\end{align*}
Beide Grenzwerte m\"ussen \"ubereinstimmen, damit $g''(1)$ sinnvoll definiert ist, also gilt
$$b=\frac{-2}3\,\Rightarrow \, c=\frac 32\,\Rightarrow\, d=-b-c=\frac{-5}6.$$
Man hat dann 
$$g''(x)=\left\{\begin{array}{lll}
-4x+3&\quad &\text{ f\"ur }x<1\\
-1&&\text{ f\"ur } x=1\\
-\frac 1{x^2}&&\text{ f\"ur } x>1\end{array}\right.$$
Die linksseitige Ableitung dieser Funktion im Punkt $x=1$ ist 
\begin{align*}
g'''_-(1)&=\underset{\tiny\begin{array}{l}x\to 1\\x<1\end{array}}\lim \frac{g''(x)-g''(1)}{x-1}
= \underset{\tiny\begin{array}{l}x\to 1\\x<1\end{array}}\lim \frac{-4x+3-(-1)}{x-1}\\
&= \underset{\tiny\begin{array}{l}x\to 1\\x<1\end{array}}\lim \frac{-4x+4}{x-1}
= \underset{\tiny\begin{array}{l}x\to 1\\x<1\end{array}}\lim \left( -4 \frac{x-1}{x-1}\right)
 = -4
\end{align*}
und die rechtsseitige Ableitung: 
\begin{align*}
g'''_+(1)&=\underset{\tiny\begin{array}{l}x\to 1\\x>1\end{array}}\lim \frac{g''(x)-g''(1)}{x-1}
= \underset{\tiny\begin{array}{l}x\to 1\\x>1\end{array}}\lim \frac{-\frac 1{x^2}-(-1)}{x-1}\\
&= \underset{\tiny\begin{array}{l}x\to 1\\x>1\end{array}}\lim \frac{x^2-1}{x^2(x-1)}
 = \underset{\tiny\begin{array}{l}x\to 1\\x>1\end{array}}\lim \frac{x+1}{x^2}=2
\end{align*}
Da die beiden Werte nicht \"ubereinstimmen ($-4\neq 2$), ist $g''$ im Punkt $x=1$ nicht
differenzierbar. 
\item  Die Funktion ist stetig in $x=0$, denn $|\sin(1/x)| \le 1$
  und somit ist $|f(x)-f(0)|=|f(x)| \le |x| \underset{x \to 0}{\longrightarrow} 0$. \\
  Der Differenzenquotient in $x_0=0$ ist
  $$ Q(x) := \dfrac{f(x)-f(x_0)}{x-x_0} = \dfrac{1}{x} f(x)=\sin(1/x). $$
  Dieser Ausdruck konvergiert \textbf{nicht} f"ur $x \to 0$.
  Setzt man zum Beispiel $x_n = \dfrac{1}{n \pi+\pi/2}$, dann gilt 
  $x_n \underset{n \to \infty}{\longrightarrow} 0$ und
 $$Q(x_n)=\sin(n \pi+\pi/2)=\left\{\begin{array}{rl}
+1,\,&\text{ f\"ur gerade }n\\
-1,\,&\text{ f\"ur ungerade }n
\end{array}\right.$$
\item  Der Differenzenquotient in $x_0=0$ ist
 $$ Q(x) := \dfrac{f(x)-f(x_0)}{x-x_0} = \dfrac{1}{x} f(x)=x\sin(1/x). $$
 Dies konvergiert f"ur $x \to 0$ gegen $0$, wie schon in Teil a) gezeigt
 wurde. Damit ist $f$ in $x_0=0$ differenzierbar und $f^\prime(0)=0$. 
 Die Ableitung f"ur $x \ne 0$ ist
 $$ f^\prime(x)= 2x \sin(1/x) + x^2 \cos(1/x) \left(-\dfrac{1}{x^2}\right)
   = 2x \sin(1/x) - \cos(1/x) . $$
 Sie konvergiert nicht in $x=0$. Sei dazu $x_n=\frac
   1{2n\pi}\underset{n\rightarrow \infty}\longrightarrow 0$. Damit gilt
$$f'(x_n)=\frac 1{2n\pi} \sin(2n\pi) - \cos(2n\pi)=1\neq f'(0),$$
also ist $f'$ nicht stetig in Null. 
\end{abc}
}

%\newcounter{AufganalysDiffRech004}
%\setcounter{AufganalysDiffRech004}{\theAufg}
%\Ergebnis{\subsubsection*{Ergebnisse zu Aufgabe \arabic{Blatt}.\arabic{AufganalysDiffRech004}:}
%
%}
