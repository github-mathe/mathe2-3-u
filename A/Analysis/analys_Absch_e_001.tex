\Aufgabe[e]{Taylor-Entwicklung in einer Variablen}{
Bestimmen Sie die Taylor-Entwicklung zweiter Ordnung der Exponentialfunktion $\operatorname{e}^x$ 
um den Punkt $x_0 = 0$ in dem Intervall $0\leq x\leq 1$ einschlie\ss lich des Restgliedtermes.
Zeigen Sie damit die Absch\"atzung:
$$
\operatorname{e} \leq 3.
$$
}

\Loesung{
Wir beginnen mit der Taylor-Entwicklung der Exponentialfunktion in dem Intervall $0\leq x \leq 1$:
$$
\operatorname{e}^x = 1 + x + \frac{1}{2}x^2 + \frac{\operatorname{e}^\xi}{3!} x^3, 
$$
mit $0\leq\xi \leq1$.

Da die Exponentialfunktion monoton steigend ist und $\xi\le 1$, gilt $\operatorname{e}^\xi \leq \operatorname{e}^1$. 
Daher gilt die folgende Absch\"atzung
$$
\operatorname{e}^x \leq 1 + x + \frac{1}{2}x^2 + \frac{\operatorname{e}}{3!}x^3,
$$
f\"ur $0\leq x \leq 1$. Der obige Ausdruck an der Stelle $x=1$ ausgewertet f\"uhrt zu
\begin{align*}
\operatorname{e} &\leq 1 + 1 + \frac{1}{2} + \frac{\operatorname{e}}{6}, \\ 
\frac{5}{6}\operatorname{e} &\leq \frac{5}{2},\\
\operatorname{e} &\leq 3.\\
\end{align*}
}
