\Aufgabe[e]{Kurvendiskussion}{
F\"uhren Sie eine Kurvendiskussion f\"ur die Funktion 
$$f(x)=\ln (3x^2+2x+1)$$
durch. Bestimmen Sie dazu: 
\begin{abc}
\item den maximalen Definitionsbereich von $f$, 
\item die Symmetrieachsen von $f$, d. h. Werte $\alpha\in \R$, so dass $f(\alpha+x)=f(\alpha-x)$, 
\item das Verhalten von $f$ im Unendlichen, 
\item die Nullstellen von $f$, 
\item die Extrema und das Monotonieverhalten von $f$, 
\item sowie die Wendepunkte und das Kr\"ummungsverhalten von $f$. 
\item Skizzieren Sie den Graphen von $f$. 
\end{abc}
}

\Loesung{
\begin{abc}
\item Die Funktion ist f\"ur alle $x\in \R$ definiert, weil das Argument der Logarithmusfunktion
immer positiv ist: 
$$3x^2+2x+1=\left(\sqrt 3 x+ \frac 1{\sqrt 3}\right)^2 +1-\frac 13>1-\frac 13>0.$$
\item Gesucht ist ein $\alpha\in \R$, so dass f\"ur alle $x\in \R$ gilt: 
\begin{align*}
&& f(\alpha+x)=& f(\alpha-x)\\
\Leftrightarrow&& \ln (3(\alpha+x)^2+2(\alpha+x)+1)=& \ln (3(\alpha-x)^2+2(\alpha-x)+1)\\
\Leftrightarrow&& 3(\alpha+x)^2+2(\alpha+x)+1=& 3(\alpha-x)^2+2(\alpha-x)+1&\text{ (Monotonie
von $\ln$)}\\
\Leftrightarrow&&6x\alpha+2x=&-6x\alpha-2x\\
\Leftrightarrow&&(12\alpha+4)x=&0\\
\Leftrightarrow&&12\alpha+4=& 0
\end{align*}
Also liegt die Symmetrieachse bei $\alpha=-1/3$. 
\item Es ist 
$$\underset{x\to \pm\infty}\lim f(x)=\infty.$$
\item Die einzige Nullstelle des Logarithmus liegt bei $1$, also muss f\"ur $f(x)=0$ gelten:
$$1=3x^2+2x+1\Rightarrow x\in\{0,-2/3\}.$$
\item Die Nullstellen der Ableitung berechnen sich zu: 
\begin{align*}
0=& f'(x)= \frac 1{3x^2+2x+1}\cdot (6x+2)\,\Rightarrow \, x=-\frac 13.
\end{align*}
Dies ist die einzige Nullstelle der Ableitung. Da die Funktion bez\"uglich dieser Achse symmetrisch
ist, und f\"ur $x\rightarrow \pm \infty$ gegen $\infty$ geht, liegt bei $x=-1/3$ ein Minimum. 
\item Die Wendepunkte liegen an den Nullstellen der zweiten Ableitung:
\begin{align*}
&&0=& f''(x)=\frac{6(3x^2+2x+1)-(6x+2)^2}{(3x^2+2x+1)^2}=\frac{-18x^2-12x+2}{(3x^2+2x+1)^2}\\
\Rightarrow&&x=&\frac{-1\pm \sqrt 2}3.
\end{align*}
Links von $-1/3-\sqrt{2}/3$ ist $f''(x)<0$, also ist die Funktion dort konkav, rechts von
$-1/3-\sqrt 2/3 $ ist $f''(x)>0$
und die Funktion $f$ ist dort konvex. Rechts von $-1/3+\sqrt 2/3$ ist die Funktion wegen der
Symmetrie wiederum konkav. 
\item \quad\\
\begin{minipage}{.4\textwidth}
\psset{xunit=1.8cm, yunit=1.8cm, runit=1cm}
\begin{pspicture}(-3,-1)(2,3)
\psgrid[subgriddiv=1,griddots=10,gridlabels=.3](-3,-1)(2,3)
\psline{-}(-.333,-1)(-.333,3)
\psplot[plotpoints=200, plotstyle=curve]
{-2.333}{2}
{3 x mul x mul 2 x mul add 1 add ln}
\psdot(0,0)
\psdot(-.666,0)
\psdot(-.333,-.405)
\psdot(.138,.288)
\psdot(-.805,.288)
\end{pspicture}
\end{minipage}
\end{abc}
}

\ErgebnisC{AufganalysKurvDisk004}
{
{\textbf{ b)}} $\alpha=-1/3$,
{\textbf{ d)}} $0$, $-2/3$,
{\textbf{ e)}} Minimum bei $-1/3$,
{\textbf{ f)}} Wendepunkte bei $-1/3\pm \sqrt 2/3$
}
