\Aufgabe[e]{~}{
Bestimmen Sie den Fl\"acheninhalt der Fl\"ache, die durch die Gerade $f_1(x)=x+2$ und die Parabel $f_2(x)=4-x^2$ begrenzt wird. 
}

\Loesung{
Zun\"achst bestimmt man die Schnittpunkte der beiden Funktionsgraphen, indem man \(f_1(x) = f_2(x)\) l\"ost:
\[
 x+2=4-x^2 \Rightarrow x^2+x-2=0 \Rightarrow x=-2 \text{ oder } x=1
\]
Die beiden Funktionsgraphen schneiden sich in \(x=-2\) und \(x=1\). 
Skizziert man die Funktionsgraphen, erkennt man, wie die Fl\"ache von den beiden Funktionsgraphen begrenzt wird, \(f_2(x)\) liegt hierbei \"uber \(f_1(x)\).


\begin{center}
\psset{yunit=1cm}
\begin{pspicture}(-3,-1)(3,4)
\psgrid(-3,-1)(3,4)
\psplot[plotpoints=100,plotstyle=curve]
{-2.21}{2.21}
{4 x neg x mul add}
\psplot[plotpoints=100,plotstyle=curve, fillstyle=solid, fillcolor=lightgray]
{-2}{1}
{4 x neg x mul add}
\psline[showpoints=true](-2,0)(1,3)
\psline(-3,-1)(2,4)
\put(2.1,.1){$f_2(x)$}
\put(1.6,3.3){$f_1(x)$}
\end{pspicture}
\end{center}

Die Funktionen schneiden sich lediglich in zwei Punkten, also gibt es nur ein zu
ber\"ucksichtigendes Fl\"achenst\"uck. Dessen Fl\"acheninhalt wird berechnet, indem man die Differenz der Integrale der beiden Funktionen bestimmt.
\begin{align*}
A & = \int\limits_{-2}^1 f_2(x)\,\d x- \int_{-2}^1 f_1(x) \,\d x = \int_{-2}^1 \left(f_2(x)-f_1(x) \right) \,\d x\\[2ex]
&= \int\limits_{-2}^1(-x^2-x+2) \,\d x = \left[-\dfrac{1}{3} x^3-\dfrac{1}{2}x^2 +2x\right]_{-2}^1
= \frac 92\,.
\end{align*}

}


\ErgebnisC{b-2011-K2-A5}{
 $9/2$
}
