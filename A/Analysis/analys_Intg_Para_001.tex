\Aufgabe[e]{Parametrisierung von Integrationsbereichen}
{
Gegeben seien die folgenden drei K\"orper im $\R^3$: 
\begin{itemize}
\item ein Quader $Q=\{\vec x\in\R^3|\, 0\leq x_1\leq 1,\, 0\leq x_2\leq 1,\, -3\leq x_3\leq 3\}$
\item eine Kugel $K=\{\vec x\in\R^3|\, \norm{\vec x}\leq 1\}$
\item ein Zylinder $Z=\{\vec x\in\R^3|\, x_1^2+x_2^2\leq 1,\ 0\leq x_3\leq 3\}$
\end{itemize}
Skizzieren und parametrisieren Sie die Schnittmenge eines jeden einzelnen dieser K\"orper mit den beiden folgenden Mengen: 
$$M_1=\{\vec x\in\R^3|\, 0\leq \vec x_3\},\, M_2=\{\vec x|\,3x_1\leq x_3\}.$$ 
D. h. geben Sie die Integrationsgrenzen der zugeh\"origen Volumenintegrale \"uber die Bereiche $Q\cap M_1,\, Q\cap M_2,\, K\cap M_1,\, \hdots$ an.
 
}

\Loesung{
$M_1$ beschreibt den oberen Halbraum $z\geq 0$. $M_2$ beschreibt die Menge der Punkte oberhalb der Ebene $z=3x$. F\"ur die Schnittmengen mit den drei K\"orpern hat man jeweils: 
\begin{itemize}
\item F\"ur den Quader: 
\begin{align*}
Q\cap M_1=&\{\vec x\in \R^3|\, 0\leq x_1\leq 1,\, 0\leq x_2\leq 1,\, 0\leq x_3\leq 3\}\\
Q\cap M_2=&\{\vec x\in \R^3|\, 0\leq x_1\leq 1,\, 0\leq x_2\leq 1,\, 3x_1\leq x_3\leq 3\}
\end{align*}
F\"ur $Q\cap M_2$ muss man keine Fallunterscheidung der $x_3$-Grenzen vornehmen, da die Obergrenze des Quaders ($z=3$) die Ebene $3y=z$ nur an der Kante des Quaders schneidet. 
\begin{center}
\begin{pspicture}(-5,-5)(5,5)

\psline[linewidth=1pt, linecolor=lightgray, fillstyle=solid, fillcolor=lightgray]
(0,0)
(1,0)
(1.70711,0.70711)
(   1.70711,   3.70711)
(0.70711,0.70711)
(0,0)

\psline[linewidth=2pt, linecolor=black]
(0,0)
(1,0)
(1.70711,0.70711)
(.70711,.70711)
(0,0)

\psline[linewidth=1pt, linecolor=gray, fillstyle=solid, fillcolor=gray]
(0,0)
(.70711,.70711)
(1.70711,3.70711)
(0.70711,3.70711)
(0,3)
(0,0)

\psline[linewidth=2pt, linecolor=black]
(0,0)
(.70711,.70711)
(1.70711,3.70711)
(1,3)
(0,0)

\psline[linewidth=2pt]
(   0.00000,  -3.00000)
(   1.00000,  -3.00000)
(   1.00000,   3.00000)
(   0.00000,   3.00000)
(   0.00000,  -3.00000)
(   0.70711,  -2.29289)

\psline[linewidth=2pt]
(   0.70711,  -2.29289)
(   1.70711,  -2.29289)
(   1.70711,   3.70711)
(   0.70711,   3.70711)
(   0.70711,  -2.29289)

\psline[linewidth=2pt](   1.00000,  -3.00000)(   1.70711,  -2.29289)
\psline[linewidth=2pt](   1.00000,   3.00000)(   1.70711,   3.70711)
\psline[linewidth=2pt](   0.00000,   3.00000)(   0.70711,   3.70711)

\psline[linecolor=black]{->}(-5,0)(5,0)
\psline[linecolor=black]{->}(0,-4)(0,5)
\psline[linecolor=black]{->}(-2,-2)(3,3)
\put(5,.1){$x$}
\put(3,3.1){$y$}
\put(.1,5){$z$}

\end{pspicture}
\end{center}

\item F\"ur die Kugel: 
\begin{align*}
K\cap M_1=&\Bigl\{\vec x\in \R^3|\, -1\leq x_1\leq +1,\, -\sqrt{1-x_1^2}\leq x_2\leq +\sqrt{1-x_1^2},\, 0\leq x_3\leq \sqrt{1-x_2^2-x_3^2}\Bigr\}
\end{align*}
Die zweite Schnittmenge $K\cap M_2$ besteht aus zwei Bereichen: 
$B_1$ der Bereich, der von oben durch die Kugeloberfl\"ache und von unten durch die Ebene $3x=z$ begrenzt wird. \\
$B_2$, der von oben und von unten durch die Kugeloberfl\"ache begrenzt wird, da die Ebenbe dort außerhalb der Kugel liegt. \\
F\"ur die Schnittkurve der Kugeloberfl\"ache $x^2+y^2+z^2=1$ mit der Ebene $3x=z$ gilt
\begin{align*}
&&x^2+y^2+9x^2=&1\\
\Leftrightarrow&&x=&\pm\sqrt{\frac{1-y^2}{10}}
\end{align*}
Damit darf $y$ nur Werte zwischen $-1$ und $+1$ annehmen.\\
$B_1$ l\"asst sich somit parametrisieren als
\begin{align*}
B_1=&\Bigl\{\vec x\in \R^3|\, -1\leq x_2\leq 1,\, -\sqrt{\frac{1-x_2^2}{10}}\leq x_1\leq \sqrt{\frac{1-x_2^2}{10}},\\
&\qquad \,3x_1\leq x_3\leq \sqrt{1-x_1^2-x_2^2}\Bigr\}\\
\end{align*}
F\"ur den zweiten Teil von $K\cap M_2$ ergibt sich
\begin{align*}
B_2=&\Bigl\{\vec x\in \R^3|\, -1\leq x_2\leq 1,\, -\sqrt{1-x_2^2}\leq x_1\leq -\sqrt{\frac{1-x_2^2}{10}},\\
&\qquad \,-\sqrt{1-x_1^2-x_2^2}\leq x_3\leq \sqrt{1-x_1^2-x_2^2}\Bigr\}\\
\end{align*}

Eine Parametrisierung in Kugelkoordinaten, deren $z$-Achse ($\tilde z$ in der Skizze) senkrecht auf der Ebene $3x=z$ steht, w\"are f\"ur diesen K\"orper deutlich einfacher. Die entsprechende Rotation um die $y$-Achse wird durch die (orthogonale) Matrix 
$$\Vec R = \frac 1{\sqrt{10}}\begin{pmatrix}
1 &0 &-3 \\
0 &\sqrt{10} & 0 \\
3 &0 & 1 
\end{pmatrix}$$
beschrieben. Damit ergibt sich dann
\begin{align*}
\vec x(r,\theta,\varphi)=&\frac 1{\sqrt{10}}\begin{pmatrix}
1 &0 &-3 \\
0 &1 & 0 \\
3 &0 & 1 
\end{pmatrix}\begin{pmatrix}r\sin\theta\cos\varphi\\r\sin\theta\sin\varphi\\r\cos\theta\end{pmatrix}\\
=& \frac r{\sqrt 10} \begin{pmatrix}
\sin\theta\cos\varphi-3\cos\theta\\
\sin\theta\sin\varphi\\
3\sin\theta\cos\varphi+\cos\theta
\end{pmatrix}
\end{align*}
und weiter
$$K\cap M_2=\left\{\vec x(r,\theta,\varphi)|\, 0\leq r\leq 1,\, 0\leq \theta\leq \pi/2,\, 0\leq \varphi\leq 2\pi\right\}.$$     

\begin{center}
\begin{pspicture}(-5,-3)(5,3.5)

\psellipticarc[linecolor=black, linewidth=2pt](0,0)(2,2){0}{180}
\psellipse[linecolor=black, linewidth=2pt](0,0)(2,1.1)
\psellipticarc[linecolor=black, linewidth=2pt](0,0)(.8,2){55}{238}

\psline[linecolor=black]{->}(-5,0)(5,0)
\psline[linecolor=black]{->}(0,-2)(0,3)
\psline[linecolor=black]{->}(-2,-3)(2,3)
\put(5,.1){$x$}
\put(2,3.1){$y$}
\put(.1,3){$z$}
\put(2,2){$K\cap M_1$}
\end{pspicture}
\end{center}
\begin{center}
\begin{pspicture}(-5,-3)(5,3.5)

%\psellipticarc[linecolor=black, linewidth=2pt](0,0)(2,2){70}{250}
%\psellipticarc[linecolor=black, linewidth=2pt](0,0)(2,1.1){57}{235}
%\psellipticarc[linecolor=black, linewidth=2pt](0,0)(.8,2){0}{360}
\rput{60}(0,0){
\psellipticarc[linecolor=black, linewidth=2pt](0,0)(2,2){0}{180}
\psellipse[linecolor=black, linewidth=2pt](0,0)(2,1.1)
\psellipticarc[linecolor=black, linewidth=2pt](0,0)(.8,2){-56}{124}
%\psellipse[linecolor=black, linewidth=2pt](0,0)(2,.45)
\psline{->}(0,0)(0,4)
}
\psarc(0,0){0.5}{90}{152}
\put(-.6,.4){$\vec R$}
\put(-3.5,1.5){$\tilde z$}




\psline[linecolor=black]{->}(-5,0)(5,0)
\psline[linecolor=black]{->}(0,-2)(0,3)
\psline[linecolor=black]{->}(-2,-3)(2,3)
\put(5,.1){$x$}
\put(2,3.1){$y$}
\put(.1,3){$z$}
\put(2,2){$K\cap M_2$}
\end{pspicture}
\end{center}
\item F\"ur den Zylinder nutzen wir die Parametrisierung in Zylinderkoordinaten 
$$\vec x(r,\varphi,z)=\begin{pmatrix} r\cos\varphi\\r\sin\varphi\\z\end{pmatrix}$$
Die erste Menge $Z\cap M_1$ stimmt mit dem Zylinder \"uberein. 
\begin{align*}
Z=Z\cap M_1=&\left\{\vec x(r,\varphi,z)|0\leq r\leq 1,\, 0\leq \varphi\leq 2\pi,\, 0\leq z\leq 3\right\}\\
Z\cap M_2=& \left\{\vec x(r,\varphi,z)|0\leq r\leq 1,\, 0\leq \varphi\leq 2\pi,\, z_0(r,\varphi)\leq z\leq 3\right\}
\end{align*}
Dabei ber\"ucksichtigt $z_0(r,\varphi)$, dass die Ebene $3x=z$ den Zylinderboden in der Mitte schneidet. Dies f\"uhrt dazu, dass f\"ur positive $x$ die Untergrenze des Integrationsbereichs von der Ebene beschrieben wird und f\"ur  negative $x$ durch den Zylinderboden $z=0$: 
$$z_0(r,\varphi)=\left\{\begin{array}{ll}
0,& \text{ f\"ur } \frac{\pi}2 \leq \varphi\leq \frac{3\pi}2\\
3r\cos(\varphi),&\text{ sonst }
\end{array}.\right.$$
\end{itemize}
\begin{center}
\begin{pspicture}(-5,-3)(5,5)

\psellipticarc[linecolor=lightgray, fillcolor=lightgray, fillstyle=solid](0,0)(1,.6){80}{240}
\psline[fillstyle=solid, fillcolor=lightgray, linecolor=lightgray](-1,0)(0,0)(1,3)(-1,3)(-1,0)
\rput{71}(0,0){\psellipticarc[linecolor=black, linewidth=2pt, fillstyle=solid, fillcolor=lightgray](0,0)(3.1,.2){-17}{165}}
\psellipse[linecolor=black, linewidth=2pt](0,0)(1,.6)
\psellipse[linecolor=black, linewidth=2pt, fillstyle=solid, fillcolor=lightgray](0,3)(1,.6)
\psline[linecolor=black, linewidth=2pt](1,0)(1,3)
\psline[linecolor=black, linewidth=2pt](-1,0)(-1,3)
\psline[linecolor=black]{->}(-5,0)(5,0)
\psline[linecolor=black]{->}(0,-2)(0,3)
\psline[linecolor=black]{->}(-2,-3)(3,4.5)
\put(5,.1){$x$}
\put(3,4.6){$y$}
\put(.1,3){$z$}
\put(2,2){$Z\cap M_2$}
\end{pspicture}
\end{center}

}

%\ErgebnisC{analysIntgZyln002}{}
