\Aufgabe[e]{Kurven im $\R^n$}{
Ein Satellit bewege sich auf der Bahn 
$$\vec r(t) = \bigl( \sqrt{1+t^2},\, 3t \bigr)^\top$$.
\begin{abc}
\item Berechnen Sie die Geschwindigkeit $\dot{ \vec r}(t)$ des Satelliten. 
\item Geben Sie die Grenzwerte (f\"ur $t\rightarrow +\infty$ und f\"ur $t\rightarrow -\infty$)  der
Bahnrichtung $r_2/r_1$  und Geschwindigkeit $\dot {\vec r}$ des Satelliten an. \\
\end{abc}

}

\Loesung{
\begin{abc}
\item Die Geschwindigkeit ist
$$\dot{\vec r}(t) = \left( \frac{t}{\sqrt{1+t^2}}, 3\right)^\top.$$
\item Es ist 
\begin{align*}
\underset{t\rightarrow \pm \infty}\lim\frac {r_2}{r_1} =
&\underset{t\to \pm \infty}\lim \frac{3t}{\sqrt{1+t^2}} \\
=& \underset{t\to \pm\infty}\lim \frac{3\text{ sign}(t)}{\sqrt{\frac 1{|t|^2}+1}}=\pm 3
\end{align*}
Die Bahn verl\"auft also asymptotisch in Richtung der beiden Geraden $\Spn\{(1,\pm 3)^\top\}$. \\
F\"ur die Geschwindigkeit gilt passend dazu 
$$\underset{t\rightarrow \pm\infty}\lim \dot{\vec r}(t)=(\pm 1,3)^\top.$$

Der Graph zeigt die Bahnkurve des Satelliten sowie deren Asymptoten.

\psset{xunit=1cm, yunit=1cm, runit=1cm}
\begin{pspicture}(-1,-6)(5,6)
\psgrid[subgriddiv=1,griddots=10,gridlabels=0](-1,-6)(5,6)
\psline{-}(2,-6)(0,0)(2,6)
\parametricplot[plotpoints=200, plotstyle=curve]
{-4}{4}
{1 t t mul add sqrt .5 mul  3 t mul .5 mul}
\rput[l](1,0){$\vec r(t)$}
\psline{->}(-1,-6.5)(-1,6.5)
\psline{->}(-1.5,-6)(5.5,-6)
\rput[r](-1,6){$r_1$}
\rput[t](5,-6){$r_2$}
\end{pspicture}


\end{abc}
}


\ErgebnisC{AufganalysSatlBahn001}
{
$\underset{t\to\infty}\lim \frac{r_2(t)}{r_1(t)}=3$, $\underset{t\to\pm\infty}\lim \dot{\vec r}(t)=(\pm 1,3)^\top$
}
