\Aufgabe[e]{Newton-Verfahren}
{
Gegeben seien die beiden Kurven in der oberen Halbebene
\[
	(y-3)^2-x=3 \text{ und } xy=4\ ,\ \ y\ge 0\ .
\]
\begin{itemize}
\item[a)] Skizzieren Sie die beiden Kurven und lesen Sie aus Ihrer Skizze eine 
nicht-ganzzahlige Näherung für den Schnittpunkt ab.
\item[b)] Um den Schnittpunkt numerisch zu bestimmen, f\"uhren Sie \textbf{einen} 
Iterationsschritt mit dem \textbf{zweidimensionalen} Newton--Verfahren durch, 
wobei Sie den Punkt \ $(x_0,y_0):=(1,5)\in\N^2$ \  als Startwert benutzen.
\item[c)] Hat sich die Näherung durch den Newtonschritt aus Teil b) gegen\"uber 
dem Startwert \ $(x_0,y_0)$ \ verbessert\,? Kurze Begründung!
\end{itemize}
}
\Loesung{
\textbf{Zu a)} \\
	

\begin{center}
\begin{pspicture}(-4,0)(6,6)
\psgrid[subgriddiv=5,subgridcolor=gray,gridcolor=black,gridlabels=.3](-4,0)(6,6)
\psplot[plotpoints=100,plotstyle=curve]
{.7}{6}
{4 x div}

\psparametricplot[plotpoints=100,plotstyle=curve]
{0}{6}
{t -3 add 2 exp -3 add
t}

\end{pspicture}\\
\end{center}


Der Schnittpunkt der beiden Kurven liegt in der Nähe von \ $(x,y)=(0.80\,,\,4.9)$\,.

\bigskip
\textbf{Zu b)} \ Der Startwert sei \ $(x_0\,,\,y_0)=(1,5)$\,. Das nichtlineare 
Gleichungssystem ist
\[
	\vec f(x,y)=\begin{pmatrix} (y-3)^2-x-3 \\ xy-4 \end{pmatrix} = \begin{pmatrix} 0 \\ 0 \end{pmatrix}\ .
\]
Damit ist die Matrix der ersten partiellen Ableitungen
\[
	\vec J_{\vec f}(x,y) = \begin{pmatrix} -1 & 2y-6 \\ y & x \end{pmatrix}\ .
\]
Der erste Newton--Schritt lautet mit eingesetzten Zahlenwerten
\begin{align*}
\begin{pmatrix} x_1 \\ y_1 \end{pmatrix} & = \begin{pmatrix} 1 \\ 5 \end{pmatrix} + \vec{\Delta x}
\end{align*}
Dabei ist $\vec{\Delta x}$ L\"osung des Gleichungssystems
$$\vec J_{\vec f}(\vec x_0)\vec{\Delta x}=-\vec f(\vec x_0):$$
$$\begin{array}{rr|l|l}
-1 & 4 & 0\\ 
5 & 1 & -1 & + 5\times I\\\hline

-1 & 4 & 0\\ 
0 &21 & -1 
\end{array}$$
Daraus ergibt sich $\vec{\Delta x}=\frac 1{21}\begin{pmatrix}-4\\-1\end{pmatrix}$
und damit 
$$\vec x_1=\begin{pmatrix}1\\5\end{pmatrix}-\frac{1}{21}\begin{pmatrix}4\\1\end{pmatrix} = \dfrac{1}{21}\begin{pmatrix}17\\104\end{pmatrix}.$$


\bigskip
\textbf{Zu c)} Die (euklidische) Norm des Funktionswertes ist kleiner geworden: \ 
 \[
\vec f(x_0,y_0) =  \begin{pmatrix} 0 \\ 1 \end{pmatrix} \,, \qquad  
\vec f(x_1\,,\,y_1) = \begin{pmatrix} 0.0023 \\ 0.0091 \end{pmatrix} \ .
\]

\medskip
Alternativ: Ja, denn die verbesserte Näherung liegt viel dichter an dem aus der 
Zeichnung abgelesenen Wert als die Ausgangsnäherung.
}


\ErgebnisC{Aufgb-2011A-K-A3}
{
\textbf{Zu b)} $\text D\, \vec f(x,y) = \begin{pmatrix} -1 & 2y-6 \\ y & x \end{pmatrix}\ .$
}
