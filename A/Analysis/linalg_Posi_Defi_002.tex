\Aufgabe[e]{Positive Definitheit}{
\begin{abc}
\item Berechnen Sie die Eigenwerte der Matrizen
$$
\vec A = \begin{pmatrix}-3& 0& 1\\ 0& -1 &0\\ 1&0&-3\end{pmatrix},\qquad
\vec B= \begin{pmatrix} 4/3& \sqrt 5/3    \\ \sqrt 5/3    & 7/6\end{pmatrix} \qquad
\text{ und }\qquad\vec C=\begin{pmatrix}0& 1\\ 1&0 \end{pmatrix}
$$
\item Welche der Matrizen sind positiv oder negativ definit, welche sind indefinit?
\end{abc}
}

\Loesung{
\begin{abc}
\item \begin{iii}
\item Das charakteristische Polynom der Matrix $\vec A$ ist:
\begin{align*}
\det(\vec A-\lambda\vec E)=&\det\begin{pmatrix}
-3-\lambda&0&1\\0&-1-\lambda&0\\1&0&-3-\lambda\end{pmatrix}\\
=&(-3-\lambda)^2(-1-\lambda)-(-1-\lambda)
\end{align*}
und seine Nullstellen 
$$\lambda_1=-1,\, \lambda_2=-2,\, \lambda_3=-4$$
sind die Eigenwerte der Matrix $\vec A$. 
\item Das charakteristische Polynom der Matrix $\vec B$ ist
$$\det(\vec B-\lambda \vec E)=\left(\frac 43 -\lambda\right)\left( \frac
76-\lambda\right)-\frac{5}9$$
und seine Nullstellen
$$\lambda_{1,2}=\frac 54\pm \sqrt{\frac{25}{16}+\frac 59-\frac{14}{9}}=\frac{5\pm 3}4\in\{2,\,
1/2\}$$
 sind die Eigenwerte von $\vec B$. 
\item Es gilt $\vec C\begin{pmatrix}1\\1\end{pmatrix}=\begin{pmatrix}1\\1\end{pmatrix}$ und $\vec
C\begin{pmatrix}1\\-1\end{pmatrix}=\begin{pmatrix}-1\\1\end{pmatrix}=-1\cdot \begin{pmatrix}1\\-1\end{pmatrix}$. \\

Also hat $\vec C$ die Eigenwerte $1$ und $-1$. 
\end{iii}
\item Alle drei Matrizen sind symmetrisch. \\
Zudem sind die Eigenwerte von $\vec A$  negativ, also ist $\vec A$ negativ definit. \\
Die Eigenwerte von $\vec B$ sind alle  positiv, also ist $\vec B$  positiv definit. \\
$\vec C$ besitzt einen positiven und einen negativen Eigenwert, ist also indefinit. 
\end{abc}
}

\ErgebnisC{AufglinalgPosiDefi002}
{
\textbf{ a)} $\vec A$: $-1,\, -2,\, -4$, $\vec B$: $1/2,\, 2$, $\vec C$: $-1,\, 1$

}
