\Aufgabe[e]{}
{

Gegeben sei die Funktion $f(x) = \cos \left(\dfrac{\pi}{2} \sin x\right)$.

\begin{iii}
 
 \item Bestimmen Sie den Definitionsbereich und den Wertebereich von $f$. 
 
 \item Zeigen Sie, dass die Funktion $f$ die Periodizität $\pi$ besitzt, d.h.\ zeigen Sie, dass $f(x+\pi) = f(x)$ für alle $x\in \R$ gilt.
 
 \item Bestimmen Sie alle Nullstellen von $f$.\\[0.5ex]
 \textbf{Hinweis:} Beachten Sie die Periodizität von $f$.
 
 \item Bestimmen Sie alle Extrema von $f$ und charakterisieren Sie diese.\\[0.5ex]
 \textbf{Hinweis:} Beachten Sie die Periodizität von $f$.
 
 \item Skizzieren Sie den Graphen von $f$ im Intervall $[-\pi,2\pi]$. 
 
\end{iii}
}


\Loesung{
\begin{iii}
\item  Der Wertebereich der Sinusfunktion ist $[-1,1]$. Auf $[-\pi/2,\pi/2]$ 
nimmt der Kosinus alle Werte von $0$ bis $1$ an. Somit gilt 
\[
D(f) =\R \qquad \text{ und } \qquad W(f) = [0,1]\,.
\]

\medskip
\item  Es gilt
\begin{align*}
 f(x+\pi) &  = \cos \left(\frac{\pi}{2} \sin (x+\pi)\right)\\[1ex]
 & = \cos \left(-\frac{\pi}{2} \sin x \right)\\[1ex]
 & = \cos \left(\frac{\pi}{2} \sin x\right)\\[1ex]
 & = f(x) 
\end{align*}
für alle $x\in \R$. Somit hat $f$ die Periode $\pi$. Hinweis: Es lässt sich zeigen, dass $f$ keine kleinere Periode als $\pi$ besitzt. 

\medskip
\item Aufgrund der Periodizität reicht es aus, die Nullstellen im Intervall $[0,\pi]$ zu bestimmen. Es gilt 
\begin{align*}
 f(x) & = \cos \left(\frac{\pi}{2} \sin x\right) = 0 \\[1ex]
 \Longleftrightarrow \quad \frac{\pi}{2} \sin x & = \frac{\pi}{2}\,.
\end{align*}
Es ergibt sich $x=\pi/2$. Die Menge aller Nullstellen ist somit
\[
N = \left\{\frac{\pi}{2} + k\pi \;\; \Big| \;\; k\in \mathbb Z \right\}\,.
\]

\medskip
\item Es gilt 
\[
f'(x) = - \sin \left(\frac{\pi}{2}\sin x\right)\frac{\pi}{2}\cos x\,.
\]
Es folgt 
\begin{align*}
 f'(x) & = - \sin \left(\frac{\pi}{2}\sin x\right)\frac{\pi}{2}\cos x = 0 \\[1ex]
 \Longleftrightarrow \quad \sin \left(\frac{\pi}{2}\sin x\right) & = 0 \quad \text{oder} \quad \cos x = 0\,.
\end{align*}
Der erste Fall liefert $x=0$ und $x=\pi$. Der zweite Fall ergibt $x=\pi/2$. Da $f$ nur Werte zwischen $0$ und $1$ annimmt, sind $x_1=(0,1)$ und $x_3=(\pi,1)$ Maxima, während $x_2 = (\pi/2,0)$ Minimum ist. Die Menge aller Maxima ist 
\[
E_{\max} = \{(k\pi,1) \mid k\in \mathbb Z \}\,.
\]
Die Menge aller Minima ist
\[
E_{\max} = \left\{ \left(\frac{\pi}{2}+k\pi,0\right) \mid k\in \mathbb Z \right\}\,.
\]

\medskip
\item
\end{iii}
\begin{center}
	\begin{pspicture}(-4,-1)(7,2)
	\psgrid[griddots=8,subgriddiv=0](-4,-1)(7,2)
	\psline[linewidth=1.2pt]{->}(-4,0)(7,0)
	\psline[linewidth=1.2pt]{->}(0,-1)(0,2)
        \psplot[plotpoints=100, plotstyle=curve]{-4}{7}
        {
        x 180 mul 3.14159 div sin 90 mul cos
        }
        \psdot(-1.5708,0)
        \psdot(1.5708,0)
        \psdot(4.7124,0)
        \psdot(  -3.14159,   1.00000)
        \psdot(   0.00000,   1.00000)
        \psdot(   3.14159,   1.00000)
        \psdot(   6.283  ,   1.00000)
        \psdot(  -1.57080,   0.00000)
        \psdot(   1.57080,   0.00000)
        \psdot(   4.71239,   0.00000)

%\cos \left(\dfrac{\pi}{2} \sin x\right)$.



	\end{pspicture} 

%\boxed{\includegraphics[width =0.25\textwidth]{./fig_f.eps}}
\end{center}

}

