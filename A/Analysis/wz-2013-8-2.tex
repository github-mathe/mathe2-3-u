\Aufgabe[e]{Taylor--Polynom  \& Extrema in 2 Dimensionen}{
\begin{itemize}
	\item[ a)] Berechnen Sie das Taylor--Polynom 2. Grades der Funktion $f(x,y) = (2x-3y)\cdot\sin(3x-2y)$ zum Entwicklungspunkt $ \vec x_0=(0,0)^{\text {T} }$ \,.
	
	\item[ b)] Ermitteln Sie die Extrema der Funktion \ $f(x,y)=2x^3-3xy+2y^3-3$\,.
\end{itemize}
}

\Loesung{
\textbf{a)} 
%NOTE: Die Sytnax der Lösung muss überarbeitet werden.
% % \textbf{1. Lösungsweg:}
% 	\[
% 	\ZAB{1.5}\bARY{rclcrcr}
% 	f & = & (2x-3y)\cdot\sin(3x-2y) & \PFEIL & f(0,0) & = & 0\\[1ex]
% 	f_x & = & 2\,\sin(3x-2y)+3\,(2x-3y)\cdot\cos(3x-2y) & \PFEIL & f_x(0,0) & = & 0 \\
% 	f_y & = & -3\,\sin(3x-2y)-2\,(2x-3y)\cdot\cos(3x-2y) & \PFEIL & f_y(0,0) & = & 0 \\[1ex]
% 	f_{xx} & = & 12\,\cos(3x-2y)-9\,(2x-3y)\cdot\sin(3x-2y) & \PFEIL & f_{xx}(0,0) & = & 12 \\
% 	f_{xy} & = & -13\,\cos(3x-2y)+6\,(2x-3y)\cdot\sin(3x-2y) & \PFEIL & f_{xy}(0,0) & = & -13 \\
% 	f_{yy} & = & 12\,\cos(3x-2y)-4\,(2x-3y)\cdot\sin(3x-2y) & \PFEIL & f_{yy}(0,0) & = & 12 \\
% 	\eARY
% \]
% Damit erhält man für das gesuche Taylor--Polynom die Darstellung:
% 	\[
% 	T_2(x,y) = 6\,x^2 -13\,xy+6\,y^2\ .
% \]
% 
% % \textbf{2. Lösungsweg:}
% % 
% % Da der Entwicklungspunkt der Ursprung des Koordinatensystems ist, bietet sich ein sehr viel einfacheres Vorgehen an:
% % Mit
% % 	\[
% % 	\sin(t)=t-\dfrac{t^3}{6}+... \UND t=(3x-2y) \PFEIL \sin(3x-2y)=(3x-2y)-\dfrac{(3x-2y)^3}{6}+...
% % \]
% % Das gesuchte Taylor--Poynom der Funktion \ $f$ \ erhält man jetzt als das Produkt von dem Faktor \ $(2x-3y)$ \ mit dem Anfang der Taylor--Entwicklung der Sinus--Funktion, wobei nur Terme bis zur zweiten Ordnung berücksichtigt werden müssen:
% % 	\[
% % 	T_2(x,y) = (2x-3y)\cdot (3x-2y) = 6\,x^2 -13\,xy+6\,y^2\ .
% % \]
% 
% \VS
% \textbf{b)} Die stationären Punkte erhält man aus
% \begin{align*}
% 	\textbf{grad}\,f(x,y) &= \lARY{c} 6\,x^2 -3\,y \\ -3\,x+6y^2 \rARY = \textbf{0}\\ 
% 	\Rightarrow \begin{cases}y=2x^2 \\ -3x+24x^4=0 \end{cases} 
% 	&\Rightarrow \begin{cases} y = 2x^2\\ 3x\left(8x^3-1\right)=0 \end{cases}
% 	\Rightarrow  \begin{cases}y = 2x^2\\ x = 0  \vee x = \frac12 \end{cases}
% \end{align*}
% zu \ $P_1 = (0,0)$ \ und \ $P_2 = \big(\frac 12,\frac 12\big)$\ .
% 
% Die Hesse--Matrix ist
% 	\[
% 	\operatorname{Hess}_f(x,y) = \lARY{cc} 12x & -3 \\ -3 & 12y \rARY .
% \]
% 
% Für \ $P_1 = (0,0)$ \ erhält man
% 	\[
% 	\operatorname{Hess}_f(0,0) = \lARY{rr} 0 & -3 \\ -3 & 0 \rARY \PFEIL \lambda_1 \cdot \lambda_2 = \det(\operatorname{Hess}_f(0,0)) = -9 < 0\ .
% \]
% Damit haben die Eigenwerte verschiedene Vorzeichen und es handelt sich um einen Sattelpunkt.
% 
% Für \ $P_2 = \big(\frac 12,\frac 12\big)$ \ erhält man
% 	\[
% 	\operatorname{Hess}_f\big(\tfrac 12,\tfrac 12\big) = \lARY{rr} 6 & -3 \\ -3 & 6 \rARY \PFEIL \lambda_1 \cdot \lambda_2 = \det((\operatorname{Hess}_f\big(\tfrac 12,\tfrac 12\big)) = 27 > 0
% \]
% sowie 
% \[
% f_{xx}\big(\tfrac 12,\tfrac 12\big)= 6 > 0\,.
% \]
% Damit sind beide Eigenwerte positiv und es handelt sich um ein Minimum.
}

\ErgebnisC{Aufgwz-2013-8-2}
{
\[
T_2(x,y) = 6\,x^2 -13\,xy+6\,y^2\,.
\]
Kritische Punkte: $P_1 = (0,0)$ und $P_2 = \big(\frac 12,\frac 12\big)$.
 }

