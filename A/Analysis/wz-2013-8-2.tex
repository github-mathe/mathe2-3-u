\Aufgabe[e]{Taylor--Polynom  \& Extrema in 2 Dimensionen}{
\begin{itemize}
	\item[ a)] Berechnen Sie das Taylor--Polynom 2. Grades der Funktion $f(x,y) = (2x-3y)\cdot\sin(3x-2y)$ zum Entwicklungspunkt $ \vec x_0=(0,0)^{\text {T} }$ \,.
	
	\item[ b)] Ermitteln Sie die Extrema der Funktion \ $f(x,y)=2x^3-3xy+2y^3-3$\,.
\end{itemize}
}

\Loesung{
\begin{abc}
%NOTE: Die Sytnax der Lösung muss überarbeitet werden.

\item
\begin{align*}
 	f & = & (2x-3y)\cdot\sin(3x-2y) & \Rightarrow & f(0,0) & = & 0\\[1ex]
 	f_x & = & 2\,\sin(3x-2y)+3\,(2x-3y)\cdot\cos(3x-2y) &  \Rightarrow & f_x(0,0) & = & 0 \\
 	f_y & = & -3\,\sin(3x-2y)-2\,(2x-3y)\cdot\cos(3x-2y) &  \Rightarrow & f_y(0,0) & = & 0 \\[1ex]
 	f_{xx} & = & 12\,\cos(3x-2y)-9\,(2x-3y)\cdot\sin(3x-2y) & \Rightarrow & f_{xx}(0,0) & = & 12 \\
 	f_{xy} & = & -13\,\cos(3x-2y)+6\,(2x-3y)\cdot\sin(3x-2y) &  \Rightarrow & f_{xy}(0,0) & = & -13 \\
 	f_{yy} & = & 12\,\cos(3x-2y)-4\,(2x-3y)\cdot\sin(3x-2y) & \Rightarrow & f_{yy}(0,0) & = & 12 \\
\end{align*}
 Damit erhält man für das gesuche Taylor--Polynom die Darstellung:
\[
 	T_2(x,y) = 6\,x^2 -13\,xy+6\,y^2\ .
 \]
 
 % \textbf{2. Lösungsweg:}
 % 
 % Da der Entwicklungspunkt der Ursprung des Koordinatensystems ist, bietet sich ein sehr viel einfacheres Vorgehen an:
 % Mit
 % 	\[
 % 	\sin(t)=t-\dfrac{t^3}{6}+... \UND t=(3x-2y) \PFEIL \sin(3x-2y)=(3x-2y)-\dfrac{(3x-2y)^3}{6}+...
 % \]
 % Das gesuchte Taylor--Poynom der Funktion \ $f$ \ erhält man jetzt als das Produkt von dem Faktor \ $(2x-3y)$ \ mit dem Anfang der Taylor--Entwicklung der Sinus--Funktion, wobei nur Terme bis zur zweiten Ordnung berücksichtigt werden müssen:
 % 	\[
 % 	T_2(x,y) = (2x-3y)\cdot (3x-2y) = 6\,x^2 -13\,xy+6\,y^2\ .
 % \]
 
\item Die stationären Punkte erhält man aus
 \begin{align*}
 	\textbf{grad}\,f(x,y) &= \begin{pmatrix} 6\,x^2 -3\,y \\ -3\,x+6y^2 \end{pmatrix} = \textbf{0}\\ 
 	\Rightarrow \begin{cases}y=2x^2 \\ -3x+24x^4=0 \end{cases} 
 	&\Rightarrow \begin{cases} y = 2x^2\\ 3x\left(8x^3-1\right)=0 \end{cases}
 	\Rightarrow  \begin{cases}y = 2x^2\\ x = 0  \vee x = \frac12 \end{cases}
 \end{align*}
 zu \ $P_1 = (0,0)$ \ und \ $P_2 = \big(\frac 12,\frac 12\big)$\ .
 
 Die Hesse--Matrix ist
 	\[
 	H(x,y) = \begin{pmatrix}12x & -3 \\ -3 & 12y \end{pmatrix}.
 \]
 
 Für \ $P_1 = (0,0)$ \ erhält man
 	\[
 	H(0,0) = \begin{pmatrix} 0 & -3 \\ -3 & 0 \end{pmatrix} \Rightarrow \lambda_1 \cdot \lambda_2 = \det(H(0,0)) = -9 < 0\ .
 \]
 Damit haben die Eigenwerte verschiedene Vorzeichen und es handelt sich um einen Sattelpunkt.
  
 Für \ $P_2 = \big(\frac 12,\frac 12\big)$ \ erhält man
 	\[
 	H\big(\tfrac 12,\tfrac 12\big) = \begin{pmatrix} 6 & -3 \\ -3 & 6 \end{pmatrix} \Rightarrow \lambda_1 \cdot \lambda_2 = \det(H\big(\tfrac 12,\tfrac 12\big)) = 27 > 0
 \]
% sowie 
% \[
% f_{xx}\big(\tfrac 12,\tfrac 12\big)= 6 > 0\,.
% \]
 Damit sind beide Eigenwerte positiv und es handelt sich um ein Minimum.

\end{abc}
}

\ErgebnisC{Aufgwz-2013-8-2}
{
\[
T_2(x,y) = 6\,x^2 -13\,xy+6\,y^2\,.
\]
Kritische Punkte: $P_1 = (0,0)$ und $P_2 = \big(\frac 12,\frac 12\big)$.
 }

