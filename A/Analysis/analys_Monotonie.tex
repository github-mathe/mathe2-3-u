\Aufgabe[e]{Monotonieverhalten}{
Bestimmen Sie jeweils das Monotonieverhalten der folgenden 
Funktionen. Geben Sie dazu die Intervalle an, in denen die 
Funktion monoton steigend bzw. monoton fallend ist.

\begin{abc}
\item $f(x) = 2x^3 - 3x^2 + 1$
\item $g(x) = -\cos(x) - 2\sin(x/2)$
\end{abc}
}
\Loesung{
\begin{abc}
\item

\item
Um alle station\"aren Punkte zu bestimmen, berechnen wir die Nullstellen der 
ersten Ableitung
$$
g'(x) = \sin(x) -\cos(x/2)  \overset{!}{=} 0.
$$
Mit der Substitution $y = x/2$ erhalten wir
$$
0 = \sin(2y) - \cos(y).
$$
Wir wenden das Additionstheorem $\sin(2\alpha) = 2 \sin(\alpha)\cos(\alpha)$ an.
$$
0 = 2\sin(y) \cos(y) - \cos(y) = (2\sin(y) - 1)\cos(y).
$$
Es muss also gelten 
\begin{align*}
\sin(y) = 1/2 \quad \text{ oder } \quad \cos(y) = 0.
\end{align*}
Damit erhalten wir die Nullstellen
\begin{align*}
y &= \frac{\pi}{6} + 2 \pi n \, \text{ f\"ur } n \in \mathbb{Z}\\
y &= \frac{5\pi}{6} + 2 \pi n \, \text{ f\"ur } n \in \mathbb{Z}\\
y &= \frac{\pi}{2} +  \pi n \, \text{ f\"ur } n \in \mathbb{Z}
\end{align*}
Durch R\"ucksubstitution erhalten wir die station\"aren Punkte:
\begin{align*}
x &= \frac{\pi}{3} + 4 \pi n \, \text{ f\"ur } n \in \mathbb{Z}\\
x &= \frac{5\pi}{3} + 4 \pi n \, \text{ f\"ur } n \in \mathbb{Z}\\
x &= \pi +  4 \pi n \, \text{ f\"ur } n \in \mathbb{Z}\\
x &= 3\pi +  4 \pi n \, \text{ f\"ur } n \in \mathbb{Z}\\
\end{align*}
Wir sehen, dass $g(x)$ eine Periodenl\"ange von $4\pi$ besitzt.
Die Funktion $g(x)$ ist zwischen den station\"aren Punkten monoton.

In den Intervallen 
$$I_1 = (\frac{\pi}{3}+ 4 \pi n , \pi+ 4 \pi n ) \text{ f\"ur }
n \in \mathbb{Z}
$$
ist $g(x)$ monoton steigend.

In den Intervallen 
$$I_2 =(\pi +  4 \pi n,\frac{5\pi}{3} + 4 \pi n) \text{ f\"ur }
n \in \mathbb{Z}$$ ist $g(x)$ monoton fallend.

In den Intervallen 
$$I_3 = (\frac{5\pi}{3}+ 4 \pi n , 3\pi+ 4 \pi n ) \text{ f\"ur }
n \in \mathbb{Z}$$ ist $g(x)$ monoton steigend.

In den Intervallen 
$$I_4 =(3\pi +  4 \pi n,\frac{\pi}{3} + 4 \pi n) \text{ f\"ur }
n \in \mathbb{Z}$$ ist $g(x)$ monoton fallend

\end{abc}


}
