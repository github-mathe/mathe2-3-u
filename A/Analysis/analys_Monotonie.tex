\Aufgabe[e]{Monotonieverhalten}{
Untersuchen Sie das Monotonieverhalten der folgenden Funktionen.
Geben Sie dazu alle Intervalle an, in denen die 
Funktion monoton steigend bzw. monoton fallend ist.

\begin{abc}
\item $f(x) = 2x^3 - 3x^2 + 1$
\item $g(x) = -\cos(x) - 2\sin(x/2)$
\end{abc}
}
\Loesung{
\begin{abc}
\item
Die erste Ableitung ist
$$
f'(x) = 6x^2 -6x = x(6x-6)
$$
Damit sind die station\"aren Punkte bei 
$$
x_1 = 0 \quad \text{ und } \quad x_2 = 1.
$$
Mit der zweiten Ableitung erhalten wir f\"ur $x_1 = 0$
$$
f''(0) = -6
$$
und f\"ur $x_2 = 1$
$$
f''(1) = 6.
$$
Damit ist bei $x_1 = 0$ ein Maximum und bei $x_2 = 1$ Minimum.
In den Intervallen $(-\infty, 0 )$ und $(1,\infty)$ ist $f(x)$ monoton steigend
und in dem Intervall $(0,1)$ ist $f(x)$ monoton fallend.
\item
Um alle station\"aren Punkte zu bestimmen, berechnen wir die Nullstellen der 
ersten Ableitung
$$
g'(x) = \sin(x) -\cos(x/2)  \overset{!}{=} 0.
$$
Mit der Substitution $y = x/2$ erhalten wir
$$
0 = \sin(2y) - \cos(y).
$$
Wir wenden das Additionstheorem $\sin(2\alpha) = 2 \sin(\alpha)\cos(\alpha)$ an.
$$
0 = 2\sin(y) \cos(y) - \cos(y) = (2\sin(y) - 1)\cos(y).
$$
Es muss also gelten 
\begin{align*}
\sin(y) = 1/2 \quad \text{ oder } \quad \cos(y) = 0.
\end{align*}
Damit erhalten wir die Nullstellen
\begin{align*}
y_1 &= \frac{\pi}{6} + 2 \pi n \, \text{ f\"ur } n \in \mathbb{Z}\\
y_2 &= \frac{5\pi}{6} + 2 \pi n \, \text{ f\"ur } n \in \mathbb{Z}\\
y_3 &= \frac{\pi}{2} +  \pi n \, \text{ f\"ur } n \in \mathbb{Z}
\end{align*}
Durch R\"ucksubstitution erhalten wir die station\"aren Punkte:
\begin{align*}
x_1 &= \frac{\pi}{3} + 4 \pi n \, \text{ f\"ur } n \in \mathbb{Z}\\
x_2 &= \frac{5\pi}{3} + 4 \pi n \, \text{ f\"ur } n \in \mathbb{Z}
\end{align*}
Da f\"ur $y_3$ die Perioden $\pi$ ist, erhalten wir nach der 
R\"ucksubstitution im Intervall $[0,4\pi)$
die zwei station\"aren Punkte $x_3$ und $x_4$.
\begin{align*}
x_3 &= \pi +  4 \pi n \, \text{ f\"ur } n \in \mathbb{Z}\\
x_4 &= 3\pi +  4 \pi n \, \text{ f\"ur } n \in \mathbb{Z}
\end{align*}


Wir sehen, dass $g(x)$ eine Periodenl\"ange von $4\pi$ besitzt.
Die Funktion $g(x)$ ist zwischen den station\"aren Punkten monoton.

In den Intervallen 
$$I_1 = (\frac{\pi}{3}+ 4 \pi n , \pi+ 4 \pi n ) \text{ f\"ur }
n \in \mathbb{Z}
$$
ist $g(x)$ monoton steigend.

In den Intervallen 
$$I_2 =(\pi +  4 \pi n,\frac{5\pi}{3} + 4 \pi n) \text{ f\"ur }
n \in \mathbb{Z}$$ ist $g(x)$ monoton fallend.

In den Intervallen 
$$I_3 = (\frac{5\pi}{3}+ 4 \pi n , 3\pi+ 4 \pi n ) \text{ f\"ur }
n \in \mathbb{Z}$$ ist $g(x)$ monoton steigend.

In den Intervallen 
$$I_4 =(3\pi +  4 \pi n,\frac{\pi}{3} + 4 \pi n) \text{ f\"ur }
n \in \mathbb{Z}$$ ist $g(x)$ monoton fallend

\begin{tikzpicture}
  \draw[->] (-2, 0) -- (8, 0) node[right] {$x$};
  \draw[->] (0, -2) -- (0, 2) node[above] {$y$};
  \node[red] at (1.5,1) {$f(x)$};
  \node[blue] at (-1.5,1) {$g(x)$};
  \draw[scale=0.5, domain=-2:16, smooth, variable=\x, blue] plot (\x,{-cos(\x r)-2*sin(\x/2 r)});
  \draw[scale=0.5, domain=-1:1.8, smooth, variable=\y, red]  plot ({\y}, {2*\y*\y*\y - 3*\y*\y + 1});
\end{tikzpicture}


\end{abc}


}
