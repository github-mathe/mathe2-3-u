\Aufgabe[e]{Newton-Verfahren}
{
Gegeben sei die Funktion 
$$f(x)=x^3+2x^2-15x-36.$$
\begin{abc}
\item Bestimmen Sie alle lokalen Extrema der Funktion $f(x)$. 
\item Begr\"unden Sie, weshalb $f$ zwei Nullstellen besitzt. 
\item F\"uhren Sie das Newton-Verfahren mit der Funktion $f$ zwei mal durch.  
W\"ahlen Sie im ersten Durchlauf den Startwert $x_0=1$ und
im  zweiten Durchlauf $x_0=-1$. F\"uhren Sie jeweils drei Iterationsschritte durch. 
\end{abc}
}

\Loesung{
\begin{abc}
\item Die Ableitung der Funktion $f$ ist
$$f'(x)=3x^2+4x-15.$$
Ihre Nullstellen 
$$x_{1/2}=\frac{-2\pm\sqrt{49}}3=\left\{\begin{array}{l}5/3\\-3\end{array}\right.$$
sind die station\"aren Punkte der Funktion $f$. \\
Die zweite Ableitung der Funktion ist $f''(x)=6x+4$ und wegen 
$$f''(x_1)=14>0\text{ und } f''(x_2)=-14<0$$
liegt bei $(x_1,f(x_1))=(5/3,-1372/27)$ ein Minimum und bei 
$(x_2,f(x_2))=(-3,0)$ ein Maximum der Funktion $f$ vor. 
\item Da $f$ links der Maximalstelle $x_2=-3$ (die auch Nullstelle ist) sowie rechts der
Minimalstelle $x_1=5/3$ streng monoton steigt, und sonst streng monoton f\"allt, besitzt $f$
lediglich die Nullstelle $x_2$ sowie (wegen $f(x_1)<0$) eine weitere Nullstelle rechts von $x_1$. 
\item Mit der oben berechneten Ableitung $f'(x)=3x^2+4x-15$  lautet die Iterationsvorschrift f\"ur das Newton-Verfahren: 
$$x_{k+1}=x_k-(f'(x_k))^{-1}\cdot f(x_k).$$
F\"ur die gegebenen Startwerte sind die ersten drei Iterationsschritte:\\


\begin{tabular}{r|r|r}
$k$&$x_k$    & $f(x_k)$\\\hline
 0 & 1.0000  &  -48.0000 \\
 1 & -5.0000 &  -36.0000 \\
 2 & -4.1000 &   -9.8010 \\
 3 & -3.5850 &   -2.5955 \\
\end{tabular}
\qquad\qquad
\begin{tabular}{r|r|r}
$k$&$x_k$    & $f(x_k)$\\\hline
    0 &  -1.00000 &  -20.00000 \\
    1 &  -2.25000 &  -3.51562  \\
    2 &  -2.64894 &  -0.81945  \\
    3 &  -2.82923 &  -0.19916  \\
\end{tabular}\\

Obwohl der erste Startwert $x_0=1$ dichter an der Nullstelle $x=4$ der Funktion liegt, konvergiert
das Verfahren trotzdem gegen die Nullstelle $x=-3$. \\
Um die Nullstelle $x=4$ dennoch zu korrekt zu
finden, muss $x_0$ geeignet gew\"ahlt werden. 
\end{abc}
}

\ErgebnisC{analysNewtVerf002}{
Es ist $f(-3)=f(4)=0$.} 
 
 
