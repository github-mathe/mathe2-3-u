\Aufgabe[e]{Mittelwertsatz}{
Sei $s(t)$ die Gesamtzahl der Kilometer, die Sie auf Ihrer Reise auf einer Autobahn mit einer Geschwindigkeitsbegrenzung von 120 km/h nach der Zeit t Stunden zurückgelegt haben. Nehmen Sie außerdem an, dass $s(1/4)=10$ Kilometer und $s(5/4)=160$ Kilometer beträgt. An einer Kontrollstelle entlang der Autobahn gestehen Sie diese Tatsachen einem Beamten der Autobahnpolizei, der mit dem Mittelwertsatz vertraut ist. Der Beamte führt ein paar schnelle Berechnungen durch, lächelt und bereitet sich dann höflich darauf vor, Ihnen einen Strafzettel auszustellen. Erklären Sie, warum.

}

\Loesung{
Wir wissen, dass $s(t)$ die Entfernung (Kilometer) zum Zeitpunkt $t$ (Stunden) ist. Wir nehmen an, dass $s(t)$ differenzierbar ist, wobei $s'(t)$ die Geschwindigkeit (km/h) zum Zeitpunkt $t$ (Std.) ist. Wenden wir den Mittelwertsatz auf die Funktion $s$ auf dem geschlossenen Intervall $[1/4,5/4]$ an. Dann 
%
$$
s'(\xi) =\frac{160-10}{5/4-1/4} = 150 \text{ km/h}.
$$
%
Der Mittelwertsatz beweist, dass Ihre Geschwindigkeit zum unbekannten Zeitpunkt xi mit Sicherheit größer war als die zulässige Höchstgeschwindigkeit.

}
