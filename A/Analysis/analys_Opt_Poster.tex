\Aufgabe[e]{Optimierungsproblem}{
%
Ein Poster muss mit einer Gesamtfläche von $\bar A = 64000 \text{ mm}^2$ gedruckt werden. Es muss 10 mm Seitenränder und 25 mm obere und untere Ränder haben. Welche Höhe und Breite ergeben die maximale Druckfläche?
%

}


\Loesung{
%
Seien $x$ und $y$ die beiden Dimensionen des Plakats und $s$ die Seitenränder und $t$ die oberen und unteren Ränder. Die Gesamtfläche ist $\bar A = 64000 \text{ mm}^2= xy$ mm$^2$. Die gedruckte Fläche beträgt
\begin{align*}
A(x) &= (x-2s)(y-2t) \\
&= (x-2s)\left(\frac{\bar A}{x} -2t\right) \quad (\text{unter Verwendung der Nebenbedingung der Gesamtfläche})\\\
& = \bar A - \frac{2s}{x} \bar A - 2tx + 4 st \text{ mm}^2.
\end{align*}
Um die Fläche zu maximieren, suchen wir die stationären Punkte:
$$
A'(x) = 2s\bar A \frac{1}{x^2} -2 t.
$$
Die stationären Punkte sind:
$$
x_c = \pm \sqrt{\frac{s\bar A}{t}}.
$$
Nur der positive Wert ist sinnvoll, da wir nach physikalischen Größen suchen.
Die zweite Ableitung ist
$$
A''(x) = -\frac{4s\bar A}{x^3}
$$
die in $x_c$ negativ ist. Daher ist $x_c$ ein lokales Maximum.
Wir haben also
$$
x_c = 160 \text{ mm}
$$
und
$$
y_c = \frac{\bar A}{x_c} = 400 \text{ mm}.
$$
Die Gesamtfl\"ache des Posters ergibt sich somit aus
$$
\bar A = x_c\, y_c = 64000 \text{ mm}^2.
$$
Die maximale Druckfläche beträgt
$$
A_{\text{max}} = (x_c-2s)\, (y_c-2t) = 49000 \text{ mm}^2.
$$
Der Graph der Fläche $A(x)$ ist  \\ \\ 

\begin{minipage}{\linewidth}
\centering

\begin{tikzpicture}
\begin{axis}[
axis lines=middle,clip=false,
xmin=-200,
xmax=800,
ymin=-10000,
ymax=50000,
xticklabel style={black},
xlabel=$x$,
ylabel=$y$]

\addplot[domain=18:600,samples=200,blue]{(x-20)*(64000/x-50)}
node[right,pos=1.,font=\footnotesize]{$A(x)$};

\end{axis}
\end{tikzpicture}
\end{minipage}

}

\ErgebnisC{AufganalysOptPoster}
{
Die maximale Druckfläche beträgt:
$
A_{\text{max}} = 49000 \text{ mm}^2.
$
}
