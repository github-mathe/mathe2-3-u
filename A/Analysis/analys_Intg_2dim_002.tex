\Aufgabe[e]{Schwerpunkt und Tr\"agheitsmoment einer Kreisscheibe}
{
Gegeben sei die Kreisscheibe
$$K =\{\vec x\in \R^2|\quad \norm{\vec x - (0,2)^\top}_2\leq 2\}$$
mit der Massendichte $\rho(x,y)=x^2+4$. 
\begin{abc}
\item Berechnen Sie die Masse $M=\int\limits_K\rho(\vec x)\mathrm{d} \vec x$ der Kreisscheibe. F\"uhren Sie
  die Rechnung in kartesischen Koordinaten durch. 
\item Berechnen Sie ebenso den Schwerpunkt 
$$\vec s=\frac 1 M \int\limits_K \rho(\vec x)\vec x \mathrm{d} \vec x.$$
F\"uhren Sie die Rechnung in den verschobenen Polarkoordinaten \\
$\vec x(r,\varphi)=(r\cos \varphi,\,2+r\sin\varphi)^\top$ aus. 
\item Berechnen Sie das Tr\"agheitsmoment bez\"uglich der $x$-Achse
$$\Theta_x = \int\limits_K \rho(x,y)y^2\mathrm{d} (x,y).$$
\end{abc}
}

\Loesung{
\begin{abc}
\item Das Integrationsgebiet wird definiert durch 
$$2\geq \norm{\vec x - (0,2)^\top}_2=\sqrt{x^2+(y-2)^2}.$$
Wir stellen dies nach $y$ um: 
\begin{align*}
&&4\geq& x^2+(y-2)^2\\
\Rightarrow&&4-x^2\geq& (y-2)^2\\
\Rightarrow&&\sqrt{4-x^2}\geq y-2\text{ und } -\sqrt{4-x^2}\leq y-2\\
\Rightarrow&&2+\sqrt{4-x^2}\geq y \text{ und } 2-\sqrt{4-x^2}\leq y.
\end{align*}
Die zul\"assigen $x$-Werte sind damit durch $-2\leq x\leq 2$ gegeben. Das Integral f\"ur die Masse
ist nun
\begin{align*}
M=& \int\limits_K\rho(\vec x)\mathrm{d}x
= \int\limits_{x=-2}^2\int\limits_{y=2-\sqrt{4-x^2}}^{2+\sqrt{4-x^2}}(x^2+4)\mathrm{d} y\mathrm{d} x\\
=& \int\limits_{x=-2}^2(x^2+4)\left[2+\sqrt{4-x^2}-(2-\sqrt{4-x^2})\right]\mathrm{d} x\\
=& 2\int\limits_{x=-2}^2(x^2+4)\sqrt{4-x^2}\mathrm{d}x.
\end{align*}
Dieses Integral l\"asst sich durch die Substitution 
$$x=2\sin u, \, \mathrm{d} x = 2 \cos u \mathrm{d} u,\, u\in[-\pi/2,\pi/2]$$ 
umformen zu
\begin{align*}
M=&2\int\limits_{u=-\frac \pi 2}^{\frac \pi 2}(4\sin^2 u + 4)\sqrt{4-4\sin^2 u}\cdot 2 \cos u \mathrm{d} u\\
=& 32\int\limits_{u=-\frac \pi 2}^{\frac \pi 2}(\sin^2 u + 1)\cos u \cos u\mathrm{d} u
= 32\int\limits_{u=-\frac \pi 2}^{\frac \pi 2}\Bigl( (\sin u \cos u)^2 +\cos^2 u\Bigr)\mathrm{d} u\\
=& 32 \int\limits_{u=-\frac \pi 2}^{\frac \pi 2}\Bigl( \left( \frac 12 \sin(2u)\right)^2 + \frac
12(\cos^2 u+1-\sin^2 u)\Bigr)\mathrm{d} u\\
=& 32\int\limits_{u=-\frac \pi 2}^{\frac \pi 2}\Bigl( \frac{\sin^2( 2u)+1-\cos^2(2u)}8 + \frac
{\cos(2u)+1}2\Bigr)\mathrm{d} u\\
=& 32\int\limits_{u=-\frac \pi 2}^{\frac \pi 2}\Bigl( \frac{1-\cos(4u)}8
+ \frac{\cos(2u)+1}2\Bigr)\mathrm{d} u=32\cdot \left[ \frac {5u}8
- \frac{\sin(4u)}{32}+ \frac{\sin(2u)}4\right]_{-\frac{\pi}2}^{\frac \pi 2}\\
=& 20\pi
\end{align*}
\item Die Rechnung in Polarkoordinaten ist einfacher. Wir belassen die Gleichung in ihrer
vektoriellen Form:
\begin{align*}
\vec s=& \frac 1M \int\limits_K\rho(\vec
x)\begin{pmatrix}r\cos\varphi\\2+r\sin\varphi\end{pmatrix}\mathrm{d}\vec x
= \frac
1M \int\limits_{\varphi=0}^{2\pi}\int\limits_{r=0}^2(r^2\cos^2\varphi+4)\begin{pmatrix}r\cos\varphi\\2+r\sin\varphi\end{pmatrix}
r\mathrm{d} r \mathrm{d} \varphi\\
=& \frac 1M \int\limits_{\varphi=0}^{2\pi}\left( \left( \frac{2^4}4\cos^2\varphi+
4 \frac{2^2}2\right)\begin{pmatrix}0\\2\end{pmatrix} + \left( \frac{2^5}5\cos^2\varphi +
4\frac{2^3}3\right)\begin{pmatrix}\cos\varphi\\\sin\varphi\end{pmatrix}\right)\mathrm{d} \varphi\\
=&\frac 1M \left( 4 \cdot \pi + 8\cdot 2\pi \right)\begin{pmatrix}0\\2\end{pmatrix}
+ \frac{32}{5M}\int\limits_{0}^{2\pi}\begin{pmatrix}\cos^3\varphi\\\cos^2\varphi\sin\varphi\end{pmatrix}\mathrm{d}\varphi\\
=& \frac{20\pi}M\begin{pmatrix}0\\2\end{pmatrix}
+ \frac{32}{5M}\int\limits_{0}^{2\pi}\begin{pmatrix}\cos\varphi
- \cos\varphi\sin^2\varphi\\\cos^2\varphi\sin\varphi\end{pmatrix}\mathrm{d}\varphi\\
=& \begin{pmatrix}0\\2\end{pmatrix}
+ \frac{32}{5M}\left.\begin{pmatrix}\sin\varphi-\frac{\sin^3\varphi}3\\-\frac{\cos^3\varphi}3\end{pmatrix}\right|_0^{2\pi}=\begin{pmatrix}0\\2\end{pmatrix}.
\end{align*}
Dieses Ergebnis war zu erwarten, da die Massendichte nicht von $y$ abh\"angt und bez\"uglich $x$
symmetrisch ist ($\rho(x)=\rho(-x)$). 
\item Alternativ zur obigen Integrationsreihenfolge  integrieren wir hier zuerst nach  $\varphi$: 
\begin{align*}
\Theta_y=& \int\limits_0^2\int\limits_0^{2\pi}(r^2\cos^2\varphi+4)(2+r\sin\varphi)^2r \mathrm{d} \varphi\mathrm{d} r\\
=&\int\limits_0^2\int\limits_0^{2\pi}\Bigl(4r^2\cos^2\varphi+4r^3\sin\varphi\cos^2\varphi+r^4\cos^2\varphi\sin^2\varphi+
\\
&\qquad +16+16r\sin\varphi+4r^2\sin^2\varphi\Bigr)\mathrm{d}\varphi
r \mathrm{d} r\\
=& \int\limits_0^2\left( 4r^2\pi + 0 + r^4\int\limits_0^{2\pi}\frac 14 \sin^2(2\varphi) \mathrm{d} \varphi +
16\cdot 2\pi +0+4r^2\pi\right)r \mathrm{d} r\\
=&\int\limits_0^2\left( 8\pi r^3+\frac 14 r^5 \pi + 32\pi r\right)\mathrm{d} r=2\pi 2^4 + \frac \pi
4\cdot \frac{2^6}6+16\pi \cdot 2^2\\
=&96\pi + \frac 83\pi = \frac{296\pi}3.
\end{align*}
\end{abc}
}

\ErgebnisC{analysIntg2dim002}
{
{{\textbf{a)}} $M = 20\pi$,  {\textbf{b)}} $\vec s = \begin{pmatrix}0\\2\end{pmatrix}$, {\textbf{c)}} $\Theta_y = \frac{296\pi}3$
}
}
