\Aufgabe[e]{Stetigkeit}{
Betrachten Sie die Funktion $y=f(x)$ mit
\begin{align*}
f(x) = 
\begin{cases}
\frac{1}{x} \quad &\text{ für } x \in (-\infty, -1],\\
\frac{3}{x} \quad &\text{ für } x \in (-1,0),\\
\frac{x^2-1}{x-1}  \quad &\text{ für } x \in [0,1) \cup (1,\infty),\\
3  \quad &\text{ für } x = 1.
\end{cases}
\end{align*}

und deren Graphen

\centering
\tikzset{
  jumpdot/.style={mark=*,solid},
  excl/.append style={jumpdot,fill=white},
  incl/.append style={jumpdot,fill=black},
}
\begin{tikzpicture}
    \begin{axis}[
     axis lines=middle,clip=false,
            xmin=-4.5,xmax=4.5, ymin=-5,ymax=5,
            xticklabel style={black},
            xlabel=$x$,
            ylabel=$y$]
    \addplot[domain=-5:-1,samples=200,red]{1/x};
    \addplot[domain=-1:-0.5,samples=200,red]{3/x};
    \addplot[incl] coordinates {(-1,-1)};
    \addplot[excl] coordinates {(-1,-3)};
    \addplot[domain=0:3,samples=200,red]{(x^2-1)/(x-1)};
    \addplot[incl] coordinates {(0,1)};
    \addplot[excl] coordinates {(1,2)};
    \addplot[incl] coordinates {(1,3)};
    \end{axis}
  \end{tikzpicture}
%
\begin{abc}
\item Finden Sie alle Werte an denen die Funktion unstetig ist.
\item Begründen Sie für jeden dieser Werte,  weshalb die formale Definition der Stetigkeit verletzt ist.
\item Klassifizieren Sie jede der Untetigkeitsstellen als \textbf{Sprungstelle}, \textbf{hebbare Unstetigkeit} oder \textbf{Polstelle}.
\end{abc}
}
\Loesung{
Die Funktion ist unstetig bei
\begin{iii}
\item $x = -1$,
\item $x = 0$,
\item $x = 1$.
\end{iii}
\begin{iii}
\item
Die Funktion ist unstetig für 
$x=-1$. Diese Unstetigkeit entspricht einer Sprungstelle, da
die links- und rechtsseitigen Grenzwerte existieren (sprich, auf einen endlichen Wert konvergieren), diese aber nicht übereinstimmen:
$$
\lim_{x \to -1^-} f(x) = \lim_{x \to -1^-} \frac{1}{x} = -1,
$$
$$
\lim_{x \to -1^+} f(x) = \lim_{x \to -1^+} \frac{3}{x} = -3.
$$
\item
Die Funktion ist unstetig bei $x=0$. Dies ist eine Polstelle, da der
linksseitiger Grenzwert nicht existiert.
$$
\lim_{x \to 0^-} f(x) = \lim_{x\to 0^-} \frac{3}{x} = -\infty,
$$
$$
\lim_{x \to 0^+} f(x) = \lim_{x \to 0^+} \frac{x^2-1}{x-1} = \lim_{x\to 0^+} x+1 =1.
$$
%
\item
Die Unstetigkeit bei $x =1$ ist hebbar, da deren links- und rechtsseitige Grenzwerte existieren und übereinstimmen
$$
\lim_{x \to 1^-} f(x) = \lim_{x \to 1^-} \frac{x^2-1}{x-1} = \lim_{x\to 1^-} x+1 =2,
$$
$$
\lim_{x \to 1^+} f(x) = \lim_{x \to 1^+} \frac{x^2-1}{x-1} = \lim_{x\to 1^+} x+1 =2,
$$
jedoch vom Funktionswert $f(1) = 3$ an der Stelle abweichen. 
\end{iii}
}
