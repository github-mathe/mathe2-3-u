\Aufgabe[e]{Folgenkonvergenz}{
Untersuchen Sie die nachfolgenden Folgen auf Konvergenz und bestimmten Sie, wenn m\"oglich, den
Grenzwert: 
\begin{align*}
a_n=& \frac{2n^2+3n}{2n^2+7}&&&b_n=&\frac{2n^3-2}{5n^2-1}\\
c_n=& \frac{2n^2+7n+(-1)^n}{5n+2} - \frac{2n^3-2}{5n^2-1},&\qquad\qquad&&d_n=& \left( 1 + \frac
1{\sqrt n}\right)^n\\
e_n =& n\left( \sqrt{n^2+1}-\sqrt{n^2-1}\right)&&& f_n=&\left( 1 + \frac x n \right)^n\text{ (mit ganzzahligem $x$)}\\
g_n=& \left( \frac{n-1}{n+1}\right)^{n+1} &&&&
\end{align*}
\textbf{Hinweise}: \\
- Setzen Sie voraus, dass $f_n$ konvergiert und ermitteln Sie den Grenzwert. \\
- Benutzen Sie zur Untersuchung von $g_n$ das Ergebnis f\"ur $f_n$.
}

\Loesung{
\begin{abc}
\item Der Bruch wird zun\"achst mit dem Kehrwert der h\"ochsten auftretenden $n$-Potenz ($1/n^2$) erweitert:
\begin{align*}
a_n=& \frac {2 + \frac 3{n}}{2+\frac 7{n^2}} 
\end{align*}
Die beiden Ausdr\"ucke $\frac 3n$ und $\frac 7{n^2}$, die noch von $n$ abh\"angen, konvergieren
beide gegen Null, damit konvergieren Z\"ahler und Nenner jeweils gegen $2$ und man hat insgesamt: 
\begin{align*}
a_n=& \frac{2+\frac 3n}{2+\frac
7{n^2}}\overset{n\rightarrow \infty}\longrightarrow \frac{2+\underset{n\rightarrow \infty}\lim \frac
3n}{2+\underset{n\rightarrow\infty}\lim \frac 7{n^2}}=\frac {2+0}{2+0}=1.
\end{align*}

\item Da die $n$-Potenz des Z\"ahlers gr\"oßer als die des Nenners ist, nehmen wir an, dass $b_n$
divergiert. Um dies zu zeigen, sch\"atzen wir ab: 
\begin{align*}
b_n=& \frac{2n^3-2}{5n^2-1}= \frac{\frac 25 (5n^2 -1)n + \frac 25\cdot n -2 }{5n^2-1}= \frac 25 n
+ \underbrace{\frac{2n-10}{25n^2-5}}_{\geq 0 \text{ f\"ur }n\geq 5}\\
\geq& \frac 25 n \underset{n\rightarrow \infty}\longrightarrow \infty
\end{align*}
Also ist $b_n$ gr\"o\ss er als die divergente Folge $\frac 25 n$ und divergiert ebenfalls. 

\item Der Ausdruck f\"ur $c_n$ wird zun\"achst auf einen Bruchstrich gebracht und dann mit dem
Kehrwert der h\"ochsten $n$-Potenz ($1/n^3$) erweitert: 
\begin{align*}
c_n=& \frac{(2n^2+7n+(-1)^n)(5n^2-1)-(2n^3-2)(5n+2)}{(5n+2)(5n^2-1)}\\
=& \frac{31n^3+(5\cdot(-1)^n-2)n^2+ 3n-(-1)^n+4}{25n^3+10n^2-5n-2}\\
=& \frac{31 + \frac{5\cdot(-1)^n-2}{n} + \frac 3{n^2} + \frac{4-(-1)^n}{n^3}}{25
+ \frac{10}n-\frac{5}{n^2} - \frac 2{n^3}}\underset{n\rightarrow\infty}\longrightarrow \frac
{31}{25}
\end{align*}
Alle Ausdr\"ucke, die noch von $n$ abh\"angen, haben die Form $\frac z{n^l}$, wobei $z$ entweder
konstant oder zumindest beschr\"ankt ist, deswegen konvergieren diese Ausdr\"ucke $z/n^l$ gegen
Null. \\
Bringt man beide Ausdr\"ucke nicht auf einen Hauptnenner, kann man nichts weiter \"uber die
Konvergenzeigenschaften sagen, da die einzelnen Br\"uche divergieren. (siehe $b_n$)
\item Wir zeigen, dass $d_n$ durch eine divergente Folge nach unten abgesch\"atzt werden kann und
folgern die Divergenz von $d_n$: F\"ur $n\geq 2$ gilt 
\begin{align*}
d_n=& \left( 1 + \frac
1{\sqrt{n}}\right)^n=\left(1+\frac 1{\sqrt n}\right)^2\cdot \left( 1+\frac 1{\sqrt{n}}\right)^{n-2}\\
=& \left( 1+ \frac{2}{\sqrt n}+\frac 1n\right)\left(1+\frac 1{\sqrt n}\right)\cdot \left(1+\frac 1{\sqrt n}\right)^{n-3}\\
=& \left( 1+ \frac 2{\sqrt n}+\frac 1n+\frac 1{\sqrt n}+\frac 2n+\frac 1{n\sqrt n}\right)\left( 1+\frac 1{\sqrt n}\right)^{n-3}\\
=& \left( 1+\frac{3}{\sqrt n}+\hdots\right)\left( 1+\frac 1{\sqrt n}\right)^{n-3}\\
\geq& 1+\frac{n}{\sqrt n}\\
%&\sum\limits_{j=0}^n\begin{pmatrix}n\\j\end{pmatrix} \left( \frac
%1{\sqrt n}\right)^j =1 + n\cdot \frac 1 {\sqrt n}
%+ \sum\limits_{j=2}^n\begin{pmatrix}n\\j\end{pmatrix} \left( \frac 1{\sqrt n}\right)^j 
\end{align*}
Die Ungleichung gilt, da die wegfallenden Summanden auf jeden Fall positiv sind, also kann man absch\"atzen:
\begin{align*}
d_n\geq & 1 + \frac n{\sqrt n} = 1 + \sqrt n\underset{n\rightarrow \infty}\longrightarrow \infty
\end{align*}
Damit muss auch f\"ur $d_n$ gelten $\underset{n\rightarrow \infty }\lim d_n = \infty$. 
\item Wir erweitern den Ausdruck f\"ur die Folge $e_n$ mit $(\sqrt{n^2+1}+\sqrt{n^2-1})$ um die
dritte binomische Formel anzuwenden: 
\begin{align*}
e_n=& \frac{n(\sqrt{n^2+1}-\sqrt{n^2-1})(\sqrt{n^2+1}+\sqrt{n^2-1})}{\sqrt{n^2+1}+\sqrt{n^2-1}}
= \frac{n(\sqrt{n^2+1}^2-\sqrt{n^2-1}^2)}{\sqrt{n^2+1}+\sqrt{n^2-1}}\\
=& \frac{n(n^2+1-n^2+1)}{\sqrt{n^2+1}+\sqrt{n^2-1}}
= \frac{2n}{\sqrt{n^2+1}+\sqrt{n^2-1}}
\end{align*}
Erweitern mit dem Kehrwert der h\"ochsten $n$-Potenz $\frac 1n$ liefert nun
$$e_n=\frac 2{\sqrt{1+\frac 1{n^2}}+ \sqrt{1-\frac 1{n^2}}}\underset{n\to\infty}\longrightarrow \frac{2}{1+1}=1.$$
\item Im ersten Schritt werden nur $x>0$ zugelassen. \\
Der Vollst\"andigkeit halber \"uberpr\"ufen wir zun\"achst als Erg\"anzung der Aufgabenstellung, dass die Folge $f_n$ konvergiert, dies geschieht in zwei
Schritten und unter Nutzung der Bernoulli-Ungleichung
$$(1+y)^m\geq 1+my\qquad\qquad \text{ f\"ur }m\in\N\text{ und } y\geq -1$$
\begin{iii}
\item Die Folge $f_n$ steigt monoton: 
\begin{align*}
&&\frac{f_{n+1}}{f_n}=& \frac{\left(1+\frac x{n+1}\right)^{n+1}}{\left(1+\frac xn\right)^n}
= \left( \frac{1+\frac x{n+1}}{1+\frac xn}\right)^{n+1}\left( 1+\frac x{n}\right)\\
&&=&\left( 1+\frac{-\frac{x}n+\frac x{n+1}}{1+\frac xn}\right)^{n+1} \left( 1 + \frac x{n}\right)
= \left( 1 + \frac{-x}{(n+x)(n+1)}\right)^{n+1}\cdot \frac{n+x}{n}\\
&&\geq&\left(1+(n+1)\cdot \frac{-x}{(n+x)(n+1)}\right)\cdot \frac{n+x}{n}\qquad\qquad\text{(Bernoulli-Ungleichung)}\\
&&=&\frac{n+x-x}{n+x}\cdot \frac{n+x}n=1\\
\Rightarrow&&f_{n+1}\geq & f_n
\end{align*}
\item Die Folge $f_n$ ist beschr\"ankt:\\
Wegen der Monotonie muss lediglich gezeigt werden, dass $f_n$ eine obere Schranke besitzt: 
\begin{align*}
&&1 + \frac x n<& 1 + \frac {2x}n\\
&&\leq&\left( 1 + \frac 1 n\right)^{2x} \qquad\qquad\text{(Bernoulli-Ungleichung)}\\
\Rightarrow&&\left( 1 + \frac xn\right)^n\leq& \left( 1 + \frac 1n\right)^{2x\cdot n}\\
&&=& \left[\left( 1 + \frac 1n \right)^n\right]^{2x}\underset{n\to\infty}{\longrightarrow}\EH{2x}
\end{align*}
Damit ist $f_n$ begrenzt durch eine konvergente Folge und somit selbst auch beschr\"ankt. 
\end{iii}
Als beschr\"ankte, monotone Folge muss $f_n$ auch konvergieren. \\

F\"ur $x> 0$ l\"asst sich $f_n$ umformen zu 
\begin{align*}
f_n=&\left( 1 + \frac x n \right)^n=\left( \left( 1 + \frac{1}{n/x}\right)^{n/x}\right)^x
\end{align*}
$f_n$ hat mit $n_k=k\cdot x$ die konvergente Teilfolge 
$$f_{n_k}=\left( \left( 1+\frac 1 {k\cdot x/x}\right)^{kx/x}\right)^x=\left(\left( 1 + \frac 1
k\right)^k\right)^x \underset{k\rightarrow \infty}\longrightarrow e^x.$$
Wenn also $f_n$ konvergiert muss der Grenzwert mit dem der Teilfolge \"ubereinstimmen und es gilt
$$\underset{n\rightarrow \infty}\lim f_n = e^x.$$

Der Fall negativer $x<0$ l\"asst sich mit Hilfe des Falls $x>0$ behandeln: 
Wir untersuchen dazu mit $y=-x>0$ die Folge
$$h_n:=f_n\cdot \left( 1 + \frac y n\right)^n=\left( \left( 1-\frac yn\right) \left( 1 + \frac
yn\right)\right)^n=\left( 1-\frac{y^2}{n^2}\right)^n.$$
Es gilt einerseits $h_n<1$. Andererseits liefert die Bernoulli-Ungleichung zumindest f\"ur $n>y$: 
$$h_n \geq 1 -n\cdot \frac{y^2}{n^2}=1-\frac{y^2}{n}\underset{n\to\infty}\longrightarrow 1.$$
Wegen dieser beiden Ungleichungen muss $h_n$ konvergieren:
$$\underset{n\to\infty}\lim h_n = 1.$$
Nach Definition von $h_n$ ist 
$$f_n=\frac{h_n}{\left( 1 + \frac y n\right)^n}.$$
Z\"ahler und Nenner dieses Ausdruckes konvergieren und somit konvergiert auch $f_n$: 
$$\underset{n\to\infty}{\lim} f_n = \frac{\underset{n\to\infty}\lim
h_n}{\underset{n\to\infty}\lim \left( 1 + \frac yn\right)^n}=\frac 1 {\EH y} = \EH{-y}=\EH x.$$
F\"ur $x=0$ hat man ohnehin 
$$\underset{n\to\infty}\lim f_n = \underset{n\to\infty}\lim 1^n=1=\EH 0.$$
Es gilt also f\"ur alle $x\in\Z$:
$$\underset{n\to\infty}\lim f_n = \EH{x}.$$
\item Es ist 
\begin{align*}
g_n = \left(\frac{n+1-2}{n+1}\right)^{n+1} = \left( 1 + \frac{-2}{n+1}\right)^{n+1}.
\end{align*}
Also ist $g_n=f_{n+1}$ mit $x=-2$ und es gilt
$$\underset{n\rightarrow \infty}\lim g_n = \underset{n\rightarrow \infty}\lim f_n = \EH{-2}.$$
\end{abc}
}

\ErgebnisC{alalysFolgLims001}{
$a_n\underset{n\to\infty}\longrightarrow 1$, 
$b_n\underset{n\to\infty}\longrightarrow \infty$,
$c_n\underset{n\to\infty}\longrightarrow \frac{31}{25}$,
$d_n\underset{n\to\infty}\longrightarrow \infty$,
$e_n\underset{n\to\infty}\longrightarrow 1$, 
$f_n\underset{n\to\infty}\longrightarrow \EH{x}$, 
}
