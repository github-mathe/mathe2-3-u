\Aufgabe[e]{Fl\"acheninhalt einer Ellipse}
{
  Gegeben sei die Ellipse durch die Punktmenge
  \[
    E = \left\{ (x, y) \in \mathbb{R}^2 : \frac{x^2}{a^2} + \frac{y^2}{b^2} \leq 1, \, \text{ mit } a > 0, b > 0 \right\}
  \]
  \begin{abc}
    \item Skizzieren Sie die Region $E$ für die konkreten Werte $a = 2$, $b = 1$. 

    \item Führen Sie eine geeignete Koordinatentransformation ein, um die Fläche $A$ der Ellipse zu berechnen.
    \item Bestimmen Sie die Jacobi-Determinante der Transformation und formulieren Sie das entsprechende Flächenintegral zur Berechnung von $A$ in den neuen Koordinaten.

    \item Berechnen Sie die Fläche der Ellipse durch Auswertung des Integrals.

    \item Diskutieren Sie, warum die direkte Verwendung der üblichen Polarkoordinaten ($x = r \cos\varphi$, $y = r \sin\varphi$) zu einer komplexeren Integrationsaufgabe führt. Worin besteht die Schwierigkeit? 
  \end{abc}

}

\Loesung{

  \begin{abc}
\item 
\begin{tikzpicture}
  % Gitterlinien optional (könnte entfernt werden)
  % \draw[step=1cm,gray!20,very thin] (-3.1,-2.1) grid (3.1,2.1);

  % Achsen
  \draw[->] (-3.2,0) -- (3.2,0) node[right] {$x$};
  \draw[->] (0,-2.2) -- (0,2.2) node[above] {$y$};

  % Zahlen auf den Achsen
  \foreach \x in {-3,-2,-1,1,2,3}
    \draw (\x,0.1) -- (\x,-0.1) node[below] {\x};
  \foreach \y in {-2,-1,1,2}
    \draw (0.1,\y) -- (-0.1,\y) node[left] {\y};
    

  % Buchstabe im Inneren
  \node at (0.5,0.5) {\large $E$};
  
  % Ellipse a=2, b=1
  \draw[thick, blue] (0,0) ellipse (2 and 1);

  % Mittelpunkt
  \fill (0,0) circle (0.05);
\end{tikzpicture} 
  
\item Geeignete Koordinaten zur Berechnung des Fl\"acheninhaltes einer Ellipse sind skalierte Polarkoordinaten: 
\[
 x = a r \cos\varphi,\quad y = b r \sin\varphi,\quad r \in [0, 1],\ \varphi \in [0, 2\pi).
 \]

\item Die Jacobi-Determinante wird wie folgt bestimmt:
\begin{align*}
\text{det} \dfrac{\partial (x,y)}{\partial (r,\varphi)} &= \text{det} \begin{pmatrix}
a \cos(\varphi) & -ar \sin(\varphi) \\
b \sin(\varphi) & br \cos(\varphi)
\end{pmatrix}\\
&= abr \cos^2(\varphi) + abr \sin^2(\varphi)\\
&= abr (\cos^2(\varphi) +\sin^2(\varphi)) \\
&= abr
\end{align*}
Damit folgt
$$
\mathrm{d}A = \mid \text{det} \dfrac{\partial (x,y)}{\partial (r,\varphi)} \mid \mathrm{d}r \, \mathrm{d} \varphi = abr \, \mathrm{d}r\, \mathrm{d} \varphi.
$$
    
F\"ur das Integral ergibt sich:
$$
A = \int_{\varphi=0}^{2\pi} \int_{r=0}^1 abr \, \mathrm{d}r\, \mathrm{d} \varphi.
$$    
    
\item 
Wenn man die gewöhnlichen Polarkoordinaten verwendet, also
\[
x = r \cos\varphi,\quad y = r \sin\varphi,
\]
und diese in die Ungleichung der Ellipse einsetzt, ergibt sich:
\[
\frac{r^2 \cos^2\varphi}{a^2} + \frac{r^2 \sin^2\varphi}{b^2} \leq 1.
\]
Dies kann umgeformt werden zu:
\[
r^2 \left( \frac{\cos^2\varphi}{a^2} + \frac{\sin^2\varphi}{b^2} \right) \leq 1,
\quad\text{also}\quad
r \leq \left( \frac{\cos^2\varphi}{a^2} + \frac{\sin^2\varphi}{b^2} \right)^{-\frac{1}{2}}.
\]

Damit ergibt sich für das Integral:
\[
A = \int_0^{2\pi} \int_0^{r(\varphi)} r\, \mathrm{d}r\, \mathrm{d}\varphi,
\quad\text{wobei}\quad
r(\varphi) = \left( \frac{\cos^2\varphi}{a^2} + \frac{\sin^2\varphi}{b^2} \right)^{-\frac{1}{2}}.
\]

\textbf{Fazit:} Die Verwendung gewöhnlicher Polarkoordinaten ist prinzipiell möglich, führt jedoch zu einem aufwändigeren Integrationsbereich, da die obere Grenze von $r$ abhängig vom Winkel $\varphi$ ist. Die Integration ist damit deutlich komplizierter.

Im Gegensatz dazu erlaubt die Transformation
\[
x = a r \cos\varphi,\quad y = b r \sin\varphi,\quad r \in [0,1],\ \varphi \in [0,2\pi)
\]
eine Beschreibung der Ellipse durch ein einfaches Rechteck im $(r, \varphi)$-Koordinatensystem, was die Berechnung der Fläche erheblich vereinfacht.

  \end{abc}

}

\ErgebnisC{analysIntegEllipse}
{
\begin{abc}
\item
\item
\item
\item
\item
\end{abc}
}
