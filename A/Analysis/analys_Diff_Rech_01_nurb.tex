\Aufgabe[e]{Differenzieren}{


% \textbf{Hinweis: }Es gilt $\dfrac{d}{dx}(\ln(x))=\dfrac 1x$. 
% \item Bestimmen Sie die vierte Ableitung folgender Funktionen, wobei Sie das geeignete Zusammenfassen von Termen nicht vergessen sollten. 
% \begin{align*}
% f_{17}(t)=&(t-3)^4-(2t+1)^5    ,&\qquad  && 
% f_{18}(t)=&(t+1)\,\sin (2t)    \\            
% f_{19}(t)=&(t^3-1)\,\text{e}^{2t}     ,&\qquad  && 
% f_{20}(t)=&\sin (3t)\,\text{e}^{-t}   \\           
% \end{align*}
% \textbf{Hinweis: }Nutzen Sie gegebenen Falls die Leibniz-Regel zur Berechnung h\"oherer Ableitungen. Diese hat dieselbe Gestalt wie der binomische Lehrsatz:  
% \begin{align*}
% (f(t)\cdot g(t))^{(n)}=&\sum\limits_{k=0}^n\begin{pmatrix}n\\k\end{pmatrix}f^{(k)}(t)\cdot g^{(n-k)}(t)\\
% \text{Z. B. f\"ur } n=2: 
% \, (f(t)\cdot g(t))''=& f(t)g''(t)+2f'(t)g'(t)+f''(t)g(t)
% \end{align*} 
Bestimmen Sie die\ $n$--te Ableitung folgender Funktionen. 
\begin{align*}
f_{21}(t)=&\sin (3t)        ,&\qquad  && 
f_{22}(t)=&t\,\text{e}^{2t} \\           
f_{23}(t)=&t\cdot \cos (t) ,&\qquad  && 
f_{24}(t)=&t\,\ln (2t)     \\           
\end{align*}

}

\Loesung{


% \item 
% \begin{align*}
% f_{17}^{(4)}= & 4\cdot 3 \cdot 2 \cdot 1 - 2^4\cdot 5 \cdot 4 \cdot 3 \cdot 2 \cdot (2t+1)\\
% =& 24\cdot \left( 1 - 80 (2t+1)\right) = -24(160t+79)\\
% f_{18}^{(4)}= &\left( \frac d{dt}\right)^3\left( \sin(2t)+(t+1)2\cos(2t)\right)\\
% =&-8\cos(2t)+ 2\left( \frac d{dt}\right)^2\left( \cos(2t)-2(t+1)\sin(2t)\right)\\
% =& -8\cos(2t) -8\cos(2t) -4  \frac d{dt}\left( \sin(2t)+2(t+1)\cos(2t)\right)\\
% =& -16\cos(2t) -8\cos(2t) -8\cos(2t)+16(t+1)\sin(2t)\\
% =&-32\cos(2t)+16(t+1)\sin(2t)\\
% f_{19}^{(4)}= &\left( \frac d{dt}\right)^3\left( (3t^2+2(t^3-1))\EH{2t}\right)\\
% =& \left( \frac d{dt}\right)^2\left((6t+6t^2+2(3t^2+2(t^3-1)))\EH{2t}\right)\\
% =& \left( \frac d{dt}\right)^2\left( (4t^3+12t^2+6t-4)\EH{2t}\right) \\
% =& \frac d{dt} \left((12t^2+24t+6+8t^3+24t^2+12t-8)\EH{2t}\right)\\
% =& (24t + 24 + 24t^2 + 48t+12+24t^2+48t+12+16t^3+48t^2+24t-16)\EH{2t}\\
% =& (16t^3+96t^2+144t+32)\EH{2t}\\
% f_{20}^{(4)}= &\left( \frac d{dt}\right)^3\left( 3\cos(3t)\EH{-t}-\sin(3t)\EH{-t}\right)\\
% =& \left( \frac d{dt}\right)^2\left((-9\sin(3t)-3\cos(3t)-3\cos(3t)+\sin(3t))\EH{-t}\right)\\
% =& \left( \frac d{dt}\right)^2\left( (-8\sin(3t)-6\cos(3t))\EH{-t}\right) \\
% =& \frac d{dt} \left((-24\cos(3t)+8\sin(3t)+18\sin(3t)+6\cos(3t))\EH{-t}\right) \\
% =& \frac d{dt} \left((-18\cos(3t)+26\sin(3t))\EH{-t}\right)\\
% =&(54\sin(3t)+18\cos(3t)+78\cos(3t)-26\sin(3t))\EH{-t} \\
% =& (28\sin(3t)+96\cos(3t))\EH{-t}
% \end{align*}
 \begin{iii}
\item Jedes Ableiten der Funktion $f_{21}$ f\"uhrt wegen der inneren Ableitung aus der Kettenregel
zu einem Faktor $3$. Außerdem ergibt jedes zweite Ableiten der \"außeren Funktion (immer die
Ableitung einer $\cos$-Funktion) einen weiteren Faktor $(-1)$. Die \"außere Funktion wird nach gradzahligem
Ableiten zu einem Sinus, bei ungradzahligen Ableitungen zu einem Kosinus. Insgesamt ergibt sich
also: 
$$f_{21}^{(n)}(t)=\left\{\begin{array}{lll}
3^n \sin(3t)&\quad&\text{ f\"ur } n=4k+0\, (k\in\N_0)\\
3^n \cos(3t)&\quad&\text{ f\"ur } n=4k+1\, (k\in\N_0)\\
-3^n \sin(3t)&\quad&\text{ f\"ur } n=4k+2\, (k\in\N_0)\\
-3^n \cos(3t)&\quad&\text{ f\"ur } n=4k+3\, (k\in\N_0)
\end{array}\right.$$
\item Die erste Ableitung von $f_{22}$ ist
$$f_{22}'(t)=\EH{2t}+2t\EH{2t}=\EH{2t} + 2 f_{22}(t).$$
Setzt man dies sukzessive fort, ergibt sich
\begin{align*}
f_{22}^{(n)}(t)=&n\cdot 2^{n-1} \EH{2t} + 2^n f_{22}(t)\\
=&\left( n\cdot 2^{n-1} + 2^n\cdot t\right)\EH{2t}=(n+2t)2^{n-1}\EH{2t}
\end{align*}
\item Die ersten vier Ableitungen von $f_{23}$ sind: 
\begin{align*}
f_{23}(t)=&
t\cdot \cos(t),\quad&f_{23}'(t)=&\cos(t)-t\sin(t)\\
f_{23}''(t)=&-2\sin(t)-t\cos(t),\quad&f_{23}'''(t)=&-3\cos(t)+t\sin(t),\\
f_{23}^{(4)}(t)=&4\sin(t)+t\cos(t),\quad&f_{23}^{(5)} =& 5\cos(t)-t\sin(t)
\end{align*}
Daraus ergibt sich f\"ur $n\geq 1$:
$$f_{23}^{(n)} = \left\{\begin{array}{lll}
+n\cos(t)-t\sin(t) &\, & \text{ f\"ur } n=4k+1\, (k\in\N_0)\\
-n\sin(t)-t\cos(t) &\, & \text{ f\"ur } n=4k+2\, (k\in\N_0)\\
-n\cos(t)+t\sin(t) &\, & \text{ f\"ur } n=4k+3\, (k\in\N_0)\\
+n\sin(t)+t\cos(t) &\, & \text{ f\"ur } n=4k+4\, (k\in\N_0)
\end{array}\right..
$$
\item Die ersten beiden Ableitungen von $f_{24}$ sind
\begin{align*}
f_{24}'(t)=& \ln (2t) + t\cdot \frac 2{2t} = \ln (2t)+1\\
f_{24}''(t)=& \frac 1 t .
\end{align*}
Alle weiteren Ableitungen ($n=2,3,\hdots$) ergeben sich zu 
$$f_{24}^{(n)}(t)=(-1)\cdot(-2)\cdot \hdots\cdot (-(n-2)) t^{-(n-1)}=(-1)^{n}(n-2)!t^{-(n-1)}.$$
\end{iii}
}

\ErgebnisC{AufganalysDiffRech001}
{
% \textbf{b)} $
% f_{17}^{(4)}=  -24(160t+79),\, 
% f_{18}^{(4)}= -32\cos(2t)+16(t+1)\sin(2t),\, 
% f_{19}^{(4)}=  (16t^3+96t^2+144t+32)\EH{2t},$\\
% $f_{20}^{(4)}=  (28\sin(3t)+96\cos(3t))\EH{-t}$
$
f_{22}^{(n)}(t) = (n+2t)2^{n-1}\EH{2t} , \, 
f_{24}^{(n)}(t) =  (-1)^{n}(n-2)!t^{-(n-1)}.
$

}
