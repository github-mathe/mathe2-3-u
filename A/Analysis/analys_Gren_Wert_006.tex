\Aufgabe[e]{Definition of limit}{
\begin{abc}
\item Recapitulate the definition of limit and show that the sequence $a_n = \displaystyle\frac{1}{n}$ has the limit $a=0$, i.e.\ $\forall k \in \N$ there is a number $N\in \mathbb R$ such that for all $n\in \N$ with $n>N$ it holds $|a_n-a|< 10^{-k}$.
\item Compute the limit $a = \lim\limits_{n\rightarrow \infty}a_n$ of the following sequences and specify which rule for the limit has been used for the computation (e.g.\ addition or product rule, sandwich theorem, product of bounded and zero sequence etc.):
%\setlength{\columnsep}{0.2cm}
%\begin{multicols}{2}
%\begin{iii}
%\item $a_n=\dfrac{n^2+5n}{3n^2+1}$.
%\item $a_n=\log_{10}(10n^2-2n)-\log_{10}(n^2+1)$.
%\item $a_n=\dfrac{(n+1)!}{n!-(n+1)!}$.
%\item $a_n=\left(1+\dfrac{1}{n}\right)^{3n}$.
%\item $a_n=\dfrac{\cos{n}}{n}$.
%\item $a_n=\sqrt{n+1}-\sqrt{n}$.
%\item $a_n=\dfrac{2^n}{n!}$.
%\end{iii}
%\end{multicols}
	$$
	\begin{array}{lllll}
	\textbf{i)} & a_n = \dfrac{n^2+5n}{3n^2+1} & &
	\textbf{ii)} &a_n = \log_{10}(10n^2-2n)-\log_{10}(n^2+1)\\
	& & & &\\
	\textbf{iii)} & a_n=\dfrac{(n+1)!}{n!-(n+1)!} & &
	\textbf{iv)} & a_n=\left(1+\dfrac{1}{n}\right)^{3n}\\
	& & & &\\
	\textbf{v)} & a_n=\dfrac{\cos{n}}{n} & &
	\textbf{vi)} & a_n=\sqrt{n+1}-\sqrt{n} \\
	& & & &\\
	\textbf{vii)} & a_n=\dfrac{2^n}{n!} 
	\end{array}
        $$
\end{abc}
}

\Loesung{
\begin{abc}
\item
It has to be shown that for all $k \in \N$ there is a $N \in \mathbb R$, such that
\begin{align*}
|n^{-1} - 0| < 10^{-k}, \, \forall n>N.
\end{align*}
Take $N =10^k$.
\item
It has been shown above that the limit
\begin{align*}
\lim\limits_{n\rightarrow\infty} \frac{1}{n} = 0.
\end{align*}
Adapting the result obtined in {\textbf a)}) it is also 
\begin{align*}
\lim\limits_{n\rightarrow\infty} \frac{1}{n^\alpha} = 0, \, \text{ for all } \alpha \in \mathbb R, \, \alpha > 0.
\end{align*}
We use also the continuity of some functions in the exercises, which is assumed as given.
\begin{iii}
\item
Dividing numerator and denominator of the two terms by $n^2$
\begin{align*}
\dfrac{1 + \frac{5}{n}}{3+\frac{1}{n^2}}
\end{align*}
	The limit of the nominator is 1 and the limit of the denominator is 3. The quotient rule for limits yields the limit $a=\frac{1}{3}$.
\item
\begin{align*}
\log_{10}(10n^2-2n)-\log_{10}(n^2+1) &= \log_{10} {\frac{10n^2-2n}{n^2+1}}\\
&= \log_{10} {\frac{10-\frac{2}{n^2}}{1+\frac{1}{n^2}}}.
\end{align*}
Using the product and addition rule you can show the limit of the sequence $b_n = \frac{10n^2-2n}{n^2+1}$
\begin{align*}
b = \lim\limits_{n\rightarrow \infty} {\frac{10-\frac{2}{n^2}}{1+\frac{1}{n^2}}}=10.
\end{align*}
Since the function $\log_{10}(x)$ is continuous in its domain, i.e.\ for $x>0$, the limit is
\begin{align*}
a = \lim\limits_{n\rightarrow \infty}\log_{10} b_n = \log_{10} b = 1.
\end{align*}
\item
Manipulating the factorial function it is
\begin{align*}
\frac{(n+1)!}{n!-(n+1)!} = \frac{(n+1)n!}{n!-(n+1)n!} = \frac{(n+1)}{-n} = -(1+\frac{1}{n}).
\end{align*}
Using the addition rule, the limit is
\begin{align*}
a = \lim\limits_{n\rightarrow \infty}\frac{(n+1)!}{n!-(n+1)!} = \lim\limits_{n\rightarrow \infty}-(1+\frac{1}{n}) = -1.
\end{align*}
\item
It is
\begin{align*}
\left(1+\dfrac{1}{n}\right)^{3n} = \left(\left(1+\dfrac{1}{n}\right)^n\right)^3
\end{align*}
The limit of the sequence $b_n = \left(1+\dfrac{1}{n}\right)^n$ is
\begin{align*}
b = \lim\limits_{n\rightarrow \infty}\left(1+\dfrac{1}{n}\right)^n = \operatorname{e}.
\end{align*}
The product rule gives
\begin{align*}
\lim\limits_{n\rightarrow \infty} b_n^3 = \lim\limits_{n\rightarrow \infty} b_n \lim\limits_{n\rightarrow \infty} b_n \lim\limits_{n\rightarrow \infty} b_n = b^3 = \operatorname{e}^3.
\end{align*}
It can be also used the continuity of the function $x^3$ to show the same result.
\item
The sequence
\begin{align*}
a_n = \frac{\cos{n}}{n}
\end{align*}
can be interpreted as the product $a_n = b_n c_n$ of a bounded sequence $b_n = \cos{n}$ and a zero sequence $c_n = \frac{1}{n}$. Therefore, the product is a zero sequence and the limit is $a=0$.
\item
It is
\begin{align*}
\sqrt{n+1}-\sqrt{n} &= \frac{\left(\sqrt{n+1}-\sqrt{n}\right) \left(\sqrt{n+1}+\sqrt{n}\right)}{\sqrt{n+1}+\sqrt{n}}\\
&=\frac{1}{\sqrt{n+1}+\sqrt{n}} = \frac{\frac{1}{\sqrt{n}}}{\sqrt{1+\frac{1}{n}}+1.
\end{align*}
Using the product and addition rule we get the linit $a=0$.
\item
The sequence $a_n = \dfrac{2^n}{n!}$ can be bounded from above (for $n>3$) and below in the following way
\begin{align*}
0 \leq \frac{2^n}{n!} = \frac{\overbrace{2\cdot 2\cdot 2\dots 2}^{n}}{1\cdot 2\cdot 3\dots n} \leq \frac{2}{1}\cdot \frac{2}{2}\cdot \underbrace{\frac{2}{3}\cdot\frac{2}{4}\dots\frac{2}{n-1}}_{\leq 1} \cdot \frac{2}{n} \leq \frac{4}{n}.
\end{align*}
Since it is bounded from above and below by two zero sequences, the sandwich theorem gives the limit $a=0$.
\end{iii}
\end{abc}
}

