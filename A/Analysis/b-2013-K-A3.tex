\Aufgabe[e]{ }{
% \mbox{}\\[1ex]
\begin{abc}
\item Entwickeln Sie die Funktion
$$
f(x,y,z) = \cosh(x) + \EH{2\,xy} - \ln(-yz) + \frac{1}{y} + \frac{z}{2}
$$
nach der Taylor'schen Formel um den Punkt \ $P=(0\,,\,-1\,,\,2)$ \ bis zu einschließlich zweiter Ordnung (ohne Restglied).
\item Bestimmen Sie die Taylor-Entwicklung 2.\ Grades um den Punkt $y_0 = \dfrac{4}{3}$ f\"ur die Umkehrfunktion $g^{-1}(y)$ von $g(x) = \dfrac{x^3}{3} + x$\  (ohne Restglied). \\[2ex]
\textbf{Hinweis:} Die Umkehrfunktion ist \textbf{nicht} explizit zu bestimmen!
\item F\"uhren Sie zur Bestimmung einer N\"aherungsl\"osung des Gleichungssystems
$$
y = \EH{-x^2}  \,, \quad y = -(x-1)^2+2
$$
einen Iterationsschritt des \textbf{zweidimensionalen} Newton-Verfahrens mit dem Startvektor \ $(x_0,y_0) = (0,0)$ \ durch.
\end{abc}
}

\Loesung{
\bigskip
\textbf{L\"osung:}\\[1ex]

\bigskip
\textbf{Zu a)} Mit \ $\vec x_0 =(0\,,\,-1\,,\,2)^\top$ \ gilt:
	\[
	\begin{array}{rcllrcl}
	f &=& \cosh(x) + \EH{2\,xy} - \ln(-yz) + \dfrac 1y + \dfrac z2 &\Rightarrow& f(\vec x_0) &=& 2 -\ln(2)\,. \\
	\\ 
	f_x &=& \sinh(x) + 2y\,\EH{2\,xy} &\Rightarrow& f_x(\vec x_0) &=& -2 \,, \\
	\\
	f_y &=& 2x\,\EH{2\,xy} - \dfrac 1y - \dfrac{1}{y^2} &\Rightarrow& f_y(\vec x_0) &=& 0\,, \\
	\\
	f_z &=& -\dfrac 1z + \dfrac 12 &\Rightarrow& f_z(\vec x_0) &=& 0\,, \\
	\\
	f_{xx} &=& \cosh(x) + 4\,y^2\,\EH{2\,xy} &\Rightarrow& f_{xx}(\vec x_0) &=& 5\,, \\
	\\
	f_{xy} &=&  4\,xy\,\EH{2\,xy} +2\EH{2\,xy}&\Rightarrow& f_{xy}(\vec x_0) &=& 2\,, \\
	\\
	f_{xz} &=& 0 &\Rightarrow& f_{xz}(\vec x_0 ) &=& 0\,, \\
	\\
	f_{yy} &=& 4\,x^2\,\EH{2\,xy} + \dfrac{1}{y^2} + \dfrac{2}{y^3} &\Rightarrow& f_{yy}(\vec x_0) &=& -1\,, \\
	\\
	f_{yz} &=& 0 &\Rightarrow& f_{yz}(\vec x_0) &=& 0\,, \\
	\\
	f_{zz} &=& \dfrac{1}{z^2} &\Rightarrow& f_{yz}(\vec x_0) &=& \dfrac 14\,.
	\end{array}
\]
Damit lautet das Taylor--Polynom 2. Grades
\begin{align*}
T_2(x,y,z)  = & 2-\ln(2) - 2\,x + \dfrac 52\,x^2 + 2 x(y+1) \\[2ex]
& -\dfrac{1}{2} (y+1)^2 + \dfrac 18\,(z-2)^2 \ .
\end{align*}

\bigskip
\textbf{Zu b)}  Mit 
$$
g^{-1}\left(\frac{4}{3}\right) = 1\,, \left(g^{-1}\right)'\left(\frac{4}{3}\right) = \dfrac{1}{g'(1)} = \frac{1}{2}
$$ 
und
$$
 \left(g^{-1}\right)''\left(\frac{4}{3}\right) = \dfrac{-g''(1)}{\left(g'(1)\right)^3} = -\frac{1}{4}
$$
folgt
$$
T_2\left(y; \frac{4}{3}\right) = 1 + \frac{1}{2}\left(y - \frac{4}{3} \right) -\frac{1}{8}\left(y - \frac{4}{3} \right)^2\,.
$$

\bigskip
\textbf{Zu c)} F\"ur $\vec F(x,y)= (y -\EH{-x^2},y +(x-1)^2-2)^\top$ gilt
$$
\vec J_{\vec{F}}(x,y) = \begin{pmatrix} 2x\EH{-x^2} & 1 \\ 2(x-1)  & 1 \end{pmatrix}\,, \quad \vec J_{\vec{F}}(0,0) = \begin{pmatrix}  0 & 1 \\ -2  & 1 \end{pmatrix}. 
$$
Das Newton-Verfahren liefert damit f\"ur den Iterationsschritt
$$\vec x_1=\vec x_0 + \vec {\Delta x}$$
mit der L\"osung $\vec{\Delta x}$ des linearen Gleichungssystems $\vec J_{\vec F}(0,0)\vec{\Delta x}=-\vec F(0,0):$
$$\begin{array}{rr|r|l}
  0  &  1  &  1 & \\
 -2  &  1  &  1 & 
\end{array}$$
die Schrittweite $$\vec{\Delta x}=\begin{pmatrix}0 \\1\end{pmatrix}$$
und damit
$$
\vec x_1 = \vec x_0+ \vec{\Delta x} = \boxed{\begin{pmatrix} 0\\ 1 \end{pmatrix}}\,.
$$
}


% \ErgebnisC{b-2013-K-A1}{
% 
% }


