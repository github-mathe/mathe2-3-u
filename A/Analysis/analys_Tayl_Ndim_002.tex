\Aufgabe[e]{Taylor-Entwicklung} {
\begin{abc}
\item Geben Sie zu der Funktion 
$$f(x,y)=x \sin(xy)$$
die Taylor-Entwicklung ersten Grades $T_1(\vec x)$ um den Punkt $\vec x_0=(1,0)^\top$ an. 
\item Berechnen Sie $f(\vec x)$ und $T_1(\vec x)$ an der Stelle $\vec x_1=(0,1)^\top$. Berechnen Sie
außerdem den Fehler der Taylor-Approximation $T_1(\vec x_1)-f(\vec x_1)$. 
\item Geben Sie das Lagrange-Restglied an.
% \item Geben Sie mit Hilfe des Lagrange-Restgliedes eine Absch\"atzung f\"ur den Fehler
% $$\left|f(\vec x)-T_1(\vec x)\right|$$
% an. 
\end{abc}
% \textbf{Hinweis}: F\"ur das Restglied erhalten Sie einen Ausdruck der Form 
% $$(3\theta-2)\cos(\theta^2-\theta) + \frac 12 (3\theta^3-7\theta^2+5\theta-1)\cos(\theta^2-\theta)$$
% mit $0\leq \theta\leq 1$. 
% Diesen Ausdruck k\"onnen Sie absch\"atzen, indem Sie die Extrema der beiden Vorfaktoren ($3\theta-2$
% und $\frac 12 (3\theta^3-7\theta^2+5\theta-1)$) auf dem Intervall $[0,1]$ und eine obere Schranke
% der Kosinus- bzw. Sinus-Funktion bestimmen. 
}


\Loesung{
\begin{abc}
\item Der Wert der Funktion selbst sowie der der ersten Ableitung im Punkt $\vec x_0=(1,0)^\top$ sind:
\begin{align*}
f(x,y)=&x\sin(xy)&\quad\Rightarrow&\quad&f(1,0)=&0\\
\nabla f(x,y)=& (\sin(xy)+xy\cos(xy),\, x^2\cos(xy))^\top&\Rightarrow&&\nabla f(1,0)=& (0,\, 1)^\top.
\end{align*}
Damit ergibt sich die Taylor-Summe (mit $\vec x = (x,y)^\top$):
$$T_1(\vec x)=f(\vec x_0) + \skalar{\nabla f(\vec x_0),\, (\vec x - \vec x_0)}= 0
+ \skalar{(0,\, 1)^\top,\, (x-1,\, y-0)^\top}=y.$$
\item An der Stelle $\vec x_1=(0,1)^\top$ hat man dann 
\begin{align*}
f(0,1)=& 0\\
T_1(0,1)=& 1\\
f(0,1)-T_1(0,1)=& -1.
\end{align*}
\item Die Lagrange-Darstellung des Restgliedes ist.
\begin{align*}
&&f(\vec x)-T_1(\vec x)=& \frac 1{2!} (\skalar{\vec x-\vec x_0,\, \nabla})^2 f(\vec x_0+\theta(\vec
x-\vec x_0)) \text{ mit }\theta\in [0,1]\\
\Rightarrow&&|f(\vec x)-T_1(\vec x)|=& \frac 12 \Bigl|
(\skalar{(-1,1)^\top,\, \nabla})^2f(1-\theta,\, \theta)\Bigr|\\
&&=&\frac 12 \left| \left( \frac{\partial^2}{\partial x^2}-2\frac{\partial^2}{\partial x\partial y}
+ \frac{\partial^2}{\partial y^2}\right)f(1-\theta,\theta)\right|
\end{align*}
Wir berechnen zun\"achst die zweiten partiellen Ableitungen von $f$:
\begin{align*}
\frac{\partial^2 f}{\partial x^2}(x,y)=& y\cos(xy)+y\cos(xy)-xy^2\sin(xy)=2y\cos(xy)-xy^2\sin(xy)\\
\frac{\partial^2 f}{\partial y^2}(x,y)=& -x^3\sin(xy)\\
\frac{\partial^2 f}{\partial x\partial y}(x,y)=& 2x\cos(xy)-x^2y\sin(xy)
\end{align*}
Durch Einsetzen der zweiten partiellen Ableitungen erhalten wir das Restglied.
% \item Die Lagrange-Darstellung des Fehlers ist 
% \begin{align*}
% &&f(\vec x)-T_1(\vec x)=& \frac 1{2!} (\skalar{\vec x-\vec x_0,\, \nabla})^2 f(\vec x_0+\theta(\vec
% x-\vec x_0)) \text{ mit }\theta\in [0,1]\\
% \Rightarrow&&|f(\vec x)-T_1(\vec x)|=& \frac 12 \Bigl|
% (\skalar{(-1,1)^\top,\, \nabla})^2f(1-\theta,\, \theta)\Bigr|\\
% &&=&\frac 12 \left| \left( \frac{\partial^2}{\partial x^2}-2\frac{\partial^2}{\partial x\partial y}
% + \frac{\partial^2}{\partial y^2}\right)f(1-\theta,\theta)\right|
% \end{align*}
% Wir berechnen zun\"achst die zweiten partiellen Ableitungen von $f$:
% \begin{align*}
% \frac{\partial^2 f}{\partial x^2}(x,y)=& y\cos(xy)+y\cos(xy)-xy^2\sin(xy)=2y\cos(xy)-xy^2\sin(xy)\\
% \frac{\partial^2 f}{\partial y^2}(x,y)=& -x^3\sin(xy)\\
% \frac{\partial^2 f}{\partial x\partial y}(x,y)=& 2x\cos(xy)-x^2y\sin(xy)
% \end{align*}
% und setzen ein:
% \begin{align*}
% |f(\vec x)-T_1(\vec x)|=& \frac 12 \Bigl|
%  2\theta\cos(\theta(1-\theta))-(1-\theta)\theta^2\sin(\theta-\theta^2) +\\
% &\qquad\quad -2\bigl( 2 (1-\theta)\cos(\theta-\theta^2)-
%  (1-\theta)^2\theta\sin(\theta-\theta^2) \bigr)+ \\
% &\qquad\qquad\qquad-(1-\theta)^3\sin(\theta-\theta^2)\Bigr|\\
% =& \frac 12\Bigl| (2\theta-4(1-\theta))\cos(\theta-\theta^2)+\\
% &\qquad\quad +
%  (1-\theta)(-\theta^2-2\theta^2+2\theta-1-\theta^2+2\theta)\sin(\theta-\theta^2)\Bigr|\\
% \leq& \frac 12 \Bigl( 2 \cdot |3\theta-2||\cos(\theta-\theta^2)| + \\
% &+|1-\theta|\cdot |-4\theta^2+4\theta-1|\cdot |\sin(\theta-\theta^2)|\Bigr)
% \end{align*}                             
% Da sowohl Kosinus und Sinus als auch $1-\theta$ betraglich stets kleiner als 1 sind, m\"ussen nur
% noch die Extrema von $|3\theta-2|$ und
% $|(1-\theta)\cdot(-4\theta^2+4\theta-1)|=|4\theta^3-8\theta^2+5\theta-1|$ auf dem Intervall $[0,1]$
% ermittelt werden:\\
% Das Maximum von $|3\theta-2|$ ist $2$. \\
% Das Maximum des zweiten Koeffizienten $\varphi(\theta)=4\theta^3-8\theta^2+5\theta-1$ liegt entweder an den Intervallgrenzen $\theta=0$ oder
% $\theta=1$ oder an einem kritischen Punkt der Funktion, also an einer Nullstelle der Ableitung: 
% \begin{align*}
% &&0\overset !=&\frac{d}{d\theta}(4\theta^3-8\theta^2+5\theta-1)=12\theta^2-16\theta+5\\
% \Rightarrow&& \theta_{1,2}=& \frac 23\pm \sqrt{\frac 49-\frac{5}{12}}=\frac{2}3 \pm \frac
% 14 \in\left\{\frac 5{12},\, \frac {11}{12}\right\}.
% \end{align*}
% Insgesamt ist also 
% \begin{align*}
% |\varphi(\theta)|\leq&\text{max }\{|\varphi(0)|,\, |\varphi(1)|,\, |\varphi(5/12)|,\, |\varphi(11/12)|\}\\
% =&\text{max }\{1,\, 0,\, 7/432,\, 25/432\}=1.
% \end{align*}
% Insgesamt ergibt sich so die Fehlerabsch\"atzung:
% $$|f(\vec x)-T_1(\vec x)|\leq \frac 12 \left( 2\cdot 2  + 1\right)=\frac 52.$$
% Diese Absch\"atzung ist mehr als doppelt so groß wie der in Aufgabenteil \textbf{b)} ermittelte tats\"achliche Fehler. 
\end{abc}
}

\ErgebnisC{AufganalysTaylNdim002}
{
$T_1(x,y)=y$
}
