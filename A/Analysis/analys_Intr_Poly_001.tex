\Aufgabe[v]{Interpolation}{
Gesucht ist ein Polynom vierten Grades 
$$p(x)=a_0 + a_1 x + a_2x^2+a_3x^3+a_4 x^4$$
mit folgenden Eigenschaften:
\begin{itemize}
\item Der Funktionswert bei $0$ ist $p(0)=0$. 
\item $p$ hat ein Minimum bei $(1,-1)$. 
\item $p$ hat einen Sattelpunkt bei $(2,0)$. 
\end{itemize}
\begin{abc}
\item Geben Sie die Bedingungen, die der Koeffizientenvektor  $(a_0,\hdots,\, a_4)^\top$ erf\"ullen
muss, in Form eines linearen Gleichungssystems an. (Es sollten sich sechs lineare Gleichungen mit
f\"unf Unbekannten ergeben.)
\item Berechnen Sie den Rang der Koeffizientenmatrix und der erweiterten Koeffizientenmatrix. Ist
das Gleichungssystem l\"osbar?
\item Welchen Wert $p(0)$ muss die Funktion bei $0$ annehmen, damit das System doch l\"osbar ist?
\item L\"osen Sie das so ge\"anderte Gleichungssystem. 
\item Skizzieren Sie die Funktion $p(x)$. 
\end{abc}

}


\Loesung{
\begin{abc}
\item Die angegebenen Eigenschaften der Funktion ergeben folgende Bedingungen:
\begin{align*}
p(0)=0                   &               &                          &\,\Rightarrow\,& a_0
          =& 0\\
p(1)=-1 \text{ minimal } &\,\Rightarrow\,& p(1)=-1,\, p'(1)=0       &\,\Rightarrow\,& a_0+a_1+a_2+a_3+a_4  =& -1\\
                         &               &                          &               & a_1+2a_2+3a_3+4a_4       =& 0\\
p(2)=0 \text{ Sattelpkt.}&\,\Rightarrow\,& p(2)=0,\, p'(2)=p''(2)=0 &\,\Rightarrow\,&
a_0+2a_1+4a_2+8a_3+16a_4 =& 0\\
                         &               &                          &               & a_1+4a_2+12a_3+32a_4     =& 0\\
                         &               &                          &               &
2a_2+12a_3+48a_4     =& 0
\end{align*}
Als Matrixgleichung ergibt sich 
$$\vec A \vec a = \vec b$$
mit 
$$\vec A = \begin{pmatrix}
1  &  0&  0&  0&  0 \\
1  &  1&  1&  1&  1 \\
0  &  1&  2&  3&  4 \\
1  &  2&  4&  8& 16 \\
0  &  1&  4& 12& 32 \\
0  &  0&  2& 12& 48 \end{pmatrix}
\text{ und }\vec b = \begin{pmatrix}
0\\-1\\0\\0\\0\\0\end{pmatrix}.$$
\item Der Rang von $\vec A$ ist h\"ochstens 5, da die Matrix nur f\"unf Spalten hat. Die Determinante
der ersten f\"unf Zeilen der Matrix ist
\begin{align*}
\det \begin{pmatrix}
1  &  0&  0&  0&  0 \\
1  &  1&  1&  1&  1 \\
0  &  1&  2&  3&  4 \\
1  &  2&  4&  8& 16 \\
0  &  1&  4& 12& 32 \end{pmatrix}
=& \det\begin{pmatrix}
  1&  1&  1&  1 \\           
  1&  2&  3&  4 \\           
  2&  4&  8& 16 \\           
  1&  4& 12& 32 \end{pmatrix}=\det \begin{pmatrix}
 1 & 0 & 0 & 0\\
 1 & 1 & 2 & 3\\
 1 & 2 & 6 &14\\
 1 & 3 & 11&31\end{pmatrix}\\
=& \det \begin{pmatrix}
  1 & 2 & 3\\             
  2 & 6 &14\\             
  3 & 11&31\end{pmatrix} = \det\begin{pmatrix}
  1 & 0 & 0\\           
  2 & 2 & 8\\           
  3 &  5&22\end{pmatrix}\\
=& \det\begin{pmatrix} 2 & 8 \\ 5 & 22\end{pmatrix} = 4\neq 0
\end{align*}
Also haben die ersten f\"unf Zeilen den Rang 5 und damit auch 
$$\Rang \vec A = 5.$$
Die Determinante der erweiterten Systemmatrix $(\vec A|\vec b)$ ist 
\begin{align*}
\det(\vec A|\vec b)=& \det\begin{pmatrix}
1  &  0&  0&  0&  0 &  0 \\
1  &  1&  1&  1&  1 & -1 \\
0  &  1&  2&  3&  4 &  0 \\
1  &  2&  4&  8& 16 &  0 \\
0  &  1&  4& 12& 32 &  0 \\
0  &  0&  2& 12& 48 &  0 \end{pmatrix}= \det \begin{pmatrix}
  1&  1&  1&  1 & -1 \\           
  1&  2&  3&  4 &  0 \\           
  2&  4&  8& 16 &  0 \\           
  1&  4& 12& 32 &  0 \\           
  0&  2& 12& 48 &  0 \end{pmatrix}\\
=& -\det \begin{pmatrix}
  1&  2&  3&  4  \\           
  2&  4&  8& 16  \\           
  1&  4& 12& 32  \\           
  0&  2& 12& 48  \end{pmatrix} = -\det\begin{pmatrix}
  1&  2&  3&  4  \\           
  0&  0&  2&  8  \\           
  0&  2&  9& 28  \\           
  0&  2& 12& 48  \end{pmatrix} \\
=& -\det\begin{pmatrix}
0&  2&  8  \\              
2&  9& 28  \\              
2& 12& 48  \end{pmatrix} = -\det \begin{pmatrix}
0&  2&  8  \\              
2&  9& 28  \\              
0&  3& 20  \end{pmatrix} \\
=& 2\cdot \det \begin{pmatrix} 2 & 8 \\ 3 & 20\end{pmatrix} = 32\neq 0
\end{align*}
Also hat $(\vec A | \vec b)$ Vollrang, $\Rang (\vec A|\vec b)=6$ und wegen 
$$\Rang \vec A \neq \Rang (\vec A|\vec b)$$
ist das Gleichungssystem $\vec A \vec a=\vec b$ nicht l\"osbar. 
\item Der Funktionswert $p(0)=c$ taucht in der oberen rechten Ecke der Matrix $(\vec A|\tilde {\vec  b})$
auf. Dabei ist $\tilde {\vec b} = (c,-1,0,0,0,0)^\top$. \\

Die Determinante \"andert sich dadurch wie folgt: 
\begin{align*}
\det (\vec A|\tilde {\vec b})=& \det(\vec A|\vec b)-c\cdot \det \begin{pmatrix}
1  &  1&  1&  1&  1  \\
0  &  1&  2&  3&  4  \\
1  &  2&  4&  8& 16  \\
0  &  1&  4& 12& 32  \\
0  &  0&  2& 12& 48  \end{pmatrix}\\
=&32-c\cdot \det \begin{pmatrix}
  1&  2&  3&  4  \\             
  2&  4&  8& 16  \\             
  1&  4& 12& 32  \\             
  0&  2& 12& 48  \end{pmatrix} - c \cdot \det\begin{pmatrix}
  1&  1&  1&  1  \\             
  1&  2&  3&  4  \\             
  1&  4& 12& 32  \\             
  0&  2& 12& 48  \end{pmatrix}\\
=& 32 -c\cdot (-32) - c \cdot \det\begin{pmatrix}
  1&  1&  1&  1  \\             
  0&  1&  2&  3  \\             
  0&  3& 11& 31  \\             
  0&  2& 12& 48  \end{pmatrix}= 32+32c -c\cdot \det\begin{pmatrix}
  1&  2&  3  \\           
  3& 11& 31  \\           
  2& 12& 48  \end{pmatrix}\\
=& 32+32c-c\cdot \begin{pmatrix} 5 & 22\\ 8 & 42\end{pmatrix}=32-2c
\end{align*}
Die Determinante verschwindet also f\"ur $p(0)=c= 16$. Damit gilt dann
$$\Rang \vec A = \Rang (\vec A|\tilde{\vec b})=5$$
und das System besitzt eine eindeutige L\"osung. 
\item Die L\"osung dieses Systems ist
$$\begin{array}{rrrrr|r|l}
1  &  0&  0&  0&  0 & 16 & \text{                     }\\
1  &  1&  1&  1&  1 & -1 & \text{ -1. Zeile           }\\
0  &  1&  2&  3&  4 &  0 & \text{                     }\\
1  &  2&  4&  8& 16 &  0 & \text{ -1. Zeile           }\\
0  &  1&  4& 12& 32 &  0 & \text{                     }\\
0  &  0&  2& 12& 48 &  0 & \text{                     }\\\hline

1  &  0&  0&  0&  0 & 16 & \text{                     }\\
0  &  1&  1&  1&  1 &-17 & \text{                     }\\
0  &  1&  2&  3&  4 &  0 & \text{ -2. Zeile           }\\
0  &  2&  4&  8& 16 &-16 & \text{ -2$\cdot$ 2. Zeile  }\\
0  &  1&  4& 12& 32 &  0 & \text{ -2. Zeile           }\\
0  &  0&  2& 12& 48 &  0 & \text{                     }\\\hline

1  &  0&  0&  0&  0 & 16 & \text{                     }\\
0  &  1&  1&  1&  1 &-17 & \text{                     }\\
0  &  0&  1&  2&  3 & 17 & \text{                     }\\
0  &  0&  2&  6& 14 & 18 & \text{ -2$\cdot$ 3. Zeile  }\\
0  &  0&  3& 11& 31 & 17 & \text{ -3 $\cdot$ 3. Zeile }\\
0  &  0&  2& 12& 48 &  0 & \text{ -2$\cdot$ 3. Zeile  }\\\hline

1  &  0&  0&  0&  0 & 16 & \text{                     }\\
0  &  1&  1&  1&  1 &-17 & \text{                     }\\
0  &  0&  1&  2&  3 & 17 & \text{                     }\\
0  &  0&  0&  2&  8 &-16 & \text{ $\cdot 1/2$         }\\
0  &  0&  0&  5& 22 &-34 & \text{-5/2$\cdot$ 4. Zeile }\\
0  &  0&  0&  8& 42 &-34 & \text{-4 $\cdot$ 4. Zeile  }\\\hline

1  &  0&  0&  0&  0 & 16 & \text{                     }\\
0  &  1&  1&  1&  1 &-17 & \text{                     }\\
0  &  0&  1&  2&  3 & 17 & \text{                     }\\
0  &  0&  0&  1&  4 & -8 & \text{                     }\\
0  &  0&  0&  0&  2 &  6 & \text{    $\cdot 1/2$      }\\
0  &  0&  0&  0& 10 & 30 & \text{-5 $\cdot$ 5. Zeile  }\\\hline

1  &  0&  0&  0&  0 & 16 & \text{                     }\\
0  &  1&  1&  1&  1 &-17 & \text{                     }\\
0  &  0&  1&  2&  3 & 17 & \text{                     }\\
0  &  0&  0&  1&  4 & -8 & \text{                     }\\
0  &  0&  0&  0&  1 &  3 & \text{                     }\\
0  &  0&  0&  0&  0 &  0 & \text{                     }
\end{array}$$
Es ergibt sich die L\"osung 
$$\vec a =( 16,\, -48,\, 48,\, -20,\, 3)^\top.$$
\item \quad\\
\begin{minipage}{.4\textwidth}
\psset{xunit=1cm, yunit=.5cm, runit=1cm}
\begin{pspicture}(-1,-1)(3,17)
\psgrid[subgriddiv=1,griddots=10,gridlabels=.3](-1,-1)(3,17)
\psplot[plotpoints=200, plotstyle=curve]
{0}{3}
{16 -48 x mul add 48 x mul x mul add -20 x mul x mul x mul add 3 x mul x mul x mul x mul add}
\psdot(0,16)
\psdot(1,-1)
\psdot(2,0)
\end{pspicture}
\end{minipage}

\end{abc}
}

\ErgebnisC{AufganalysIntrPoly001}
{
{\textbf{a)}} Die Systemmatrix ist $\Vek A = \begin{pmatrix} 
1 & 0 & 0 & 0 & 0 \\  
1 & 1 & 1 & 1 & 1 \\  
0 & 1 & 2 & 3 & 4 \\  
1 & 2 & 4 & 8 & 16\\  
0 & 1 & 4 & 12& 32\\  
0 & 0 & 2 & 12& 48\\  
\end{pmatrix}$, die rechte Seite des Systems $\vec b=(0,-1,0,0,0,0)^\top$. 
\textbf{d)} $\vec a=(16,\, -48,\, 48,\, -20,\, 3)^\top$
}
