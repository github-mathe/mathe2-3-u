\Aufgabe[e]{Divergenz von Folgen}{
    Betrachten Sie die Folge:
    \[
    a_n = \frac{n!}{\sqrt{n}}.
    \]
    Zeigen Sie, dass die Folge \((a_n)\) gegen \(\infty\) strebt, indem Sie die Definition 1.10 des Skriptes nutzen. Zeigen Sie dazu, dass für jedes \(M > 0\) ein \(N \in \mathbb{N}\) existiert, sodass:
        \[
        a_n > M \quad \text{für alle } n > N.
        \]
 
        \textbf{Hinweis:}
	Verwenden Sie die Stirling-Approximation für \(n!\):
        \[
        n! \sim \sqrt{2\pi n} \left(\frac{n}{e}\right)^n,
        \]
        um das asymptotische Verhalten von \(a_n\) zu bestimmen, und verwenden Sie die Abschätzung \(n (\ln n - 1) > n\) für \(n > 8\), um ein geeignetes \(N\) für ein gegebenes \(M > 0\) zu finden.
}

\Loesung{
    Mit der Stirling-Approximation:
    \[
    n! \sim \sqrt{2\pi n} \left(\frac{n}{e}\right)^n,
    \]
    ergibt sich:
    \[
    a_n = \frac{n!}{\sqrt{n}} \sim \frac{\sqrt{2\pi n} \left(\frac{n}{e}\right)^n}{\sqrt{n}} = \sqrt{2\pi} \cdot \frac{n^n}{e^n}.
    \]

    \bigskip
    \textbf{Definition der Divergenz:} \\
    Wir zeigen, dass \(a_n \to \infty\), indem wir für jedes \(M > 0\) ein \(N\) finden, sodass:
    \[
    a_n > M.
    \]

    Da \(a_n \sim \sqrt{2\pi} \cdot \frac{n^n}{e^n}\), genügt es zu zeigen, dass:
    \[
    \frac{n^n}{e^n} > \frac{M}{\sqrt{2\pi}}.
    \]

    Nehmen wir den Logarithmus:
    \[
    n \ln n - n > \ln\left(\frac{M}{\sqrt{2\pi}}\right).
    \]

    \bigskip
    \textbf{Abschätzung für \(n\):} \\
    Für \(n > 8\) gilt die Ungleichung:
    \[
    n (\ln n - 1) > n.
    \]
    Daher genügt es, ein \(n\) zu finden, sodass:
    \[
    n > \ln\left(\frac{M}{\sqrt{2\pi}}\right).
    \]

    \textbf{Beispiel:} Für \(M = 1000\) ergibt sich:
    \[
    \ln\left(\frac{1000}{\sqrt{2\pi}}\right) \approx \ln(398.94) \approx 5.99.
    \]
    Also muss \(n > 5.99\). Wählen wir \(N = 8\), dann gilt für alle \(n > N\):
    \[
    a_n > 1000.
    \]

    Somit gilt allgemein:
    \[
    n > \max(8, \ln(M)/\sqrt{2\pi}).
    \]
    
    Die Folge \(a_n = \frac{n!}{\sqrt{n}}\) strebt nach \(\infty\).
}
\ErgebnisC{Divergenz}
{
$N = \max(8, \ln(M)/\sqrt{2\pi})$
}
