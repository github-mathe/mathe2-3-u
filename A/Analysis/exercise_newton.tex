\Aufgabe[]{Newton-Verfahren (alte Klausuraufgabe)}{
Gegeben sei die Funktion $f:\R ^3\rightarrow \R $ durch 
$$f(x,y,z)=z\cdot \cos(\pi\cdot(x+y)) +z^2+y^4.$$
Gesucht sind die station\"aren Punkte dieser Funktion. 
\begin{abc}
\item Geben Sie an, welche Bedingung ein station\"arer Punkt $\vec x\in\R^3$ erf\"ullen muss. 
\item Um eine N\"aherung f\"ur einen solchen Punkt zu berechnen, soll das dreidimensionale Newton-Verfahren angewendet werden. Auf welche Funktion wird das Newton-Verfahren angewendet?\\
Geben Sie die Iterationsvorschrift an. 
\item F\"uhren Sie f\"ur den Startvektor $\vec x_0=\begin{pmatrix}1/2\\1\\0\end{pmatrix}$ einen Iterationsschritt durch. 
\item Geben Sie ein geeignetes Abbruchkriterium des Newton-Verfahrens an. (Dieses Kriteriums ist \textbf{nicht} auszuwerten.)
\end{abc}
}

\Loesung{
\begin{abc}
\item In einem station\"aren Punkt muss der Gradient der Funktion $f$ verschwinden: 
$$\vec 0\overset != \nabla f(x,y,z)=\begin{pmatrix}
-\pi z \sin(\pi(x+y))\\
-\pi z \sin(\pi(x+y))+4y^3\\
\cos(\pi(x+y))+2z
\end{pmatrix}$$

\item Das Newton-Verfahren wird auf die Funktion $\vec F(x,y,z)=\nabla f(x,y,z)$ angewendet. Die Ableitung dieser Funktion ist
\begin{align*}
&\vec F'(x,y,z)\Bigl(=H_f(x,y,z)\Bigr)\\
=&\begin{pmatrix}
-\pi^2 z\cos(\pi(x+y)) & -\pi^2 z \cos(\pi(x+y)) & -\pi \sin(\pi(x+y))\\
-\pi^2 z\cos(\pi(x+y)) & -\pi^2 z \cos(\pi(x+y))+12y^2 & -\pi\sin(\pi(x+y))\\
-\pi\sin(\pi(x+y)) & -\pi \sin(\pi(x+y)) & 2\end{pmatrix}.
\end{align*}
Zu gegebenem Startwert $\vec x_0$ wird die folgende Iteration durchgef\"uhrt: \\
F\"ur $j=0,1,2,\hdots$: 

\begin{iii}
\item L\"ose das lineare Gleichungssystem $\vec F'(\vec x_j)\Delta \vec x = -\vec F(\vec x_j)$ nach $\Delta \vec x$ auf.
\item Berechne n\"achste Iteration $\vec x_{j+1}=\vec x_j+\Delta \vec x$. 
\end{iii}

\item F\"ur den gegebenen Startvektor $\vec x_0=\left(\frac 12,1,0\right)^\top$ ergibt sich 
zun\"achst 
$$\vec F\left(\frac 12,1,0\right)=\begin{pmatrix}0 \\4\\0 \end{pmatrix}\text{ und }\vec F'\left(\frac 12,1,0\right)=\begin{pmatrix}
 0&  0 & \pi \\
0 & 12 & \pi \\
\pi & \pi & 2\end{pmatrix}.$$
Die L\"osung des Gleichungssystems $\vec F'(\vec x_0) \Delta \vec x = -\vec F(\vec x_0)$ 
ergibt
$$\Delta \vec x= \begin{pmatrix}\frac 13\\-\frac 13\\0 \end{pmatrix}$$
und als n\"achsten Iterationsschritt
$$\vec x_1=\vec x_0+\Delta \vec x = \begin{pmatrix}\frac 56\\\frac 23\\0 \end{pmatrix}.$$

\item Geeignete Abbruchkriterien sind etwa
$\Vert{\vec F(\vec x_k)}\Vert<\varepsilon$ oder $\Vert{\vec x_k-\vec x_{k-1}}\Vert<\varepsilon$ mit fest vorgegebenem $\varepsilon>0$. \\
Desweiteren empfiehlt es sich, die Iteration nach $N$ Schritten (z. B. $N=1000$) abzubrechen, auch wenn das Abbruchkriterium nicht erf\"ullt ist.  
\end{abc}
}

