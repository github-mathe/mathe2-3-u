\Aufgabe[e]{Lokale Extrema in 2 Dim.}
{
Gegeben sei die Funktion $f\in {\text{Abb}}\,(\R^2,\R)$ mit
\[
f(x,y)= (x+y)^2-4\,(x+y-2)+y^3-3y .
\]
Bestimmen Sie alle station\"aren (kritischen) Punkte dieser Funktion und geben Sie
an, ob es sich dabei um lokale Minima, Maxima oder Sattelpunkte handelt, oder ob
der Charakter des station\"aren Punktes nicht durch die Hesse--Matrix entschieden 
werden kann.
}
\Loesung{
Die station\"aren Punkte erh\"alt man aus
\[
	\textbf{grad}\,f(x,y) = \begin{pmatrix}
	                        2\,(x+y)-4 \\
	                        2\,(x+y)-4 +3\,y^2-3
	                        \end{pmatrix} = \begin{pmatrix}  0 \\ 0 \end{pmatrix} 
\]
zu \ $P_1 = (3,-1)$ \ und \ $P_2 = (1,1)$. Die Hesse--Matrix der Funktion 
\ $f$ \ lautet:
\[
	\vec{H}_f(x,y) = \begin{pmatrix}
	                     2 & & 2 \\
	                     2 & & 2+6\,y
	                     \end{pmatrix}\,.
\]
Für den Punkt \ $P_1$ \ ergibt sich damit
\[
	\vec{H}_f(3,-1) = \begin{pmatrix} 2 & 2 \\ 2 & -4 \end{pmatrix} 
	\quad \Longrightarrow \lambda_1 \cdot \lambda_2 = \det(\vec H_f) = -12 < 0\ ,                   
\]
Damit haben die Eigenwerte verschiedene Vorzeichen und es handelt sich um einen Sattelpunkt. Entsprechend erh\"alt man für den 
Punkt \ $P_2$, dass 
	\[
	\vec{H}_f(1,1) = \begin{pmatrix} 2 & 2 \\ 2 & 8\end{pmatrix} 
	\quad \Longrightarrow \lambda_1 \cdot \lambda_2 = \det(\vec H_f) = 12 > 0,\, \operatorname{Sp}(\vec H_f) = 10 > 0\,.                   
\]
Damit sind beide Eigenwerte positiv und es handelt sich um ein Minimum.
}


\ErgebnisC{Aufgb-2011A-K-A4a}
{
Die station\"aren Punkte lauten  $P_1 = (3,-1)$ und $P_2 = (1,1)\,.$
}
