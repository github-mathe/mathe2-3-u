\Aufgabe[e]{}{
\begin{abc}
\item Bestimmen Sie die folgenden Integrale
\begin{multicols}{2}
\begin{iii}
\item $\int\limits_{t=0}^x (t^2+3t-4)\mathrm{d} t$
\item $\int\limits_{x=-4}^4 (x^3-x)\mathrm{d} x$
\item $\int\limits_{x=-1}^3 \left(5 x^4 + \frac{x^3}3 +2\right) \mathrm{d} x$
\item $\int \frac{1}{(x+1)^3}\mathrm{d} x$.
\end{iii} 
\end{multicols}
\item Bestimmen Sie desweiteren
\begin{multicols}{2}
\begin{iii}
\item $\int \cos(x)\mathrm{d} x$
\item $\int\limits_{x=2}^8 \frac 1x \mathrm{d} x$
\item $\int\limits_{x=0}^{2\pi} \sin(x) \mathrm{d} x$
\item $\int\limits_{x=0}^{\pi/2}\cos(x) \mathrm{d} x$.
\end{iii}
\end{multicols}
\end{abc}
}

\Loesung{
\begin{abc}
\item 
\begin{iii}
\item 
\begin{align*}
\int\limits_{t=0}^x\left( t^2+3t-4\right) \mathrm{d} t 
=& \left[\frac{t^3}3+\frac {3t^2}2 - 4t\right]_{t=0}^x\\
=& \frac{x^3}3+\frac {3x^2}2 - 4x
\end{align*}
\item
\begin{align*}
\int\limits_{x=-4}^4\left(x^3-x\right)\mathrm{d} x=& \left[\frac{x^4}4-\frac{x^2}2\right]_{x=-4}^4=0
\end{align*}
(Dasselbe Ergebnis kann man auch ohne Rechnung
 begr\"unden, da eine ungerade Funktion auf 
 einem symmetrischen Intervall integriert wird.)
\item
\begin{align*}
\int\limits_{x=-1}^3\left(5x^4+\frac{x^3}3+2\right) \mathrm{d} x =& \left[x^5+\frac{x^4}{12}+2x\right]_{x=-1}^3\\
=& 3^5+\frac{3^4}{12}+2\cdot 3-\left( (-1)^5+\frac{(-1)^4}{12}+2\cdot(-1)\right)\\
=& 243+\frac{27}4+6+1-\frac 1{12}+2 \\
=& 252+\frac{81-1}{12} = \frac{776}3
\end{align*}
\item
\begin{align*}
\int \frac{1}{(x+1)^3}\mathrm{d} x=&-\frac{1}{2(x+1)^2}+C
\end{align*}
\end{iii}
\item 
\begin{iii}
\item
\begin{align*}
\int \cos(x)\mathrm{d} x=&\sin(x)+C
\end{align*}
\item
\begin{align*}
 \int\limits_{x=2}^8 \frac 1x\mathrm{d} x=& \Bigl.\ln|x|\Bigr|_{x=2}^8=\ln(8)-\ln(2)=\ln\frac{8}2=\ln(4)
 \end{align*}
\item
\begin{align*}
 \int\limits_{x=0}^{2\pi} \sin(x) \mathrm{d} x=& \Bigl.-\cos(x)\Bigr|_{x=0}^{2\pi}=-\cos(2\pi)+\cos(0)=0\\
\end{align*}
(Auch hier h\"atte man mit der Symmetrie
 der Sinusfunktion argumentieren k\"onnen.)
\item
\begin{align*}
 \int\limits_{x=0}^{\pi/2}\cos(x)\mathrm{d} x=& \Bigl.\sin(x)\Bigr|_{x=0}^{\pi/2}=\sin\left(\frac\pi 2\right)-\sin(0)=1-0=1.
 \end{align*}
\end{iii}
\end{abc}
}

 \ErgebnisC{analysInteGral013}
 {
\textbf{a) }$\frac{x^3}3+\frac {3x^2}2 - 4x $, 
$0$, 
$\frac{776}3$,
$-\frac{1}{2(x+1)^2}+C$\\
\textbf{b) }$\sin(x)+C$, 
$\ln(4)$,
$0$,
$1$
 }


