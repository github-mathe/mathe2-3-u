\Aufgabe[e]{Kurvendiskussion}{
Bestimmen Sie den maximalen Definitionsbereich, die Symmetrie, alle Nullstellen, sowie Art und Lage
der kritischen Punkte und Wendepunkte der rellen Funktion
$$  f(x) =x \sqrt{16-x^2}.$$

}

\Loesung{
\begin{iii}
\item Der maximale Definitionsbereich ist \ $D = [-4,4]$ .
\item Die Funktion \ $f$ \ ist ungerade bzw. punktsymmetrisch zum Ursprung:
$$
  f(x) = x\sqrt{16-x^2} = -(-x)\sqrt{16-(-x)^2}=-f(-x)\ .
$$
\item  Die Nullstellen sind 
\begin{align*}
&&0=& x \sqrt{16-x^2}\\
\Leftrightarrow&&x=&0\,\text{ oder }\, 16=x^2\\
\Leftrightarrow&&x\in& \{0,\, 4,\, -4\}
\end{align*}
\item Kritische Punkte liegen bei $x\in D$ mit: 
\begin{align*}
&&0=& f'(x) = \sqrt{16-x^2}-\frac{x\cdot x}{\sqrt{16-x^2}}=\frac{16-2x^2}{\sqrt{16-x^2}}\\
\Leftrightarrow &&\pm 4 =& \sqrt 2 x \quad\Leftrightarrow \quad x=\pm 2\sqrt{2}
\end{align*}
Die Funktion $f$ selbst hat an den Grenzen des Definitionsbereiches $D$ sowie im Ursprung den Wert
$f(0)=f(\pm 4)=0$ 
F\"ur alle anderen $x>0$ ist $f(x)>0$. Also muss im Punkt $x=+2\sqrt 2$ das absolute (und damit auch
ein relatives) Maximum $f(2\sqrt 2)=8$ der Funktion liegen. \\
Mit der Symmetrie der Funktion folgt, dass in $x=-2\sqrt 2$ ein Minimum $f(-2\sqrt 2)=-8$ liegt. 
\item  Wendepunkte und Konvexität:\\
Zun\"achst ist 
\begin{align*}
f''(x)=& \frac{-4x\sqrt{16-x^2}-(16-2x^2)\cdot (-2x)\frac 12(16-x^2)^{-1/2}}{16-x^2}\\
=& \frac{-4x(16-x^2)+x(16-2x^2)}{(16-x^2)^{3/2} }\\
=& \frac{2x^3-48x}{(16-x^2)^{3/2}}
\end{align*}
Die einzige reelle Nullstelle des Z\"ahlers im Definitionsbereich $]-4,4[$ ist $x=0$ und es gilt
$$
   f''(x)=	\begin{cases}
					>0 & \text{für }x \in ]-4,0[\\
					<0 & \text{für }x \in ]0,4[ 
				\end{cases}
$$
Also liegt in $(0,0)$ ein Wendepunkt, links davon ist $f$ konvex und rechts davon konkav. \\

\unitlength1cm
\begin{picture}(2.5,9)
\put(5.5,4){\begin{picture}(0,0)\unitlength1cm
\thinlines
\put( -4.5,0){\vector(1,0){10}}
\multiput( -4 , -0.1)(1,0){9}{\line(0,1){0.2}}
\put(.9,-0.4){{ 1}}
\put( 0,-4.25){\vector(0,1){8.75}}
\multiput( -0.1,-4)(0,1){9}{\line(1,0){0.2}}
\put(-0.5,.9){{ 2}}

\thicklines
\bezier{200}( -4.000, -0.002)( -4.000, -2.748)( -3.333, -3.685)
\bezier{200}( -3.333, -3.685)( -3.047, -4.089)( -2.667, -3.975)
\bezier{200}( -2.667, -3.975)( -2.365, -3.885)( -2.000, -3.464)
\bezier{200}( -2.000, -3.464)( -1.697, -3.114)( -1.333, -2.514)
\bezier{200}( -1.333, -2.514)( -1.041, -2.032)( -0.667, -1.315)
\bezier{200}( -0.667, -1.315)( -0.445, -0.891)(  0.000,  0.000)
\bezier{200}(  0.000,  0.000)(  0.445,  0.891)(  0.667,  1.315)
\bezier{200}(  0.667,  1.315)(  1.041,  2.032)(  1.333,  2.514)
\bezier{200}(  1.333,  2.514)(  1.697,  3.114)(  2.000,  3.464)
\bezier{200}(  2.000,  3.464)(  2.365,  3.885)(  2.667,  3.975)
\bezier{200}(  2.667,  3.975)(  3.047,  4.089)(  3.333,  3.685)
\bezier{200}(  3.333,  3.685)(  4.000,  2.748)(  4.000,  0.002)

\end{picture}}
\end{picture}

\end{iii}
}

\ErgebnisC{AufganalysKurvDisk001}
{
$D(f)= [-4,4]$, $f$ ist ungerade, Nullstellen: $x=0,\pm 4$, Extrema bei $x=\pm 2\sqrt 2$, \\
Wendepunkt
bei $x=0$

}
