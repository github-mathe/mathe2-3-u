\Aufgabe[e]{Wiederholung: Schwerpunkt und Tr\"agheitsmoment}
{
Gegeben sei der Kreissektor \ $B$ \ in der $x$--$z$--Ebene in Abhängigkeit von den Parametern 
  \[ 
  R>0  \text{ und } 0<\alpha\le\pi\ .
\]

Durch die Rotation der Fläche \ $B$ \ um die $z$--Achse wird ein Kugelsegment \ $K$ \ gebildet.\\[1ex]
Bestimmen Sie den Schwerpunkt und das Tr\"agheitsmoment bez\"uglich der $z$-Achse des homogenen Rotationskörpers.\\[2ex]
\textbf{Hinweis}: In Bezug auf die Masse bedeutet Homogenit\"at, dass die Massendichte $\rho(x,y,z)$ des K\"orpers konstant ist. Es kann also beispielsweise $\rho\equiv 1$  angenommen werden. 
\begin{center}
\begin{pspicture}(-1,-.4)(4,4)
\psline[fillstyle=solid, fillcolor=lightgray, linecolor=lightgray](0,0)(2.6,1.5)(0,3)(0,0)
\psline{->}(-1,0)(4,0)
\put(3.6,.1){$x$}

\psline{->}(0,-.4)(0,4)
\put(.1,3.6){$z$}

\psline(0,0)(2.60,1.50)
\psarc[fillstyle=solid, fillcolor=lightgray](0,0){3}{30}{90}

\psarc(0,0){.8}{30}{90}

\put(1,1.4){$B$}
\put(-.4,3.1){$R$}
\put(.2,.4){$\alpha$}

\psellipse[linestyle=dashed, linecolor=gray](0,1.5)(2.6,.4)
\psline[linestyle=dashed, linecolor=gray](0,0)(-2.6,1.5)
\psarc[linestyle=dashed, linecolor=gray](0,0){3}{90}{150}

\end{pspicture}
\end{center}
}
\Loesung{
Zun\"achst ben\"otigen wir das Volumen des Rotationsk\"orpers. Wir f\"uhren die Berechnung in
Kugelkoordinaten durch: 
\begin{align*}
V=&\int\limits_K \d V
=\int\limits_0^{2\pi}\int\limits_0^R\int\limits_0^\alpha r^2\sin\theta\d\theta\d r\d\varphi\\
=& 2\pi \frac{R^3}3 \int\limits_0^\alpha\sin\theta\d\theta
=\frac{2\pi R^3}3 \left( -\cos(\alpha)+\cos(0)\right)=\frac{2\pi R^3}3 (1-\cos(\alpha)).
\end{align*}

Wegen der Symmetrie des Rotationskörpers liegt der Schwerpunkt auf der $z$--Achse des Koordinatensystems. Es ist also nur die $z$--Komponente \ $z_{\text S}$ \ zu berechnen:
\begin{align*}
	z_{\text S} =& \frac 1 V \int\limits_K z\d V
=\frac 1V \int\limits_0^{2\pi}\int\limits_0^R\int\limits_0^\alpha r\cos\theta
r^2\sin\theta\d\theta\d r \d\varphi\\
=& \frac{2\pi}{V}\cdot\frac{R^4}4 \int\limits_0^\alpha \sin\theta\cos\theta\d\theta
= \frac{3R}{4(1-\cos\alpha)}\cdot \left.\frac{\sin^2\theta}2\right|_0^\alpha\\
=& \frac{3R\sin^2\alpha}{8(1-\cos\alpha)}=\frac{3R}8\cdot \frac{1-\cos^2\alpha}{1-\cos\alpha}
= \frac {3R}8 (1+\cos\alpha).
\end{align*}

Das Tr\"agheitsmoment bez\"uglich der $z$-Achse ergibt sich zu: 
\begin{align*}
\Theta_z=&  \int\limits_K r_\perp^2\d V 
=  \int\limits_0^{2\pi}\int\limits_0^R \int\limits_0^\alpha (r\sin\theta)^2 r^2\sin\theta\d \theta\d
r\d\varphi\\
=& \frac{2\pi R^5}5\int\limits_0^\alpha\sin^2\theta\cdot \sin\theta\d\theta
= \frac{2\pi R^5}5 \int\limits_0^\alpha(1-\cos^2\theta)\sin\theta\d \theta\\
=& \frac{2\pi R^5}5 \left[ -\cos\theta+\frac{\cos^3\theta}3\right]_0^\alpha
= \frac{2\pi R^5}5 \left( 1-\cos\alpha + \frac{\cos^3\alpha-1}3\right)\\
=& \frac{2\pi R^5}{15}(\cos^3\alpha-3\cos\alpha+2).
\end{align*}
}

\ErgebnisC{AufganalysIntgNdim017}
{
$z_S=\frac{3R}8(1+\cos\alpha)$, $\Theta_z=\frac{2\pi R^5}{15}(\cos^3\alpha-3\cos\alpha+2)$
}
