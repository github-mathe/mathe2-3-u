\Aufgabe[e]{Stationäre Punkte}{
Gegeben sei die Funktion $$f(x,y) = x^3+x^2y-y^2-4y.$$ Finde alle stationären Punkte und bestimme, ob sie lokale Minima, Maxima oder Sattelpunkte sind.
}
\Loesung{
Die stationären Punkte sind die Lösungen des folgenden Gleichungssystems%
\begin{eqnarray*}
\frac{\partial}{\partial x} f(x,y) = 0 \quad&\Rightarrow& \quad x(3x+2y)= 0 \\\
\frac{\partial}{\partial y} f(x,y) = 0 \quad&\Rightarrow& \quad x^2-2y-4= 0
\end{eqnarray*}
%
Die erste Gleichung ist für $x=0$ und für $3x+2y=0$ erfüllt, d.h. wenn $x=0$ oder wenn $y=-\frac32 x$. In beiden Fällen wird der $y$-Wert durch die zweite Gleichung im System bestimmt.
\begin{enumerate}
\item Fall $x=0$: Aus der zweiten Gleichung ergibt sich $0-2y-4=0$, d.h. \ $y=-2$. Der erste kritische Punkt ist $\vec P_1 = (0,-2)$.
\item Fall $y=-\frac32 x$. Setzt man dies in die zweite Gleichung ein, erhält man $x^2-2(-\frac32 x) - 4 =0$. Dies ergibt
$$
x^2+3x-4=0 \longrightarrow (x-1)(x+4)=0.
$$
Die Lösungen sind $x=1$ und $x=-4$. Die beiden weiteren kritischen Punkte sind dann: $\vec P_2=(1,-\frac32)$ und $\vec P_3=(-4, 6)$.
\end{enumerate}

Zusammengefasst lauten die drei kritischen Punkte
$$
\vec P_1 = (0,-2),\qquad \vec P_2 = (1,-\frac32), \qquad \vec P_3 = (-4,6).
$$

Die Hessische Matrix ist:
$$
H=
\begin{pmatrix}
f_{xx}(x,y) & f_{xy}(x,y) \\
f_{xy}(x,y) & f_{yy}(x,y) 
\end{pmatrix}
=
\begin{pmatrix}
6x+2y & 2x \\
2x & -2
\end{pmatrix}.
$$


Charakterisierung der stationären Punkte:
\begin{align*}
\boldsymbol{P}_1 &= \begin{pmatrix} 0 \\\ -2 \end{pmatrix}, \quad \begin{pmatrix} -4 & 0 \\\\ 0 & -2 \end{pmatrix}, \quad h_{11}<0, \det(H)=8>0 \rightarrow \mbox{Maximum}\\
\boldsymbol{P}_2 &= \begin{pmatrix} 1 \\\ -\frac32 \end{pmatrix}, \quad \begin{pmatrix} 3 & 2 \\\ 2 & -2 \end{pmatrix}, \quad h_{11}>0, \det(H)=-10<0 \rightarrow \mbox{Sattelpunkt}\\
\boldsymbol{P}_3 &= \begin{pmatrix} -4 \\\ 6 \end{pmatrix}, \quad \begin{pmatrix} -12 & -8 \\\ -8 & -2 \end{pmatrix}, \quad h_{11}<0, \det(H)=-40<0 \rightarrow \mbox{Sattelpunkt}
\end{align*}

}
