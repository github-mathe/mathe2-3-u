\Aufgabe[e]{Umkehrfunktion}{
Gegeben seien die folgenden Funktionen. Geben Sie den Definitionsbereich an (betrachten Sie dabei den Hauptwert der Funktion) 
und überprüfen Sie, ob die Funktionen invertierbar sind. Bestimmen Sie jeweils die inverse Funktion.
\begin{iii}
\item  
$f(x)=2x -1 $
\item 
$f(x) = x^\frac{1}{3} $
\item 
$f(x) = (x-1)^\frac{1}{3}$
\item 
$f(x) = \dfrac{x}{x+1}$
\item 
$f(x) = \log_2(x+3)$
\item 
$f(x) = 2+\operatorname{e}^{x-1}$
\item 
$f(x) = \arccos(x^{-2})$
\end{iii}
}
\Loesung{
\begin{iii}
\item Die Funktion $f:\mathbb R \to \mathbb R$, 
$f(x)=2x -1$ ist bijektiv und daher invertierbar. Um die Umkehrfunktion zu bestimmen, l\"ost man die Gleichung
$y=\dfrac{x+1}{2}$ f\"ur die Variable $x$. Es gilt
\begin{align*}
2x-1 &= y,\\
2x &= y+1,\\
x&= \frac{y+1}{2}.
\end{align*}
Das Ergebnis erhalten wir durch Vertauschen der Variablennamen. Beide Funktionen sind auf dem Definitionsbereich $\mathbb R$
definiert.
\item Die Funktion
$x^\frac{1}{3}$ ist definiert f\"ur $x\in \mathbb R$. F\"ur die erste Ableitung gilt $\frac{1}{3}x^{-2/3}\ge 0$ f\"ur $x\in \mathbb R$, 
daher ist die Funktion streng monoton steigend und folglich injektiv. Au\ss erdem ist die Funktion surjektiv auf $\mathbb R$, weil sie nicht beschr\"ankt ist.
Daher ist die Funktion bijektiv und damit invertierbar. Die Umkehrfunktion $x^3$ hat den gleichen Definitionsbereich $\mathbb R$.
\item Die Funktion $(x-1)^{1/3}$ ist ebenso wie die vorherige Funktion bijektiv. Die Umkehrfunktion kann berechnet werden durch
\begin{align*}
(x-1)^\frac{1}{3} &= y,\\
x-1 & = y^3,\\
x = y^3+1
\end{align*}
mit dem Definitionsbereich $\mathbb R$.
\item Die Funktion $\displaystyle \frac{x}{x+1}$ ist definiert f\"ur $\mathbb R \setminus \{-1\}$. Die Funktion hat die Grenzwerte
$$\lim_{x\to\pm\infty} f(x)=1.$$
Daher ist sie surjektiv auf dem Wertebereich $(-\infty,1) \cup (1, +\infty)$. Die Funktion $f$ ist eine bijektive Abbildung
$f: (-\infty,-1) \cup (-1, +\infty)\to(-\infty,1) \cup (1, +\infty)$. Da die Ableitung $$f'(x)=\frac{1}{(x+1)^2} > 0, \quad \mathbb R \setminus \{-1\}$$
stets positiv ist, ist die Funktion streng monoton steigend auf dem Definitionsbereich.
Die Umkehrfunktion wird berechnet durch
\begin{align*}
\frac{x}{x+1} &= y\\
x & = (x+1)y\\
x(1-y) & = y\\
x &= \frac{y}{1-y},
\end{align*}
Die Funktion ist definiert f\"ur $x \in \mathbb R \setminus \{+1\}$.
\item Die Funktion $\log_2(x+3)$ ist definiert f\"ur $x+3 > 0 \Rightarrow x > -3$. Sie ist streng monoton steigend. Die Abbildung $f: (-3, +\infty) \to \mathbb R$ 
ist bijektiv.
Die Umkehrfunktion kann berechnet werden als
\begin{align*}
\log_2(x+3) &= y,\\
2^{\log_2(x+3)} & = 2^y,\\
x+3 = 2^y,\\
x = 2^y-3,
\end{align*}
Die Funktion ist definiert auf $\mathbb R$ und der Wertebereich ist $(-3, +\infty)$.
\item Die Funktion $2+\operatorname{e}^{x-1}$ ist definiert auf $x\in \mathbb R$. Die Abbildung $f: \mathbb R \to (2, +\infty)$ ist bijektiv,
weil die Ableitung stets positiv ist $f'(x) = \operatorname{e}^{x-1} > 0$.
%
Die Umkehrfunktion ist
\begin{align*}
2+\operatorname{e}^{x-1} &= y\\
\operatorname{e}^{x-1} & = y-2\\
\ln{\operatorname{e}^{x-1}}& = \ln{(y-2)}\\
x & = \ln{(y-2)}+1,
\end{align*}
welche definiert ist auf $x\in (2, +\infty)$ und der Wertebereich ist $\mathbb R$.
\item Die Funktion $\arccos(x)$ ist die Umkehrfunktion von $\cos(x)$, wenn man nur den Hauptwert der Funktion betrachtet. Nach Definition beschr\"anken wir den 
Definitionsbereich von $\cos(x)$ auf $0 \leq x \leq \pi$, dann ist der Wertebereich $[-1,1]$. Damit ist der Definitionsbereich von $\arccos(x)$ gegeben 
durch $[-1,1]$ und der Wertebereich ist $[0,\pi]$. F\"ur die Funktion $\arccos(x^{-2})$ ist der Definitionsbereich der Definitionsbereich der 
zusammengesetzten Funktion. Der Definitionsbereich der inneren Funktion $x^{-2}$ ist $\mathbb R \setminus \{0\}$. Der Definitionsbereich der 
\"au\ss eren Funktion $\arccos(w)$ wie oben bemerkt, ist $w\in[-1,1]$, daher muss $w:=x^{-2}\geq-1$ gelten, was immer wahr ist. Es gilt
$x^{-2} \leq 1 \Rightarrow x^2 \ge 1 \Rightarrow x \ge 1 \land x \leq -1$. Der Wertebereich der zusammengesetzten Funktion $\arccos(x^{-2})$ ist $[0,\pi]$, weil das 
der Wertebereich der \"au\ss eren Funktion ist.\\
Zu untersuchen bleibt die Monotonie der zusammengesetzten Funktion. Da die Funktion $x^{-2}$ nicht-monoton ist, k\"onnen wir daraus keine Aussage \"uber 
die Monotonie der zusammengesetzten Funktion treffen. Die Ableitung von $\arccos$ ist
$$(\arccos(w))' = -\frac{1}{\sqrt{1-w^2}}.$$
F\"ur die Ableitung der zusammengesetzten Funktion benutzen wir die Kettenregel. Damit erhalten wir die Ableitung
\begin{align*}
\left(\arccos(x^{-2})\right)^\prime & = -\frac{1}{\sqrt{1-(x^{-2})^2}}\, (-2 x^{-3}),\\
& = \frac{2}{\sqrt{1-(x^{-4})}}\, x^{-3},
\end{align*}
welche ungerade ist. Daher ist die zusammengesetzte Funktion $\arccos(x^{-2})$ nicht-monoton. Um die Umkehrfunktion zu bestimmen m\"ussen wir den Definitionsbereich
in zwei Teile teilen, in denen die Funktion jeweils monoton ist. Damit definieren wir zwei Zweige der Funktion. Die Funktion ist monoton, wenn wir sie auf den Definitionsbereich 
f\"ur den $x$-positiven Zweig auf $[1, \infty)$ einschr\"anken und f\"ur den $x$-negativen Zweig auf $(-\infty, -1]$ einschr\"anken. 
Die Umkehrfunktion wird berechnet durch
\begin{align*}
\arccos(x^{-2}) &= y,\\
\cos(\arccos{x^{-2}}) & = \cos(y),\\
x^{-2} & = \cos(y),\\
x & = \pm \frac{1}{\sqrt{\cos(y)}}.
\end{align*}
Durch Vertauschen der Variablennamen erhalten wir die Umkehrfunktion
$$
y = \pm \frac{1}{\sqrt{\cos(x)}}.
$$
Diese Funktion ist die zusammengesetzte Funktion mit dem Definitionsbereich, der die Bedingungen $$\cos(x)>0$$ und $x\in [0, \pi]$ erf\"ullt. 
Dies ist der Definitionsbereich f\"ur den Hauptwert von $\cos(x)$. Daraus ergibt sich die Bedingung $x\in [0,\frac{\pi}{2})$.  Der Wertebereich ist der Wertebereich der \"au\ss eren Funktion.
Also $[1, +\infty]$ f\"ur den positiven Zweig und  $(-\infty, -1]$ f\"ur den negativen Zweig.
\end{iii}
}

\ErgebnisC{InverseFunktion}{
\textbf{i}) $f^{-1}(x)=\dfrac{x+1}{2}.$
\textbf{ii}) $f^{-1}(x) = x^3.$
\textbf{iii}) $f^{-1}(x) = x^3+1.$
\textbf{iv}) $f^{-1}(x) = \dfrac{x}{1-x}.$
\textbf{v}) $f^{-1}(x) = 2^x-3.$
\textbf{vi}) $f^{-1}(x) = \ln(x-2) + 1.$
\textbf{vii}) $f^{-1}(x) = \pm\dfrac{1}{\sqrt{\cos{x}}}.$
}
