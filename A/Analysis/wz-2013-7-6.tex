\Aufgabe[e]{Taylor--Entwicklung in 2 Dim.}{
Bestimmen Sie das Taylor-Polynom 2. Ordnung der Funktion 
$$
f(x,y) = \EH x\cos(\pi(x+2y))
$$
um den Entwicklungspunkt \ $\vec x_0 = (1,-1)^{\top}$ .
}

\Loesung{
Die partiellen Ableitungen lauten:
\begin{align*}
f(x,y) =& \EH x\cdot\cos\big(\pi(x+2y)\big) \\
&\,\Rightarrow\, f(1,-1) =-\EH{ } \\
\\
f_x(x,y) =& \EH x\cdot \Big[\cos\big(\pi(x+2y)\big) -\pi\sin\big(\pi(x+2y)\big) \Big] \\
&\,\Rightarrow\, f_x(1,-1) = -\EH{ } \\
\\
f_y(x,y) =& -2\pi\cdot\EH x\cdot \sin\big(\pi(x+2y)\big) \\
&\,\Rightarrow\, f_y(1,-1) = 0\\
\\
f_{xx}(x,y) =& \EH x \Big[(1-\pi^2)\cos\big(\pi(x+2y)\big) - 2\pi\sin\big(\pi(x+2y)\big)\Big] \\
&\,\Rightarrow\, f_{xx}(1,-1) =\EH{ }\,(\pi^2-1) \\
\\
f_{xy}(x,y) =& -2\pi\cdot \EH x \Big[\sin\big(\pi(x+2y)\big) + \pi\cos\big(\pi(x+2y)\big)\Big] \\
&\,\Rightarrow\, f_{xy}(1,-1) = 2\,\pi^2\,\EH{ } \\
\\
f_{yy}(x,y) =& -4\,\pi^2\cdot \EH x\cdot \cos\big(\pi(x+2y)\big) \\
&\,\Rightarrow\, f_{yy}(1,-1) = 4\,\pi^2\,\EH{ } \\
\end{align*}

Damit lautet das Taylorpolynom 2. Grades am Entwicklungspunkt \ $\vec x_0 =(1,-1)^{\top}$
\[
T_2(x,y) =  -\EH{ } - \EH{ }\,(x-1) + (\pi^2-1)\,\EH{ }\,\dfrac{(x-1)^2}{2} + 2\pi^2\,\EH{ }\,(x-1)(y+1)+ 4\pi^2\,\EH{ }\,\dfrac{(y+1)^2}{2} \ .
\]
}

% \ErgebnisC{dummy}
% {
% 
% }

