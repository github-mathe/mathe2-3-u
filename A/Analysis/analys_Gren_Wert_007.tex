\Aufgabe[e]{Grenzwert}
{
Bestimmen Sie den Grenzwert 
\[
\lim_{x\rightarrow \infty} (x+3)\left(\operatorname e^{2/x}-1\right)\,.
\]

}

\Loesung{
 Es gilt 
\[
(x+3)\left(\operatorname e^{2/x}-1\right) = \frac{\operatorname 
e^{2/x}-1}{\dfrac{1}{x+3}} =: \frac{f(x)}{g(x)}
\]
mit $\lim_{x\rightarrow \infty} f(x) = 0$ und $\lim_{x\rightarrow \infty} g(x) = 0$. Mit 
dem Satz von L'Hospital folgt dann
\begin{align*}
 \lim_{x\rightarrow \infty} \dfrac{f(x)}{g(x)} & = \lim_{x\rightarrow \infty }
\dfrac{f'(x)}{g'(x)} = \lim_{x\rightarrow \infty} \dfrac{2\operatorname e^{2/x} 
(x+3)^2}{x^2} \\[1ex]
& = \lim_{x\rightarrow \infty} 2\operatorname e^{2/x} \, \lim_{x\rightarrow \infty} 
\dfrac{(x+3)^2}{x^2} = 2\,.
\end{align*}
}
