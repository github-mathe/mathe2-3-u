\subsection*{Umwandlung einer lin. homogenen Dgl. h\"oherer Ordnung in ein Differentialgleichungssystem 1. Ordnung}
Jede homogene lineare Dgl. \emph{n}-ter Ordnung mit konstanten Koeffizienten
\[ 
L_n y := y^{(n)} + a_{n-1} y^{(n-1)} + ... + a_1 y'+ a_0 y = 0
\]
kann in ein Differentialgleichungssystem 1. Ordnung \"uberf\"uhrt werden.
Dazu werden neue Funktionen definiert:
\begin{align*}
  y_1(t) &= y(t) \\
  y_2(t) &= y'(t) \\
  y_3(t) &= y''(t) \\
        &\vdots \\
  y_n(t) &= y^{(n-1)}(t),
\end{align*}
welche sich in einem Vektor $\vec y$ zusammenfassen lassen:
$$
\vec y(x) = \begin{pmatrix} y_1(x) \\ y_2(x) \\ . \\ . \\ y_n(x)  \end{pmatrix}.
$$
\noindent
Durch Ableiten ergeben sich $n$ Differentialgleichungen 1. Ordnung
\begin{align*}
  y_1'(t) &= y'(t) = y_2(t) \\
  y_2'(t) &= y''(t) = y_3(t) \\
        &\vdots \\
  y'_n(t) &= y^{(n)}(t) = - \sum_{k=1}^{n} a_{k-1}y_k(t).
\end{align*}

\noindent
In Matrixschreibweise ergibt sich
\begin{align*}
\vec y'(t) &= \vec A \, \vec y(t)  \\
 \begin{pmatrix} y'_1(t) \\ y'_2(t) \\ . \\ y'_{n-1}(t) \\ y'_n(t)  \end{pmatrix}     &=    \begin{pmatrix}
0 & 1 & 0 & ... & 0 \\
0 & 0 & 1 & ... & 0 \\
. & . & . & ... & . \\
0 & 0 & 0 & ....& 1 \\
-a_0 & -a_1 & -a_2 & ... & -a_{n-1}
\end{pmatrix} 
\begin{pmatrix} y_1(t) \\ y_2(t) \\ . \\ y_{n-1}(t)\\ y_n(t)  \end{pmatrix}
\end{align*}

\noindent
\textbf{Beispiel:}
Gegeben sei die lineare homogene Dgl.:
\[ y'' + 4y' + 5y = 0 \]
\noindent
Es werden die Funktionen
\begin{align*}
  y_1 &= y \\
  y_2 &= y'
\end{align*}
definiert.
\noindent
Durch Ableiten ergibt sich das System
\begin{align*}
  y_1' &= y' = y_2 \\
  y_2' &= y'' =  - 5y_1-4y_2 .
\end{align*}
\noindent
Es resultiert in Matrixschreibweise
$$
\vec y' = \begin{pmatrix}  0 & 1 \\ -5 & -4 \end{pmatrix} \, \vec y
$$


\subsection*{L\"osen eines homogenen lin. Differentialgleichungssystems 1. Ordnung}
Es sei $\vec y(t) = \left( y_1(t), y_2(t), ..., y_n(t)\right)^{\text{T}}$ mit $\vec y'(t) = \vec A\, \vec y(t)$ gegeben.
Es wird der Ansatz
$$
\vec y(t) = \vec v e^{\lambda t} 
$$
gew\"ahlt. Es ergibt sich hiermit
\begin{align*}
\lambda \vec v e^{\lambda t}   &= \vec A \vec v e^{\lambda t}  \\
(\vec A \vec v - \lambda \vec v)e^{\lambda t}  &= 0 \\
(\vec A -\lambda \vec E)\vec v &= 0,
\end{align*}
das dem Eigenwertproblem entspricht, wobei $\lambda$ ein Eigenwert von $\vec A$ und $\vec v$ der dazugeh\"orige Eigenvektor ist. Unter der Annahme, dass die Matrix $\vec A$ diagonalisierbar ist, ergibt sich die allgemeine L\"osung
\begin{align*}
\vec y(t) &= C_1 e^{\lambda_1 t } \vec v_1 +C_2 e^{\lambda_2 t} \vec v_2 + ... \\
          &= \vec Y(t) \, \vec C,
\end{align*}
wobei $\vec Y(t)$ die sogenannte Fundamentalmatrix oder Wronski-Matrix ist.\\

\noindent
\textbf{Beispiel:}\\
Gegeben sei
$$
\vec y'(t) = \begin{pmatrix} 3 & 1 \\ 2 & 2 \end{pmatrix} \vec y(t).
$$
Um die Eigenwerte $\lambda$ zu bestimmen, werden die Nullstellen des charakteristischen Polynoms bestimmt:
\begin{align*}
P(\lambda) &= \text{det}\begin{pmatrix} 3-\lambda & 1 \\ 2 & 2-\lambda \end{pmatrix} \\
           &= (3-\lambda)(2-\lambda) -2 \\
           &= \lambda^2 -5\lambda +4.
\end{align*}
Es ergeben sich $\lambda_1 = 1$ und $\lambda_2 =4$. Zu den Eigenwerten werden nun die dazugeh\"origen Eigenvektoren $v$ bestimmt. \\
Es ergeben sich folgendes Gleichungssysteme:
\begin{align*}
\lambda_1=1:&&\begin{array}{rr|l|l}
2 & 1 & 0 \\
2 & 1 & 0 & \text{II' = II- I}\\\hline
2 & 1 & 0 \\
0 & 0 & 0 & \\
\end{array}&\Leftarrow&&\vec v_1= \begin{pmatrix}-1\\ 2\end{pmatrix}\\
\lambda_2=4 :&&\begin{array}{rr|l|l}
-1&  1 &  0 & \\
 2& -2 &  0 & \text{II'= II+2\,I}\\\hline
-1&  1 &  0 \\
0& 0 & 0 & \\
\end{array}&\Leftarrow&&\vec v_2= \begin{pmatrix}1   \\1\end{pmatrix}\\
\end{align*}
Damit ergibt sich die allgemeine L\"osung
\begin{align*}
\vec y(t) &= C_1 \begin{pmatrix}-1 \\ 2 \end{pmatrix} e^t + C_2  \begin{pmatrix}1 \\ 1 \end{pmatrix} e^{4t} \\
          &=  \begin{pmatrix}-e^x & e^{4x} \\ 2e^x & e^{4x} \end{pmatrix}  \begin{pmatrix} C_1 \\ C_2 \end{pmatrix} \\
          &= \vec Y(t) \begin{pmatrix} C_1 \\ C_2 \end{pmatrix}.
\end{align*}
Die Konstanten $C_1$ und $C_2$ k\"onnen mit Anfangsbedingungen bestimmt werden.

\subsection*{Hauptvektoren}
Ist die Matrix $\vec A$ nicht diagonalisierbar, d.h. ist die algebraische Vielfachheit $k$ des Eigenwertes gr\"o\ss er als die geometrische Vielfachheit zum dazugeh\"origem Eigenwert, so m\"ussen die sogenannten \textbf{Hauptvektoren} $\vec w$ ermittelt werden, um die allgemeine L\"osung bestimmen zu k\"onnen. \\

\noindent
Eine Kette von Hauptvektoren $\vec w_j$ der Stufe $j=1,2,...,k$ zum Eigenwert $\lambda$ lassen sich aus
$$
(\vec A - \lambda \vec E) \vec w_j = \vec w_{j-1}
$$
bestimmen. Und damit auch die allgemeine L\"osung 
$$
\vec y(t) = e^{\lambda t} \sum_{j=1-1}^{k} C_j \dfrac{t^r}{r!} w_{j-r}.
$$\\

\noindent
\textbf{Beispiel:}\\
Gegeben sei die Matrix
$$
\vec A = \begin{pmatrix}
2 & 0 & 0 \\
-1 & -1 & 9 \\
0 & -1 & 5 
\end{pmatrix}.
$$

Die Eigenwerte der Matrix lauten $\lambda_{1,2,3} = 2$, d.h. es liegt eine algebraische Vielfachheit von 3 vor.
Es werden nun die dazugeh\"origen Eigenvektoren bestimmt.
$$
\begin{array}{rrr|l}
0 & 0 & 0 & 0 \\
-1 & -3 & 9 & 0 \\
0 & -1 & 3 & 0 
\end{array}.
$$
Die L\"osung des LGS lautet
$$
\vec w_1 = \begin{pmatrix} 0 \\ 3 \\ 1 \end{pmatrix} \, t,\, t \in \mathbb{R}.
$$
Das bedeutet, dass wir eine geometrische Vielfachheit von 1 vorliegen haben und ein erster Eigenvektor
$$
\vec v_1 = \begin{pmatrix} 0 \\ 3 \\ 1 \end{pmatrix}
$$
lauten kann. Es m\"ussen nun noch zwei weitere Vektoren, die Hauptvektoren, bestimmt werden.
Mit $(\vec A - 2 \vec E) \vec v_2 = \vec v_{1}$ ergibt sich das LGS
$$
\begin{array}{rrr|l}
0 & 0 & 0 & 0 \\
-1 & -3 & 9 & 3 \\
0 & -1 & 3 & 1 
\end{array}.
$$
Die L\"osung des LGS lautet
$$
\vec w_2 = \begin{pmatrix} 0 \\ 3 \\ 1 \end{pmatrix} \, t + \begin{pmatrix} 0 \\ -1 \\ 0 \end{pmatrix} ,\, t \in \mathbb{R}.
$$
Der Hauptvektor 2. Stufe lautet z.B.
$$
\vec v_2 = \begin{pmatrix} 0 \\ -1 \\ 0 \end{pmatrix}.
$$
Der dritte Eigenvektor ergibt sich aus $(\vec A - 2 \vec E) \vec v_3 = \vec v_{2}$  und dem LGS
$$
\begin{array}{rrr|l}
0 & 0 & 0 & 0 \\
-1 & -3 & 9 & -1 \\
0 & -1 & 3 & 0 
\end{array}.
$$
Die L\"osung des LGS lautet
$$
\vec w_3 = \begin{pmatrix} 18 \\ 3 \\ 1 \end{pmatrix} \, t + \begin{pmatrix} 1 \\ 0 \\ 0 \end{pmatrix} ,\, t \in \mathbb{R}.
$$
Der Hauptvektor 3. Stufe lautet z.B.
$$
\vec v_2 = \begin{pmatrix} 1 \\ 0 \\ 0 \end{pmatrix}.
$$
Damit kann die allgemeine L\"osung bestimmt werden:
$$
\vec y(t) = C_1 \vec v_1 e^{2t} + C_2 (\vec v_2 + \vec v_1 t )e^{2t} + C_3 (\vec v_3 + \vec v_2 t + \vec v_1 \frac{t^2}{2}) e^{2t}\\
$$