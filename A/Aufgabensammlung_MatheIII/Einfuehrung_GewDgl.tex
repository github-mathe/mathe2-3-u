Eine gew\"ohnliche Differentialgleichung (Dgl.) ist eine Gleichung, die aus einer unbekannten Funktion $y(x)$ und ihren Ableitungen $y'(x)$, $y''(x)$,..., $y^m(x)$ besteht, wobei die Funktion $y$ nur von einer Variablen $x$ abh\"angt und nur nach dieser abgeleitet wird. Es wird zwischen einer \textbf{impliziten} Darstellung der Differentialgleichung \textbf{$n$-ter Ordnung}
$$
F(x, y'(x), y''(x),..., y^n(x)) =0
$$
und einer \textbf{expliziten} Darstellung der Differentialgleichung $n$-ter Ordnung
$$
y^n(x) = f(x, y'(x), y''(x),..., y^{n-1}(x)) 
$$
unterschieden. Die Differentialgleichung hei\ss t \textbf{autonom}, wenn $F$ bzw. $f$ nicht explizit von $x$ abh\"angt. \\

\noindent
Eine Differentialgleichung ist \textbf{linear}, wenn sie die Form
$$
a_n(x)\, y^{(n)} + a_{n-1}(x)\, y^{(n-1)} + \dots + a_1(x)\, y' + a_0(x)\, y = f(x)
$$
besitzt, wobei $a_n(x)$ \textbf{variable Koeffizienten} sind. H\"angen die Koeffizienten nicht von $x$ ab, so handelt es sich um \textbf{konstante Koeffizienten}.
Man spricht von einer \textbf{homogene} Dgl., wenn $f(x)$=0 ist. Gilt $f(x) \neq 0$, dann handelt es sich um eine \textbf{inhomogene} Differentialgleichung.\\

\noindent
Eine Differentialgleichung zusammen mit einer Anfangsbedingung hei\ss t \textbf{Anfangswertproblem} (AWP) und haben z.B. die Form
$$
y'(x)=f(x,y(x)), \,  y(x_0) = y_0,
$$
wobei $x_0$ und $y_0$ gegebene Werte sind.\\

\noindent
Beispiele f\"ur \textbf{nichtlineare} Differentialgleichungen sind:
\begin{align*}
y' &= y^2, \\
y'' +y \, y' &= 0,\\
y' &= \sin(y).
\end{align*}

\subsection*{Explizite Differentialgleichung 1. Ordnung}
Eine explizite Dgl. 1. Ordnung besitzt die Form
$$
\dfrac{\mathrm{d}y}{\mathrm{d}x}:= y' = f(x,y),
$$
wobei $f$ im Allgemeinen eine nichtlineare Funktion ist. \\
\noindent
\textbf{Typ A: Differentialgleichungen mit getrennten Ver\"anderlichen}
Liegt die Differentialgleichung in der Form
$$
y' =f(x)\cdot g(y)
$$
vor, wobei $f$ und $g$ stetige Funktionen sind, so kann das L\"osungeverfahren der \textbf{Trennung der Ver\"anderlichen} (TdV) angewendet werden. Die L\"osung ist dann gegeben durch
$$
\int \dfrac{\mathrm{d}y}{g(y)} = \int f(x) \, \mathrm{d}x +C.
$$

\textbf{Typ B: Homogene Differentialgleichung}


\textbf{Typ C:$y' = f(ax+by+c)$}

\textbf{Typ D: Lineare Differentialgleichung 1. Ordnung}

\textbf{Typ E: Bernoulli-Differentialgleichung}

\textbf{Typ F: Riccati-Differentialgleichung}

\subsection*{Lineare Differentialgleichung h\"oherer Ordnung}

\subsubsection*{Homogene lineare Differentialgleichung}

\subsubsection*{Inhomogene lineare Differentialgleichung}