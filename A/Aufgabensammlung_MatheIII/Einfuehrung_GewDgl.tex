Eine gew\"ohnliche Differentialgleichung (Dgl.) ist eine Gleichung, die aus einer unbekannten Funktion $y(x)$ und ihren Ableitungen $y'(x)$, $y''(x)$,..., $y^m(x)$ besteht, wobei die Funktion $y$ nur von einer Variablen $x$ abh\"angt und nur nach dieser abgeleitet wird. Es wird zwischen einer \textbf{impliziten} Darstellung der Differentialgleichung \textbf{$n$-ter Ordnung}
$$
F(x, y'(x), y''(x),..., y^n(x)) =0
$$
und einer \textbf{expliziten} Darstellung der Differentialgleichung $n$-ter Ordnung
$$
y^n(x) = f(x, y'(x), y''(x),..., y^{n-1}(x)) 
$$
unterschieden. Die Differentialgleichung hei\ss t \textbf{autonom}, wenn $F$ bzw. $f$ nicht explizit von $x$ abh\"angt. \\

\noindent
Eine Differentialgleichung ist \textbf{linear}, wenn sie die Form
$$
a_n(x)\, y^{(n)} + a_{n-1}(x)\, y^{(n-1)} + \dots + a_1(x)\, y' + a_0(x)\, y = f(x)
$$
besitzt, wobei $a_n(x)$ \textbf{variable Koeffizienten} sind. H\"angen die Koeffizienten nicht von $x$ ab, so handelt es sich um \textbf{konstante Koeffizienten}.
Man spricht von einer \textbf{homogene} Dgl., wenn $f(x)$=0 ist. Gilt $f(x) \neq 0$, dann handelt es sich um eine \textbf{inhomogene} Differentialgleichung.\\

\noindent
Eine Differentialgleichung zusammen mit einer Anfangsbedingung hei\ss t \textbf{Anfangswertproblem} (AWP) und hat z.B. die Form
$$
y'(x)=f(x,y(x)), \,  y(x_0) = y_0,
$$
wobei $x_0$ und $y_0$ gegebene Werte sind.\\

\noindent
Beispiele f\"ur \textbf{nichtlineare} Differentialgleichungen sind:
\begin{align*}
y' &= y^2, \\
y'' +y \, y' &= 0,\\
y' &= \sin(y).
\end{align*}

\subsection*{Explizite Differentialgleichung 1. Ordnung}
Eine explizite Dgl. 1. Ordnung besitzt die Form
$$
\dfrac{\mathrm{d}y}{\mathrm{d}x}:= y' = f(x,y),
$$
wobei $f$ im Allgemeinen eine nichtlineare Funktion ist. \\

\noindent
\textbf{Typ A: Differentialgleichungen mit getrennten Ver\"anderlichen}\\
Liegt die Differentialgleichung in der Form
$$
y' =f(x)\cdot g(y)
$$
vor, wobei $f$ und $g$ stetige Funktionen sind, so kann das L\"osungeverfahren der \textbf{Trennung der Ver\"anderlichen} (TdV) angewendet werden. Die L\"osung ist dann gegeben durch
$$
\int \dfrac{\mathrm{d}y}{g(y)} = \int f(x) \, \mathrm{d}x +C.
$$
\textbf{Beispiel:} 
\begin{align*}
y'-x^2y &= 0 \\
\frac{\mathrm{d}y}{\mathrm{d}x} &= x^2y \\
\int \frac{1}{y} \, \mathrm{d}y &= \int x^2 \, \mathrm{d}x \\
\ln(|y|) &= \frac{1}{3} x^3 +C \\
y        &= e^{\frac{1}{3}x^3+C} = e^{\frac{1}{3}x^3} \, C
\end{align*}

\noindent
\textbf{Typ B: Homogene Differentialgleichung}\\
Ist eine Differentialgleichung der Form
$$
y' =g \left( \dfrac{y}{x} \right), \, x \neq 0 
$$
gegeben, so kann die Dgl. mit der Substitution $u(x) = y(x)/x$ in eine Differentialgleichung mit getrennten Variablen f\"ur $u(x)$ transformiert werden.\\
\newpage
\noindent
\textbf{Beispiel:} 
$$
y' = \left( \frac{y}{x} \right)^2 +\frac{y}{x} \\
$$
Mit $u(x) = y(x)/x$ ergibt sich:
\begin{align*}
y &= u(x) \, x \\
y'   &= u' \, x + u 
\end{align*}
Einsetzen in die Dgl. ergibt:
\begin{align*}
u' x + u &= u^2 +u \\
u' x &= u^2 \\
\frac{\mathrm{d}u}{\mathrm{d}x} &= \frac{u^2}{x} \\
\int \frac{1}{u^2}\, \mathrm{d}u &= \int \frac{1}{x} \, \mathrm{d}x \\
-\frac{1}{u} &= \ln(|x|)+C \\
u &= - \frac{1}{\ln(|x| +C)} \\
y &= - \frac{x}{\ln(|x| +C)}
\end{align*}



\noindent
\textbf{Typ C: $y' = f(ax+by+c)$}\\
Ist die Differentialgleichung in der Form 
$$
y' = f(ax+by+c), \, \text{mit} \, a,b,c \in \mathbb{R}, \, b \neq 0
$$
gegeben, so kann die Dgl. mit dem Ansatz $u(x):= ax+by+c$ in eine Dgl. mit getrennten Variablen
$$
u' = a +b f(u)
$$
transformiert werden. Diese Dgl. kann mit dem Verfahren der Trennung der Ver\"anderlichen gel\"ost werden.\\

\noindent
\textbf{Typ D: Lineare Differentialgleichung 1. Ordnung}\\
Handelt es sich um eine lineare Differentialgleichung 1. Ordnung
$$
L_1 y:= y' +p(x)y = q(x),
$$
so besteht die allgemeine L\"osung der Dgl. aus der L\"osung $y_{\text{h}}(x)$ der homogenen Differentialgleichung
$$
L_1 y := y' +p(x)y = 0
$$
und einer partikul\"aren L\"osung $y_{\text{p}}$ der inhomogenen Dgl.,
sodass
$$
\mathcal{L}(\text{Dgl.}) = y_{\text{p}}(x) + y_{\text{h}}(x)
$$
gilt. Die partikul\"are L\"osung kann mit dem Verfahren der Variation der Konstanten bestimmt werden.\\

\noindent
\textbf{Beispiel:} \\
Gegeben sei die Differentialgleichung
$$
y'-3y=6.
$$
Zun\"achst wird die hom. Dgl. $y'-3y=0$ mit Trennung der Variablen gel\"ost:
\begin{align*}
\frac{\mathrm{d}y}{\mathrm{d}x} &= 3x \\
\int \frac{1}{y} \, \mathrm{d}y &= \int 3 \, \mathrm{d}x \\
\ln(|y|) &= 3x+C \\
y &= e^{3x} \, C.
\end{align*}
Damit lautet die L\"osung der hom. Dgl. $y_{\text{h}}(x)= e^{3x} \, C$.
F\"ur die L\"osung der inhomogenen Dgl. wird das Verfahren der Variation der Konstanten angewendet. Damit lautet der Ansatz f\"ur die partikul\"are L\"osung:
\begin{align*}
y_{\text{p}} &= C(x) e^{3x} \\
y'_{\text{p}} &= C' e^{3x} + 3 C e^{3x}.
\end{align*}
Einsetzen in die Dgl. ergibt:
\begin{align*}
C' e^{3x} + 3 C e^{3x} - 3Ce^{3x}  &= 6 \\
C' e^{3x} &= 6 \\
\Rightarrow C' &= 6 e^{-3x}. 
\end{align*}
Eine Integration liefert:
\begin{align*}
\frac{\mathrm{d}C}{\mathrm{d}x} &= 6 e^{-3x} \\
\int  \, \mathrm{d}C &= \int 6 e^{-3x} \, \mathrm{d}x \\
C &= -2 e^{-3x} + C'
\end{align*}
Damit gilt:
$$
y_{\text{p}}(x) = C(x) e^{3x} = -2 + C' e^{3x}.
$$
Damit gilt f\"ur die allgemeine L\"osung der inhomogenen Differentialgleichung:
\begin{align*}
y(x) &=  y_{\text{h}}(x) + y_{\text{p}}(x)\\
     &= C e^{3x} -2.
\end{align*}

\noindent
\textbf{Typ E: Bernoulli-Differentialgleichung}\\
Ist eine Differentialgleichung der Form
$$
y' + p(x)y = q(x) y^r
$$
gegeben, so spricht man von der Bernoulli-Differentialgleichung.
Diese kann mit $z(x):= y^{1-r}(x)$ in eine lineare Dgl. 1. Ordnung 
$$
z' +(1-r)p(x)z = (1-r)q(x)
$$
transformiert werden.  \\


\noindent
\textbf{Typ F: Riccati-Differentialgleichung}\\
Abgesehen von dem Spezialfall $r=0$ (Bernoulli-Dgl.) hat die Riccati-Dgl. im Allgemeinen nicht geschlossen l\"osbar. In einigen F\"allen kann eine partikul\"are L\"osung geraten werden, sodass eine allgemeine L\"osung angegeben werden kann. \\


\newpage
\subsection*{Lineare Differentialgleichung h\"oherer Ordnung}
Eine lineare Differentialgleichung besitzt die Form
$$
L_n y :=a_n(x)\, y^{(n)} + a_{n-1}(x)\, y^{(n-1)} + \dots + a_1(x)\, y' + a_0(x)\, y = f(x),
$$
wobei $L_n$ ein linearer gew\"ohnlicher Differentialoperator $n$-ter Ordnung ist.  \\


\subsubsection*{Homogene lineare Differentialgleichung}
Ist $f(x) = 0$, so handelt es sich um eine homogene Differentialgleichung.
Die allgemeine L\"osung der homogenen Dgl. lautet dann
$$
y_{\text{h}}(x) = C_1  y_1(x) + C_2 y_2(x) + ... + C_n y_n(x).
$$
F\"ur eine homogene Dgl. mit konstanten Koeffizienten wird als der Exponentialansatz $y(x) = e^{\lambda x }$ gew\"ahlt. Der gesuchte Exponent wird ermittelt, indem die Nullstellen $\lambda_i$ des charakteristischen Polynoms $P(\lambda)$ bestimmt werden. 
Liegen nur einfach Nullstellen vor, so ist die allgemeine L\"osung
$$
 y_{\text{h}}(x) = C_1  e^{\lambda_1 x} + C_2 e^{\lambda_2 x}  + ... + C_n e^{\lambda_n x},
$$
wobei $y_1(x)$, $y_2(x)$,..., $y_n(x)$ ein sogenanntes \textbf{Fundamentalsystem} bilden. \\
Liegen mehrfache Nullstellen $\lambda_i$ mit Vielfachheiten $k_i$ vor, so sieht das charakteristische Polynom wie folgt aus:
$$
P(\lambda) = a_n (\lambda-\lambda_1)^{k_1}(\lambda-\lambda_2)^{k_2}...(\lambda-\lambda_n)^{k_n}.
$$
Dann bilden die Funktionen
\begin{align*}
& x^{k_1-1}e^{\lambda_1x}, ..., xe^{\lambda_1x}, e^{\lambda_1x} \\
& x^{k_2-1}e^{\lambda_2x}, ..., xe^{\lambda_2x}, e^{\lambda_2x} \\
& x^{k_n-1}e^{\lambda_nx}, ..., xe^{\lambda_nx}, e^{\lambda_nx}
\end{align*}
ein Fundamentalsystem.\\


\noindent
\textbf{Beispiel f\"ur einfache Nullstellen von $P(\lambda)$:}\\
Gegeben ist die homogene lineare Differentialgleichung 3. Ordnung
$$
y''' -2y''-5y'+6y =0.
$$
Das charakteristische Polynom ergibt sich durch Einsetzen des Exponentialansatz $y(x) = e^{\lambda x }$ in die Dgl. und lautet:
$$
P(\lambda) = \lambda^3 -2 \lambda^2 -5 \lambda +6 =0.
$$
Die Nullstellen von $P(\lambda)$ sind $\lambda_1 = 1$, $\lambda_2=-2$ und $\lambda_3 = 3$.
Damit ergeben sich $n$ linear unabh\"angige L\"osungen
$$
y_1(x) = e^x, \, y_2(x) = e^{-2x}, \, y_3(x)= e^{3x}.
$$
Damit lautet die allgemeine L\"osung der homogen Dgl.
$$
y(x) = C_1 e^{x} + C_2 e^{-2x} + C_3 e^{3x}.
$$

\noindent
\textbf{Beispiel f\"ur Nullstellen von $P(\lambda)$ mit Vielfachheit $\neq$ 1:}\\
Gegeben sei die Differentialgleichung
$$
y'''+3y''+3y'+y =0.
$$
Das charakteristische Polynom ist
$$
P(\lambda) =  \lambda^3 +3\lambda^2+3\lambda+1 = (\lambda+1)^3.
$$
Damit liegt eine dreifache Nullstelle $\lambda_{1,2,3} = -1$ vor. Es ergeben sich die unabh\"angigen L\"osungen
$$
y_1(x) = e^{-x}, \, y_2(x) = xe^{-x}, \, y_3(x)= x^2e^{-x}.
$$
Damit lautet die allgemeine L\"osung der homogen Dgl.
$$
y(x) = C_1 e^{-x} + C_2 xe^{-x} + C_3 x^2e^{-x}.
$$

\subsubsection*{Inhomogene lineare Differentialgleichung}
Um die inhomogenen linearen Differentialgleichung $L_n y = f(x)$ zu l\"osen, wird zun\"achst die homogene L\"osung bestimmt. Im Anschluss wird die inhomogene Dgl. betrachtet und ein geeigneter Ansatz f\"ur die partikul\"are L\"osung bestimmt, welcher der St\"orfunktion $f(x)$ \"ahnlich ist. Nach Einsetzen des Ansatzes in die Dgl. und Bestimmung der der Koeffizienten, ergibt sich die allgemeine L\"osung:
$$
y(x)   =y_{\text{p}}(x)+ C_1  e^{\lambda_1 x} + C_2 e^{\lambda_2 x}  + ... + C_n e^{\lambda_n x} .
$$
\newpage
\noindent
\textbf{Beispiel:}\\
Gegeben sei die Differentialgleichung
$$
y'''+y'' = x.
$$
Zun\"achst wird die hom. Dgl. $y'''+y'' =0$ gel\"ost.
Das charakteristische Polynom $P(\lambda) = \lambda^3 +\lambda^2$ besitzt die Nullstellen $\lambda_{1,2} = 0$ und $\lambda_3=-1$. Es liegt also eine doppelte Nullstelle vor, sodass die allgemeine L\"osung der hom. Dgl.
$$
y_{\text{h}}(x) = C_1 e^{0x} + C_2 x e^{0x}+ C_3 e^{-x} = C_1 + C_2 x + C_3 e^{-x}
$$
lautet.
Als Ansatz f\"ur die partikul\"are L\"osung wird ein Polynom verwendet:
$$
y_{\text{p}}(x) = A_1 + A_2 x.
$$
Hier liegt jedoch eine zweifache Resonanz vor, sodass der Ansatz modifiziert wird:
\begin{align*}
y_{\text{p}}(x) &= A_1 x^2 + A_2 x^3,\\
y'_{\text{p}}(x) &= 2 A_1x +3 A_2x^2, \\
y''_{\text{p}}(x) &= 2 A_1 +6 A_2 x, \\
y'''_{\text{p}}(x) &= 6 A_2.
\end{align*}
Einsetzen in die Dgl. ergibt:
$$
6 A_2 +2 A_1 +6 A_2 x = x.
$$
Durchf\"uhren eines Koeffizientenvergleiches:
\begin{align*}
6 A_2 &= 1 \Rightarrow A_2 = \frac{1}{6}\\
6A_2+2A_1 &= 0 \Rightarrow A_1 = - \frac{1}{2}
\end{align*}
Damit lautet die partikul\"are L\"osung
$$
y_{\text{p}}(x) = - \frac{1}{2} x^2 +  \frac{1}{6} x^3.
$$
Somit ist die allgemeine L\"osung der inhomogenen Differentialgleichung
$$
y(x) = C_1 + C_2 x + C_3 e^{-x}- \frac{1}{2} x^2 +  \frac{1}{6} x^3.
$$
