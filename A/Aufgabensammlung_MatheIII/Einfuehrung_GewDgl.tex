Eine gew\"ohnliche Differentialgleichung (Dgl.) ist eine Gleichung, die aus einer unbekannten Funktion $y(x)$ und ihren Ableitungen $y'(x)$, $y''(x)$,..., $y^m(x)$ besteht, wobei die Funktion $y$ nur von einer Variablen $x$ abh\"angt und nur nach dieser abgeleitet wird. Es wird zwischen einer \textbf{impliziten} Darstellung der Differentialgleichung \textbf{$n$-ter Ordnung}
$$
F(x, y'(x), y''(x),..., y^n(x)) =0
$$
und einer \textbf{expliziten} Darstellung der Differentialgleichung $n$-ter Ordnung
$$
y^n(x) = f(x, y'(x), y''(x),..., y^{n-1}(x)) 
$$
unterschieden. Die Differentialgleichung hei\ss t \textbf{autonom}, wenn $F$ bzw. $f$ nicht explizit von $x$ abh\"angt. \\

\noindent
Eine Differentialgleichung ist \textbf{linear}, wenn sie die Form
$$
a_n(x)\, y^{(n)} + a_{n-1}(x)\, y^{(n-1)} + \dots + a_1(x)\, y' + a_0(x)\, y = f(x)
$$
besitzt, wobei $a_n(x)$ \textbf{variable Koeffizienten} sind. H\"angen die Koeffizienten nicht von $x$ ab, so handelt es sich um \textbf{konstante Koeffizienten}.
Man spricht von einer \textbf{homogene} Dgl., wenn $f(x)$=0 ist. Gilt $f(x) \neq 0$, dann handelt es sich um eine \textbf{inhomogene} Differentialgleichung.\\

\noindent
Eine Differentialgleichung zusammen mit einer Anfangsbedingung hei\ss t \textbf{Anfangswertproblem} (AWP) und haben z.B. die Form
$$
y'(x)=f(x,y(x)), \,  y(x_0) = y_0,
$$
wobei $x_0$ und $y_0$ gegebene Werte sind.\\

\noindent
Beispiele f\"ur \textbf{nichtlineare} Differentialgleichungen sind:
\begin{align*}
y' &= y^2, \\
y'' +y \, y' &= 0,\\
y' &= \sin(y).
\end{align*}

\subsection*{Explizite Differentialgleichung 1. Ordnung}
Eine explizite Dgl. 1. Ordnung besitzt die Form
$$
\dfrac{\mathrm{d}y}{\mathrm{d}x}:= y' = f(x,y),
$$
wobei $f$ im Allgemeinen eine nichtlineare Funktion ist. \\

\noindent
\textbf{Typ A: Differentialgleichungen mit getrennten Ver\"anderlichen}\\
Liegt die Differentialgleichung in der Form
$$
y' =f(x)\cdot g(y)
$$
vor, wobei $f$ und $g$ stetige Funktionen sind, so kann das L\"osungeverfahren der \textbf{Trennung der Ver\"anderlichen} (TdV) angewendet werden. Die L\"osung ist dann gegeben durch
$$
\int \dfrac{\mathrm{d}y}{g(y)} = \int f(x) \, \mathrm{d}x +C.
$$

\noindent
\textbf{Typ B: Homogene Differentialgleichung}\\
Ist eine Differentialgleichung der Form
$$
y' =g \left( \dfrac{y}{x} \right), \, x \neq 0 
$$
gegeben, so kann die Dgl. mit der Substitution $u(x) = y(x)/x$ in eine Differentialgleichung mit getrennten Variablen f\"ur $u(x)$ transformiert werden.\\

\textbf{Beispiel:} \\

\noindent
\textbf{Typ C: $y' = f(ax+by+c)$}\\
Ist die Differentialgleichung in der Form 
$$
y' = f(ax+by+c), \, \text{mit} \, a,b,c \in \mathbb{R}, \, b \neq 0
$$
gegeben, so kann die Dgl. mit dem Ansatz $u(x):= ax+by+c$ in eine Dgl. mit getrennten Variablen
$$
u' = a +b f(u)
$$
transformiert werden. Diese Dgl. kann mit dem Verfahren der Trennung der Ver\"anderlichen gel\"ost werden.\\

\noindent
\textbf{Typ D: Lineare Differentialgleichung 1. Ordnung}\\
Handelt es sich um eine lineare Differentialgleichung 1. Ordnung
$$
L_1 y:= y' +p(x)y = q(x),
$$
so besteht die allgemeine L\"osung der Dgl. aus der L\"osung $y_{\text{h}}(x)$ der homogenen Differentialgleichung
$$
L_1 y := y' +p(x)y = 0
$$
und einer partikul\"aren L\"osung $y_{\text{p}}$ der inhomogenen Dgl.,
sodass
$$
\mathcal{L}(\text{Dgl.}) = y_{\text{p}} + y_{\text{h}}(x)
$$
gilt. \\
\textbf{Beispiel:} \\

\noindent
\textbf{Typ E: Bernoulli-Differentialgleichung}\\
Ist eine Differentialgleichung der Form
$$
y' + p(x)y = q(x) y^r
$$
gegeben, so spricht man von der Bernoulli-Differentialgleichung.
Diese kann mit $z(x):= y^{1-r}(x)$ in eine lineare Dgl. 1. Ordnung 
$$
z' +(1-r)p(x)z = (1-r)q(x)
$$
transformiert werden.  \\


\noindent
\textbf{Typ F: Riccati-Differentialgleichung}\\
Abgesehen von dem Spezialfall $r=0$ (Bernoulli-Dgl.) hat die Riccati-Dgl. im Allgemeinen nicht geschlossen l\"osbar. In einigen F\"allen kann eine partikul\"are L\"osung geraten werden, sodass eine allgemeine L\"osung angegeben werden kann. \\


\noindent
\subsection*{Lineare Differentialgleichung h\"oherer Ordnung}
Eine lineare Differentialgleichung besitzt die Form
$$
L_n y :=a_n(x)\, y^{(n)} + a_{n-1}(x)\, y^{(n-1)} + \dots + a_1(x)\, y' + a_0(x)\, y = f(x),
$$
wobei $L_n$ ein linearer gew\"ohnlicher Differentialoperator $n$-ter Ordnung ist.  \\

\subsubsection*{Homogene lineare Differentialgleichung}
Ist $f(x) = 0$, so handelt es sich um eine homogene Differentialgleichung.
Die allgemeine L\"osung der Dgl. lautet dann
$$
y_{\text{h}}(x)= C_1  y_1(x) + C_2 y_2(x) + ... + C_n y_n(x).
$$


\subsubsection*{Inhomogene lineare Differentialgleichung}
Die allgemeine L\"osung der inhomogenen linearen Differentialgleichung $L_n y = f(x)$ lautet
\begin{align*}
y(x) & =y_{\text{p}}(x)+ C_1  y_1(x) + C_2 y_2(x) + ... + C_n y_n(x)\\
     & =y_{\text{p}}(x)+ C_1  e^{\lambda_1 x} + C_2 e^{\lambda_2 x}  + ... + C_n e^{\lambda_n x} .
\end{align*}
Das bedeutet, dass zun\"achst die homogene L\"osung bestimmt wird. F\"ur eine Dgl. mit konstanten Koeffizienten wird dazu der Exponentialansatz $y(x) = e^{\lambda x }$ gew\"ahlt. Der gesuchte Exponent wird ermittelt, indem die Nullstellen des charakteristischen Polynoms $P(\lambda)$ bestimmt werden.