Eine gew\"ohnliche Differentialgleichung (Dgl.) ist eine Gleichung, die aus einer unbekannten Funktion $y(x)$ und ihren Ableitungen $y'(x)$, $y''(x)$,..., $y^m(x)$ besteht, wobei die Funktion $y$ nur von einer Variablen $x$ abh\"angt und nur nach dieser abgeleitet wird. Es wird zwischen einer \textbf{impliziten} Darstellung der Differentialgleichung \textbf{$n$-ter Ordnung}
$$
F(x, y'(x), y''(x),..., y^n(x)) =0
$$
und einer \textbf{expliziten} Darstellung der Differentialgleichung $n$-ter Ordnung
$$
y^n(x) = f(x, y'(x), y''(x),..., y^{n-1}(x)) 
$$
unterschieden. \\
\noindent
Eine Differentialgleichung hei\ss t linear, wenn .... und nicht-linear, wenn ..... .
Variable/konstante Koeffizienten.... \\

Man spricht von einer homogene Dgl., wenn ....
Eine Differentialgleichung zusammen mit einer Anfangsbedingung hei\ss t \textbf{Anfangswertproblem} (AWP) und haben z.B. die Form
$$
y'(x)=f(x,y(x)), \,  y(x_0) = y_0,
$$
wobei $x_0$ und $y_0$ gegebene Werte sind.

\subsection*{Explizite Differentialgleichung 1. Ordnung}
Eine explizite Dgl. 1. Ordnung besitzt die Form
$$
\dfrac{\mathrm{d}y}{\mathrm{d}x}:= y' = f(x,y),
$$
wobei $f$ im Allgemeinen eine nichtlineare Funktion ist. \\
\noindent
\textbf{Typ A: Differentialgleichungen mit getrennten Ver\"anderlichen}
Liegt die Differentialgleichung in der Form
$$
y' =f(x)\cdot g(y)
$$
vor, wobei $f$ und $g$ stetige Funktionen sind, so kann das L\"osungeverfahren der \textbf{Trennung der Ver\"anderlichen} (TdV) angewendet werden. Die L\"osung ist dann gegeben durch
$$
\int \dfrac{\mathrm{d}y}{g(y)} = \int f(x) \, \mathrm{d}x +C.
$$




\subsection*{Lineare Differentialgleichung}

\subsubsection*{Homogene lineare Differentialgleichung}

\subsubsection*{Inhomogene lineare Differentialgleichung}