\Aufgabe[e]{Differentialgleichgungssystem, Hauptvektoren}
{
Gegeben sei die Matrix 
$$\vec A=\begin{pmatrix}
4  & 1 & -1 &  2  &   1 \\
0  & 4 &  1 &  1  &   2 \\
0  & 0 &  3 &  0  &  -1 \\
0  & 0 &  0 &  4  &   1 \\
0  & 0 & -1 &  0  &   3 
\end{pmatrix}$$
sowie die Vektoren
$$\vec u_1=\begin{pmatrix}1\\0\\0\\0\\0\end{pmatrix},\, \vec u_2=\begin{pmatrix}0\\1\\0\\0\\0\end{pmatrix},\, \vec u_3=\begin{pmatrix}0\\-2\\0\\1\\0\end{pmatrix},\,
\vec u_4=\begin{pmatrix}0\\4\\-1\\-3\\1\end{pmatrix} \text{ und } \vec u_5=\begin{pmatrix}9\\-10\\8\\-4\\8\end{pmatrix}.$$
 
\begin{abc}
\item Berechnen Sie die Matrix-Vektor-Produkte $\vec A \vec u_j$, $j=1,2,3,4,5$. Welche Eigenwerte und Hauptvektoren hat $\vec A$?
\item L\"osen Sie das Differentialgleichungssystem 
$$\vec y'(x)=\vec A\vec y(x).$$
\item L\"osen Sie das zugeh\"orige Anfangswertproblem mit den Anfangswerten 
$$\vec y(0)=\vec y_0=(0,9,-8,5,-8)^\top.$$
\end{abc}
}

\Loesung{
\begin{abc}
\item Die gefragten Produkte ergeben sich zu 
\begin{align*}
\vec A\vec u_1 =& 4\vec u_1\\
\vec A \vec u_2 =& (1,4,0,0,0)^\top=4\vec u_2+\vec u_1\\
\vec A \vec u_3 =& (0,-7,0,4,0)^\top=4\vec u_3+\vec u_2\\
\vec A \vec u_4 =& (0,14,-4,-11,4)^\top=4\vec u_4+\vec u_3\\
\vec A \vec u_5 =& (18,-20,16,-8,16)^\top=2\vec u_5.
\end{align*}
Damit hat $\vec A$ den vierfachen Eigenwert $\lambda_1=4$ mit Eigenvektor $\vec u_1$ und den einfachen Eigenwert $\lambda_5=2$ mit Eigenvektor $\vec u_5$. 
Die erweiterten Eigenvektoren zu $\lambda_1$ sind $\vec u_2$, $\vec u_3$ und $\vec u_4$, da jeweils gilt

$$\vec A \vec u_j = \lambda_1 \vec u_j + \vec u_{j-1},\qquad j=2,3,4.$$
\item Das Differentialgleichungssystem $\vec y'=\vec A\vec y$ hat dann die L\"osung
\begin{align*}
\vec y(x)=&\EH{\lambda_1 x}\Bigl( \alpha \vec u_1 + \beta \left( \vec u_2 + x \vec u_1\right) + \gamma \left( \vec u_3 + x \vec u_2 + \frac{x^2}2\vec u_1\right) + \\
&\qquad+ \delta\left(\vec u_4 + x \vec u_3 + \frac{x^2}2 \vec u_2+\frac{x^3}{6}\vec u_1\right)  \Bigr) + 
\varepsilon \EH{\lambda_5 x}\vec u_5\\
=&\EH{4x}\left( \alpha \begin{pmatrix}
1 \\0\\0\\0\\0
\end{pmatrix} + 
\beta \begin{pmatrix}
x\\1\\0\\0\\0\end{pmatrix} 
+ \gamma\begin{pmatrix}
x^2/2\\-2+x\\0\\1\\0\end{pmatrix}
+ \delta\begin{pmatrix}
x^3/6\\
4-2x+x^2/2\\
-1\\
-3+x\\
1\end{pmatrix}\right) 
+ \varepsilon\EH{2x}\begin{pmatrix}9\\-10\\8\\-4\\8\end{pmatrix}
\end{align*} 
\item Setzt man die gegebenen Anfangswerte ein, ergibt sich ein Gleichungssystem f\"ur die Integrationskonstanten: 
$$\begin{array}{rrrrr|r|l}
\alpha&\beta&\gamma&\delta&\varepsilon&&\\\hline
1 & 0 & 0  & 0  & 9   & 0  &   \\
0 & 1 & -2 & 4  & -10 & 9  &   \\
0 & 0 & 0  & -1 &  8  & -8 & \text{4. Zeile}  \\
0 & 0 & 1  &  -3&  -4 & 5  & \text{3. Zeile}  \\
0 & 0 & 0  &  1 &  8  & -8 & \text{+3. Zeile}  \\\hline

1 & 0 & 0  & 0  & 9   & 0  &   \\
0 & 1 & -2 & 4  & -10 & 9  &   \\
0 & 0 & 1  &  -3&  -4 & 5  & \text{}  \\
0 & 0 & 0  & -1 &  8  & -8 & \text{}  \\
0 & 0 & 0  &  0 &  16  & -16 &   \\\end{array}$$
Hieraus ergeben sich die Koeffizienten 
$$\varepsilon=-1,\, \delta=0,\, \gamma=1,\, \beta=1,\, \alpha=9$$
und damit 
\begin{align*}
\vec y(x)=&\EH{4x} \begin{pmatrix}
9+x+x^2/2\\
-1+x\\
0\\
1\\
0\end{pmatrix}
-\EH{2x}\begin{pmatrix}9\\-10\\8\\-4\\8\end{pmatrix}
\end{align*} 

 

\end{abc}
}

\ErgebnisC{AufgsysdglEignHptv002}
{
$\lambda_1=4$, $\lambda_2=2$
}

