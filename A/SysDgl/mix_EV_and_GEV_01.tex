\Aufgabe[e]{Differentialgleichungssysteme erster Ordnung}
{
Berechnen Sie die allgemeine Lösungen der folgenden Differentialgleichungssysteme:
\begin{abc}
\item
\begin{align*}
y'_1 &= 2y_1 - 3y_2,\\
y'_2 &= -y_1 + 4y_2.
\end{align*}
\item
\begin{align*}
y'_1 &= -y_1,\\
y'_2 &= -y_2.
\end{align*}
\item
\begin{align*}
y'_1 &= 2y_1 - y_2,\\
y'_2 &= y_1 + 4y_2.
\end{align*}
\end{abc}
}


\Loesung{
\begin{abc}
\item
Wir können das Differentialgleichungssystem schreiben als:
$$
\vec y' = \begin{bmatrix} y'_1 \\ y'_2 \end{bmatrix} =
\begin{bmatrix}
2 & -3 \\ -1 & 4
\end{bmatrix}
\begin{bmatrix}
y_1 \\ y_2
\end{bmatrix}.
$$
Das charakteristische Polynom der Systemmatrix ist:
$$
p(\lambda) = \lambda^2 - 6\lambda + 5.
$$
Die Nullstellen des Polynoms sind $\lambda_1 = 1$ und $\lambda_2 = 5$.
Der Eigenvektor $\vec v_1$ zu dem Eigenwert $\lambda_1 = 1$ wird bestimmt durch das Lösen des 
linearen Gleichungssystems:
$$
\begin{bmatrix}
1 & -3 \\
-1 & 3
\end{bmatrix}
\begin{bmatrix}
v_{11} \\ v_{12}
\end{bmatrix}
=\begin{bmatrix}
0\\0
\end{bmatrix}
$$
Das führt zu
$$
\vec v_1 = \begin{bmatrix}3\\1 \end{bmatrix}.
$$
Der Eigenvektor $\vec v_2$ zu dem Eigenwert $\lambda_2 = 5$ wird bestimmt durch das Lösen des 
linearen Gleichungssystems:
$$
\begin{bmatrix}
-3 & -3 \\
-1 & -1
\end{bmatrix}
\begin{bmatrix}
v_{21} \\ v_{22}
\end{bmatrix}
=\begin{bmatrix}
0\\0
\end{bmatrix}
$$
Das führt zu
$$
\vec v_2 = \begin{bmatrix}1\\-1 \end{bmatrix}.
$$
Die allgemeine Lösung ist:
$$
\vec y = C_1 \operatorname{e}^{t} \begin{bmatrix} 3\\1 \end{bmatrix}
        + C_2 \operatorname{e}^{5t} \begin{bmatrix} 1\\-1 \end{bmatrix},
        \quad C_1,C_2 \in \mathbb{R}.
$$
\item
Wir können das Differentialgleichungssystem schreiben als:
$$
\vec y' = \begin{bmatrix} y'_1 \\ y'_2 \end{bmatrix} =
\begin{bmatrix}
-1 & 0 \\ 0 & -1
\end{bmatrix}
\begin{bmatrix}
y_1 \\ y_2
\end{bmatrix}.
$$
Das charakteristische Polynom der Systemmatrix ist:$$
p(\lambda) = (\lambda + 1)^2.
$$
Die Nullstellen des Polynoms sind $\lambda_1 = \lambda_2 = -1$.
Die Eigenvektoren $\vec v_1,2$ können bestimmt werden durch lösen des Gleichungssystems:
$$
\begin{bmatrix}
0 & 0 \\
0 & 0
\end{bmatrix}
\vec v
=\begin{bmatrix}
0\\0
\end{bmatrix}
$$
Daher ist jeder Vektor $\vec v \in \mathbb{R}^2$ ein Eigenwert. Wir wählen
$\vec v_1 = \begin{bmatrix} 1\\0 \end{bmatrix}$ and $\vec v_2 \begin{bmatrix} 0\\ 1\end{bmatrix}$
Die allgemeine Lösung ist
$$
\vec y = C_1 \operatorname{e}^{-t} \begin{bmatrix}1\\0 \end{bmatrix}
       + C_2 \operatorname{e}^{-t} \begin{bmatrix}0\\1 \end{bmatrix}.
$$
\item
Wir können das Differentialgleichungssystem schreiben als:
$$
\vec y' = \begin{bmatrix} y'_1 \\ y'_2 \end{bmatrix} =
\begin{bmatrix}
2 & -1 \\
1 & 4
\end{bmatrix}
\begin{bmatrix}
y_1 \\ y_2
\end{bmatrix}.
$$
Das charakteristische Polynom der Systemmatrix ist:
$$
p(\lambda) = (\lambda - 3)^2.
$$
Da die Systemmatrix nicht diagonalisierbar ist, müssen wir den Eigenvektor $\vec v_1$ und den 
Hauptvektor $\vec v_2$ bestimmen. Wir lösen dafür das lineare Gleichungssystem:
$$
\begin{bmatrix}
-1 & -1\\
1 & 1
\end{bmatrix}
\vec v_1 = 
\begin{bmatrix}
0\\0
\end{bmatrix}
$$
 und 
$$\begin{bmatrix}
-1 & -1\\
1 & 1
\end{bmatrix}
\vec v_2 = \vec v_1
$$
Das führt zu:
$$
\vec v_1 =\begin{bmatrix} -1\\1 \end{bmatrix} \quad \text{ and } \quad 
\vec v_2 =\begin{bmatrix} 1\\0 \end{bmatrix}
$$
Daher ist die allgemeine Lösung:
$$
\vec y = C_1 \operatorname{e}^{3t} \begin{bmatrix}-1\\1 \end{bmatrix} + 
         C_2 \operatorname{e}^{3t}\begin{bmatrix} 1-t \\ t \end{bmatrix}.
$$
\end{abc}
}

\ErgebnisC{mixEvandGEV01}{
\begin{abc}
\item  $\vec y(t) = C_1 \operatorname{e}^{t} \begin{bmatrix} 3\\1 \end{bmatrix}
        + C_2 \operatorname{e}^{5t} \begin{bmatrix} 1\\-1 \end{bmatrix}$
\item  $\vec y(t) = C_1 \operatorname{e}^{-t} \begin{bmatrix}1\\0 \end{bmatrix}
       + C_2 \operatorname{e}^{-t} \begin{bmatrix}0\\1 \end{bmatrix}$
\item  $\vec y(t) = C_1 \operatorname{e}^{3t} \begin{bmatrix}-1\\1 \end{bmatrix} + 
         C_2 \operatorname{e}^{3t}\begin{bmatrix} 1-t \\ t \end{bmatrix}$
\end{abc}
}
