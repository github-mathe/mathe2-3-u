\Aufgabe[e]{Inhomogenes lineares System von DGLn}{
\begin{abc}
\item Gegeben sei das Anfangswertproblem 
$$u''(x)-16u(x)=16x\qquad \text{ mit }\qquad u(0)=1,\; u'(0)=4.$$
\begin{iii}
\item \"Uberf\"uhren Sie diese gew\"ohnliche Differentialgleichung zweiter Ordnung in ein
System aus zwei Differentialgleichungen erster Ordnung,
$$\vec y'(x)=\vec A\vec y(x)+\vec g(x),\qquad \text{ mit } \qquad \vec y(x)=\begin{pmatrix}y_1(x)\\y_2(x)\end{pmatrix}.$$
Geben Sie hierf\"ur auch die Anfangsbedingung an. 
\item Bestimmen Sie die L\"osung $\vec y(x)$ dieses Anfangswertproblems. 
\item Bestimmen Sie daraus die L\"osung $u(x)$ der urspr\"unglichen Anfangswertaufgabe. 
\end{iii}
%\item Der Spielstand eines Fußballspiels der Mannschaft D gegen die Mannschaft E sei  in der 90. Minute unentschieden. \\
%Der Ball befinde sich zu diesem Zeitpunkt ($t=0$) am Ort $\vec x(0)=\vec x_0=(0,0)^\top$. Er werde mit der Geschwindigkeit $\vec x'(0)=\vec v_0$ in Richtung des Tores der Mannschaft E geschossen.\\
%Die Abwehrspieler der Manschaft E stehen so, dass nur ein Schuss innerhalb der $(x_1,x_2)-$Ebene m\"oglich ist. 
%Die Bahn des Balles werde durch das folgende Differentialgleichungssystem beschrieben:
%$$\vec x''(t)=-k\vec x'(t)-\begin{pmatrix}0\\g\end{pmatrix}.$$
%\begin{iii}
%\item Zeigen Sie, dass dieses Anfangswertproblem umgeschrieben werden kann in ein Differentialgleichungssystem 1. Ordnung mit folgender Gestalt:
%$$\frac{\d}{\d t}\begin{pmatrix}u_1\\u_2\\u_3\\u_4\end{pmatrix}(t)=\begin{pmatrix}
%  0  &  0  &  1  &  0  \\
%  0  &  0  &  0  &  1  \\
%  0  &  0  & -k  &  0  \\
%  0  &  0  &  0  & -k  \end{pmatrix}\begin{pmatrix}u_1(t)\\u_2(t)\\u_3(t)\\u_4(t)\end{pmatrix}+\begin{pmatrix}0\\0\\0\\-g\end{pmatrix},\qquad \vec u(0)=\begin{pmatrix}0\\0\\v_{0,1}\\v_{0,2}\end{pmatrix}$$
%\item L\"osen Sie das vierdimensionale Anfangswertproblem. 
%\item Zu welchem Zeitpunkt $t_1$ \"uberfliegt der Ball die Torlinie (diese liege bei $x_1=10$)?
%\item Wie muss die Startgeschwindigkeit $\vec v_0$ gew\"ahlt werden, dass der Ball die Torlinie in der H\"ohe $x_2=2$ genau zwischen Torlatte und Torwarthand der Mannschaft N \"uberfliegt?
%\end{iii}
\item Gegeben sei das Differentialgleichungssystem 
$$\vec y'(x)=\vec A \vec y(x)+\vec b,\qquad \text{ mit }\qquad  \vec
A=\begin{pmatrix}1&0&0\\3&1&-2\\2&2&1\end{pmatrix}\quad \text{ und }\quad \vec
b=\begin{pmatrix}4\\-4\\6\end{pmatrix}.$$
\begin{iii}
\item Bestimmen Sie ein \textbf{reelles} Fundamentalsystem des zugeh\"origen homogenen Problems. 
\item Bestimmen Sie die allgemeine Lösung des inhomogenen Differentialgleichungssystems 
$$\vec y'(x)=\vec A \vec y(x) + \vec b.$$
\end{iii}
\end{abc}

}

\Loesung{
\begin{abc}
%\item \begin{iii}
%\item Durch die Setzung $\vec u=\begin{pmatrix}\vec x\\\vec
%x'\end{pmatrix}=\begin{pmatrix}x_1\\x_2\\x_1'\\x_2'\end{pmatrix}$ folgt aus der
%Differentialgleichung zweiter Ordnung: 
%$$\frac{\d}{\d
%t} \begin{pmatrix}u_1(t)\\u_2(t)\\u_3(t)\\u_4(t)\end{pmatrix}=\begin{pmatrix}x_1'(t)\\x_2'(t)\\x_1''(t)\\x_2''(t)\end{pmatrix}=\begin{pmatrix}
%0 & 0 & 1 & 0 \\
%0 & 0 & 0 & 1 \\
%0 & 0 & -k& 0 \\
%0 & 0 & 0
%&-k\end{pmatrix}\begin{pmatrix}x_1(t)\\x_2(t)\\x_1'(t)\\x_2'(t)\end{pmatrix}-\begin{pmatrix}0\\0\\0\\g\end{pmatrix}=\vec
%A\vec u -\begin{pmatrix}0\\0\\0\\g\end{pmatrix}.$$ 
%\item Die Systemmatrix $\vec A = \begin{pmatrix}
%0 & 0 & 1 & 0\\
%0 & 0 & 0 & 1\\
%0 & 0 & -k& 0\\
%0 & 0 & 0 &-k\end{pmatrix}$
%hat die Eigenwerte 
%$$\lambda_1=\lambda_2=0\qquad\text{ und } \qquad \lambda_3=\lambda_4=-k.$$
%Eigenvektoren zum doppelten Eigenwert $\lambda_1=0$  sind 
%$$\vec v_1=\begin{pmatrix}1\\0\\0\\0\end{pmatrix},\, 
%\vec v_2=\begin{pmatrix}0\\1\\0\\0\end{pmatrix}. $$
%Eigenvektoren zum Eigenwert $\lambda_3=-k$ sind L\"osungen des Gleichungssystems
%$$(\vec A-(-k)\vec E_4)\vec v=\begin{pmatrix}k&0&1&0\\0&k&0&1\\0&0&0&0\\0&0&0&0\end{pmatrix}\vec v\overset !=\vec 0$$
%Dieses Gleichungssystem hat den Rang 2, es gibt also zwei linear unabh\"angige Eigenvektoren, etwa: 
%$$\vec v_3=\begin{pmatrix}1\\0\\-k\\0\end{pmatrix},\, 
%\vec v_4=\begin{pmatrix}0\\1\\0\\-k\end{pmatrix}.$$
%Damit ist die allgemeine L\"osung der homogenen Differentialgleichung
%$${\vec u}_h(t)=c_1\vec v_1+c_2\vec v_2 + c_3\EH{-kt}\vec v_3+c_4\EH{-kt}\vec v_4.$$
%Eine Partikul\"arl\"osung der inhomogenen Gleichung hat die Gestalt 
%$$\vec u_p(t)=\vec a + t\vec b.$$
%Eingesetzt in die Differentialgleichung ergibt sich 
%\begin{align*}
%&&\vec b =& \vec A \vec a + t \vec A \vec b - \begin{pmatrix}0\\0\\0\\g\end{pmatrix}\\
%\Rightarrow&& \begin{pmatrix}b_1\\b_2\\b_3\\b_4\end{pmatrix}=& \begin{pmatrix}a_3+tb_3-0\\a_4+tb_4-0\\-ka_3-ktb_3-0\\-ka_4-ktb_4-g\end{pmatrix}\\
%\Rightarrow&&b_3=&b_4=0\\
%&&b_1=&a_3\\
%&&b_2=&a_4\\
%&& b_3=&-ka_3\\
%&&b_4=&-ka_4-g\\
%\Rightarrow&&b_3=&b_4=a_3=b_1=0\\
%&&a_4=&b_2=-\frac gk\\
%\end{align*}
%F\"ur $a_1$ und $a_2$ gibt es keine Einschr\"ankungen. Diese Koeffizienten werden $a_1=a_2=0$
%gew\"ahlt. 
%$$\vec u_p(t)=-\frac{g}k\begin{pmatrix}0\\t\\0\\1\end{pmatrix}.$$
%Damit hat das inhomogene System die allgemeine L\"osung 
%$$\vec u(t)=\vec u_h(t)+\vec u_p(t).$$
%Einsetzen der Anfangswerte ergibt die Parameter $c_1,\, c_2,\, c_3$ und $c_4$: 
%$$\vec u(0)=\begin{pmatrix}c_1+c_3\\c_2+c_4\\-kc_3\\-kc_4-g/k\end{pmatrix}\overset!= \begin{pmatrix}0\\0\\ v_{0,1}\\ v_{0,2}\end{pmatrix}\,\Rightarrow\, \begin{pmatrix}
%c_1\\c_2\\c_3\\c_4\end{pmatrix} = \begin{pmatrix}v_{0,1}/k\\(v_{0,2}+g/k)/k\\-v_{0,1}/k\\-(v_{0,2}+g/k)/k\end{pmatrix}.$$
%\item Es ist 
%$$\vec x(t)=\begin{pmatrix}u_1(t)\\u_2(t)\end{pmatrix}=\frac 1k \begin{pmatrix}
%v_{0,1}-v_{0,1}\EH{-kt}\\(v_{0,2}+\frac gk)(1-\EH{-kt})-tg
%\end{pmatrix}.
%$$
%Daraus ergibt sich f\"ur den Zeitpunkt $t_1$: 
%$$x_1(t_1)=\frac 1k v_{0,1}(1+k\EH{-kt_1})\overset!=10\qquad \Rightarrow\qquad 
% t_1=-\frac 1k \ln\left( \frac{10}{v_{0,1}}-\frac 1k\right).$$
%\item Weiter ergibt sich mit diesem Wert f\"ur $t_1$: 
%$$x_2(t_1)=\frac {v_{0,2}}k\left( 1 + k \EH{-kt_1}\right)=\frac{v_{0,2}}k\left( 1+k\left(\frac{10}{v_{0,1}}-\frac 1k\right)\right)=\frac{v_{0,2}\cdot 10k}{v_{0,1}k}=\frac{10 v_{0,2}}{v_{0,1}}$$
%und mit der Bedingung $x_2(t_1)=2$ folgt
%$$\frac{v_{0,1}}{v_{0,2}}=5.$$ 
%\end{iii}
\item 
\begin{iii}
\item Mit 
$$\vec y(x)=\begin{pmatrix}u(x)\\u'(x)\end{pmatrix}$$
ergibt sich 
$$\vec y'(x)=\begin{pmatrix}0&1\\
16 & 0 \end{pmatrix} \vec y(x) + \begin{pmatrix}0\\16x\end{pmatrix}.$$
Die zugeh\"origen Anfangswerte sind $\vec y(0)=(1,4)^\top$. 
\item Die Systemmatrix hat die Eigenwerte $\lambda_1=+4$ und $\lambda_2=-4$ mit den Eigenvektoren 
$$\vec v_1=\begin{pmatrix}1\\4\end{pmatrix}\text{ und }\vec v_2=\begin{pmatrix}-1\\4\end{pmatrix}.$$
Die Fundamentalmatrix ist damit 
$$\vec Y(x)=\begin{pmatrix}\EH{4x}&-\EH{-4x}\\4\EH{4x}&4\EH{-4x}\end{pmatrix}.$$
Die Inverse der Matrix $\vec Y(0)=\begin{pmatrix}1&-1\\4&4\end{pmatrix}$ ist 
$$\vec Y(0)^{-1}=\frac 18\begin{pmatrix}
4  & 1\\
-4 & 1
\end{pmatrix}.$$
Eine L\"osung des Anfangswertproblems ergibt sich dann zu 
\begin{align*}
\vec y(x)=& \vec Y(x)\vec Y(0)^{-1}\vec y(0)+\int\limits_{t=0}^x\vec Y(x-t)\vec Y(0)^{-1}\vec g(t)\d t,\qquad \text{mit } \vec g(t)=\begin{pmatrix}0\\16 t\end{pmatrix}.\\
=&\vec Y(x)\begin{pmatrix}1\\0\end{pmatrix}+\int\limits_{t=0}^x\vec Y(x-t)\begin{pmatrix}2t\\2t\end{pmatrix}\d t\\
=& \EH{4x}\begin{pmatrix} 1\\4\end{pmatrix} + \int\limits_{t=0}^x2t\begin{pmatrix}\EH{4(x-t)}-\EH{-4(x-t)}\\4\EH{4(x-t)}+4\EH{-4(x-t)}\end{pmatrix}\d t\\
=& \EH{4x}\begin{pmatrix} 1\\4\end{pmatrix} + \left[2t\cdot \frac 14\begin{pmatrix}-\EH{4(x-t)}-\EH{-4(x-t)}\\-4\EH{4(x-t)}+4\EH{-4(x-t)}\end{pmatrix}\right]_{t=0}^x+ \\
&- \int\limits_{t=0}^x2\cdot \frac 14\begin{pmatrix}-\EH{4(x-t)}-\EH{-4(x-t)}\\-4\EH{4(x-t)}+4\EH{-4(x-t)}\end{pmatrix}\d t\\
=& \EH{4x}\begin{pmatrix}1\\4\end{pmatrix}+ \frac x2\begin{pmatrix}-2\\0\end{pmatrix}-\frac 18\left.\begin{pmatrix}\EH{4(x-t)}-\EH{-4(x-t)}\\4\EH{4(x-t)}+4\EH{-4(x-t)}\end{pmatrix}\right|_{t=0}^x\\
=& \EH{4x}\begin{pmatrix}1\\4\end{pmatrix}+ \begin{pmatrix}-x\\0\end{pmatrix} + \begin{pmatrix}0\\-1\end{pmatrix} +\frac 18 \begin{pmatrix}\EH{4x}-\EH{-4x}\\4\EH{4x}+4\EH{-4x}\end{pmatrix}\\
=&\begin{pmatrix}-x\\-1\end{pmatrix} + \frac{9\EH{4x}}8\begin{pmatrix}1\\4\end{pmatrix}+ \frac{\EH{-4x}}8\begin{pmatrix}-1\\4\end{pmatrix}
\end{align*}
\end{iii}

\item 
\begin{iii}
\item Die Eigenwerte der Matrix $\vec A$ ergeben sich aus dem charakteristischen Polynom: 
\begin{align*}
&&0=& \det\begin{pmatrix}1-\lambda & 0 & 0 \\ 3 & 1-\lambda & -2\\ 2 & 2 & 1-\lambda\end{pmatrix} =
(1-\lambda)((1-\lambda)^2+4)\\
\Rightarrow&&\lambda_1=&1,\, \lambda_{2/3}=1\pm 2\imag.
\end{align*}
Die Eigenvektoren zu $\lambda_1$ und $\lambda_2$ ergeben sich aus dem charakteristischen
Gleichungssystem: 
\begin{align*}
\lambda_1=1:&&\begin{array}{rrr|l}
0 & 0 & 0 & 0 \\
3 & 0 & -2& 0 \\
2 & 2 & 0 & 0\end{array}&\Leftarrow&&\vec v_1= \begin{pmatrix}2\\-2\\3\end{pmatrix}\\
\lambda_2=1+2\imag :&&\begin{array}{rrr|l}
-2\imag & 0 & 0 & 0 \\
3 & -2\imag & -2& 0 \\
2 & 2 & 0 & -2\imag \end{array}&\Leftarrow&&\vec v_2= \begin{pmatrix}0\\\imag\\1\end{pmatrix}
\end{align*}
Ein reelles Fundamentalsystem ist damit gegeben durch 
\begin{align*}
\vec y_1=&\EH{\lambda_1x}\vec v_1=\EH{x}\begin{pmatrix}2\\-2\\3\end{pmatrix},\,\\
\vec y_2=&\Re\left( \EH{\lambda_2x}\vec v_2\right)=\EH
x \begin{pmatrix}0\\-\sin(2x)\\\cos(2x)\end{pmatrix}\\
\text{ und }\vec y_3=&\Im\left( \EH{\lambda_2x}\vec v_2\right)=\EH
x\begin{pmatrix}0\\\cos(2x)\\\sin(2x)\end{pmatrix}.
\end{align*}
\item Da die Inhomogenit\"at des Differentialgleichungssystems 
$$\vec b=\begin{pmatrix}4\\-4\\6\end{pmatrix}=2\vec v_1$$
ein Eigenvektor der Systemmatrix $\vec A$ ist,
kann man als Partikul\"arl\"osung $\vec y_p(x)=\alpha \vec v_1$ ansetzen. Einsetzen in die Differentialgleichung
ergibt: 
$$\vec y_p'(x)=\vec 0\overset!=\alpha \vec A\vec v_1 + 2\vec v_1=\alpha \cdot 1\vec v_1+2\vec
v_1\,\Rightarrow\, \alpha=-2.$$
Damit hat man als allgemeine L\"osung der inhomogenen Differentialgleichung: 
$$\vec y(x)=\vec y_h(x)+\vec y_p(x)=c_1\vec y_1(x) + c_2\vec y_2(x) + c_3\vec y_3(x)-2\vec v_1.$$
\end{iii}
\end{abc}
}

\ErgebnisC{sysdglMixdKlau011}
{
\begin{abc}
\item 
\begin{iii}
\item $\vec y'(x)=\begin{pmatrix}0&1\\
16 & 0 \end{pmatrix} \vec y(x) + \begin{pmatrix}0\\16x\end{pmatrix}$ mit $\vec y(0)=(1,4)^\top$
\item $y(x) =\begin{pmatrix}-x\\-1\end{pmatrix} + \frac{9\EH{4x}}8\begin{pmatrix}1\\4\end{pmatrix}+ \frac{\EH{-4x}}8\begin{pmatrix}-1\\4\end{pmatrix}$
\item 
\end{iii}
\item 
\begin{iii}
\item $\mathcal{F} = \left\lbrace \vec y_1, \vec y_2, \vec y_3 \right\rbrace  ==\left\lbrace \EH{x}\begin{pmatrix}2\\-2\\3\end{pmatrix}, \, \EH x \begin{pmatrix}0\\-\sin(2x)\\\cos(2x)\end{pmatrix}, \, \EH x\begin{pmatrix}0\\\cos(2x)\\\sin(2x)\end{pmatrix}\right\rbrace $
\item $y(x) = c_1\vec y_1(x) + c_2\vec y_2(x) + c_3\vec y_3(x)-2\vec v_1$ mit $\begin{pmatrix}4\\-4\\6\end{pmatrix}=2\vec v_1$
\end{iii}
\end{abc}
}