\Aufgabe{}{
Gegeben sei das folgende Differentialgleichungssystem

$$
\boldsymbol y'(t) = \begin{pmatrix}2 & 1 & 1\\ -1 & 0 & -1 \\ 1 & 1 & 2\end{pmatrix}
                    \boldsymbol y(t) + \begin{pmatrix} 1\\ 3\\ 9\end{pmatrix}\operatorname{e}^{-3t}
$$
mit dem Anfangswert $\boldsymbol y(0) = \begin{pmatrix} -3\\2\\1 \end{pmatrix}$.

\begin{abc}
\item
Ermitteln Sie die Fundamentalmatrix $\boldsymbol Y(t) $ des homogenen Systems und geben Sie die Lösung des homogenen Systems.
\item
Berechnen Sie die allgemeine Lösung des inhomogenen Systems und lösen Sie das Anfangswertproblem.
\end{abc}
}

\Loesung{

\begin{abc}
\item
Das charakteristische Polynom ist
$$
p(\lambda) = (2-\lambda)(1-\lambda)^2
$$
Die Eigenwerte sind also $\lambda_1 = 2$ und $\lambda_{2/3} = 1$.
Die Eigenvektor $\boldsymbol v_1$ zu dem Eigenwert $\lambda_1 = 2$ ist
$$\begin{array}{rrr|rl}
0  &  1 &  1 & 0 & \text{tausche 1. Zeile und 3.Zeile} \\
-1 & -2 & -1 & 0 &        \\
1  &  1 &  0 & 0 &        \\\hline

 1 &  1 &  0 & 0 &         \\
-1 & -2 & -1 & 0 & + \text{1. Zeile}   \\
 0 &  1 &  1 & 0 &         \\\hline

 1 &  1 &  0 & 0  &         \\
 0 & -1 & -1 & 0  &   \\
 0 &  1 &  1 & 0  & + \text{2. Zeile}\\\hline

 1 &  1 &  0 & 0  &         \\
 0 & -1 & -1 & 0  &   \\
 0 &  0 &  0 & 0  &   \\
\end{array}$$
Damit erhalten den Eigenvektor $\boldsymbol v_1 = \begin{pmatrix}1\\-1\\1 \end{pmatrix}$.

Die Eigenvektoren $\boldsymbol v_2$ und $\boldsymbol v_3$ zu dem Eigenwert $\lambda = 1$

$$\begin{array}{rrr|rl}
1  &  1 &  1 & 0 &  \\
-1 & -1 & -1 & 0 & + \text{1. Zeile}       \\
1  &  1 &  1 & 0 & - \text{1. Zeile}      \\\hline

 1 &  1 &  1 & 0  &         \\
 0 &  0 &  0 & 0  &    \\
 0 &  0 &  0 & 0  &         \\

\end{array}$$
Daraus erhalten wir die Eigenvektoren $\boldsymbol v_2 = \begin{pmatrix} -1\\1\\0 \end{pmatrix}$
und $\boldsymbol v_3 = \begin{pmatrix} -1\\ 0 \\1 \end{pmatrix}$.

Die Fundamentalmatrix ist dann gegeben durch
$$
\boldsymbol Y(t) =
\begin{pmatrix}
\operatorname{e}^{2t} & -\operatorname{e}^{t} & -\operatorname{e}^{t}\\
-\operatorname{e}^{2t}& \operatorname{e}^{t}  & 0 \\
\operatorname{e}^{2t} & 0 & \operatorname{e}^{t}
\end{pmatrix}
$$
Damit ist die allgemeine Lösung der homogenen Gleichung
$$
\boldsymbol y_h(t) = \boldsymbol Y(t) \boldsymbol c \quad \text{ mit } \, \boldsymbol c \in \mathbb{R}^3.
$$
\item
Durch Variation der Konstanten erhalten wir die partikuläre Lösung $\boldsymbol y_p(t) = \boldsymbol Y(t) \boldsymbol c(t)$
Wir lösen das Gleichungssystem $\boldsymbol Y(t)\boldsymbol c' = \begin{pmatrix} 1\\3\\9 \end{pmatrix}\operatorname{e}^{-3t}$.

$$\begin{array}{rrr|rl}
 \operatorname{e}^{2t}  &  -\operatorname{e}^{t} &  -\operatorname{e}^{t} &  \operatorname{e}^{-3t} &  \\
-\operatorname{e}^{2t}  &   \operatorname{e}^{t} &  0                    & 3\operatorname{e}^{-3t} & + \text{ 1. Zeile} \\
 \operatorname{e}^{2t}  &    0                   &  \operatorname{e}^{t} & 9\operatorname{e}^{-3t} & - \text{ 1. Zeile} \\ \hline

 \operatorname{e}^{2t}  &  -\operatorname{e}^{t} &  -\operatorname{e}^{t} &  \operatorname{e}^{-3t} &  \\
 0                      & 0                      & -\operatorname{e}^{t}  & 4\operatorname{e}^{-3t} & \\
 0                      & \operatorname{e}^{t}   & 2\operatorname{e}^{t}  & 8\operatorname{e}^{-3t}
\end{array}
$$
Wir erhalten
$$
\boldsymbol c'(t) = \begin{pmatrix} 13 \operatorname{e}^{-5t} \\ 16\operatorname{e}^{-4t} \\ -4\operatorname{e}^{-4t}\end{pmatrix}
$$
und damit
$$
\boldsymbol c(t)= \begin{pmatrix} -\frac{13}{5}\operatorname{e}^{-5t} \\-4\operatorname{e}^{-4t} \\ \operatorname{e}^{-4t} \end{pmatrix}
$$
Die allgemeine Lösung ist nun
$$
\boldsymbol y_{allg}(t) = \boldsymbol y_p(t) + \boldsymbol y_h(t) = \boldsymbol Y(t)(\boldsymbol c(t) + \boldsymbol c)
$$
Der konstante Vektor wird durch den Anfangswert festgelegt:
\begin{align*}
\begin{pmatrix} -3\\2\\1 \end{pmatrix} &= \boldsymbol Y(0)(\boldsymbol c(0) +\boldsymbol c)\\
                                       &= \begin{pmatrix} 1 & -1 & -1\\-1 & 1 & 0\\ 1 & 0 & 1 \end{pmatrix} \begin{pmatrix} -\frac{13}{5}+c_1\\-4+c_2\\1+c_3 \end{pmatrix}
\end{align*}
Wir erhalten
$$
\boldsymbol c = \begin{pmatrix} 6\\ \frac{13}{5}\\0 \end{pmatrix}
$$
Die Lösung für das Anfangswertproblem lautet
$$
\boldsymbol y_{AWP} (t) =
\begin{pmatrix}
\operatorname{e}^{2t} & -\operatorname{e}^{t} & -\operatorname{e}^{t}\\
-\operatorname{e}^{2t}& \operatorname{e}^{t}  & 0 \\
\operatorname{e}^{2t} & 0 & \operatorname{e}^{t}
\end{pmatrix}
\left(
\begin{pmatrix} -\frac{13}{5}\operatorname{e}^{-5t} \\-4\operatorname{e}^{-4t} \\ \operatorname{e}^{-4t} \end{pmatrix}
+ \begin{pmatrix} 6\\ \frac{13}{5}\\0 \end{pmatrix}
\right)
$$
\end{abc}

}
