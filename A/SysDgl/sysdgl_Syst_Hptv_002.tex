\Aufgabe[f]{Differentialgleichungssystem}
{
Die Matrix $\vec A\in{\R}^{(3,3)}$ und die Vektoren $\vec{v},\,\vec{w},\,\vec{z}\in{\R}^3$ seien wie folgt gegeben:
$$\vec A=\begin{pmatrix} 5 & 1 & -3 \\
      4 & 4 & -4 \\ 1 & 1 & 1\end{pmatrix},\;\;
\vec{v}:=\begin{pmatrix}1\\2\\1\end{pmatrix},\,
\vec{w}:=\begin{pmatrix}1/2\\1/2\\a\end{pmatrix},\,
\vec{z}:=\begin{pmatrix}1\\0\\1\end{pmatrix}.$$

\begin{abc}
\item
Offenbar gilt \ $\vec A\vec{v}= 4\vec{v}$\,. Berechnen Sie
$$\mbox{(i)}\quad\mbox{$\lambda\in{\R}$ so, dass \ $\vec A\vec{z}=\lambda\,\vec{z}$ \ gilt},\qquad
\mbox{(ii)}\quad\mbox{die Spur \ Sp$(\vec A)$ \ von \ $\vec A$}.$$
Bestimmen Sie mit diesen Ergebnissen alle Eigenwerte der Matrix \ $\vec A$.

(Selbstverständlich sollen Sie \textbf{nicht} das charakteristische Polynom bestimmen und lösen!)

\item Berechnen Sie \ $a\in{\R}$ \ so, dass \ $(\vec A-4\vec E)\vec{w}=\vec{v}$ \ gilt. Bestimmen Sie nun alle Haupt-- und Eigenvektoren von \ $\vec A$\,.

\item Bestimmen Sie die allgemeine Lösung \ $\vec{y}(t)$ \ des DGl--Systems
$$\vec{y}{\,'}(t)=\vec A\vec{y}(t),\quad t\in {\R}\,.$$   
\end{abc}
}


\Loesung{
\begin{abc}
\item
 i) \ Man berechnet ohne Schwierigkeit
$$\vec A\vec{z}=(2,0,2)^\top =2 \vec{z},\quad \mbox{also}\quad \fbox{$\lambda=2$.}$$

ii) \ Es gilt \ $\operatorname{Sp}\,(\vec A)= 10$.

Mit den schon bekannten Eigenwerten \ \fbox{$\lambda_1:=4$}, \fbox{$\lambda_2=2$} \ folgt noch \ $\lambda_3={\operatorname{Sp}}\,(\vec A)- \lambda_1-\lambda_2$\,, also\  \fbox{$\lambda_3=4\ .$}


\item Die Gleichung
  \[
  (\vec A-4\vec E)\vec{w} = \begin{pmatrix} 1 & 1 & -3\\ 4 & 0 & -4 \\ 1 & 1 & -3  \end{pmatrix}\,
                               \begin{pmatrix} 1/2 \\ 1/2 \\ a  \end{pmatrix} =
                               \begin{pmatrix} 1 \\ 2 \\ 1 \end{pmatrix} -a \begin{pmatrix} 3 \\ 4 \\ 3 \end{pmatrix} =
                               \vec{v}-a \begin{pmatrix} 3 \\ 4 \\3 \end{pmatrix} \stackrel{!}{=}\vec{v}
\]
führt auf \ \fbox{$a=0$\,.}

Mit diesem Wert \ ($a=0$) \ erfüllt \ $\vec w$ \ die Hauptvektorgleichung, ist also ein \textbf{Hauptvektor 2.~Stufe} zum Eigenwert \ $\lambda=4$\,.


Eigenvektoren sind \ $\vec{v}$ \ zum EW \ $\lambda=4$ \ sowie \ $\vec{z}$ \ zum EW \ $\lambda=2$.


\item \ Die allgemeine Lösung des DGl--Systems lautet
$$\fbox{$\displaystyle
\vec{y}(t)=\EH{2t}C_1\,\vec{z}+\EH{4t}\,\big(C_2\,\vec{v}+C_3(\vec{w}+t\vec{v})\big),\quad C_k\in{\R}\,.$}$$
\end{abc}
}
