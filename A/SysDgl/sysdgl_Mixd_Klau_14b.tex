\Aufgabe[e]{Inhomogenes lineares System von DGLn}{
Gegeben seien 
$$\vec A=\begin{pmatrix}
  -7&  -5&   6\\
   9&   8&  -9\\
   0&   1&  -1
\end{pmatrix},\quad \vec
g(x)=\begin{pmatrix}x\,\EH{-x}\\\EH{-x}\\(x+1)\EH{-x}\end{pmatrix}\quad \text{ und } \quad \vec y_0=\begin{pmatrix}1\\0\\2\end{pmatrix}.$$
Zu untersuchen ist das Anfangswertproblem  
$$\vec y'(x)=\vec A \vec y(x)+\vec g(x)\quad \text{ mit } \vec y(0)=\vec y_0.$$
\textbf{Hinweis}: Ein Eigenwert der Matrix $\vec A$ ist $\lambda=2$. 
\begin{abc}
\item Bestimmen Sie ein Fundamentalsystem des zugeh\"origen homogenen Problems. 
\item Bestimmen Sie die L\"osung des inhomogenen Anfangswertproblems. 
\end{abc}

}

\Loesung{
\begin{abc}
\item Die Eigenwerte der Matrix $\vec A$ ergeben sich aus dem charakteristischen Polynom: 
\begin{align*}
&&0=& \det\begin{pmatrix}-7-\lambda &-5 & 6 \\ 9 & 8-\lambda & -9\\ 0 & 1 &-1-\lambda\end{pmatrix}\\
&&=&(-7-\lambda)\det\begin{pmatrix}8-\lambda&-9\\1&-1-\lambda\end{pmatrix}
-9\det\begin{pmatrix}-5&6\\1&-1-\lambda\end{pmatrix}\\
&&=&(-7-\lambda)((8-\lambda)(-1-\lambda)+9)-9(-5(-1-\lambda)-6)\\
&&=&(-7-\lambda)(\lambda^2-7\lambda+1)-9(5\lambda-1)\\
&&=&-\lambda^3+3\lambda+2
\end{align*}
Ein Eigenwert ist $\lambda_1=-1$: 
$$\begin{array}{r|r|r|r|r}
            &   -1    &    0   &     3   &     2  \\\hline
            &\setminus&    1   &    -1   &    -2  \\\hline
\lambda=-1  &   -1    &    1   &     2   &     0  
\end{array}$$
Die verbliebenen Eigenwerte ergeben sich als Nullstellen des Restpolynoms \\
$-\lambda^2+\lambda+2$: 
$$\lambda_{2/3}=\frac 12\pm\sqrt{\frac 14+2}=\left\{\begin{array}{l}2\\-1\end{array}\right..$$
Die Eigenvektoren ergeben sich aus dem jeweiligen charakteristischen
Gleichungssystem: 
\begin{align*}
\lambda_2=2:&&\begin{array}{rrr|l|l}
-9&-5 & 6 & 0 \\
9 & 6 & -9& 0 & + I\\
 0& 1 & -3& 0\\\hline
-9&-5 & 6 & 0 \\
0 & 1 & -3& 0 &    \\
 0& 1 & -3& 0&-II\\\hline
-9&-5 & 6 & 0 \\
0 & 1 & -3& 0 &    \\
 0& 0 &  0& 0&   \\
\end{array}&\Leftarrow&&\vec v_2= \begin{pmatrix}-1\\ 3\\1\end{pmatrix}\\
\lambda_1=\lambda_3=-1 :&&\begin{array}{rrr|l|l}
-6&-5 & 6 & 0 & \\
9 & 9 & -9& 0 & +3/2\times \,I\\
 0& 1 &  0& 0\\\hline
-6&-5 & 6 & 0 & \\
0 &3/2&  0& 0 &\\
 0& 1 &  0& 0\\
\end{array}&\Leftarrow&&\vec v_1= \begin{pmatrix}1\\0    \\1\end{pmatrix}\\
\end{align*}
Weitere linear unabh\"angige Eigenvektoren gibt es nicht. Es muss also ein Hauptvektor ermittelt werden, dieser ist L\"osung des Gleichungssystems 
$$(\vec A-(-1)\vec E)\vec w=\vec v_1:$$
$$\begin{array}{rrr|l|l}
-6&-5 & 6 & 1 & \\
9 & 9 & -9& 0 & +3/2\times \,I\\
 0& 1 &  0& 1\\\hline
-6&-5 & 6 & 1 & \\
0 &3/2&  0&3/2&\\
 0& 1 &  0& 1\\
\end{array}\,\Leftarrow\,\vec w= \begin{pmatrix}0\\1    \\1\end{pmatrix}
$$
Ein  Fundamentalsystem ist damit gegeben durch 
\begin{align*}
\vec y_1=&\EH{\lambda_1x}\vec v_1=\EH{-x}\begin{pmatrix}1\\0\\1\end{pmatrix},\,\\
\vec y_2=&\EH{\lambda_1x}(\vec w + x\vec v_1)=\EH{-x}\begin{pmatrix}x\\1\\1+x\end{pmatrix},\,\\
\vec y_3=&\EH{\lambda_2x}\vec v_2=\EH{2x}\begin{pmatrix}-1\\ 3\\1\end{pmatrix}.
\end{align*}
\item Eine Fundamentalmatrix des Systems ergibt sich aus dem Fundamentalsystem: 
$$\vec Y(x)=\left( \vec y_1,\, \vec y_2,\, \vec y_3\right)
=\begin{pmatrix}
\EH{-x} & x\EH{-x} & -\EH{2x}\\
0       & \EH{-x}  & 3 \EH{2x}\\
\EH{-x} & (1+x)\EH{-x} & \EH{2x}
  \end{pmatrix}.$$
An der Stelle $x=0$ hat man
$$\vec Y(0)=\begin{pmatrix}
1 & 0 &-1\\
0 & 1 & 3\\
1 & 1 & 1\end{pmatrix}
$$
Um $\vec Y(0)^{-1}\vec g(t)$ und $\vec Y(0)^{-1}\vec y(0)$ zu ermitteln wird das zugeh\"orige Gleichungssystem gel\"ost: 
$$\begin{array}{rrr|l|l|l}
1 & 0 & -1 & t\EH{-t}     & 1       \\
0 & 1 & 3  & \EH{-t}      & 0      \\
1 & 1 & 1  & (t+1)\EH{-t} & 2 & - I\\\hline

1 & 0 & -1 & t\EH{-t}     & 1   \\
0 & 1 & 3  & \EH{-t}      & 0   \\
0 & 1 & 2  & \EH{-t}      & 1   & - II\\\hline

1 & 0 & -1 & t\EH{-t}     & 1  \\
0 & 1 & 3  & \EH{-t}      & 0  \\
0 & 0 &-1  &     0        & 1  & \\

\end{array}$$
L\"osung des Systems ist 
$$\vec Y(0)^{-1}\vec g(t)=\begin{pmatrix}
t\EH{-t}\\
\EH{-t}\\
0
\end{pmatrix}
$$
und 
$$\vec Y(0)^{-t}\vec y(0)=\begin{pmatrix}
0\\
3\\
-1\end{pmatrix}.$$
Damit ergibt sich dann die L\"osung des Anfangswertproblems zu: 
\begin{align*}
&\vec y_{AWP}(x)\\
=&\vec Y(x)\vec Y(0)^{-1}\vec y(0)+\int\limits_{t=0}^x\vec Y(x-t)\vec Y(0)^{-1}\vec g(t)\d t\\
=& \vec Y(x)\begin{pmatrix}0\\3\\-1\end{pmatrix} + \\
&+\int\limits_{t=0}^x\begin{pmatrix}
\EH{-(x-t)} & (x-t)\EH{-(x-t)}    & -\EH{2(x-t)}  \\
   0        &      \EH{-(x-t)}    & 3\EH{2(x-t)}  \\
\EH{-(x-t)} & (1+x-t)\EH{-(x-t)}    &  \EH{2(x-t)}
\end{pmatrix}
\EH{-t}\begin{pmatrix}
t\\1\\0\end{pmatrix}\d t\\
=&3\EH{-x}(\vec w+x\vec v_1) - \EH{2x}\vec v_2 + \\
&+\EH{-x}\int\limits_{t=0}^x
\begin{pmatrix}
x\\
1\\
1+x
\end{pmatrix}\d t\\
=& \EH{-x}\begin{pmatrix}
0+x\\
1+0\\
1+x\end{pmatrix}-\EH{2x}\begin{pmatrix}
-1\\3\\1\end{pmatrix}
+ x\EH{-x}\begin{pmatrix}x\\1\\1+x\end{pmatrix}\\
=& \EH{-x}\begin{pmatrix}x^2+x\\x+1\\x^2+2x+1\end{pmatrix} + \EH{2x}\begin{pmatrix} 1\\-3\\-1\end{pmatrix}
\end{align*}
\end{abc}
}

\ErgebnisC{sysdglMixdKlau14b}
{
\begin{abc}
\item Fundamentalsystemmatrix: $\vec Y(x)=\left( \vec y_1,\, \vec y_2,\, \vec y_3\right)
=\begin{pmatrix}
\EH{-x} & x\EH{-x} & -\EH{2x}\\
0       & \EH{-x}  & 3 \EH{2x}\\
\EH{-x} & (1+x)\EH{-x} & \EH{2x}
  \end{pmatrix}$
\item $y_{\text{AWP}}(x)=\EH{-x}\begin{pmatrix}x^2+x\\x+1\\x^2+2x+1\end{pmatrix} + \EH{2x}\begin{pmatrix} 1\\-3\\-1\end{pmatrix}$
\end{abc}
}