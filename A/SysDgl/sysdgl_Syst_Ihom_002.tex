\Aufgabe[e]{Inhomogene lineare Systeme (Hauptvektoren)}
{
Gegeben sei das Differentialgleichungssystem 

$$\vec y'(t) = \begin{pmatrix}
0  & 1 & -1 \\
-2 & 3 & -1 \\
-1 & 1 &  1 \end{pmatrix}\vec y(t) + \begin{pmatrix}-1\\2\\1\end{pmatrix} \EH{5t}$$
 mit den Anfangswerten $\vec y(0)=(0,1,2)^\top$. 
\begin{abc}
\item Ermitteln Sie die Fundamentalmatrix $\vec Y(t)$ (des homogenen Systems). 
\item Berechnen Sie die L\"osung  des Anfangswertproblemes, indem Sie die Schritte der Variation der Konstanten explizit ausf\"uhren. 
\item Berechnen Sie zus\"atzlich die L\"osung des Anfangswertproblemes unter Nutzung der entsprechenden Formel aus der Vorlesung. 
\end{abc}
}

\Loesung{
\begin{abc}
\item Die Eigenwerte der Systemmatrix 
$$\vec A = \begin{pmatrix}
0  & 1 & -1 \\
-2 & 3 & -1 \\
-1 & 1 &  1 \end{pmatrix}$$
werden als Nullstellen des charakteristischen Polynoms berechnet: 
\begin{align*}
&&0\overset!=& \det\begin{pmatrix}
0-\lambda  & 1 & -1 \\
-2 & 3-\lambda & -1 \\
-1 & 1 &  1-\lambda \end{pmatrix}
= \det\begin{pmatrix}
0          & 1-\lambda & -1-\lambda+\lambda^2\\
 0         & 1-\lambda & -3+2\lambda\\
-1         & 1         &  1-\lambda \end{pmatrix}\\
&&=&-1\cdot \det\begin{pmatrix}
1-\lambda & -1-\lambda+\lambda^2\\
1-\lambda & -3+2\lambda\end{pmatrix}
= -(1-\lambda)(-3+2\lambda+1+\lambda-\lambda^2)\\
&&=& (1-\lambda)(\lambda^2-3\lambda+2)
= (1-\lambda)(\lambda-2)(\lambda-1)\\
\Rightarrow&&\lambda_1=&\lambda_2=1,\, \lambda_3=2
\end{align*}
Die Eigenvektoren sind L\"osungen der charakteristischen Gleichungssysteme: 
\begin{iii}
\item $\lambda_{1/2}=1$ (weiße rechte Seite) 
$$
\begin{array}{rrr|r|r|l}
-1 & 1 & -1 & 0 & \cellcolor{lightgray} 1 &         \\
-2 & 2 & -1 & 0 & \cellcolor{lightgray} 1 &  \text{II'=II-2I}      \\   
-1 & 1 & 0  & 0 & \cellcolor{lightgray} 0 &   \text{III'=III-I}     \\\hline   

-1 & 1 & -1 & 0 & \cellcolor{lightgray} 1 &         \\
 0 & 0 &  1 & 0 & \cellcolor{lightgray}-1 &   \\
 0 & 0 & 1  & 0 & \cellcolor{lightgray}-1 &  \text{III''=III'-2II'}      \\\hline   

-1 & 1 & -1 & 0 & \cellcolor{lightgray} 1 &         \\
 0 & 0 &  1 & 0 & \cellcolor{lightgray}-1 &   \\
 0 & 0 &  0 & 0 & \cellcolor{lightgray} 0 &  \\
\end{array}
$$
Die einzige (linear unabh\"angige L\"osung ist $\vec v_1=(1,1,0)^\top$. Daher wird noch ein Hauptvektor $\vec v_2$ gesucht. Dieser ist L\"osung desselben Gleichungssystems mit der rechten Seite $\vec v_1$ (graue Spalte). Es ergibt sich $\vec v_2=(0,0,-1)^\top$ als m\"oglicher Hauptvektor. 
\item $\lambda_3=2$: 
$$\begin{array}{rrr|r|l}
-2 & 1 & -1 & 0  &         \\
-2 & 1 & -1 & 0  &       \text{II'=II-I}         \\
-1 & 1 & -1 & 0  & \text{III'=III-1/2I}         \\\hline

-2 & 1 & -1 & 0  &         \\
 0 & 0 &  0 & 0  &         \\
 0 &1/2&-1/2& 0  &         \\
\end{array}$$
Es ergibt sich $\vec v_3=(0,1,1)^\top$. Damit ist die Fundamentalmatrix
$$\vec Y(t)=\left( \EH t \vec v_1,\, \EH t (\vec v_2+\frac{t}{1!}\vec v_1),\,\EH {2t}\vec v_3\right) = 
\begin{pmatrix}
\EH t & \EH t t & 0 \\
\EH t & \EH t t & \EH{2t}\\
0     & -\EH t  & \EH{2t}
\end{pmatrix}$$
\end{iii}
\item Die L\"osung des homogenen Systems ist damit 
$$\vec y_h(t)=\vec Y(t)\vec c \text{ mit dem konstanten Vektor }\vec c\in\R^3.$$
Zur L\"osung des inhomogenen Systems setzen wir den Ansatz $\vec y_p(t)=\vec Y(t) \vec c(t)$  mit der vektorwertigen Funktion $\vec c(t)$ in das inhomogene System ein: 
\begin{align*}
&&(\vec Y(t)\vec c(t))'=& \vec A\vec Y(t)\vec c(t) + \begin{pmatrix}-1\\2\\1\end{pmatrix}\EH{5t}\\
\Rightarrow&&\vec Y'(t)\vec c(t) + \vec Y(t)\vec c'(t)=& \vec A\vec Y(t)\vec c(t)+ \begin{pmatrix}-1\\2\\1\end{pmatrix}\EH{5t}\\
\Rightarrow&&(\underbrace{\vec Y'(t)-\vec A\vec Y(t)}_{=\vec 0})\vec c(t)+ \vec Y(t)\vec c'(t)=& \begin{pmatrix}-1\\2\\1\end{pmatrix}\EH{5t}\\
\Rightarrow&&\vec c'(t)=& \vec Y(t)^{-1}\begin{pmatrix}-1\\2\\1\end{pmatrix}\EH{5t}
\end{align*}
Dabei wird der Ausdruck $\vec Y'(t)-\vec A\vec Y(t)$ Null, da die Spalten von $\vec Y(t)$ bereits L\"osungen der homogenen Gleichung ($\vec y'(t)=\vec A\vec y(t)$) sind. \\
Auf die Berechnung der inversen Matrix $\vec Y(t)^{-1}$ wird verzichtet, stattdessen l\"osen wir das Gleichungssystem $\vec Y(t) \vec c'=\begin{pmatrix}-1\\2\\1\end{pmatrix}\EH{5t}$: 
$$\begin{array}{rrr|r|l}
\EH t & \EH t t & 0       & -1 \EH{5t}  & \\
\EH t & \EH t t & \EH{2t} &  2 \EH{5t}  & \text{II'=II-I}\\
0     & -\EH t  & \EH{2t} &  1 \EH{5t} & \\\hline

\EH t & \EH t t & 0       & -1 \EH{5t}   & \\
   0  &   0     & \EH{2t} &  3 \EH{5t}   & \text{II''=III}\\
0     & -\EH t  & \EH{2t} &  1 \EH{5t}   & \text{III'=II'}\\\hline

\EH t & \EH t t & 0       & -1 \EH{5t}   & \\
0     & -\EH t  & \EH{2t} &  1 \EH{5t}   &  \text{II'''=II''-III'}\\
   0  &   0     & \EH{2t} &  3 \EH{5t}   & \\\hline
   
\EH t & \EH t t & 0       & -1 \EH{5t}   & \\
0     & -\EH t  & 0       &  -2 \EH{5t}   &  \\
   0  &   0     & \EH{2t} &  3 \EH{5t}   & \\\hline
\end{array}$$
Daraus ergibt sich 
\begin{align*}
&&\vec c'(t)=& \begin{pmatrix}
-(2t+1)\EH{4t}\\
2\EH{4t}\\
3\EH{3t}\end{pmatrix}\\
\Rightarrow&&\vec c(t)= & \begin{pmatrix}
-\frac{4 t+ 1}8\EH{4t}\\
\frac 12 \EH{4t}\\
\EH{3t}\end{pmatrix}
\end{align*}
Die Integrationskonstanten wurden hier zu Null gew\"ahlt. \\
Damit ist 
$$\vec y_p(t)=\vec Y(t) \vec c(t)$$
eine L\"osung des inhomogenen Systems. Die allgemeine L\"osung des inhomogenen Systems ist dann 
$$\vec y_{allg}(t)=\vec y_p(t) + \vec y_h(t) = \vec Y(t)(\vec c(t) + \vec c). $$
Der konstante Vektor $\vec c$ wird durch die Anfangsbedingungen festgelegt: 
\begin{align*}
&&\begin{pmatrix}0\\1\\2\end{pmatrix}\overset !=& \vec y(0)=\vec Y(0)(\vec c(0)+\vec c)\\
&&=& \begin{pmatrix}
1 & 0 & 0 \\
1 & 0 & 1 \\
0 & -1& 1 \end{pmatrix}
\begin{pmatrix}-1/8+c_1\\1/2+c_2\\1+c_3\end{pmatrix} \\
\Rightarrow&& c_1=&\frac 18\\
&&c_3=& 0\\
&&c_2=& -\frac 32
\end{align*}
Damit ist dann 
\begin{align*}
\vec y_{AWP}(t)&=\vec Y(t)\begin{pmatrix}\frac 18((-4t-1)\EH{4t}+1)\\
\frac 12(\EH{4t}-3)\\
\EH{3t}\end{pmatrix}
= \begin{pmatrix}
\frac 18(-\EH{5t}+(1-12t)\EH{t})\\
\frac 18(7\EH{5t}+(1-12t)\EH{t})\\
\frac 12(\EH{5t}+3\EH t)\end{pmatrix} \\
&= \frac {\EH{5t}}8\begin{pmatrix}-1\\7\\4\end{pmatrix}+\frac{\EH t}8\begin{pmatrix}1-12t\\1-12t\\12\end{pmatrix}.
\end{align*}

\item Mit der Formel 
$$\vec y_{AWP}(t)=\vec Y(t)\vec Y(0)^{-1}\vec y_0 + \int\limits_0^t\vec Y(t-s)\vec Y(0)^{-1}\vec f(s)\d s$$ 
ergibt sich 
mit 
$$\vec Y(0)^{-1}\vec y_0=\begin{pmatrix}1 & 0 & 0 \\
1 & 0 & 1 \\
0 & -1& 1 \end{pmatrix}^{-1}\begin{pmatrix}0\\1\\2\end{pmatrix}
= \begin{pmatrix}0\\-1\\1\end{pmatrix}$$
sowie 
$$\vec Y(0)^{-1}\vec f(s) = \begin{pmatrix}1 & 0 & 0 \\
1 & 0 & 1 \\
0 & -1& 1 \end{pmatrix}^{-1}\begin{pmatrix}-1\\2\\1\end{pmatrix}\EH{5s}=
\begin{pmatrix}
-1\\2 \\3\end{pmatrix}\EH{5s}$$
die L\"osung des Anfangswertproblems: 

\begin{align*}
\vec y_{AWP}(t)=&\begin{pmatrix}
\EH t & \EH t t & 0 \\
\EH t & \EH t t & \EH{2t}\\
0     & -\EH t  & \EH{2t}\end{pmatrix}\begin{pmatrix}0\\-1\\1\end{pmatrix} + \int\limits_0^t 
\begin{pmatrix}
\EH {t-s} & \EH {t-s} (t-s) & 0 \\
\EH {t-s} & \EH {t-s} (t-s) & \EH{2(t-s)}\\
0     & -\EH {t-s}  & \EH{2(t-s)}\end{pmatrix}\begin{pmatrix}-1\\2\\3\end{pmatrix}\EH{5s}\d s\\
=& \begin{pmatrix}
-\EH t t \\
-\EH t t + \EH{2t}\\
\EH t + \EH{2t}\end{pmatrix}
+ \int\limits_0^t\begin{pmatrix}
\EH{t-s}(-1+2t-2s)\\
\EH{t-s}(-1+2t-2s)+3\EH{2(t-s)}\\
-2\EH{t-s}+ 3 \EH{2(t-s)}\end{pmatrix}\EH{5s}\d s\\
=& \begin{pmatrix}
-\EH t t \\
-\EH t t + \EH{2t}\\
\EH t + \EH{2t}\end{pmatrix}
+ \int\limits_0^t\begin{pmatrix}
\EH{t+4s}(-1+2t-2s)\\
\EH{t+4s}(-1+2t-2s)+3\EH{2t+3s}\\
-2\EH{t+4s}+ 3 \EH{2t+3s}\end{pmatrix}\EH{5s}\d s\\
=& \begin{pmatrix}
-\EH t t \\
-\EH t t + \EH{2t}\\
\EH t + \EH{2t}\end{pmatrix}
+ \left.\frac 14 \EH{t+4s}\begin{pmatrix}-1+2t-2s\\
-1+2t-2s\\
-2
\end{pmatrix}\right|_{s=0}^t 
- \frac 14\int\limits_0^t\EH{t+4s}\begin{pmatrix}
-2\\-2\\0\end{pmatrix}\d s \\
& +\left. \frac 13\EH{ 2t+3s}\begin{pmatrix}
0\\3\\3\end{pmatrix}\right|_{s=0}^t\\
=& \begin{pmatrix}
-\EH t t \\
-\EH t t + \EH{2t}\\
\EH t + \EH{2t}\end{pmatrix}
+ \frac 14 \EH{5t}\begin{pmatrix}-1\\
-1\\
-2
\end{pmatrix}
- \frac 14 \EH{t}\begin{pmatrix}-1+2t\\
-1+2t\\
-2
\end{pmatrix}
- \frac 1{16}(\EH{5t}-\EH t)\begin{pmatrix}
-2\\-2\\0\end{pmatrix} \\
& +\frac 13(\EH{ 5t}-\EH{2t})\begin{pmatrix}
0\\3\\3\end{pmatrix}\\
=& \EH t\begin{pmatrix}
-t +\frac 14 - \frac t2-\frac 18\\
-t +\frac 14-\frac t2-\frac 18\\
1+\frac 12
\end{pmatrix}
+\EH{5t}\begin{pmatrix}
-\frac 14+\frac 18\\
-\frac 14+\frac 18+1\\
-\frac 12+1
\end{pmatrix}\\
=& \frac{\EH t}8\begin{pmatrix}1-12t\\1-12t\\12\end{pmatrix}
+ \frac{\EH{5t}}8\begin{pmatrix}-1\\7\\4\end{pmatrix}
\end{align*}
\end{abc}

} 

\ErgebnisC{AufgsysdglSystIhom002}
{
\begin{abc}
\item $\vec Y(t)=\begin{pmatrix}
\EH t & \EH t t & 0  \\
\EH t & \EH t t & \EH{2t} \\
0     & -\EH t  & \EH{2t} \end{pmatrix}$
\item $\vec y_{AWP}(t)= \dfrac {\EH{5t}}8\begin{pmatrix}-1\\7\\4\end{pmatrix}+\dfrac{\EH t}8\begin{pmatrix}1-12t\\1-12t\\12\end{pmatrix}$
\end{abc}

}

