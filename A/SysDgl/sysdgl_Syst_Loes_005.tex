\Aufgabe[e]{Systeme linearer Differentialgleichungen}
{
Bestimmen Sie die allgemeine L\"osung des Systems von Differentialgleichungen
$$ \vec{y}\,^\prime(x) =  \begin{pmatrix}
  -1 & -1 &  3\\
  -3 &  5 &  1\\
 -10 & -2 & 10
\end{pmatrix}
 \vec{y}(x). $$
 
}

\Loesung{
Das charakteristische Polynom der Matrix $\vec A = \begin{pmatrix}
  -1 & -1 &  3\\
  -3 &  5 &  1\\
 -10 & -2 & 10
\end{pmatrix}$ ist
\begin{align*}
  p_{\vec A}(\lambda) & = \det \begin{pmatrix}
  -1 -\lambda& -1         &  3        \\
  -3         &  5-\lambda &  1        \\
 -10         & -2         & 10-\lambda
 \end{pmatrix}\\
=& (-1-\lambda)\bigl( (5-\lambda)(10-\lambda)+2\bigr)
-(-1)\bigl(-3 (10-\lambda)-1\cdot (-10)\bigr)
+  3\bigl( -3\cdot(-2)-(5-\lambda)\cdot(-10)\bigr)\\
=& (-1-\lambda)\bigl(\lambda^2-15\lambda+52\bigr)
+ \bigl( 3\lambda-20\bigr)
+ 3\bigl(-10\lambda+56\bigr)
= -\lambda^3+14\lambda^2-64\lambda+96
\end{align*}
Eine Nullstelle dieses Polynoms ist $\lambda_1=4$:
$$\begin{array}{r|r|r|r|r|r}
    &  -1       &  14  &  -64  &  96  \\\hline
x=4 & \setminus &  -4  &   40  & -96  \\\hline
    &  -1       &  10  &  -24  &   0
\end{array}$$
und das Restpolynom ist $q(\lambda)=-\lambda^2+10\lambda-24$.
Dessen Nullstellen sind
$$\lambda_{2/3}=5\pm\sqrt{25-24}=\left\{\begin{array}{r}4\\6\end{array}\right..$$

Damit hat die Matrix den doppelten Eigenwert $\lambda_1=\lambda_2=4$ und den einfachen Eigenwert $\lambda_3=6$.
Eigenvektoren zu $\lambda_1=4$ ergeben sich aus dem charakteristischen Gleichungssystem (rechte Seite Null)
$$\begin{array}{rrr|r|r|l}
  -1 -4      & -1   &  3        & 0 & 1 & \text{II}\\
  -3         &  5-4 &  1        & 0 & 1 & \text{I}-3\times \text{II}\\
 -10         & -2   & 10-4      & 0 & 2 & \text{III}-6\times\text{II}\\\hline
                                         
  -3         &  1   &  1        & 0 & 1 & \text{I}+\frac 14\times \text{II}\\
   4         & -4   &  0        & 0 &-2 &  \times 1/4               \\
   8         & -8   &  0        & 0 &-4 & \text{III}-2\times\text{II}\\\hline
                                         
  -2         &  0   &  1        & 0 &1/2& \\
   1         & -1   &  0        & 0 &-1/2& \\
   0         &  0   &  0        & 0 & 0 &
\end{array}$$
Der einzige linear unabh\"angige Eigenvektor ist
$$\vec v_1=\begin{pmatrix}1\\1\\2\end{pmatrix}.$$
Ein Hauptvektor ergibt sich aus dem obigen Gleichungssystem mit rechter Seite $\vec v_1$.
Es m\"ussen nur die Gauß-Schritte f\"ur die neue rechte Seite nachgeholt werden und ein m\"oglicher Hauptvektor ist 
$$\vec w_2=\begin{pmatrix}0\\1/2\\1/2\end{pmatrix}.$$

Ein Eigenvektor zu $\lambda_3=6$ ergibt sich aus dem charakteristischen Gleichungssystem $(\vec A-6\vec E_3)\vec v_3=\vec 0$
$$\begin{array}{rrr|r|l}
  -7  & -1  &  3  &  0  & \text{II}\\ 
  -3  & -1  &  1  &  0  & \text{I}-3\times \text{II}\\
 -10  & -2  &  4  &  0  & \text{III}-4\times \text{II}\\\hline

  -3  & -1  &  1  &  0  & \\ 
   2  &  2  &  0  &  0  & \\
   2  &  2  &  0  &  0  & \text{III}-        \text{II}\\\hline

  -3  & -1  &  1  &  0  & \\ 
   2  &  2  &  0  &  0  & \\
   0  &  0  &  0  &  0  & \\
\end{array}$$
zu $\vec v_3=\begin{pmatrix}1\\-1\\2\end{pmatrix}$.\\

Die allgemeine L\"osung des DGL.-Sytems ist schließlich 
\begin{align*}
   \vec{y}(x) & =  \EH{\lambda_1 x} \left\{ c_1 \vec{v}_1 
    + c_2 \big[ \vec{w}_2 + x \vec{v}_1 \big]\right\}
    + c_3 \EH{\lambda_3 x}\vec v_3\\
   & = \EH{4x} \left\{ c_1 \begin{pmatrix}1\\1\\2\end{pmatrix}
    + c_2 \left[\begin{pmatrix}0\\1/2\\1/2\end{pmatrix} + x \begin{pmatrix}1\\ 1\\ 2\end{pmatrix} \right]
\right\}+ c_3 \EH{6x} \begin{pmatrix}1\\-1\\2\end{pmatrix}\\
=& \begin{pmatrix}
\EH{4x} & x\EH{4x} & \EH{6x} \\
\EH{4x} & \frac{1+2x}2\EH{4x} &-\EH{6x} \\
2\EH{4x} & \frac{1+4x}2\EH{4x} &2\EH{6x}
\end{pmatrix}\begin{pmatrix}c_1\\c_2\\c_3\end{pmatrix}.
\end{align*}
mit Parametern $c_1,\,c_2,\, c_3\in\R$
und der Fundamentalmatrix
$$\vec Y(x)=  \begin{pmatrix}
\EH{4x} & x\EH{4x} & \EH{6x} \\
\EH{4x} & \frac{1+2x}2\EH{4x} &-\EH{6x} \\
2\EH{4x} & \frac{1+4x}2\EH{4x} &2\EH{6x}
\end{pmatrix}.$$


}


\ErgebnisC{AufgsysdglSystLoes005}
{

}

