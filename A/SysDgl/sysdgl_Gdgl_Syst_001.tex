\Aufgabe[e]{Differentialgleichungen und Systeme (Hauptvektoren)}
{
Gegeben sei die lineare Differentialgleichung
% \renewcommand{\theequation}{\arabic{Blatt}.\arabic{equation}}
\begin{equation}
\label{dgl}
	u'''(t) - 4\,u''(t)+4\,u'(t) = 9\,\EH{-t}\ .
\end{equation}
\begin{abc}
 \item \ Verwandeln Sie die Differentialgleichung \eqref{dgl} in ein System erster Ordnung
\[
\dot{{\vec x}}(t) = \vec A \vec x(t) + \vec f(t) \,.
\]
\item  Bestimmen Sie die allgemeine L\"osung des homogenen Systems \ $\dot{{\vec x}}(t) = \vec A \vec x(t) $ \  aus \mbox{Teil a)}\,.
\item  Bestimmen Sie die allgemeine L\"osung des Systems aus Teil a)\,. \\[1ex]
\textbf{Hinweis:} Eine spezielle L\"osung des Systems aus Teil a) kann mit dem Ansatz $\vec x_{\mathrm{p}}(t) = (\alpha,\beta,\gamma)^\top \cdot\EH{-t}$, mit $\alpha,\beta,\gamma\in \R$,  bestimmt werden.
\end{abc}

}

\Loesung{
\begin{abc}
\item Mit den Substitutionen \ $u' =: v$ \ und \ $u'' = v' =: w$ \ erhält man das System erster Ordnung:
\[
	{\left(\begin{array}{c} u(t) \\ v(t) \\ w(t) \end{array}\right)' = \left(\begin{array}{crr} 0 & 1 & 0 \\ 0 & 0 & 1 \\ 0 & -4 & 4 \end{array}\right)\left(\begin{array}{c} u(t) \\ v(t) \\ w(t) \end{array}\right) + \left(\begin{array}{c} 0 \\ 0 \\ 9 \end{array}\right)\cdot\EH{-t}\ .}
\]
Mit $\boxed{\vec x(t) := (u(t),v(t),w(t))^\top}$ und $\boxed{\vec f(t):= (0,0,9)^\top \EH{-t}}$ folgt
\[
\boxed{\dot{\vec x}(t) =  \vec A  \vec x(t) + \vec f(t)\,, \qquad \vec A := \left(\begin{array}{crr} 0 & 1 & 0 \\ 0 & 0 & 1 \\ 0 & -4 & 4 \end{array}\right)\,.}
\]

\bigskip
\item Die charakteristische Gleichung ist
\[
\det \left(\begin{array}{ccc} -\lambda & 1 & 0 \\ 0 & -\lambda & 1 \\ 0 & -4 & 4-\lambda \end{array}\right) = -\lambda\cdot\big(\lambda^2 -4\lambda+4\big) = 0 .
\]
Die Eigenwerte sind somit \ $\lambda_1=0 $ \ und \ $ \lambda_{2,3}=2$. Ein Eigenvektor f\"ur \ $\lambda_1=0$ \ ist \ $\vec v_1=\big(1\,,\,0\,,\,0\big)^{\top}$ . F\"ur \ $\lambda_{2,3}=2$ \ gibt es nur einen linear unab\"angigen Eigenvektor, z.B. \ $\vec v_2=\big(1\,,\,2\,,\,4\big)^{\top}$ . Es muss also noch ein zugeh\"origer Hauptvektor bestimmt werden:
\[
\left(\begin{array}{ccc} -2 & 1 & 0 \\ 0 & -2 & 1 \\ 0 & -4 & 2 \end{array}\right)	\left(\begin{array}{c} h_1 \\ h_2 \\ h_3 \end{array}\right) = \left(\begin{array}{c} 1 \\ 2 \\ 4 \end{array}\right) \,\Rightarrow\,  \vec h = \left(\begin{array}{c} 0 \\ 1 \\ 4 \end{array}\right)\ .
\]
Damit erh\"alt man die allgemeine L\"osung des homogenen Systems zu
\[
	\vec x_{\mathrm{hom}} (t) = c_1\left(\begin{array}{c} 1 \\ 0 \\ 0 \end{array}\right) + c_2\left(\begin{array}{c} 1 \\ 2 \\ 4 \end{array}\right)\cdot\EH{2\,t} + c_3\left[\left(\begin{array}{c} 0 \\ 1 \\ 4 \end{array}\right)+t\left(\begin{array}{c} 1 \\ 2 \\ 4 \end{array}\right)\right]\cdot\EH{2\,t}
\]


\bigskip
\item F\"ur das inhomogene System erh\"alt man mit dem Ansatz $\vec x_{\mathrm{p}}(t) = (\alpha,\beta,\gamma)^\top \cdot\EH{-t}$ nach K\"urzen durch den Exponentialterm das LGS
\[
	\left(\begin{array}{c} -\alpha \\ -\beta \\ -\gamma \end{array}\right) = \left(\begin{array}{crr} 0 & 1 & 0 \\ 0 & 0 & 1 \\ 0 & -4 & 4 \end{array}\right)\left(\begin{array}{c} \alpha \\ \beta \\ \gamma \end{array}\right)  + \left(\begin{array}{c} 0 \\ 0 \\ 9 \end{array}\right)
	\,\Rightarrow\, \left(\begin{array}{c} \alpha \\ \beta \\ \gamma \end{array}\right) = \left(\begin{array}{r} -1 \\ 1 \\ -1 \end{array}\right)\ .
\]
Eine partikul\"are L\"osung des inhomogenen Systems ist damit
\[
	\vec x_{\mathrm{p}}(t)  = \left(\begin{array}{r} -1 \\ 1 \\ -1 \end{array}\right)\cdot\EH{-t}\ .
\]

Die allgemeine L\"osung ist die Summe aus der homogenen und der partikul\"aren L\"osung:
  \[
\boxed{
\begin{array}{@{\;}rc@{\;}l}
 \vec x(t) & = & \displaystyle c_1\left(\begin{array}{c} 1 \\ 0 \\ 0 \end{array}\right) + c_2\left(\begin{array}{c} 1 \\ 2 \\ 4 \end{array}\right)\cdot\EH{2\,t} + c_3\left[\left(\begin{array}{c} 0 \\ 1 \\ 4 \end{array}\right)+t\left(\begin{array}{c} 1 \\ 2 \\ 4 \end{array}\right)\right]\cdot\EH{2\,t}\\[4ex]
&& \displaystyle + 	\left(\begin{array}{r} -1 \\ 1 \\ -1 \end{array}\right)\cdot\EH{-t}\ \ .
\end{array}}
\]
\end{abc}

}


\ErgebnisC{AufgsysdglGdglSyst001}
{
$	\vec x_{\mathrm{hom}} (t) = c_1\left(\begin{array}{c} 1 \\ 0 \\ 0 \end{array}\right) + c_2\left(\begin{array}{c} 1 \\ 2 \\ 4 \end{array}\right)\cdot\EH{2\,t} + c_3\begin{pmatrix} t\\ 1+2t \\ 4+4t \end{pmatrix}\cdot\EH{2\,t}$
}

