\Aufgabe[e]{L\"osungen eines LGS in Abh\"angigkeit von einem Parameter}{
Gegeben ist das inhomogene lineare Gleichungssystem
$$
\begin{array}{rcrcrcrcr}
x_1 & + & 2 x_2 & + &   3 x_3 & - &     x_4 & = & 1\,,\\
	&   &   x_2 &   &         & + &     x_4 & = & 0\,,\\
	&   &       &   &     x_3 & + & a\, x_4 & = & 2\,,\\
	&   &       &   & a\, x_3 & + &     x_4 & = & 2\,,\\
\end{array}
$$
mit $a\in \R$.

\begin{abc}
\item F\"ur welche Werte von $a \in \R$ hat das zugehörige \textbf{homogene}
lineare Gleichungssystem genau eine, keine oder mehrere L\"osungen?

\item F\"ur welche Werte von $a \in \R$ hat das gegebene \textbf{inhomogene}
lineare Gleichungssystem genau eine, keine oder mehrere L\"osungen?

\item Geben Sie die Lösungen
  $\boldsymbol x_1 \in \R^4$ des zugehörigen \textbf{homogenen} Systems und
  $\boldsymbol x_2 \in \R^4$ des gegebenen \textbf{inhomogenen} Systems
für $a=0$ an.

\item Geben Sie die Lösungsmengen
  $\mathbb{L}_1$ des zugehörigen \textbf{homogenen} Systems und
  $\mathbb{L}_2$ des gegebenen \textbf{inhomogenen} Systems
für $a=1$ an.

\end{abc}
}
\Loesung{
Wir bestimmen eine $a$-parameterabhängige Stufenform der gegebenen
linearen Glei\-chungssysteme (homogenes LGS und inhomogenes LGS).
Es ist ein Gau\ss{}-Schritt durch\-zu\-f\"uh\-ren.

$$
\begin{array}{rrrl|r|l|}
1 & 2 & 3 & -1 &  0 & 1\\[1ex]
0 & 1 & 0 &  1 &  0 & 0\\[1ex]
0 & 0 & 1 &  a &  0 & 2\\[1ex]
0 & 0 & a &  1 &  0 & 2\\[1ex]
\hline
\multicolumn{2}{c}{}\\
1 & 2 & 3 & -1 &  0 & 1\\[1ex]
0 & 1 & 0 &  1 &  0 & 0\\[1ex]
0 & 0 & 1 &  a &  0 & 2\\[1ex]
0 & 0 & 0 & 1-a^2 & 0 & 2-2a\\[1ex]
\hline
\end{array}
$$

\medskip
Das Gleichungssystem ist für alle rechten Seiten genau dann eindeutig lösbar,
wenn
\vskip-2ex
$$
1-a^2 \stackrel{!}{\neq} 0\quad \Leftrightarrow\quad
a \neq 1 \quad \text{und}\quad a \neq -1\,,
$$
anderenfalls hängt die Lösbarkeit von der Inhomogenität (rechten Seite) ab.


\bigskip
\textbf{a)} Das \textbf{homogene} Gleichungssystem
ist für $a=1$ oder $a=-1$ mehrdeutig lösbar,
da sich die letzte Zeile der Stufenform dann als Nullzeile ergibt, und
ist für alle $a \in \R \setminus \{ -1, 1 \}$ eindeutig lösbar,
da alle Diagonalelemente der Stufenform verschieden von Null sind.


\medskip
\textbf{b)} Das \textbf{inhomogene} Gleichungssystem
ist für $a = a_1 := 1$ mehrdeutig lösbar da $2 - 2 a_1 = 0 = 1 - a_1^2$,
ist für $a = a_2 := -1$ nicht lösbar da $2 - 2 a_2 = 4 \neq 0 = 1 - a_2^2$, und
ist für alle $a \in \R \setminus \{ -1, 1 \}$ eindeutig lösbar.

\smallskip
\textbf{c)} Für $a=0$\, ist $\boldsymbol x_1 = (0, 0, 0, 0)^\top$ und
  $\boldsymbol x_2 = (1, -2, 2, 2)^\top$\,.

\smallskip
\textbf{d)} Für $a=1$\, gilt
$$
\begin{array}{r@{\,\,}c@{\,\,}l}
\mathbb{L}_1 &=& \Big\{\quad \boldsymbol x \in \R^4 \quad \Big| \quad
\boldsymbol x = t\, (6, -1, -1, 1)^\top\,,\quad \forall t \in \R \quad \Big\}\,,\\[3ex]
% 
\quad \mathbb{L}_2 &=& \Big\{ \quad \boldsymbol x \in \R^4 \quad \Big| \quad
\boldsymbol x = (-5, 0, 2, 0)^\top + t\, (6, -1, -1, 1)^\top\,,\quad
\forall t \in \R \quad \Big\}\,.
\end{array}
$$
}

\ErgebnisC{linalg_Lgsy_Para_005}
{
Überprüfen Sie Ihre Lösungen jeweils durch eine Probe.
}
