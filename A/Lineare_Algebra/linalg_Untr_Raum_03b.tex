\Aufgabe[e]{Lineare Unterräume (fr\"uhere Klausuraufgabe)}{
Der Unterraum $\boldsymbol U \subset \R^3$ werde von den Vektoren
$$
\boldsymbol u_1 = \begin{pmatrix} 1 \\ -1 \\ 1 \end{pmatrix}\,,\quad
\boldsymbol u_2 = \begin{pmatrix} 2 \\ 1 \\ 0 \end{pmatrix}\,,\quad
\boldsymbol u_3 = \begin{pmatrix} 4 \\-1 \\ 2 \end{pmatrix}
$$
aufgespannt.

\begin{abc}
\item Zeigen Sie $\dim \boldsymbol U = 2$ und bestimmen Sie den Unterraum
  $\boldsymbol U^\perp$.

\item Es sei $\boldsymbol p := (1,0,1)^\top$.
  Geben Sie die Hessesche Normalform der Hyperebene
  $\boldsymbol H := \boldsymbol p + \boldsymbol U$ an und bestimmen Sie den
  Abstand der Ebene $\boldsymbol H$ vom Koordinatenursprung.
\end{abc}
}
\Loesung{
Im Folgenden bezeichnen wir mit der Abbildung $\langle \cdot, \cdot \rangle$ das
Standard-Skalarprodukt im $\R^3$ und mit der Abbildung $\| \cdot \|$ die durch
dieses Skalarprodukt induzierte Norm.

\begin{abc}
\item  Wir bestimmen zun\"achst $\boldsymbol U^\perp$.
% 
Dieser besteht aus allen Vektoren $\boldsymbol x\in \R^3$,
die orthogonal zu $\boldsymbol u_1$, $\boldsymbol u_2$ und $\boldsymbol u_3$ sind:
% 
$$
\langle \boldsymbol x, \boldsymbol u_1 \rangle = 0\,,\quad
\langle \boldsymbol x, \boldsymbol u_2 \rangle = 0\quad \text{und}\quad
\langle \boldsymbol x, \boldsymbol u_3 \rangle = 0\,.
$$ 

Auf dieses Gleichungssystem wird der Gau\ss{}-Algorithmus angewendet: 
$$
\begin{array}{r r r |r|l}
1 &-1 & 1 & 0 & \\
2 & 1 & 0 & 0 & -2\times \text{ 1. Zeile}\\
4 &-1 & 2 & 0 & -4\times \text{ 1. Zeile}\\\hline

1 &-1 & 1 & 0 & \\
0 & 3 &-2 & 0 &                          \\
0 & 3 &-2 & 0 & -        \text{ 2. Zeile}\\\hline


1 &-1 & 1 & 0 & \\
0 & 3 &-2 & 0 &                          \\
0 & 0 & 0 & 0 & \\
\end{array}\,.
$$

Die freie Wahl von $x_3=\lambda$ und R\"uckw\"artseinsetzen liefert
$$
\boldsymbol x = \lambda \begin{pmatrix} -1/3 \\ 2/3 \\ 1 \end{pmatrix}\,,\quad
\lambda \in \R\,.\quad
% 
\text{Folglich ist}\quad
% 
\boldsymbol U^\perp = \operatorname{span}\left\{
\begin{pmatrix} -1 \\ 2 \\ 3 \end{pmatrix} \right\}\,,
$$
und $\dim \boldsymbol U^\perp = 1$.

\medskip
Mit dem Dimensionssatz f\"ur Orthogonalkomplemente gilt
$$
3
= \dim \R^3
= \dim \boldsymbol U + \dim \boldsymbol U^\perp
= \dim \boldsymbol U + 1\,,
$$
und somit folgt $\dim \boldsymbol U = 2$.

\bigskip
\textit{Alternativ} lesen wir $\dim \boldsymbol U = 2$ an der Stufenform des
Gleichungssystems ab.

% % Original-Lösungsvorschlag (Teil a) mit Kreuzprodukt:
% % 
% Wir testen das angegebene Vektorsystem auf lineare Unabh"angigkeit:
% \[
% \left(\begin{array}{@{}c@{}}
% \boldsymbol u_1^{\top}\\
% \boldsymbol u_2^{\top}\\
% \boldsymbol u_3^{\top}\end{array}\right)
% % 
% \quad\Rightarrow \quad
% % 
% \begin{array}{rrr}
% 1 & 1 & 1\\
% 1 & 2 & 2\\
% 5 & 1 & 1
% \end{array}
% % 
% \quad\Rightarrow\quad
% % 
% \begin{array}{r@{\;}r@{\;}r}
% 1 & 1 & 1\\
% 0 & 1 &1\\
% 0 & -4 & -4
% \end{array}
% % 
% \quad\Rightarrow\quad
% % 
% \begin{array}{rrrr}
% 1 & 1 & 1 & =: \boldsymbol v_1^{\top}\\
% 0 & 1 & 1 & =: \boldsymbol v_2^{\top}\\
% 0 & 0 & 0
% \end{array}\,.
% \]
% Es folgt hieraus
% \[
% \boldsymbol U = \operatorname{span}\, \{\boldsymbol v_1, \boldsymbol v_2 \}
% =  \operatorname{span}\, \left\{
% \left(\begin{array}{@{}c@{}} 1\\ 1\\ 1\end{array}\right)\,,\,\,
% \left(\begin{array}{@{}c@{}} 0\\ 1\\ 1\end{array}\right)
% \right\}\,,\quad
% % 
% \dim \boldsymbol U = 2\,.
% \]
% Eine auf $\boldsymbol U$ senkrechte Richtung $\boldsymbol{n}$ erh"alt man aus
% dem Kreuzprodukt
% \[
% \boldsymbol{n}
% = \boldsymbol v_1 \times \boldsymbol v_2
% % =\left|\begin{array}{ccc}
% % \boldsymbol e_1&1&0\\ \boldsymbol e_2&1&1\\\boldsymbol e_3&1&1\end{array}\right|
% = \left(\begin{array}{@{\,}r} 0\\ -1\\ 1\end{array}\right)
% % 
% \quad\Rightarrow\quad
% % 
% \boldsymbol U^{\perp}
% = \operatorname{span}\, \{ \boldsymbol{n} \}
% = \operatorname{span}\, \left\{
% \left(\begin{array}{@{\,}r} 0\\ -1\\ 1\end{array}\right)
% \right\}\,.
% \]

\item Als Normalenvektor der Hyperebene muss ein Basisvektor von $\vec U^\perp$ gew\"ahlt werden: 
$$\vec n=\begin{pmatrix}-1\\2\\3\end{pmatrix}.$$
Normieren f\"uhrt auf
$$\vec n_0=\frac 1{\sqrt{14}}\begin{pmatrix} -1\\2\\3\end{pmatrix}.$$
Damit und mit $\vec p=(1,0,1)^\top$ ist die Hessesche Normalform von $\vec H$ 
$$\vec H=\{\vec x\in\R^3|\, \skalar{\vec x-\vec p,\vec n_0}=0\}.$$
Da die Hessche Normalform einen normierten Normalenvektor $\vec n_0$ enth\"alt, ergibt sich der
Abstand der Hyperebene zum Koordinatenursprung $\vec 0$ aus dem Skalarprodukt
$$d(\vec H,\vec 0)=\left|\skalar{\vec 0-\vec p,\vec
n_0}\right|=\left|\skalar{\begin{pmatrix}-1\\0\\-1\end{pmatrix},\frac 1{\sqrt{14}
}\begin{pmatrix}-1\\2\\3\end{pmatrix}}\right|=\frac 2{\sqrt{14}}.
$$
\end{abc}
}

\ErgebnisC{linalgUntrRaum03b}
{
{\textbf{b)} } $d=2/\sqrt{14}$
}
