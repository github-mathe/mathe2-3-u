\Aufgabe[e]{Dimensionssatz für Unterräume}{
Best\"atigen Sie den Dimensionssatz für Unterräume für

\hspace*{.05\linewidth} $\boldsymbol U_1 = \operatorname{span}\{\,
  (1, 0, 0)^\top\,,\,\,
  (0, 1, 0)^\top
\,\} \subset \boldsymbol V$ und

\hspace*{.05\linewidth} $\boldsymbol U_2 = \operatorname{span}\{\,
  (2, 2, 0)^\top\,,\,\,
  (0, 0, 3)^\top
\,\} \subset \boldsymbol V$
mit $\boldsymbol V = \R^3$\,.
}
\Loesung{
Die Unterräume $\boldsymbol U_1 \subset \R^3$ und $\boldsymbol U_2 \subset \R^3$
enthalten jeweils $2$ linear unabhängige Vektoren im $\R^3$
(Prüfen Sie dieses nach!). Somit gilt
$$
\dim(\boldsymbol U_1) = 2\quad \text{und}\quad
\dim(\boldsymbol U_2) = 2\,.
$$

Der Untervektorraum $\boldsymbol U_1 + \boldsymbol U_2$ enthält
$3$ linear unabhängige Vektoren im $\R^3$ (Prüfen Sie dieses nach!).
Somit gilt
$$
\dim(\boldsymbol U_1 + \boldsymbol U_2) = 3\,.
$$

Für die Bestimmung von $\dim(\boldsymbol U_1 \cap \boldsymbol U_2)$
müssen alle linear unabhängigen Vektoren des verbleibenden Unterraums gefunden
werden.
% 
Das bedeutet, für $\boldsymbol x \in \boldsymbol U_1 \cap \boldsymbol U_2$
gilt \underline{$\boldsymbol x \in \boldsymbol U_1$ und
$\boldsymbol x \in \boldsymbol U_2$}.
% 
Folglich lässt sich dieses $\boldsymbol x$ hier darstellen als
$$
\begin{array}{r@{\,\,}c@{\,\,}l}
\boldsymbol x &=& \lambda_{11}\, (1, 0, 0)^\top + \lambda_{12}\, (0, 1, 0)^\top\,,\quad
  \lambda_{11}, \lambda_{12}\in \R\,, \quad \text{und}\\[1ex]
% 
\boldsymbol x &=& \lambda_{21}\, (2, 2, 0)^\top + \lambda_{22}\, (0, 0, 3)^\top\,,\quad
  \lambda_{21}, \lambda_{22}\in \R\,.
\end{array}
$$

Also ist hier die vektorwertige Gleichung
$$
  \lambda_{11}\, (1, 0, 0)^\top
+ \lambda_{12}\, (0, 1, 0)^\top
- \lambda_{21}\, (2, 2, 0)^\top
- \lambda_{22}\, (0, 0, 3)^\top \stackrel{!}{=} \boldsymbol 0 \,,\quad
$$
% 
für die Koeffizienten
$\lambda_{11}, \lambda_{12}, \lambda_{21}, \lambda_{22} \in \R$
zu lösen\,.
% 
Eine andere Darstellung des zugehörigen linearen Gleichungssystems ergibt hier
$$
\begin{array}{rrrrr}
% \lambda_{11} & \lambda_{12} & \lambda_{21} & \lambda_{22}\\[1ex]
% \hline
1 & 0 & -2 &  0  & 0\\[1ex]
~ & 1 & -2 &  0  & 0\\[1ex]
~ & ~ &  ~ & -3  & 0\\[1ex]
\hline
\end{array}\,.
$$
% eine Stufenform ohne Anwendung des Gau\ss{}-Algorithmus.

\medskip
An der Stufenform des Gleichungssystems kann man ablesen, dass die Lösung
mehrdeutig ist (d.h. es liegt im Unterraum $\boldsymbol U_1 \cap \boldsymbol U_2$
mindestens ein linear unabhängiger Vektor).
% 
Die Lösung des Gleichungssystems ergibt
$$
\lambda_{22} = 0\,,\,\,
\lambda_{21} =: t \in \R\,,\,\,
\lambda_{12} = 2\, t\,,\,\,
\lambda_{11} = 2\, t\,.
$$

Das Einsetzen dieser $\lambda_{11}$, $\lambda_{12}$, $\lambda_{21}$,
$\lambda_{22}$ in die beiden Gleichungen der Darstellungen des Vektors
$\boldsymbol x \in \boldsymbol U_1 \cap \boldsymbol U_2$
ergibt hier
$$
\begin{array}{r@{\,\,}c@{\,\,}l@{\,\,}c@{\,\,}l}
\boldsymbol x
&=& 2\, t\, (1, 0, 0)^\top + 2\, t\, (0, 1, 0)^\top
&=& t\, (2, 2, 0)^\top
\quad \text{und}\\[1ex]
% 
\boldsymbol x
&=& t\, (2, 2, 0)^\top + 0 \cdot (0, 0, 3)^\top
&=& t\, (2, 2, 0)^\top
\end{array}
$$
für (weiterhin) beliebige $t \in \R$\,.
% 
Aus diesem folgt
$$
\boldsymbol U_1 \cap \boldsymbol U_2 =
\operatorname{span}\left\{\, (1, 1, 0)^\top \,\right\}
$$
und folglich
$$
\dim(\boldsymbol U_1 \cap \boldsymbol U_2) = 1\,,
$$
da $\boldsymbol U_1 \cap \boldsymbol U_2$
genau einen linear unabhängigen Vektor im $\R^3$ enthält.

\bigskip
Somit sind die Aussagen des Dimensionssatzes für Unterräume hier, mit
$$
\begin{array}{r@{\,\,}c@{\,\,}l}
\dim(\boldsymbol U_1 + \boldsymbol U_2)
&=& \dim(\boldsymbol U_1) + \dim(\boldsymbol U_2)
 - \dim(\boldsymbol U_1 \cap \boldsymbol U_2)\,,\\[1ex]
% 
(3 &=& 2 + 2 - 1\,,)
\end{array}
$$
% 
sowie
$$
\begin{array}{r@{\,\,}c@{\,\,}l}
\dim(\boldsymbol U_1) &=& 2 \le n = 3 < \infty\,,\\[1ex]
\dim(\boldsymbol U_2) &=& 2 \le n = 3 < \infty\,,\\[1ex]
\dim(\boldsymbol U_1 + \boldsymbol U_2) &=& 3 \le n = 3 < \infty\,,\\[1ex]
\dim(\boldsymbol U_1 \cap \boldsymbol U_2) &=& 1 \le n = 3 < \infty\,,
\end{array}
$$
mit $n=3$, da $\boldsymbol V = \R^3$,
und $\boldsymbol V$ ist ein endlichdimensionaler $\R$-Vektorraum,
gezeigt.
}

\ErgebnisC{linalg_Untr_RaumDim_001}
{
Zur Information: $\dim( \boldsymbol U_1 \cap \boldsymbol U_2 ) = 1$\,.
}
