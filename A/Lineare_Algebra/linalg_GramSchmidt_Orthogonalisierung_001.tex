\Aufgabe[e]{Schmidt'sches Orthonormalisierungsverfahren}{
Gegeben seien die Vektoren
$$
\boldsymbol u_1 = \begin{pmatrix} 2 \\ 0 \\ 2 \end{pmatrix}\,,\quad
\boldsymbol u_2 = \begin{pmatrix} 1 \\ 2 \\ 1 \end{pmatrix}\,,\quad
\boldsymbol u_3 = \begin{pmatrix} 0 \\ 3 \\ 4 \end{pmatrix}\,.
$$

Bestimmen Sie mit Hilfe des Schmidt'schen Orthonormalisierungsverfahren eine
Orthonormalbasis des $\R^3$\,.

\smallskip
\textbf{Hinweis:} Wählen Sie $\boldsymbol u_1$ als ersten Vektor
zur Bestimmung einer Orthonormalbasis mithilfe des Schmidt'schen
Orthonormalisierungsverfahren.
}
\Loesung{
Im Folgenden bezeichnen wir mit der Abbildung $\langle \cdot, \cdot \rangle$ das
Standard-Skalarprodukt im $\R^3$ und mit der Abbildung $\| \cdot \|$ die durch dieses Skalarprodukt induzierte Norm.

\bigskip
Die Normierung des Vektors \ $\boldsymbol u_1$ \ ergibt den ersten Basisvektor:
\begin{equation*}
\boldsymbol w_1
= \dfrac{\boldsymbol u_1}{\|\boldsymbol u_1\|}
= \dfrac{1}{\sqrt{2}} \begin{pmatrix} 1 \\ 0 \\ 1\end{pmatrix}\,.
\end{equation*}

Die Orthogonalisierung von \ $\boldsymbol u_2$ \ und anschließende Normierung
ergeben
\begin{equation*}
\boldsymbol v_2
= \boldsymbol u_2
- \langle \boldsymbol u_2, \boldsymbol w_1\rangle\, \boldsymbol w_1
% 
= \begin{pmatrix}1 \\ 2 \\ 1\end{pmatrix}
- \dfrac{2}{\sqrt{2}}\, \dfrac{1}{\sqrt{2}} \begin{pmatrix} 1\\ 0\\ 1\end{pmatrix}
% 
= \begin{pmatrix}0 \\ 2 \\ 0\end{pmatrix}
% 
\quad \Rightarrow \quad
% 
\boldsymbol w_2 = \dfrac{\boldsymbol v_2}{\|\boldsymbol v_2\|}
= \begin{pmatrix}0 \\ 1\\ 0\end{pmatrix}\,.
\end{equation*}

Die Orthogonalisierung von $\boldsymbol u_3$ ergibt
\begin{equation*}
\boldsymbol v_3
= \boldsymbol u_3
- \langle \boldsymbol u_3, \boldsymbol w_1\rangle\, \boldsymbol w_1
- \langle \boldsymbol u_3, \boldsymbol w_2\rangle\, \boldsymbol w_2
% 
= \begin{pmatrix}0 \\ 3 \\ 4\end{pmatrix}
- \dfrac{4}{\sqrt{2}}\, \dfrac{1}{\sqrt{2}} \begin{pmatrix}1 \\ 0 \\ 1\end{pmatrix}
- 3\, \begin{pmatrix}0 \\ 1 \\ 0 \end{pmatrix}
% 
= \begin{pmatrix}-2 \\ 0 \\ 2\end{pmatrix}\,.
\end{equation*}
% 
Die anschließende Normierung ergibt den dritten Basisvektor.
\begin{equation*}
\boldsymbol w_3
= \dfrac{\boldsymbol v_3}{\|\boldsymbol v_3\|}
= \dfrac{1}{\sqrt{2}} \begin{pmatrix} -1 \\ 0 \\ 1\end{pmatrix}\,.
\end{equation*}

Die Vektoren $ \boldsymbol w_1,\, \boldsymbol w_2,\, \boldsymbol w_3$ bilden dann eine Orthonormalbasis des $\R^3$\,.
}

\ErgebnisC{linalgGramSchmidtOrthogonalisierung001}
{
$\vec w_3= \frac 1{\sqrt 2} (-1,0,1)^\top$
}
