\Aufgabe[e]{Basiswechsel}{
Gegeben seien die Vektoren
$$
\boldsymbol v_1 =
\left(\begin{array}{r}
1\\
1\\
1
\end{array}\right)\,,\quad
%
\boldsymbol v_2 =
\left(\begin{array}{r}
0\\
1\\
3
\end{array}\right)\,,\quad
%
\boldsymbol v_3 =
\left(\begin{array}{r}
1\\
2\\
8
\end{array}\right)\quad \text{und}\quad
%
\boldsymbol v_4 =
\left(\begin{array}{r}
-1\\
 1\\
 2
\end{array}\right)
$$
bezüglich der Standardbasis von $\mathbb{R}^3$.
%
Bestimmen Sie die Koordinatendarstellung von jedem Vektor
bezüglich der Basis
$$
S = \left\{
\left(\begin{array}{r}
1\\
1\\
1
\end{array}\right),
%
\left(\begin{array}{r}
0\\
1\\
3
\end{array}\right),
%
\left(\begin{array}{r}
0\\
0\\
4
\end{array}\right)
\right\}\,.
$$
}


\Loesung{
Lösen des entsprechenden linearen Gleichungssystems, um die Koordinatendarstellung
in der gegebenen Basis zu bestimmen
$$
\begin{array}{lccl}
& & 1 \lambda_1 &= 1\,,\\
& 1 \lambda_1 &+ 1 \lambda_2 &= 1\,,\\
1 \lambda_1 &+ 3 \lambda_2 &+ 4 \lambda_3 &= 1\,,
\end{array}
$$
ergibt
$$
\boldsymbol v_1^S =
\left(\begin{array}{r}
1\\
0\\
0
\end{array}
\right)\,.
$$


$$
\begin{array}{lccl}
& &1 \lambda_1 &= 0\,,\\
&1 \lambda_1 &+ 1 \lambda_2 &= 1\,,\\
1 \lambda_1 &+ 3 \lambda_2 &+ 4 \lambda_3 &= 3\,,
\end{array}
$$
ergibt
$$
\boldsymbol v_2^S =
\left(\begin{array}{r}
0\\
1\\
0
\end{array}
\right)\,.
$$


$$
\begin{array}{lccl}
& &  1 \lambda_1 &= 1\,,\\
& 1 \lambda_1 &+ 1 \lambda_2 &= 2\,,\\
1 \lambda_1 &+ 3 \lambda_2 &+ 4 \lambda_3 &= 8\,,
\end{array}
$$
ergibt
$$
\boldsymbol v_3^S =
\left(\begin{array}{r}
1\\
1\\
1
\end{array}
\right)\,.
$$

\newpage
$$
\begin{array}{lccr}
& & 1 \lambda_1 &= -1\,,\\
& 1 \lambda_1 &+ 1 \lambda_2 &= 1\,,\\
1 \lambda_1 &+ 3 \lambda_2 &+ 4 \lambda_3 &= 2\,,
\end{array}
$$
ergibt
$$
\boldsymbol v_4^S =
\left(\begin{array}{r}
-1\\
 2\\
-3 / 4
\end{array}
\right)\,.
$$

}
