\Aufgabe[e]{Methode der kleinsten Quadrate}{
Gegeben seien die  Messwerte
\begin{center}
 \begin{tabular}{c|c|c|c} 
  $i$ & 1 & 2 & 3 \\\hline
  $t_i$ & -3 & 1 & 2   \\ \hline
  $y_i$ &  33 & -3 & -22
 \end{tabular}
\end{center}
\begin{abc}
\item Es soll ein Polynom $p(x)=a_1 + a_2 x + a_3 x^2$ bestimmt werden, welches diese Messwerte
  interpoliert: 
$$p(t_j)=y_j,\qquad j=1,2,3.$$
\begin{iii}
\item Geben Sie das lineare Gleichungssystem f\"ur den Koeffizientenvektor \\
\mbox{$\vec a=(a_1,a_2,a_3)^\top$} an. 
\item Ermitteln Sie die L\"osung des Gleichungssystems. 
\end{iii}
\item Nun soll dasselbe Interpolationsproblem f\"ur ein lineares Polynom\\
$q(x)=b_1+b_2x$ gel\"ost werden: 
$$q(t_j)=y_j,\qquad j=1,2,3.$$ 
\begin{iii}
\item Ist dieses Problem l\"osbar? (Begr\"unden Sie Ihre Antwort.)
\item Geben Sie ein Gleichungssystem f\"ur die Koeffizienten $b_1,\, b_2$ an, deren Polynom $q(x)$
das geforderte Problem bestm\"oglich l\"ost, es soll also gelten: 
$$\norm{(q(t_1)-y_1,q(t_2)-y_2,q(t_3)-y_3)^\top}_2^2=\text{min!}.$$
\textbf{Hinweis}: Die L\"osung des Gleichungssystems ist \textbf{ nicht} zu bestimmen. 
%\item Ermitteln Sie die L\"osung $\vec b=(b_1,b_2,b_3)^\top$ dieses Systems. 
\end{iii}
\end{abc}
}

\Loesung{
\begin{abc}
\item \begin{iii}
\item Aus $p(t_j)=y_j, \quad j=1,2,3$ folgen die Gleichungen: 
$$\begin{array}{rrrrrrrrr}
 a_1 & - &   3    & a_2 & + &    9    & a_3 &  = & 33  \\
 a_1 & + &   1    & a_2 & + &    1    & a_3 &  = & -3   \\
 a_1 & + &   2    & a_2 & + &    4    & a_3 &  = & -22 
\end{array}$$
\item Mit Hilfe des Gauß-Algorithmus ergibt sich 
$$\begin{array}{rrr|r|l}
1  & - 3    &  9    &  33   & \text{                        }\\
1  &   1    &  1    &  -3   & -\text{            1. Zeile}\\
1  &   2    &  4    &  -22  & -\text{ 1. Zeile}\\\hline
                        
1  & - 3    &  9    &  33   & \text{                        }\\
0  &   4    &  -8   &  -36  & \times 1/4                       \\
0  &   5    & -5    &  -55  &      \times 1/5             \\\hline
                        
1  & - 3    &  9    &  33   & \text{                        }\\
0  &   1    &  -2   &  -9   &                                  \\
0  &   1    & -1    &  -11  & -\text{ 2. Zeile}\\\hline                           

1  & - 3    &  9    &  33   & \text{                        }\\
0  &   1    &  -2   &  -9   &                                  \\
0  &   0    &  1    &  -2   &          
\end{array}$$
und daraus die L\"osung 
$$a_3=-2\, \Rightarrow\, a_2=-13    \,\Rightarrow \,  a_1=12.$$
\end{iii}
\item \begin{iii}
\item Hier ergeben sich die drei Gleichungen 
$$\begin{array}{rrrrrrrrr}
 b_1 & - &   3    & b_2 &  = & 33  \\
 b_1 & + &   1    & b_2 &  = & -3   \\
 b_1 & + &   2    & b_2 &  = & -22 
\end{array}$$
und daraus das Gauß-Tableau mit der Koeffizientenmatrix 
$$\vec B=\begin{pmatrix}
1   &  -3  \\
1   & 1    \\
1   & 2    
\end{pmatrix}$$
$$\begin{array}{rr|r|l}
1   &  -3   &  33 & \\
1   & 1     &  -3 & \text{-1. Zeile}\\
1   & 2     & -22 & \text{-1. Zeile}\\\hline

1   &  -3   &  33 & \\
0   & 4     & -36 &                 \\
0   & 5     & -55 & 4\times \text{ 3. Zeile }-5\times\text{ 2. Zeile}\\\hline

1   &  -3   &  33 & \\
0   & 4     & -36 &                 \\
0   & 0     & -40 & 
\end{array}$$
Aus der letzten Zeile folgt der Widerspruch $0=-40$, also ist das System nicht l\"osbar. 
\item Dieses Minimierungsproblem wird von der L\"osung der Normalgleichung 
$$\vec B^\top \vec B \vec b = \vec B^\top \begin{pmatrix}33\\-3\\-22\end{pmatrix}\,\Leftrightarrow\, \begin{pmatrix} 3 & 0 \\ 0 & 14\end{pmatrix}\vec b = \begin{pmatrix}8\\-146\end{pmatrix}$$
gel\"ost. 
\end{iii}


\end{abc}


}

