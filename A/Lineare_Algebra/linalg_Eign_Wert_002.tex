\Aufgabe[e]{Eigenwerte und Eigenvektoren}{
\begin{abc}
\item Berechnen Sie die Eigenwerte und Eigenr\"aume der Matrizen\\
$$\boldsymbol A=\begin{pmatrix}1&-4&-4\\0&3&2\\2&-7&-4\end{pmatrix}, \,
\boldsymbol B=\begin{pmatrix}1&-4&-4\\0&3&2\\0&-1&0\end{pmatrix}.$$%\text{ und }$$
% $$ \boldsymbol C = \begin{pmatrix}
% -1+2\operatorname{i}  & -2+2\operatorname{i}  & -1+\operatorname{i} \\
% 1-2\operatorname{i}   & 2 -2\operatorname{i}  & 1-\operatorname{i}  \\
%   2\operatorname{i}   &    2\operatorname{i}  &   \operatorname{i}  \\
% \end{pmatrix}.$$
% \item Invertieren Sie -- falls m\"oglich -- die drei Matrizen. 
% \item Sind die drei Matrizen invertierbar? Begr\"unden Sie Ihre Antwort.
\item Die Matrix $\boldsymbol M\in \R^{n\times n}$ habe den Eigenwert $\lambda$. \\
Welchen Eigenwert hat dann $\boldsymbol M + \boldsymbol E_{n}$?
\end{abc}
}
\Loesung{
\begin{abc}
\item \begin{iii} 
\item Die Eigenwerte von $\boldsymbol A$ ergeben sich zu
\begin{align*}
0&\overset != \det \begin{pmatrix}1-\lambda &-4&-4\\0&3-\lambda
  &2\\2&-7&-4-\lambda\end{pmatrix}\\
&=(1-\lambda)(3-\lambda)(-4-\lambda)-16+8(3-\lambda)+14(1-\lambda)\\
&=-\lambda^3-9\lambda+10\\
\Rightarrow\lambda_1&=1, \quad \lambda_2=-\frac 12 +\operatorname{i}\frac{\sqrt{39}}2,
\quad \lambda_3 = -\frac12 - \operatorname{i}\frac{\sqrt{39}}2
\end{align*}
Eigenvektoren ergeben sich wie \"ublich aus
\begin{align*}
(&\boldsymbol A-\lambda_j \boldsymbol E)\boldsymbol v_j\overset != 0\\ &\Rightarrow\boldsymbol v_1=\begin{pmatrix}1\\1\\-1\end{pmatrix},\,
\boldsymbol v_2=\begin{pmatrix}-8 \\4\\-7+\operatorname{i}\sqrt{39}\end{pmatrix},\,
\boldsymbol v_3=\begin{pmatrix}-8 \\4\\-7-\operatorname{i}\sqrt{39}\end{pmatrix}
\end{align*}
Diese drei Eigenvektoren spannen die zu den jeweiligen Eigenwerten
geh\"orenden Eigenr\"aume auf. 

\item Die Eigenwerte von $\boldsymbol B$ ergeben sich zu 
\begin{align*}
\quad 0&\overset != \det\begin{pmatrix}1-\lambda& -4&-4\\0&3-\lambda&2\\0&-1&-\lambda\end{pmatrix}\\
&=(1-\lambda)(-3\lambda+\lambda^2+2)\\[2ex]
\Rightarrow \quad \lambda_1&=\lambda_2=1, \quad \lambda_3=2,
\end{align*}
und die Eigenvektoren: 
\begin{align*}
&\begin{pmatrix}0&-4&-4\\0&2&2\\0&-1&-1\end{pmatrix}\boldsymbol v_{1/2}=0
\quad \Leftarrow\boldsymbol v_1 = \begin{pmatrix}0\\1\\-1\end{pmatrix}, \,  \boldsymbol v_2=\begin{pmatrix}1\\1\\-1\end{pmatrix}\\
&\begin{pmatrix}-1&-4&-4\\0&1&2\\0&-1&-2\end{pmatrix}\boldsymbol v_3=0
\quad \Leftarrow\boldsymbol v_3=\begin{pmatrix}4\\-2\\1\end{pmatrix}
\end{align*}
Die Eigenr\"aume zu den beiden Eigenvektoren sind damit
$$\boldsymbol U_{\lambda=1} = \operatorname{span}\{\boldsymbol v_1,\,  \boldsymbol v_2\}, \quad \boldsymbol U_{\lambda=2} = \operatorname{span}\{\boldsymbol v_3\}.$$
% 
% \item Die Eigenwerte von $\boldsymbol C$ ergeben sich als Nullstellen des charakteristischen Polynoms:
% \begin{align*}
% 0\overset !=&\det\begin{pmatrix}
%      -1+2\operatorname{i}-\lambda &       -2+2\operatorname{i}   & -1+\operatorname{i}   \\
%      1-2\operatorname{i}          & 2-2\operatorname{i} - \lambda&  1-\operatorname{i}   \\
%      2\operatorname{i}            &      2\operatorname{i}       &   \operatorname{i}-\lambda
% \end{pmatrix}\quad (Z_2+Z_1)\\
% =&\det\begin{pmatrix}
%      -1+2\operatorname{i}-\lambda &       -2+2\operatorname{i}   & -1+\operatorname{i}   \\
%      -\lambda          &          - \lambda&        0   \\
%      2\operatorname{i}            &      2\operatorname{i}       &   \operatorname{i}-\lambda
% \end{pmatrix}\quad \quad (S_1-S_2)\\
% =&\det\begin{pmatrix}
%       1       -\lambda &       -2+2\operatorname{i}   & -1+\operatorname{i}   \\
%             0          &          - \lambda&        0   \\
%             0          &      2\operatorname{i}       &   \operatorname{i}-\lambda
% \end{pmatrix}\\
% =&(1-\lambda)\cdot (-\lambda)(\operatorname{i}-\lambda)
% \end{align*}
% Also sind die Eigenwerte: 
% $$\lambda\in \{1,\, 0,\, \operatorname{i}\}.$$
% Die jeweiligen Eigenvektoren ergeben sich als L\"osungen der Gleichungssysteme:
% $$(\boldsymbol C - \lambda\cdot \boldsymbol E)\boldsymbol v_\lambda=\boldsymbol 0, \quad \lambda\in \{1,\, 0,\, \operatorname{i}\}.$$
% F\"ur $\lambda = 1$: 
% $$\begin{array}{rrr|r|l}
% -2+2\operatorname{i}     &-2+2\operatorname{i}     &-1+\operatorname{i}       & 0 & \text{                              }\\
% 1-2\operatorname{i}      & 1-2\operatorname{i}     & 1-\operatorname{i}       & 0 & \text{ +1. Zeile                    }\\
% 2\operatorname{i}        &   2\operatorname{i}     & \operatorname{i}-1       & 0 & \text{ -1. Zeile                    }\\\hline
% 
% -2+2\operatorname{i}     &-2+2\operatorname{i}     &-1+\operatorname{i}       & 0 & \text{                              }\\
%  -1           &-1            &        0      & 0 & \text{                              }\\
% 2             &   2          &        0      & 0 & \text{  +2$\cdot$ 2. Zeile          }\\\hline
% 
% -2+2\operatorname{i}     &-2+2\operatorname{i}     &-1+\operatorname{i}       & 0 & \text{                              }\\
%  -1           &-1            &        0      & 0 & \text{                              }\\
% 0             &   0          &        0      & 0 &
% \end{array}$$
% $$\Rightarrow \quad  \boldsymbol v_1 = \alpha (1,\, -1,\, 0)^\top, \quad \alpha\in\R.$$
% 
% F\"ur $\lambda = 0$: 
% $$\begin{array}{rrr|r|l}
% -1+2\operatorname{i}     &-2+2\operatorname{i}     &-1+\operatorname{i}       & 0 & \text{                              }\\
% 1-2\operatorname{i}      & 2-2\operatorname{i}     & 1-\operatorname{i}       & 0 & \text{ +1. Zeile                    }\\
% 2\operatorname{i}        &   2\operatorname{i}     & \operatorname{i}         & 0 & \text{ $\cdot (-1+2\operatorname{i})-2\operatorname{i}\cdot$ 1. Zeile }\\\hline
% 
% -2+2\operatorname{i}     &-2+2\operatorname{i}     &-1+\operatorname{i}       & 0 & \text{                              }\\
%   0           & 0            &        0      & 0 & \text{                              }\\
%       0       & 2\operatorname{i}       & \operatorname{i}         & 0 & \text{                              }
% \end{array}$$
% $$\Rightarrow \quad  \boldsymbol v_0 = \alpha (0,\, -1,\, 2)^\top, \quad \alpha\in\R.$$
% 
% F\"ur $\lambda = \operatorname{i}$: 
% $$\begin{array}{rrr|r|l}
% -1+ \operatorname{i}     &-2+2\operatorname{i}     &-1+\operatorname{i}       & 0 & \text{                              }\\
% 1-2\operatorname{i}      & 2-3\operatorname{i}     & 1-\operatorname{i}       & 0 & \text{ +1. Zeile                    }\\
% 2\operatorname{i}        &   2\operatorname{i}     &        0      & 0 & \text{                              }\\\hline
% 
% -1+ \operatorname{i}     &-2+2\operatorname{i}     &-1+\operatorname{i}       & 0 & \text{                              }\\
%  - \operatorname{i}      &  - \operatorname{i}     &        0      & 0 & \text{                              }\\
% 2\operatorname{i}        &   2\operatorname{i}     &        0      & 0 & \text{  +2$\cdot$ 2. Zeile          }\\\hline
% 
% -1+ \operatorname{i}     &-2+2\operatorname{i}     &-1+\operatorname{i}       & 0 & \text{                              }\\
%  - \operatorname{i}      &  - \operatorname{i}     &        0      & 0 & \text{                              }\\
%       0       &        0     &        0      & 0 &
% \end{array}$$
% $$\Rightarrow \quad  \boldsymbol v_{\operatorname{i}} = \alpha (1,\, -1,\, 1)^\top, \quad \alpha\in\R.$$
\end{iii}
Die Eigenr\"aume sind damit 
$$\boldsymbol U_{1}=\operatorname{span}\{(1,\, -1,\, 0)^\top\},\, \boldsymbol U_0=\operatorname{span}\{(0,\, -1,\, 2)^\top\},\, \boldsymbol U_{\operatorname{i}}
= \operatorname{span}\{(1,\, -1,\, 1)^\top\}.$$
% \item Die ersten beiden Matrizen sind invertierbar:
% $$\begin{array}{rrr|rrr|l}
%      1 &    -4 &    -4 &     1 &     0 &     0 & \text{                                     }\\
%      0 &     3 &     2 &     0 &     1 &     0 & \text{                                     }\\
%      2 &    -7 &    -4 &     0 &     0 &     1 & \text{  -$2\cdot$ 1. Zeile                 }\\\hline
% 
%      1 &    -4 &    -4 &     1 &     0 &     0 & \text{  $\cdot 3 + 4\cdot$ 2. Zeile        }\\
%      0 &     3 &     2 &     0 &     1 &     0 & \text{                                     }\\
%      0 &     1 &     4 &    -2 &     0 &     1 & \text{  $\cdot 3 - $ 2. Zeile              }\\\hline
% 
%      3 &     0 &    -4 &     3 &     4 &     0 & \text{ $\cdot 5$  +$2\cdot$ 3. Zeile       }\\
%      0 &     3 &     2 &     0 &     1 &     0 & \text{ $\cdot 5$    -3. Zeile              }\\
%      0 &     0 &    10 &    -6 &    -1 &     3 & \text{                                     }\\\hline
% 
%     15 &     0 &     0 &     3 &     18&     6 & \\
%      0 &    15 &     0 &     6 &     6 &    -3 & \\
%      0 &     0 &    10 &    -6 &    -1 &     3 & 
% \end{array}$$
% Damit ist 
% $$\boldsymbol A^{-1} = \frac 1 {10}\begin{pmatrix} 
%      2  &     12 &      4  \\
%      4  &      4 &     -2  \\
%     -6  &     -1 &      3   
% \end{pmatrix}.$$
% 
% $$\begin{array}{rrr|rrr|l}
%      1 &    -4 &    -4 &     1 &     0 &     0 & \text{  -4$\cdot$ 3. Zeile                 }\\
%      0 &     3 &     2 &     0 &     1 &     0 & \text{  3. Zeile                           }\\
%      0 &    -1 &     0 &     0 &     0 &     1 & \text{  2. Zeile + 3$\cdot$ 3. Zeile       }\\\hline
% 
%      1 &     0 &    -4 &     1 &     0 &    -4 & \text{  +2$\cdot$ 3. Zeile                 }\\
%      0 &    -1 &     0 &     0 &     0 &     1 & \text{  $\cdot (-1)$                       }\\
%      0 &     0 &     2 &     0 &     1 &     3 & \text{  $\cdot 1/2$                        }\\\hline
% 
%      1 &     0 &     0 &     1 &     2 &     2 &\\
%      0 &     1 &     0 &     0 &     0 &    -1 &\\
%      0 &     0 &     1 &     0 &    1/2&   3/2 &
% \end{array}$$
% Damit ist 
% $$\boldsymbol B^{-1} = \begin{pmatrix}1&2&2\\0&0&-1\\ 0&1/2&3/2\end{pmatrix}.$$
% 
% $\boldsymbol C$ hat den Eigenwert $\lambda=0$, also ist $\operatorname{Kern}
% \boldsymbol C \neq \{\boldsymbol 0\}$ und $\boldsymbol C$ nicht invertierbar. 

\item Es gibt also einen Vektor $\boldsymbol v\neq \boldsymbol 0$, mit dem gilt $\boldsymbol M \boldsymbol v=\lambda \boldsymbol
v$. Daraus folgt  
$$(\boldsymbol M + \boldsymbol E_n)\boldsymbol v = \boldsymbol M \boldsymbol v + \boldsymbol E_n \boldsymbol v = \lambda\boldsymbol v + 1 \boldsymbol v = (\lambda
+ 1 )\boldsymbol v$$
also dass $\lambda+1$ Eigenwert von $\boldsymbol M + \boldsymbol E_n$ ist.  
\end{abc}
}

\ErgebnisC{linalg_Eign_Wert_002}
{
{ a)} $\lambda_A\in \{1,\, -1/2\pm \sqrt{39}/2\operatorname{i}\}$, $\boldsymbol v_{A,1}=(1,1,-1)^\top,\, \boldsymbol
v_{A,2/3}=(-8,4, -7\pm \sqrt{39}\operatorname{i})^\top$. \\
$\lambda_B\in\{1,\, 2\}$, $\boldsymbol U_{\lambda=1}=\operatorname{span}\{(0,1,-1)^\top,\, (1,1,-1)^\top\}$, $\boldsymbol
U_{\lambda=2}=\operatorname{span}\{(4,-2,1)^\top\}$. \\
% $\lambda_C\in \{0,\, 1,\, \operatorname{i}\}$, $\boldsymbol v_{C,0}=(0,-1,2)^\top$, $\boldsymbol v_{C,1}=(1,-1,0)^\top$,
% $\boldsymbol v_{C,\operatorname{i}}=(1,-1,1)^\top$.\\
% { b)} $\boldsymbol B^{-1} = \begin{pmatrix}1&2&2\\0&0&-1\\ 0&1/2&3/2\end{pmatrix}$, $\boldsymbol A^{-1} = \frac 1 {10}\begin{pmatrix} 
%      2  &     12 &      4  \\
%      4  &      4 &     -2  \\
%     -6  &     -1 &      3   
% \end{pmatrix}.$\\
{ b)} $\lambda+1$

}
