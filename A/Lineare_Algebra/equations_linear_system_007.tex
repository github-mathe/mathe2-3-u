\Aufgabe[e]{Lineare Gleichungssysteme}{
\begin{abc}
\item Betrachten Sie ein lineares Gleichungssystem beliebiger Dimension 
      mit zwei unterschiedlichen Lösungen. Hat ein solches System immer 
      unendlich viele Lösungen?

\item Gibt es ein lineares Gleichungssystem mit weniger Gleichungen 
      als Variabelen, sodass es eine eindeutige Lösung ist?

\item Seien $a,b,c,d,r$ und $s$ reelle Zahlen. Überprüfen Sie, ob das
      lineare Gleichungssystem 

$$
\begin{array}{rcl}
a\, x_1 + b\, x_2 &=& r\\
c\, x_1 + d\, x_2 &=& s
\end{array}
$$
eindeutig lösbar ist, falls $a d - b c \neq 0$ und berechnen Sie die Lösung.
\end{abc}

}

\Loesung{
\textbf{a)} Seien $(x_1, x_2, \ldots, x_n)^T$ und
$(y_1, y_2, \ldots, y_n)^T$ zwei unterschiedliche spezielle Lösungen
desselben linearen Gleichungssystems. Dann ist
$$
(x_1 - y_1, x_2 - y_2, \ldots, x_n - y_n)^T
$$
eine Lösung des entsprechenden homogenen Systems. 
%
Dann ist auch für jedes beliebiges $t \in \R$,
$$
(x_1, x_2, \ldots, x_n)^T + t\, (x_1 - y_1, x_2 - y_2, \ldots, x_n - y_n)^T
$$
eine Lösung des ursprünglichen linearen Gleichungssystems.
Daher hat ein solches lineares Gleichungssystem immer unendlich viele Lösungen.

\bigskip
\textbf{b)}
Nach dem Überführen in Zeilenstufenform erkennt man, dass eine Variable 
frei gewählt werden kann, falls eine Lösung existiert.
Daher ist die Lösung niemals eindeutig.


\bigskip
\textbf{c)} Sei $D := a d - b c \neq 0$.
Da $D \neq 0$ folgt $a \neq 0$ oder $c \neq 0$.
Wir betrachten zuerst den Fall $a \neq 0$. Es gilt
$$
\begin{array}{cc|c}
a & b & r\\
c & d & s
\end{array}
\quad \stackrel{Z_2 - \frac{c}{a}\,Z_1}{\longrightarrow}
\quad
\begin{array}{cc|c}
a & b & r\\
0 & d-\frac{c}{a}b & s-\frac{c}{a}r
\end{array}
\quad\longrightarrow\quad
\begin{array}{cc|c}
a & b & r\\
0 & \frac{ad-bc}{a} & \frac{as-cr}{a}
\end{array}
$$

Aus der zweiten Zeile folgt
$$
\frac{a d - b c}{a}\, x_2 = \frac{a s - c r}{a} \quad
\stackrel{D \neq 0}{\longrightarrow}\quad
x_2 = \frac{a s - c r}{D}\,.
$$
% 
Die erste Zeile impliziert ($D = a d - b c \neq 0$)
$$
a\, x_1 + b\, \frac{a s - c r}{D} = r \quad \longrightarrow \quad
a\, x_1 = \frac{r D - a b s + b c r}{D} = \frac{a d r - b c r - a b s + b c r}{D}
$$
und daher
$$
x_1 = \frac{d r - b s}{D}\,.
$$
% 
Die eindutige Lösung ist dann
$$
(x_1, x_2)^T = \left( \dfrac{d r - b s}{D}, \dfrac{a s - c r}{D}\right)^T\,.
$$
% 
Der Fall $c \neq 0$ wird analog behandelt.

}
