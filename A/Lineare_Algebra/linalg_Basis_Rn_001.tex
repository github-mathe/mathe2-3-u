\Aufgabe[e]{Basis von $\mathbb{R}^3$}{
Welche der folgenden Mengen bilden eine Basis von $\mathbb{R}^3$\, ?
\setlength{\columnsep}{-6ex}
\begin{multicols}{2}
\begin{iii}
\item $ \mathcal{B}_1 =
\left\{ \left(\begin{array}{r} 1 \\ 0 \\ 1 \end{array}\right),
\left(\begin{array}{r} 0 \\ 1 \\ 0 \end{array}\right),
\left(\begin{array}{r} 1 \\ 1 \\ 1 \end{array}\right) \right\} $

\item $ \mathcal{B}_2 =
\left\{ \left(\begin{array}{r} 1 \\ 1 \\ 0 \end{array}\right),
\left(\begin{array}{r} 1 \\ 1 \\ 0 \end{array}\right) \right\}$

\item $ \mathcal{B}_3 =
\left\{ \left(\begin{array}{r} 1 \\ 1 \\ 1 \end{array}\right),
\left(\begin{array}{r} 1 \\ 0 \\ 1 \end{array}\right),
\left(\begin{array}{r} 0 \\ 0 \\ 1 \end{array}\right) \right\} $

\item $ \mathcal{B}_4 =
\left\{ \left(\begin{array}{r} 1 \\ 1 \\ 0 \end{array}\right),
\left(\begin{array}{r} 0 \\ 1 \\ 1 \end{array}\right),
\left(\begin{array}{r} 1 \\ 0 \\ 1 \end{array}\right),
\left(\begin{array}{r} 1 \\ 1 \\ 1 \end{array}\right) \right\} $
\end{iii}
\end{multicols}

}


\Loesung{
\begin{iii}

\item 
Die Vektoren sind linear abhängig. Das kann man mit der folgenden Beziehung zeigen
$$
\left(\begin{array}{r} 1\\ 0\\ 1 \end{array}\right)
+ \left(\begin{array}{r} 0\\ 1\\ 0 \end{array}\right)
= \left(\begin{array}{r} 1\\ 1\\ 1 \end{array}\right)
$$
Daher bilden sie keine Basis von $\mathbb{R}^3$\,.

\item 
Eine Basis von $\mathbb{R}^3$ hat genau $3$ linear unabhängige Vektoren.
Da die Menge $\mathcal{B}_2$ nur zwei Vektoren enthält, kann die Menge 
keine Basis von $\mathbb{R}^3$ bilden.

\item 
Wir überprüfen, ob die Elemente von $\mathcal B_3$ linear unabhängig sind.
$$
c_1\, (1, 1, 1)^\top + c_2\, (1, 0, 1)^\top + c_3\, (0, 0, 1)^\top
\stackrel{!}{=} \boldsymbol 0\,,\quad
c_1, c_2, c_3 \in \mathbb{R}\,.
$$
% 
Daraus erhalten wir das lineare Gleichungssystem 
$$
\begin{array}{r c r c r c l}
c_1 &+& c_2 & & &=& 0\,, \\
c_1 & & & & &=& 0\,, \\
c_1 &+& c_2 &+& c_3 &=& 0\,.
\end{array}
$$
% 
Das Gleichungssystem hat mit $c_1=0$, $c_2=0$ und $c_3 = 0$ eine eindeutige
Lösung. Die drei Vektoren in der Menge $\mathcal{B}_3$ sind also linear 
unabhängig und bilden daher eine Basis von $\mathbb{R}^3$.

\item
Eine Basis von $\mathbb{R}^3$ hat genau $3$ linear unabhängige Vektoren.
Da die Menge $\mathcal{B}_4$ vier Vektoren enthält, kann die Menge 
keine Basis von $\mathbb{R}^3$ bilden.
\end{iii}

}
