\Aufgabe[e]{Darstellungen von Ebenen im $\R^3$}{
Gegeben sind die Punkte im $\R^3$ mit den Ortsvektoren 
$$
\boldsymbol a = \begin{pmatrix} -1 \\ 0 \\ 1 \end{pmatrix}\,,\quad
\boldsymbol b = \begin{pmatrix} 0 \\ 0 \\ 2 \end{pmatrix}\,,\quad
\boldsymbol c = \begin{pmatrix} -1 \\ 2 \\ 0 \end{pmatrix}\,,\quad
\boldsymbol d = \begin{pmatrix} 1 \\ 2 \\ z \end{pmatrix}\,,\quad
z \in \R\,.
$$

\begin{abc}
\item Geben Sie für die durch $\boldsymbol a$, $\boldsymbol b$ und
$\boldsymbol c$ aufgespannte Ebene $\boldsymbol E$ im $\R^3$
\begin{iii}
\item eine Parameterdarstellung,
\item eine Hessesche Normalform und
\item eine allgemeine Ebenengleichung an.
\end{iii}

\item Bestimmen Sie $z \in \R$ so, dass $\boldsymbol d$ in der Ebene $\boldsymbol E$
liegt.
\end{abc}
}
\Loesung{

\begin{abc}
\item \begin{iii}
\item Zwei m\"ogliche Richtungsvektoren der Ebene werden durch 
$$\vec b - \vec a=\begin{pmatrix}1\\0\\1\end{pmatrix},\, \vec c-\vec a
  = \begin{pmatrix}0\\2\\-1\end{pmatrix}$$
angegeben. Als St\"utzvektor w\"ahlen wir $\vec a$. Damit kann man $\vec E$ schreiben als 
$$\vec E = \left\{\vec x=\begin{pmatrix}-1\\0\\1\end{pmatrix} +
  \lambda\begin{pmatrix}1\\0\\1\end{pmatrix}+ \mu \begin{pmatrix}0\\2\\-1\end{pmatrix}\Big|\, \lambda,\mu\in\R\right\}.$$

\item F\"ur eine Darstellung der Ebene in Hessescher Normalform ben\"otigen wir einen
 Normalenvektor $\vec n$ der Ebene, dieser muss also senkrecht auf den Richtungsvektoren der Ebene
 stehen. Dies f\"uhrt auf das Gleichungssystem 
$$\begin{array}{rrr|r|l}
n_1   &   n_2    &   n_3   & \\\hline
  1   &  0      &   1     &   0    &                      \\
  0   &  2      &  -1     &   0    &                      \\
\end{array}$$
Die Wahl $n_3=2$ f\"uhrt auf $n_2=1$ und $n_1=-2$. \\


Damit ist die Normalform
\begin{align*}
 \vec E &= \left\{ \vec x \in  \mathbb{R}^3 \mid \left\langle \vec x, \vec n \right\rangle = \langle \vec a, \vec n\rangle \right\} \\
    &=\left\{ \vec x \in  \mathbb{R}^3 \mid \left\langle \vec x, \begin{pmatrix} -2\\1\\2 \end{pmatrix} \right\rangle = 4 \right\}.
\end{align*}


F\"ur die Hessesche Normalform muss der Normalenvektor noch normiert werden
$$
\|\vec n\| = 3.
$$
Damit ist der normierte Normalenvektor 
$$
\vec n_0 = \frac{1}{3} \begin{pmatrix} -2\\1\\2 \end{pmatrix}
$$

Mit diesem Normalenvektor ist eine Hessesche Normalform der Ebene
$$
 \vec E = \left\{ \vec x \in  \mathbb{R}^3 \mid \left\langle \vec x, \frac{1}{3}\begin{pmatrix} -2\\1\\2 \end{pmatrix} \right\rangle =\frac{4}{3} \right\}.
$$

\item Eine allgemeine Ebenengleichung ergibt sich aus der Berechnung des obigen Skalarproduktes: 
$$0\overset!=\skalar{\vec x-\vec a,\vec n}=-2(x_1+1)+1(x_2-0)+2(x_3-1)\,\Rightarrow\,
-2x_1+x_2+2x_3=4.$$
\end{iii}
\item Wenn $\vec d$ auf der Ebene liegt, m\"ussen seine Koordinaten die allgemeine Ebenengleichung
erf\"ullen, daraus folgt
$$-2\cdot 1 + 2 + 2 z=4,$$ 
dies ergibt den Wert $z=2$. 
\end{abc}
}

\ErgebnisC{linalg_Geom_Ebene_001}
{
{\textbf{ a)}} z. B. $-2x_1+x_2+2x_3=4$
}
