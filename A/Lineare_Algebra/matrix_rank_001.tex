\Aufgabe[e]{Rang einer Matrix}{
Bestimmen Sie den Rang der folgenden Matrizen
\begin{abc}
\item
$$
\boldsymbol A =
\begin{pmatrix}
1 & 3 &  1 & -2 & -3\\
1 & 4 &  3 & -1 & -4\\
2 & 3 & -4 & -7 & -3\\
3 & 8 &  1 & -7 & -8
\end{pmatrix}\,.
$$

\item
$$
\boldsymbol B =
\begin{pmatrix}
 1 & -2 & 0\\
 1 &  1 & 1\\
-1 & -1 & 4\\
 3 &  4 & 3
\end{pmatrix}\,.
$$

\item
$$
\boldsymbol C =
\begin{pmatrix}
3 &  8 & 6 & -4 & -1 \\
0 &  0 & 1 &  1 &  3 \\
0 &  0 & 0 &  0 &  4 \\
0 & -7 & 8 &  1 &  1 \\
0 &  0 & 0 & -7 &  4 
\end{pmatrix}\,.
$$

\end{abc}
}

\Loesung{
\begin{abc}
\item
Durch Anwendung des Gauß-Verfahrens erhalten wir
$$
\begin{array}{r@{\quad}c@{\quad}l}
\begin{array}{rrrrr||}
1 &     3 &     1 &    -2 &    -3\\
1 &     4 &     3 &    -1 &    -4\\
2 &     3 &    -4 &    -7 &    -3\\
3 &     8 &     1 &    -7 &    -8\\
\hline
\end{array} &\rightarrow&
% 
\begin{array}{rrrrr||}
1 &     3 &     1 &    -2 &    -3\\
0 &     1 &     2 &     1 &    -1\\
0 &    -3 &    -6 &    -3 &     3\\
0 &    -1 &    -2 &    -1 &     1\\
\hline
\end{array}\quad \rightarrow\\[6ex]
% 
% 
\begin{array}{rrrrr||}
1 &     3 &     1 &    -2 &    -3\\
0 &     1 &     2 &     1 &    -1\\
0 &     0 &     0 &     0 &     0\\
0 &     0 &     0 &     0 &     0\\
\hline
\end{array} & .
\end{array}
$$


Daraus folgt $\operatorname{Rang}(\boldsymbol A) = 2$.

\item
Durch Anwendung des Gauß-Verfahrens erhalten wir
$$
\begin{array}{rrr||}
 1 & -2 & 0\\
 1 &  1 & 1\\
-1 & -1 & 4\\
 3 &  4 & 3\\
\hline
\end{array} \quad \rightarrow \quad
%
\begin{array}{rrr||}
 1 & -2 & 0\\
 0 &  3 & 1\\
 0 & -3 & 4\\
 0 & 10 & 3\\
\hline
\end{array} \quad \rightarrow \quad
%
\begin{array}{rrr||}
 1 & -2 & 0\\
 0 &  3 & 1\\
 0 &  0 & 5\\
 0 &  0 & 3-\frac{10}{3}\\
\hline
\end{array}
$$

$$
\rightarrow \quad
%
\begin{array}{rrr||}
 1 & -2 & 0\\
 0 &  3 & 1\\
 0 &  0 & 5\\
 0 &  0 & 0\\
\hline
\end{array}
$$

Daraus folgt $\operatorname{Rang}(\boldsymbol B) = 3$.


\item
Durch Vertauschen der Zeilen erhalten wir
$$
\begin{array}{rrrrr||l}
3 &     8 &     6 &    -4 &    -1 \\
0 &     0 &     1 &     1 &     3 &\text{ 4. Zeile}\\
0 &     0 &     0 &     0 &     4 &\text{ 2. Zeile}\\
0 &    -7 &     8 &     1 &     1 &\text{ 5. Zeile}\\
0 &     0 &     0 &    -7 &     4 &\text{ 3. Zeile}\\
\hline
\end{array}\quad\rightarrow\quad
% 
\begin{array}{rrrrr||}
3 &     8 &     6 &    -4 &    -1 \\
0 &    -7 &     8 &     1 &     1 \\
0 &     0 &     1 &     1 &     3 \\
0 &     0 &     0 &    -7 &     4 \\
0 &     0 &     0 &     0 &     4 \\
\hline
\end{array}\,.
$$

Daraus folgt $\operatorname{Rang}(\boldsymbol C) = 5$.

\end{abc}
}

\ErgebnisC{linalgMaprKmpl001}
{
\textbf{a)} $\operatorname{Rang}(\boldsymbol A) = 2$ \, ,
\textbf{b)} $\operatorname{Rang}(\boldsymbol B) = 3$ \, ,
\textbf{c)} $\operatorname{Rang}(\boldsymbol C) = 5$.
}
