\Aufgabe[e]{Vektor- und Matrix-Normen}{
\begin{abc}
\item Bestimmen Sie die drei Standard-Normen
(Betragssummen-, Euklidische- und Maximums-Norm) der Vektoren:
$$
\boldsymbol{a} = \begin{pmatrix}	3 \\ -4 \\ 12	\end{pmatrix}\,,\quad
% 
\boldsymbol{b} = \begin{pmatrix}	1 \\ 1 \\ 1	\end{pmatrix}\,.
$$

\item Bestimmen Sie die drei Standard-Normen\\
(Spaltensummen-, Zeilensummen-, Frobenius-Norm) der Matrizen:
$$
\boldsymbol{C} = \begin{pmatrix}	2 & 0 \\ -4 & 3	\end{pmatrix}\,,\quad
% 
\boldsymbol{D} = \begin{pmatrix}	3 & 0 \\ 0 & 3	\end{pmatrix}\,.
$$
% \item Sind die Vektoren 
% $$\vec a=\begin{pmatrix} 1\\2\\3\end{pmatrix},\, \vec b=\begin{pmatrix}2\\0\\2\end{pmatrix},\, \vec
% c=\begin{pmatrix}1\\1\\0\end{pmatrix}$$
% linear unabh\"angig?
% \item W\"ahlen Sie den Parameter $t\in\R$ so, dass die Vektoren $\vec a$, $\vec b$ und $\vec
% d_t=\begin{pmatrix}1\\t\\2\end{pmatrix}$ linear abh\"angig sind. 
\end{abc}
}
\Loesung{
\begin{abc}
\item 
$$
\|\boldsymbol{a}\|_{1} = 19\,,\quad
\|\boldsymbol{a}\|_{2} = 13\,,\quad
\|\boldsymbol{a}\|_{\infty} = 12\,.
$$
$$
\|\boldsymbol{b}\|_{1} = 3\,,\quad
\|\boldsymbol{b}\|_{2} = \sqrt{3}\,,\quad
\|\boldsymbol{b}\|_{\infty} = 1\,.
$$

\item 
$$
\|\boldsymbol{C}\|_{1} = 6\,,\quad
\|\boldsymbol{C}\|_{\infty} = 7\,,\quad
\|\boldsymbol{C}\|_{F} = \sqrt{29}\,.\quad
% \|\boldsymbol{C}\|_{G} = 8\,.
$$
$$
\|\boldsymbol{D}\|_{1} = \|\boldsymbol{D}\|_{\infty} = 3\,,\quad
\|\boldsymbol{D}\|_{F} = 3\sqrt{2}\,.%\quad
% \|\boldsymbol{D}\|_{G} = 6\,.
$$
% \item Der Gauß-Algorithmus f\"ur die drei Vektoren ($\vec a^\top,\, \vec b^\top,\, \vec c^\top$ in
% den Zeilen) ergibt: 
% $$\begin{array}{rrr|l}
% 1   &   2   &   3   &\text{                  }\\
% 2   &   0   &   2   &-2\times \text{ 1. Zeile                 }\\
% 1   &   1   &   0   &- \text{ 1. Zeile                  }\\\hline
% 
% 1   &   2   &   3   &\text{                  }\\
% 0   &  -4   &  -4   &                               \\
% 0   &  -1   &  -3   &-1/4\cdot \text{ 2. Zeile          }\\\hline
% 
% 1   &   2   &   3   &\text{                  }\\
% 0   &   -4  &  -4   &                                          \\
% 0   &   0   &  -2   &\\
% \end{array}$$
% Es verbleiben drei Zeilen in der Zeilensufenform , damit sind die drei Vektoren linear abh\"angig. 
% \item Wir f\"uhren den Gauß-Algorithmus f\"ur die drei Vektoren $\vec a,\, \vec b,\, \vec d_t$
% durch: 
% $$\begin{array}{rrr|l}
% 1   &   2   &   3   &\text{                  }\\
% 2   &   0   &   2   &-2\times \text{ 1. Zeile                 }\\
% 1   &   t   &   2   &- \text{ 1. Zeile                  }\\\hline
% 
% 1   &   2   &   3   &\text{                  }\\
% 0   &  -4   &  -4   &                                \\
% 0   &  t-2  &  -1   &+(t-2)/4\cdot \text{ 2. Zeile          }\\\hline
% 
% 1   &   2   &   3   &\text{                  }\\
% 0   &  -4   &  -4   &                                          \\
% 0   &   0   &-1-(t-2)  &\\
% \end{array}$$
% Die Vektoren sind genau dann linear unabh\"angig, wenn die letzte Zeile der Zeilenstufenform
% verschwindet, wenn also gilt: 
% $$-1-(t-2)=0\,\Leftrightarrow\, t=1.$$
\end{abc}
}

\ErgebnisC{linalg_VeMa_Norm_011}
{
$\|\boldsymbol{a}\|_{1} = 19\,,\,\,
\|\boldsymbol{a}\|_{2} = 13\,,\,\,
\|\boldsymbol{a}\|_{\infty} = 12\,.$

$\|\boldsymbol{b}\|_{1} = 3\,,\,\,
\|\boldsymbol{b}\|_{2} = \sqrt{3}\,,\,\,
\|\boldsymbol{b}\|_{\infty} = 1\,.$

$\|\boldsymbol{C}\|_{1} = 6\,,\,\,
\|\boldsymbol{C}\|_{\infty} = 7\,,\,\,
\|\boldsymbol{C}\|_{F} = \sqrt{29}\,.$%,\,\,
% \|\boldsymbol{C}\|_{G} = 8\,.$

$\|\boldsymbol{D}\|_{1} = \|\boldsymbol{D}\|_{\infty} = 3\,,\,\,
\|\boldsymbol{D}\|_{F} = 3\sqrt{2}\,.$%\,\,
% \|\boldsymbol{D}\|_{G} = 6\,.$
}
