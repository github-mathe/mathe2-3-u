\Aufgabe[e]{}{
Gegeben seien die Vektoren
$$
\boldsymbol x = \begin{pmatrix} 4 \\ 2 \\ 3 \\ 1 \end{pmatrix}\,,\quad
\boldsymbol b_1 = \begin{pmatrix} 1 \\ 1 \\ 0 \\ 0 \end{pmatrix}\,,\quad
\boldsymbol b_2 = \begin{pmatrix} 1 \\ 0 \\ 0 \\ 1 \end{pmatrix}\,,\quad
\boldsymbol b_3 = \begin{pmatrix} 0 \\ 1 \\ 1 \\ 0 \end{pmatrix}\,.
$$

\begin{abc}
\item Bestimmen Sie eine Orthonormalbasis
$\{ \boldsymbol a_1, \boldsymbol a_2, \boldsymbol a_3 \}$
von $\boldsymbol U
= \operatorname{span} \{ \boldsymbol b_1, \boldsymbol b_2, \boldsymbol b_3 \}$.
Dabei sei
$\operatorname{span} \{ \boldsymbol b_1 \}
= \operatorname{span} \{ \boldsymbol a_1 \}$
und
$\operatorname{span} \{ \boldsymbol b_2, \boldsymbol b_3 \}
= \operatorname{span} \{ \boldsymbol a_2, \boldsymbol a_3 \}$.
\item Pr\"ufen Sie nach, ob der Vektor $\vec x$ im Unterraum $\vec U$ enthalten ist. Vesuchen Sie dazu, den Vektor bez\"uglich der Orthogonalbasis $\vec a_1,\, \vec a_2,\, \vec a_3$ darzustellen. 
%\item Berechnen Sie die orthogonale Projektion von $\boldsymbol x$ auf $\boldsymbol U$.
%
%\item Berechnen Sie den Abstand
%$\min_{\boldsymbol u \in \boldsymbol U} \| \boldsymbol x - \boldsymbol u \| $
%von $\boldsymbol x$ zu $\boldsymbol U$.
\item Geben Sie eine Basis des Orthogonalraumes $\vec U^\perp$ an. 
\end{abc}
}
\Loesung{
\begin{abc}
\item Mit dem Schmidt'schen Orthonormalisierungsverfahren folgt: \\
\begin{eqnarray*} 
\boldsymbol{a}_{1} & = &\dfrac{1}{\|\boldsymbol{b}_{1}\|}\boldsymbol{b}_{1}
=\dfrac{1}{\sqrt{2}}\begin{pmatrix} 1 \\ 1 \\ 0 \\ 0 \end{pmatrix}, \\
\boldsymbol{w}_{2} & = & \boldsymbol{b}_{2}-\langle \boldsymbol{b}_{2},
\boldsymbol{a}_{1} \rangle 
\boldsymbol{a}_{1}=\begin{pmatrix} 1 \\ 0 \\ 0 \\ 1 \end{pmatrix}-\dfrac{1}{2}
\begin{pmatrix} 1 \\ 1 \\ 0 \\ 0 \end{pmatrix}
=\dfrac{1}{2}\begin{pmatrix} 1 \\ -1 \\ 0 \\ 2 \end{pmatrix}, \\
\boldsymbol{a}_{2} & = & \dfrac{1}{\|\boldsymbol{w}_{2}\|} \boldsymbol{w}_{2}
=\dfrac{1}{\sqrt{6}}\begin{pmatrix} 1 \\ -1 \\ 0 \\ 2 \end{pmatrix}, \\
\boldsymbol{w}_{3} & = & \boldsymbol{b}_{3}-\langle \boldsymbol{b}_{3},
\boldsymbol{a}_{1}\rangle 
\boldsymbol{a}_{1}-\langle \boldsymbol{b}_{3},\boldsymbol{a}_{2}\rangle 
\boldsymbol{a}_{2}
=\begin{pmatrix} 0 \\ 1 \\ 1 \\ 0 \end{pmatrix}-\dfrac{1}{2}
\begin{pmatrix} 1 \\ 1 \\ 0 \\ 0 \end{pmatrix}
-\dfrac{-1}{6}\begin{pmatrix} 1 \\ -1 \\ 0 \\ 2 \end{pmatrix}
=\dfrac{1}{6}\begin{pmatrix} -2 \\ 2 \\ 6 \\ 2 \end{pmatrix}, \\
\boldsymbol{a}_{3} & =& \dfrac{1}{\|\boldsymbol{w}_{3}\|} \boldsymbol{w}_{3}
=\dfrac{1}{\sqrt{12}}\begin{pmatrix} -1 \\ 1 \\ 3 \\ 1 \end{pmatrix}. 
\end{eqnarray*}

\item Die Projektion von $\boldsymbol{x}=(4,2,3,1)^{\top}$ auf 
$\boldsymbol U$ ist gegeben durch 
\begin{eqnarray*}
\boldsymbol P \boldsymbol{x}&= & \langle \boldsymbol{x},\boldsymbol{a}_{1}\rangle 
\boldsymbol{a}_{1}
+\langle \boldsymbol{x},\boldsymbol{a}_{2}\rangle \boldsymbol{a}_{2}
+\langle \boldsymbol{x},\boldsymbol{a}_{3}\rangle \boldsymbol{a}_{3} \\[1ex]
&= & \dfrac{6}{2}\begin{pmatrix} 1 \\ 1 \\ 0 \\ 0 \end{pmatrix}+\dfrac{4}{6}
\begin{pmatrix} 1 \\ -1 \\ 0 \\ 2 \end{pmatrix}
+\dfrac{8}{12}\begin{pmatrix} -1 \\ 1 \\ 3 \\ 1 \end{pmatrix}
=\begin{pmatrix} 3 \\ 3 \\ 2 \\ 2 \end{pmatrix}.
\end{eqnarray*}
$\vec x$ ist also nicht durch die Basis $\vec a_1,\,\vec a_2,\, \vec a_3$ darstellbar, es gilt $\vec x\neq \vec P\vec x$, damit gilt $\vec x\notin\vec U$. 
%\item Es gilt
%$$
%\min_{\boldsymbol u \in \boldsymbol U} \|\boldsymbol x - \boldsymbol u\| = 
%\|\boldsymbol x - \boldsymbol P \boldsymbol x\| = 
%\left\|\begin{pmatrix}1\\-1\\ 1\\ -1 \end{pmatrix}\right\| = \sqrt{4} = 2\,.
%$$
%
\item $\vec U^\perp$ enth\"alt alle Vektoren, die zu allen Vektoren $\vec u\in\vec U$ senkrecht sind. Der Orthogonalraum ist damit die L\"osung des linearen Gleichungssystems 
$$\skalar{\vec b_1,\vec x}=0,\, \skalar{\vec b_2,\vec x}=0,\, \skalar{\vec b_3,\vec x}=0.$$
Dies f\"uhrt auf
$$\begin{array}{rrrr|r|l}
   1  &  1  &  0  &  0  &  0  &                               \\
   1  &  0  &  0  &  1  &  0  & -\text{ 1. Zeile}             \\
   0  &  1  &  1  &  0  &  0  &                               \\\hline

   1  &  1  &  0  &  0  &  0  &                               \\
   0  & -1  &  0  &  1  &  0  & \\
   0  &  1  &  1  &  0  &  0  & + \text{ 2. Zeile}            \\\hline

   1  &  1  &  0  &  0  &  0  &                               \\
   0  & -1  &  0  &  1  &  0  & \\
   0  &  0  &  1  &  1  &  0  &             \\
\end{array}$$
$x_4$ ist frei w\"ahlbar, und die L\"osungsmenge ist: 
$$\vec U^\perp=\left\{\begin{pmatrix}-t\\t \\-t \\t\end{pmatrix};\, t\in\R\right\}.$$

Damit ist eine Basis von $\vec U^\perp$
$$
\mathcal B (\vec U^\perp) = \left\{ \begin{pmatrix} -1\\1\\-1\\1 \end{pmatrix} \right\}.
$$
\end{abc}
}

\ErgebnisC{linalg_Orthogonale_Projektion_001-2011-M1-KL1}
{
\textbf{c)} $\vec U^\perp=\text{span}\{(-1,1,-1,1)\}$
}
