\Aufgabe[e]{Lineare Unabh\"angigkeit in Polynomr\"aumen}{
Untersuchen Sie, ob die Elemente der Menge
$$
M_1 = \left\{
  p_1(x) = x^2\,,\quad
  p_2(x) = (x+1)^2\,,\quad
  p_3(x) = (x-1)^2
\right\}\,,\quad x \in \R\,,
$$
% 
linear unabh\"angig im Vektorraum $P_2$ der
reellwertigen Polynome vom maximalen Grad $2$ sind und
geben Sie die Dimension von $\operatorname{span} M_1$ an.
}
\Loesung{
Die Elemente der Menge $M_1$ sind linear unabhängig genau dann, wenn
$$
0 = \lambda_1\, p_1(x) + \lambda_2\, p_2(x) + \lambda_3\, p_3(x)\,,\quad
\lambda_1, \lambda_2, \lambda_3 \in \R\,,
$$
nur die eindeutige Wahl $\lambda_1 = \lambda_2 = \lambda_3 = 0$ zulässt.
% 
Vergleichen Sie auch die Definition der linearen Unabhängigkeit von Elementen
eines Vektorraums aus dem Skript.

\bigskip
Um das Problem zu lösen, setzen wir die Polynome in die obige Gleichung ein
und erhalten
$$
\begin{array}{r@{\,\,}c@{\,\,}l}
0 &=& \lambda_1\, (x^2) + \lambda_2\, (x + 1)^2 + \lambda_3\, (x - 1)^2\,,\\[1ex]
% 
0 &=& x^2 \cdot ( \lambda_1 + \lambda_2 + \lambda_3 )
  + x \cdot ( 2\, \lambda_2 -2\, \lambda_3 )
  + 1 \cdot (\lambda_2 + \lambda_3)\,,
\end{array}
$$
und durch Koeffizientenvergleich erhalten wir das Gleichungssystem
$$
\begin{array}{r@{\,\,}c@{\,\,}l}
\lambda_1 + \lambda_2 + \lambda_3 &=& 0\,,\\[1ex]
2\, \lambda_2 -2\, \lambda_3 &=& 0\,,\\[1ex]
\lambda_2 + \lambda_3 &=& 0\,,
\end{array}
$$
in den Unbekannten $\lambda_1$, $\lambda_2$ und $\lambda_3$.

Durch die Anwendung des Gau\ss{}-Algorithmus erhalten wir
$$
\begin{array}{rrrr}
1 & 1 &  1  & 0\\[1ex]
0 & 2 & -2  & 0\\[1ex]
0 & 0 &  2  & 0\\[1ex]
\hline
\end{array}\,.
$$
% 
Dieses Gleichungssystem ist eindeutig lösbar.
Somit kann nur die Wahl $\lambda_1 = \lambda_2 = \lambda_3 = 0$
getroffen werden.

\medskip
Folglich sind die Elemente der Menge $M_1$ linear unabhängig.

\medskip
Aus der Stufenform folgt $\dim \operatorname{span} M_1 = 3$\,.
}

\ErgebnisC{linalg_linUnab_Poly_001}
{
$\dim(\text{span}( M_1))=3$
}
