\Aufgabe[e]{Klausuraufgabe Dez. 2010}{
Gegeben seien
$$
\boldsymbol A =
\frac{1}{\sqrt{2}}
\begin{pmatrix}
1 & - \operatorname{i}\\
\operatorname{i} & -1
\end{pmatrix}\,,\quad
% 
\boldsymbol B =
\frac{1}{\sqrt{2}}
\begin{pmatrix}
\operatorname{i} & 1\\
1 & \operatorname{i}
\end{pmatrix}\,,\quad
% 
\boldsymbol b =
\frac{1}{\sqrt{2}}
\begin{pmatrix}
0\\
-2 + 2 \operatorname{i}
\end{pmatrix}\,.
$$

\begin{abc}
\item Berechnen Sie die Matrizenprodukte
$\boldsymbol A^\ast \boldsymbol A$,
$\boldsymbol B \boldsymbol B^\ast$,
$\boldsymbol A \boldsymbol b$,
$\boldsymbol B^2$,
$\boldsymbol B^4$,
$\boldsymbol B^8$.

\item Sind die Matrizen $\boldsymbol A$ und $\boldsymbol B$ symmetrisch,
hermitesch, orthogonal, unit\"ar oder haben sie keine der genannten Eigenschaften?

\item L\"osen Sie die Gleichungssysteme $\boldsymbol A \boldsymbol x = \boldsymbol b$,
$\boldsymbol B \boldsymbol y = \boldsymbol b$.
\end{abc}
}
\Loesung{
\textbf{Zu a)} Es ergibt sich
$$
\boldsymbol A^\ast \boldsymbol A = \boldsymbol E_2 \,, \quad \boldsymbol B \boldsymbol B^\ast = \boldsymbol E_2 \,, \quad \boldsymbol A\boldsymbol b = \begin{pmatrix} 1+\operatorname{i}\\ 1-\operatorname{i} \end{pmatrix}\,,$$ 
$$\boldsymbol B^2 =  \begin{pmatrix} 0 & \operatorname{i} \\ \operatorname{i} & 0\end{pmatrix}\,, \quad \boldsymbol B^4 = -\boldsymbol E_2\,, \quad \boldsymbol B^8 = \boldsymbol E_2\,. 
$$

\bigskip
\textbf{Zu b)} $\boldsymbol A$ ist unit\"ar und hermitesch, denn $\boldsymbol A^\ast = \boldsymbol A$ und $\boldsymbol B$ ist unit\"ar.

\bigskip
\textbf{Zu c)} Es ergibt sich
$$
\boldsymbol x = \boldsymbol A \boldsymbol b\,, \quad \boldsymbol y = \boldsymbol B^\ast\boldsymbol b = \begin{pmatrix} -1+\operatorname{i} \\ 1+\operatorname{i} \end{pmatrix}\,.
$$
}

% \ErgebnisC{xyz002}
% {
% 
% }
