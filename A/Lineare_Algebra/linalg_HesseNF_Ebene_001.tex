\Aufgabe[e]{Hessesche Normalform einer Ebene}{
Gegeben sei die Gerade
$$
\boldsymbol g = \left\{\, \boldsymbol x \in \R^3 \,\middle|\,
\boldsymbol x = (1, 0, 2)^\top + \lambda\, (2, 3, 1)^\top\,,\quad
\lambda \in \R \,\right\}
$$
und $\boldsymbol p = (1, 0, 1)^\top \in \R^3$\,.

\smallskip
Geben Sie eine Hessesche Normalform der Ebene $\boldsymbol E$, mit
$\boldsymbol p \in \boldsymbol E$ an,
welche orthogonal zu $\boldsymbol g$ liegt.
}
\Loesung{
Im Folgenden werden mit der Abbildung $\langle \cdot, \cdot \rangle$ das
Standard-Skalarprodukt des $\R^3$ und
mit der Abbildung $\| \cdot \|$ die aus dem Standard-Skalarprodukt des $\R^3$
induzierte Norm
bezeichnet.

\bigskip
Ein Normalenvektor $\boldsymbol n := (2, 3, 1)^\top$ der zu bestimmenden
Ebene $\boldsymbol E$ entspricht nach Aufgabenstellung dem Richtungsvektor der
Geraden $\boldsymbol g$.
% 
Zur Darstellung der Ebene in Hessescher Normalform muss ein Normalenvektor mit
$\| \boldsymbol n_0 \| = 1$ gewählt werden.
Einen solchen erhalten wir beispielsweise durch
$$
\boldsymbol n_0
= \frac{1}{\| \boldsymbol n \|}\, \begin{pmatrix} 2 \\ 3 \\ 1 \end{pmatrix}
= \frac{1}{\sqrt{14}}\, \begin{pmatrix} 2 \\ 3 \\ 1 \end{pmatrix}\,.
$$

Eine Hessesche Normalform der Ebene $\boldsymbol E$ ist somit bestimmt durch
$$
\begin{array}{r@{\,\,}c@{\,\,}l}
\boldsymbol E &=& \left\{\, \boldsymbol x \in \R^3 \,\middle|\,
  \left\langle \boldsymbol x, \boldsymbol n_0 \right\rangle =
  \left\langle \boldsymbol p, \boldsymbol n_0 \right\rangle\,,\quad
  \boldsymbol n_0 = \frac{1}{\sqrt{14}}\, (2, 3, 1)^\top\,,\,\,
  \boldsymbol p = (1, 0, 1)^\top
\right\}\,,\\[3ex]
% 
\boldsymbol E &=& \left\{\, \boldsymbol x \in \R^3 \,\middle|\,
\dfrac{1}{\sqrt{14}}\,
  \left\langle \boldsymbol x\,,\,\, (2, 3, 1)^\top \right\rangle =
  \dfrac{3}{\sqrt{14}}
\right\}\,.\\[1ex]
\end{array}
$$

\bigskip
\textbf{Hinweis:} Eine Hessesche Normalform ist,
bis auf das Vorzeichen des gewählten normierten Normalenvektors $\boldsymbol n_0$,
eindeutig bestimmt.
% 
Die andere mögliche Wahl wäre hier
$\boldsymbol n_0^- = (-1) \cdot \boldsymbol n_0
= \frac{1}{\sqrt{14}}\, (-2, -3, -1)^\top$
anstelle von $\boldsymbol n_0$\,.
}

\ErgebnisC{linalg_HesseNF_Ebene_001}
{
Eine zur gesuchten Ebene \textit{parallele} Ebene hat die allgemeine Ebenengleichung $2\, x_1 + 3\, x_2 + x_3 = 0$\,.
}
