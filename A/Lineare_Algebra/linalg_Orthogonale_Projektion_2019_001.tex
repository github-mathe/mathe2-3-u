\Aufgabe[e]{Orthogonale Projektion und minimaler Abstand}{
Seien $\vec V =\mathbb R^3$ und $\vec u = (1, 2, -5)^T$. Sei $\vec U \subset \vec V$ der 
Unterraum $\vec U=\operatorname{span}\{\vec u\}$ und $\vec U^\perp$ das 
Orthogonalkomplement.

\begin{abc}
\item Bestimmen Sie eine Basis von $\vec U^\perp$.
\item Bestimmen Sie eine Orthonormalbasis von $\vec U^\perp$.
\item Bestimmen Sie den Abstand von $\vec v = (3,1,7)^\top$ zu $\vec U^\perp$.
\end{abc}

}

\Loesung{
\begin{abc}
\item
Der Unterraum $\vec U^\perp$ 
besteht aus allen Vektoren, die die folgende Bedingung erfüllen:
\[
x+2y-5z = 0.
\]
Wir w\"ahlen f\"ur zwei Komponenten einen beliebigen Wert $y=s$ und $z=t$. Damit 
erhalten wir $x=-2s+5t$ und
\[
\vec x = \begin{pmatrix} -2\\1\\0\end{pmatrix} s + \begin{pmatrix} 5\\0\\1\end{pmatrix} t.
\]
Eine Basis von $\vec U^\perp$ ist
\[
\mathcal B = \left \{ \begin{pmatrix} -2\\1\\0 \end{pmatrix}, \begin{pmatrix} 5\\0\\1 \end{pmatrix}\right\}.
\]
\item 
Wir bezeichnen die beiden Vektoren der Basis $\mathcal B$ als $\vec {\tilde u_1}$ und 
$\vec {\tilde u_2}$.

Eine Orthonormalbasis $\{\vec u_1, \vec u_2\}$ erhalten wir, indem zunächst wir den Vektor 
 $\vec {\tilde u_1}$ von $\mathcal B$ normieren. 
Der erste Vektor ist dann
\[
\vec u_1 = \frac{\tilde{\vec u_1}}{\|\tilde {\vec u_1}\|} =\frac{1}{\sqrt{5}} \begin{pmatrix} -2\\1\\0 \end{pmatrix}.
\]
Der Vektor $\vec u_2$ muss in $\vec U^\perp$ liegen, d.h.\ er muss eine Linearkombination 
der Vektoren aus $\mathcal B$ sein und die Orthogonalitätsbedingung 
$\langle \vec u_1, \vec u_2 \rangle=0$ erfüllen. Weiterhin muss der Vektor normiert sein.

Um diesen Vektor $\vec u_2$ zu finden, berechnen die Orthogonalprojektion von $\vec{ \tilde u_2}$ auf $\vec u_1$ und subtrahieren die Projektion von 
$\vec {\tilde u_2}$ selbst. Damit haben wir eine orthogonalen Vektor zu $\vec u_1$ konstruiert.

Die Orthogonalprojektion von $\vec{\tilde u_2}$ auf $\vec u_1$ ist
\begin{align*}
		P_{\vec u_1}(\vec{\tilde u_2}) &= \langle \vec {\tilde u_2}, \vec u_1\rangle \vec u_1	\\
		&=\frac{1}{5} \left \langle 
		\begin{pmatrix}  5\\0\\1 \end{pmatrix}, 
		\begin{pmatrix} -2\\1\\0 \end{pmatrix} \right \rangle 
		\begin{pmatrix} -2\\1\\0 \end{pmatrix} \\
		&= \begin{pmatrix} 4\\-2\\0 \end{pmatrix}
\end{align*}
Wir erhalten den Vektor $\vec {\hat u_2}$ orthogonal zu $\vec u_1$ durch Subtraktion
\[
	\vec {\hat u_2} = \vec{\tilde u_2} - P_{\vec u_1}(\vec{\tilde u_2}) 
	= \begin{pmatrix} 5\\0\\1 \end{pmatrix} - \begin{pmatrix} 4\\-2\\0 \end{pmatrix} 
	= \begin{pmatrix} 1\\2\\1 \end{pmatrix}.
\]
Der Einheitsvektor $\vec u_2$ ist
\[
	\vec u_2 = \frac{\vec {\hat u_2}}{\|\vec {\hat u_2}\|} = \frac{1}{\sqrt{6}} 
	\begin{pmatrix} 1\\2\\1 \end{pmatrix}.
\]


%\item An orthonormal basis $\{\vec u_1, \vec u_2\}$ can be obtained by dividing one vector of $\mathcal B$ by its norm and computing the coordinates of the second vector to be orthogonal to the first and of unity norm.
%The first vector is thus
%\[
%\vec u_1 = \frac{1}{\sqrt{5}} \begin{pmatrix} -2\\1\\0\end{pmatrix}.
%\]
%The vector $\vec u_2$ needs to be in $\vec U^\perp$, i.e.\ to be a linear combination of the basis vectors in $\mathcal B$ and to sastify the orthogonality condition $\langle \vec u_1, \vec u_2 \rangle$. Furthermore, it needs to have unity norm.
%We first construct a vector $\vec {\widetilde u_2}$, which satisfies the first two conditions and then we normalize it to obtain $\vec u_2$. 
%
%To satisfy the first condition, the vector $\vec {\widetilde u_2}$ has to be expressed in terms of the basis vectors as
%\[
%\vec {\widetilde u_2} = a \begin{pmatrix} -2\\1\\0\end{pmatrix} + b \begin{pmatrix} 5\\0\\1\end{pmatrix} = \begin{pmatrix} -2a+5b\\a\\b\end{pmatrix} ,
%\]
%where $a$ and $b$ are the components of $\vec {\widetilde u_2}$ with respect to the basis $\mathbb B$.
%
%The second condition is
%\begin{align*}
%\langle \vec u_1, \vec {\widetilde u_2} \rangle = \frac{1}{\sqrt{5}} (4a-10b+a) &\stackrel{!}{=} 0.
%\end{align*}
%Setting $b = 1$, we get $a=2$ and the vector is determined as
%\[
%\vec {\widetilde u_2} = 2 \begin{pmatrix} -2\\1\\0\end{pmatrix} + \begin{pmatrix} 5\\0\\1\end{pmatrix} = \begin{pmatrix} 1\\2\\1\end{pmatrix}.
%\]
%The additional condition for the orthonormality is the unity norm, i.e.\
%\[
%\vec u_2 = \frac{\vec {\widetilde u_2}}{\|\vec {\widetilde u_2}\|} = \frac{1}{\sqrt{6}}\begin{pmatrix} 1\\2\\1\end{pmatrix}.
%\]
\item
Der Abstand von $\vec v$ zu $\vec U^\perp$ berechnen wir mit der Orthogonalprojektion 
von $\vec v$ auf $\vec U^\perp$.
Da wir schon eine Orthonormalbasis von $\vec U^\perp$ haben, können wir die folgende Formel benutzen
\[
P_{\vec U^\perp}(\vec v) = \sum_{k=1}^2 \langle \vec v, \vec u_k\rangle \vec u_k.
\]
Es gilt
\begin{align*}
\langle \vec v, \vec u_1\rangle &= -\sqrt{5},\\
\langle \vec v, \vec u_2\rangle &= 2\sqrt{6},
\end{align*}
und
\[
P_{\vec U^\perp}(\vec v) = -\sqrt{5} \frac{1}{\sqrt{5}}\begin{pmatrix} -2\\1\\0 \end{pmatrix} + 2\sqrt{6} \frac{1}{\sqrt{6}}\begin{pmatrix} 1\\2\\1 \end{pmatrix} = \begin{pmatrix} 4\\3\\2 \end{pmatrix}.
\]
Der Vektor 
\[
\vec w = \vec v - \vec P_{\vec U^\perp}(\vec v) = \begin{pmatrix} -1\\-2\\5 \end{pmatrix},
\]
ist orthogonal zu $\vec U^\perp$ und daher parallel zu $\vec U$ (das kann leicht 
überprüft werden). $\vec w$ und $\vec P_{\vec U^\perp}(\vec v)$ sind eine orthogonale 
Zerlegung des Vektors $\vec v$, daher ist der Abstand zu dem Unterraum $\vec U^\perp$ 
die Länge von $\vec w$
\[
d(\vec v, \vec U^\perp) = \|\vec w\| = \sqrt{30}.
\]
\end{abc}
}
