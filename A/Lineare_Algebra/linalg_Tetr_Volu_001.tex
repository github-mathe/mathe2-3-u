\Aufgabe[e]{Volumenberechnung mittels Determinanten}{
Im $\R^3$ sei der Tetraeder gegeben, der durch die Vektoren
$$\Vek a=\begin{pmatrix}1\\1\\4\end{pmatrix},\, \Vek b=\begin{pmatrix}2\\1\\-1\end{pmatrix},\, \Vek
c=\begin{pmatrix}2\\-1\\1\end{pmatrix}$$
aufgespannt wird. 
\begin{abc}
\item Berechnen Sie unter Nutzung einer Determinante das Volumen des Tetraeders. 
\item Wie groß ist das Volumen des Tetraeders, der durch 
$$\Vek {Aa}, \, \Vek{Ab}, \Vek{Ac}\text{ mit } \Vek A
= \begin{pmatrix}1&1&0\\0&1&-1\\2&1&0\end{pmatrix}$$
aufgespannt wird?
\end{abc}


}

\Loesung{
\begin{abc}
\item Das Tetraedervolumen ergibt sich aus 
\begin{align*}
V_T(\Vek a,\, \Vek b,\, \Vek c)=& \frac 16 \left|\det(\Vek a,\, \Vek b,\, \Vek c)\right|\\
=&\frac 16 \left| \det \begin{pmatrix} 1& 2 & 2\\1 & 1 & -1\\ 4 & -1 &
1\end{pmatrix}\right|\begin{matrix}\text{ (1. Zeile+2$\cdot$ 2. Z.)}\\\\ \text{ (3. Z.+2. Z.)}\end{matrix} \\
=&\frac 16 \left| \det\begin{pmatrix} 3 & 4 & 0 \\ 1 & 1 & -1\\ 5 & 0 &0 \end{pmatrix}\right|\\
=&\frac 16 \left|-\det \begin{pmatrix} 1&1&-1\\3&4&0\\5&0&0\end{pmatrix}\right|=\frac
16\left|\det\begin{pmatrix} -1&1&1\\0&4&3\\0&0&5\end{pmatrix}\right|\\
=&\frac 16 |-20|=\frac {10}3
\end{align*}
\item Das Volumen des Tetraeders $(\Vek{Aa},\, \Vek{Ab},\, \Vek{Ac})$ ist gegeben durch 
\begin{align*}
V_T(\Vek {Aa},\Vek{Ab},\Vek{Ac})=&\frac 16\left|\det(\Vek{Aa},\, \Vek{Ab},\, \Vek{Ac})\right|=\frac
16\left|\det(\Vek A\cdot (\Vek a,\Vek b,\Vek c))\right|\\
=&\frac 16 \left|\det \Vek A\cdot \det (\Vek
a,\Vek b,\Vek c)\right|.
\end{align*}
Mit $\det \Vek A = -2+1=-1$ bleibt das Volumen des Tetraeders also erhalten: 
$$V_T(\Vek{Aa}, \Vek {Ab},\Vek {Ac})=V_T(\Vek a,\Vek b,\Vek c)=10/3.$$
\end{abc}

}

\ErgebnisC{AufglinalgTetrVolu001}
{
\textbf{a)} $V_T(\Vek a,\Vek b,\Vek c)=10/3$
}
