\Aufgabe[e]{Vektoren}{
\begin{minipage}[c]{0.55\linewidth}
Ein Gewicht sei an zwei Seilen angeh\"angt.\newline
Bestimmen Sie zeichnerisch und rechnerisch die Kr\"afte in den Seilen.\newline

\smallskip
Dazu m\"ussen Sie den Kraftvektor in zwei Komponenten zerlegen,
die in die Richtung der Seile zeigen,
da Seile nur Zugkr\"afte \"ubertragen k\"onnen.
\end{minipage}
% 
\begin{minipage}[c]{0.45\linewidth}
\begin{pspicture}(-3,-3)(3,3)
\psline[fillcolor=lightgray,fillstyle=solid,linecolor=lightgray](-3,-3)(-2.6,-3)(-2.6,3)(-3,3)(-3,-3)
\psline(-2.6,-3)(-2.6,3)
\psline[fillcolor=lightgray, fillstyle=solid, linecolor=lightgray](3,-3)(2.6,-3)(2.6,3)(3,3)(3,-3)
\psline(2.6,-3)(2.6,3)
\psarc(0,0){1.0}{150}{45}
\psline(-2.6,1.7)(0,0)(2.6,2.6)
\psline[linewidth=3pt]{->}(0,0)(0,-3)
\put(-.8,-.4){$120^\circ$}
\put(.2,-.2){$135^\circ$}
\put(.1,-1.4){$F=65$}
\end{pspicture}
\end{minipage}
}
\Loesung{
\begin{minipage}{0.55\linewidth}
Aus der Skizze kann man abmessen:
\begin{equation*}
k_1 \approx 48 \quad \text{und} \quad k_2 \approx 58\,.
\end{equation*}

Die rechnerische L\"osung kann \"uber den Sinussatz erfolgen
\begin{equation*}
\begin{array}{cl}
 & \dfrac{\sin(75^\circ)}{65}
= \dfrac{\sin(45^\circ)}{k_1}
= \dfrac{\sin(60^\circ)}{k_2} \\[3ex]
% 
\Rightarrow &
k_1 \approx 47.58 \quad \text{und} \quad k_2 \approx 58.28 \,.
\end{array}
\end{equation*}
\end{minipage}
% 
\begin{minipage}{0.45\linewidth}
\begin{pspicture}(-1,-3)(3,0)
\psline[linewidth=3pt]{->}(0,0)(0,-3)
\psline[linewidth=2pt]{->}(0,0)(2.196,-1.098)
\psline[linewidth=2pt]{->}(2.196,-1.098)(0,-3)
\put(.1,-1.4){$F=65$}
\put(1.1,-.4){$k_1\approx 48$}
\put(1.2,-2.4){$k_2\approx 58$}
\psarc(0,0){1.0}{-90}{-30}
\psarc(0,-3){1.0}{45}{90}
\put(.1,-.7){$60^\circ$}
\put(.1,-2.5){$45^\circ$}
\end{pspicture}

\end{minipage}
}

\ErgebnisC{linalg_Vekt_Krft_010}
{
$k_1 \approx 47.58 \quad \text{und} \quad k_2 \approx 58.28 \,.$
}
