\Aufgabe[e]{Determinanten}{
Gegeben sei die Matrix
$$\vec A = \begin{pmatrix} 
1 & 2 & 3\\
4 & 5 & 6\\
7 & 8 & 9\\
\end{pmatrix}.$$
Berechnen Sie die Determinante unter Verwendung von
\begin{abc}
\item Gau{\ss}-Elimination,
\item Laplace-Entwicklung,
% \item Regel von Sarrus.
\end{abc}
}
\Loesung{
\begin{abc}
\item 
Gau{\ss}-Elimination:
\begin{align*}
\begin{pmatrix} 
1 & 2 & 3\\
4 & 5 & 6\\
7 & 8 & 9\\
\end{pmatrix}
\xrightarrow{\substack{Z2'=Z2-4Z1\\Z3'=Z2-7Z1}}
\begin{pmatrix} 
1 & 2 & 3\\
0 & -3 & -6\\
0 & -6 & -12\\
\end{pmatrix}
\xrightarrow{Z3''=Z3'-2Z2'}
\begin{pmatrix} 
1 & 2 & 3\\
0 & -3 & -6\\
0 & 0 & 0\\
\end{pmatrix}
\end{align*}
Die Determinante ist das Produkt der Diagonalelemente:
$$
\operatorname{det}(\vec A) = 1 \cdot (-3) \cdot 0 = 0
$$
\item Laplace-Entwicklung:

Wir entwickeln die Determinante entlang der ersten Spalte
\begin{align*}
\operatorname{det}(\vec A) &= 1 \cdot \operatorname{det} \begin{pmatrix} 5 & 6 \\ 8 & 9\end{pmatrix}
                            - 4 \cdot \operatorname{det} \begin{pmatrix} 2 & 3 \\ 8 & 9\end{pmatrix}
                            + 7 \cdot \operatorname{det} \begin{pmatrix} 2 & 3 \\ 5 & 6\end{pmatrix}\\
&= 1 \cdot (5 \cdot 9 - 6 \cdot 8) - 4 \cdot (2\cdot 9 - 3\cdot 8) + 7 \cdot (2\cdot 6-5\cdot 3)\\
&=-3+24-21\\ &= 0.
\end{align*}
% \item Regel von Sarrus.
% 
% Zuerst erweitern wir die Matrix
% $$
% \begin{pmatrix} 
% 1 & 2 & 3 & 1 & 2\\
% 4 & 5 & 6 & 4 & 5\\
% 7 & 8 & 9 & 7 & 8
% \end{pmatrix}
% $$
% Dann berechnen wir die Produkte der Diagonalen. Die Abw\"artsdiagonalen sind rot und die Aufw\"artsdiagonalen sind blau geschrieben:
% \begin{align*}
% &\textcolor{red}{(1 \cdot 5 \cdot 9) + (2 \cdot 6 \cdot 7) + (3 \cdot 4 \cdot 8)} - \textcolor{blue}{[(7 \cdot 5 \cdot 3) + (8 \cdot 6 \cdot 1) + (9 \cdot 4 \cdot 2)]}\\ &= \textcolor{red}{45+84+96} - \textcolor{blue}{[105+48+72]} = \textcolor{red}{225} - \textcolor{blue}{225} = 0
% \end{align*}

\end{abc}
}

\ErgebnisC{Determinanten}
{
$\operatorname{det}(\vec A) = 0$
}
