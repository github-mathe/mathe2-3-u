\Aufgabe[e]{}
{
Im $\R^3$ sind die Vektoren 
$$\vec a=\begin{pmatrix}2\\2\\1\end{pmatrix},\vec b=\begin{pmatrix} 1\\1\\1\end{pmatrix}, \vec
c=\begin{pmatrix}1\\1\\-1\end{pmatrix}$$
gegeben. \\
Es ist zu \"uberpr\"ufen, ob die Vektoren $\vec a,\,\vec b,\, \vec c$ linear unabh\"angig sind. %Desweiteren ist eine \textit{m\"oglichst einfache} Basis des von den drei Vektoren aufgespannten Unterraumes zu bestimmen. 
\begin{abc}
\item Wenden Sie dazu den Gauß-Algorithmus auf die Matrix $(\vec a,\vec b,\vec c)$ an. 
\item Wenden Sie ebenso den Gauß-Algorithmus auf die Matrix $(\vec a,\vec b,\vec c)^\top$ an. 
\item Bestimmen Sie die Dimension von $\operatorname{span} \{\boldsymbol a, \boldsymbol b, \boldsymbol c \}$.
\item Zeigen Sie, dass die Zeilenvektoren der Zeilenstufenformen aus \textbf{a)} 
und \textbf{b)} zwei unterschiedliche Untervektorräume des
$\mathbb R^3$ aufspannen.
\end{abc}

}
\Loesung{
\begin{abc}
\item Dieser Gauß-Algorithmus behandelt das Gleichungssystem, das zur Definition der linearen Unabh\"angigkeit geh\"ort: \\
$\vec a,\,\vec b,\vec c$ sind linear unabh\"angig. $\,\Leftrightarrow\,$ Das Gleichungssystem $ \alpha\vec a+\beta \vec b+\gamma \vec c=\vec 0 $
hat \textbf{nur} die L\"osung $\alpha=\beta=\gamma=0.$
$$\begin{array}{rrr|r|l}
\alpha&\beta&\gamma&       \\\hline
2     &  1  &  1   & 0  &   \\
2     &  1  &  1   & 0  & -I  \\
1     &  1  & -1   & 0  & -\frac 12 \times I  \\\hline

2     &  1  &  1   & 0  &   \\
0     &  0  &  0   & 0  &     \\
0     &  1/2& -3/2 & 0  & \text{ tausche }II\leftrightarrow III\\\hline

2     &  1  &  1   & 0  &   \\
0     &  1/2& -3/2 & 0  & \\
0     &  0  &  0   & 0  &     \\
\end{array}$$
Es ist eine Variable frei w\"ahlbar, damit gibt es mehrere L\"osungen und die Vektoren sind linear abh\"angig. 
\item Dieser Gauß-Algorithmus kombiniert die einzelnen Vektoren so, dass m\"oglichst einfache Vektoren \"ubrig bleiben (Zeilenstufenform), denen man die lineare Unabh\"angigkeit/Abh\"angigkeit ansieht. Es liegt kein Gleichungssystem zu Grunde, daher entf\"allt die rechte Seite. 
$$\begin{array}{rrr|l}
\alpha&\beta&\gamma&       \\\hline
2     &  2  &  1     &   \\
1     &  1  &  1     & -\frac 12 \times I\\
1     &  1  & -1     & -\frac 12 \times I  \\\hline

2     &  2  &  1     &   \\
0     &  0  &  1/2   &   \\
0     &  0  & -3/2   & +3\times II  \\\hline

2     &  2  &  1     &   \\
0     &  0  &  1/2   &   \\
0     &  0  &  0     & \\
\end{array}$$
Auch hier verbleiben zwei Vektoren. Diese bilden bereits eine Basis des von $\vec a,\, \vec b$ und $\vec c$ aufgespannten Unterraumes $\vec U$. 

Die Zeilenvektoren des ersten Systems haben nichts mit dem Unterraum $\vec U$ zu tun und sind  nicht in ihm enthalten.\\
Die Zeilenvektoren des zweiten Systems sind immer eine Linearkombination der vorhergehenden Zeilen. Der Unterraum $\vec U$ wird also nie verlassen. 
\item Aus der Zeilenstufenform kann man ablesen, dass die Dimension von 
$\operatorname{span} \{\boldsymbol a, \boldsymbol b, \boldsymbol c \}$ gleich 2 ist.

\item Der Untervektorraum, der durch die Zeilen der Zeilenstufenform aus \textbf{a)}
aufgespannt wird ist:

$$
\boldsymbol U_1 = \operatorname {span} \left\{ \begin{pmatrix} 2\\1\\1 \end{pmatrix} \, ,
\, \begin{pmatrix} 0\\1\\-3 \end{pmatrix}\right\}.
$$

Der Untervektorraum, der durch die Zeilen der Zeilenstufenform aus \textbf{b)}
aufgespannt wird ist:

$$
\boldsymbol U _2 = \operatorname {span} \left\{ \begin{pmatrix} 2\\2\\1 \end{pmatrix} \, ,
\, \begin{pmatrix} 0\\0\\1 \end{pmatrix}\right\}.
$$

Falls die beiden Vektorräume $\boldsymbol U_1$ und $\boldsymbol U_2$ identisch, kann man 
jeden Vektor aus $\boldsymbol U_1$ in der Basis von $\boldsymbol U_2$ darstellen und umgekehr.
Wir wählen also einen Vektor $\boldsymbol v \in \boldsymbol U_2$ zum Beispiel

$$
\boldsymbol v = \begin{pmatrix} 2\\2\\1 \end{pmatrix} - \begin{pmatrix}0\\0\\1 \end{pmatrix}
= \begin{pmatrix} 2\\2\\0 \end{pmatrix}.
$$

Wir nehmen an, dass $\boldsymbol v \in \boldsymbol U_1$. Dann lässt sich $\boldsymbol v$ 
als Linearkombination der Basisvektoren von $\boldsymbol U_1$ schreiben.

$$
\lambda \begin{pmatrix} 2\\1\\1 \end{pmatrix}
+\mu \begin{pmatrix} 0\\1\\-3 \end{pmatrix}
=\begin{pmatrix} 2\\2\\ 0 \end{pmatrix}
$$

An den ersten und zweiten Komponenten kann man sehen, dass $\lambda =1$ und 
$\mu = 1$ sein müssen, um die Gleichung zu erfüllen. Dies führt jedoch zu
einem Widerspruch in der dritten 
Komponente. Daher ist die Annahme, dass $ \boldsymbol v \in \boldsymbol U_1$, falsch. 
Daraus folgt

$$
\boldsymbol U_1 \neq \boldsymbol U_2.
$$

\end{abc}
}


