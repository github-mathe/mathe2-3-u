\Aufgabe[e]{Basis des $\R^5$}{
Im $\R^5$ sind die Vektoren
$$
\boldsymbol c_1=\begin{pmatrix} 1\\2\\3\\5\\0\end{pmatrix}\,,\,\,
\boldsymbol c_2=\begin{pmatrix} 0\\3\\5\\7\\8\end{pmatrix}\,,\,\,
\boldsymbol c_3=\begin{pmatrix} 0\\0\\4\\-9\\3\end{pmatrix}
$$
gegeben.

\begin{abc}
\item Sind die Vektoren der Menge
$\mathcal C = \{ \boldsymbol c_1, \boldsymbol c_2, \boldsymbol c_3 \}$
linear unabh\"angig?

\item Erg\"anzen Sie $\mathcal C$ zu einer Basis des $\R^5$.
\end{abc}
}
\Loesung{
\begin{abc}
\item An der Stufenstruktur der drei Vektoren ($\vec c_2$ und $\vec c_3$ haben keinen $x_1$-Anteil,
$\vec c_3$ hat keinen $x_2$-Anteil.) kann man erkennen, dass die Vektoren linear unabh\"angig sind:
$$\left( \begin{array}{r}
1\\2\\3\\5\\0\end{array}\right),
\left(\begin{array}{r} \cellcolor{lightgray}0\\3\\5\\7\\8\end{array}\right),
\left(\begin{array}{r} \cellcolor{lightgray}0\\\cellcolor{lightgray}0\\4\\-9\\3\end{array}\right)
$$
\item  Unter Fortsetzung dieser Stufenstruktur kann man 
$$\vec c_4 = \left(\begin{array}{r}\cellcolor{lightgray}0\\\cellcolor{lightgray}0\\\cellcolor{lightgray}0\\1\\0\end{array}\right),\, \vec c_5=\left(\begin{array}{r}\cellcolor{lightgray}0\\\cellcolor{lightgray}0\\\cellcolor{lightgray}0\\\cellcolor{lightgray}0\\1\end{array}\right)$$
erg\"anzen, um eine Basis zu erhalten. 
\end{abc}
}

\ErgebnisC{linalg_Basis_Rn_002}
{
\textbf{Zu b)} z.B. $\boldsymbol c_4 = \boldsymbol e_4$ und
$\boldsymbol c_5 = \boldsymbol e_5$\,.
}
