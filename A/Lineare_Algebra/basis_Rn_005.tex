\Aufgabe[e]{Basiswechsel}{
Gegeben sei der Vektor $\boldsymbol v = (2,1)^T$ bezüglich der
Standardbasis des $\mathbb{R}^2$.

Bestimmen Sie die Koordinaten von $\boldsymbol v$ bezüglich der Basis


$$
\left\{ \boldsymbol a_1, \boldsymbol a_2 \right\}
= \left\{
\begin{pmatrix}
3\\
2
\end{pmatrix},
%
\begin{pmatrix}
2\\
3
\end{pmatrix}
\right\}\,.
$$


}


\Loesung{
Wir benutzen die Notation $\boldsymbol v_a = (\lambda_1, \lambda_2)^T$,
um de Vektor $\boldsymbol v$ in den Koordinaten bezüglich der Basis
$\left\{ \boldsymbol a_1, \boldsymbol a_2 \right\}$ zu bezeichnen.

%
Die Komponenten von $\boldsymbol v_a$ erfüllen die folgende Gleichung

$$
\lambda_1\, \boldsymbol a_1 + \lambda_2\, \boldsymbol a_2
= \boldsymbol v\,.
$$

Daraus folgt das lineare Gleichungssystem
$$
\begin{array}{rcl}
3 \lambda_1 + 2 \lambda_2 = 2\,,\\
2 \lambda_1 + 3 \lambda_2 = 1\,,\\
\end{array}
$$
mit der Lösung
$$
\boldsymbol v_a =
\begin{pmatrix}
\frac{4}{5}\\[1.5ex]
-\frac{1}{5}
\end{pmatrix}\,.
$$

}
