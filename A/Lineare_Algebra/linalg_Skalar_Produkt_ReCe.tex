\Aufgabe[e]{Skalarprodukt im $\C^3$ (fr\"uhrere Klausuraufgabe)}{
\begin{abc}
\item Bestimmen Sie alle normierten Vektoren des $\R^3$, die zu 
$\vec u_1=(1,1,0)^\top$ und $\vec u_2=(1,0,1)$ bez\"uglich des Standardskalarproduktes in $\R^3$ senkrecht stehen. 
\item Bestimmen Sie alle normierten Vektoren des $\C^3$, die zu
$\boldsymbol v_1 = (1, \operatorname{i}, 0)^\top$ und
$\boldsymbol v_2 = (0, \operatorname{i}, -\operatorname{i})^\top$
bezüglich des Standardskalarproduktes in $\C^3$ senkrecht stehen.
\end{abc}
}
\Loesung{
Im Folgenden bezeichnen wir mit der Abbildung $\langle \cdot, \cdot \rangle$ das
Standard-Skalarprodukt in $\R^3$ bzw. in $\C^3$ und mit der Abbildung $\| \cdot \|$ die aus dem Skalarprodukt induzierte Norm.
\begin{abc}
\item Ein L\"osungsvektor $\vec x=(x,y,z)^\top$ muss die Bedingungen
$$\skalar{\vec x,\vec u_1}=x+y=0\text{ und }\skalar{\vec x,\vec u_2}=x+z=0$$
erf\"ullen. 
Diese f\"uhren auf das Gleichungssystem: 
$$\begin{array}{rrr|r|l}
x  &  y  &  z  &   &   \\\hline
1  &  1  & 0   & 0 &   \\
1  &  0  & 1   & 0 & -I\\\hline
1  &  1  & 0   & 0 &   \\
0  & -1  & 1   & 0 & \\
\end{array}$$
Alle L\"osungen haben also die Gestalt: 
$$\vec x=\begin{pmatrix}-t\\t\\t\end{pmatrix}\qquad\text{ mit } t\in\R.$$
Da die L\"osungsvektoren normiert sein sollen, muss gelten: 
$$1=\norm{\vec x}=3t^2\,\Rightarrow\, t=\pm\frac{1}{\sqrt 3}.$$
Die einzigen beiden L\"osungen sind also 
$$\vec x_{1/2}=\pm\frac 1{\sqrt 3}\begin{pmatrix}-1\\1\\1\end{pmatrix}.$$
\item Die Bedingungen
$\langle \boldsymbol v, \boldsymbol v_1 \rangle = 0$ und
$\langle \boldsymbol v, \boldsymbol v_2 \rangle = 0$
angewandt auf den beliebig gewählten Vektor
$\boldsymbol v = (v_1, v_2, v_3)^\top \in \C^3$
ergeben
$$
v_1 - \operatorname{i} v_2 = 0\,,\quad
-\operatorname{i} v_2 + \operatorname{i} v_3 = 0\,.
$$

Setzt man $v_2 = \lambda \in \C$, so folgt
$$
v_1 = \operatorname{i} \lambda\,,\quad
v_3 = \lambda\,,\quad
\mbox{also }
\boldsymbol v = \lambda\, \begin{pmatrix} \operatorname{i}\\ 1\\ 1\end{pmatrix} 
=: \lambda\, \boldsymbol w\,.
$$

Die Vektoren $\boldsymbol z \in \C^3$,
$\boldsymbol z := \frac{1}{\| \boldsymbol v \|}\, \boldsymbol v$,
der L\"osungsmenge sollen normiert sein, also $\| \boldsymbol z \| \overset{!}{=} 1$.
Mit
$$
\| \boldsymbol v\|
= \| \lambda\, \boldsymbol w \|
= | \lambda |\, \| \boldsymbol w \|
= | \lambda |\, \sqrt{3}\,,
$$
und der (vorteilhaften) Wahl $| \lambda | \stackrel{!}{=} 1$, also
$$
\lambda = \operatorname{e}^{\operatorname{i} \varphi}\,,\quad
\varphi \in [0, 2\,\pi)\,,
$$
folgt die L\"osungsmenge als
$$
\mathbb L =
\left\{ \boldsymbol z \in \C^3 \,\middle|\,
%
\boldsymbol z =
\dfrac{1}{\sqrt{3}}\, \operatorname{e}^{\operatorname{i} \varphi}\, 
\begin{pmatrix} \operatorname{i}\\ 1\\ 1\end{pmatrix}\,,\quad
\varphi \in [0, 2\,\pi) \right\}\,.
$$
\end{abc}
}

%\ErgebnisC{linalg_Skalar_Produkt_Komplex_001-2010-M1-KL1}
%{
%
%}
