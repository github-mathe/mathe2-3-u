\Aufgabe[e]{Lineare Gleichungssystem}{
Gegeben sei das lineare Gleichungssystem 

	\[
		\begin{pmatrix}
	    ~~2x_1 &  - & x_2  & + & ax_3 \\
	  	~-4x_1 &  &  & + & (1-2a)x_3  \\
	    -2x_1 & - & x_2 & - & 2x_3 \\
		\end{pmatrix} = \begin{pmatrix} 2b \\ -4b-4 \\ -6 \end{pmatrix}, \, \ \ a\in \mathbb R \ .
	\]  
	
\begin{abc}
\item Bringen Sie das Gleichungssystem in die Zeilenstufenform.
\item Bestimmen Sie in Abh\"angigkeit von $a$ den Rang der Systemmatrix.
\item Bestimmen Sie, f\"ur welche Werte von $a$ und $b$ das Gleichungssystem
\begin{iii}
	\item eine eindeutige L\"osung hat,
	\item unendlich viele L\"osungen hat,
	\item keine L\"osung hat.
\end{iii} 
\item Geben Sie die L\"osungsmenge f\"ur $a = 1$ und $b=-1$ an.
\item Geben Sie die L\"osungsmenge f\"ur $a = 3$ und $b=1$ an.
\item Bestimmen Sie das Bild der Systemmatrix f\"ur $a=3$.
\item Bestimmen Sie den Kern der Systemmatrix f\"ur $a=3$.
\end{abc}
}

\Loesung{
\begin{abc}
	\item ~\\ % a) Zeilenstufenform

		\begin{center}
			\begin{tabular}{|l|rrc|c|l|} \hline
	 			I    & 	2   & --1    &   $a$    &  2$b$&  \\
	 			II   & 	--4 &    0   & 1--2$a$& --4$b$--4 & \\
	 			III  & --2  &--1	 &  --2   & --6 & \\ \hline
	 			I    & 	--2 &  --1   & --2    & --6  &  \\
	 			II'  & 	0   &  --2   &  1     & --4   & II' = II + 2I \\
	 			III' &	0   &  --2   & $a$--2 & 2$b$--6& III' = III + I\\ \hline
	 			I    & 	2   & --1    &  $a$   & 2$b$   &  \\
	 			II'  & 	0   &  --2   &  1     & --4   & \\
	 			III''&	0   & 0      & $a$--3 & 2$b$--2    & III'' = III' - II'\\ \hline
			\end{tabular}
		\end{center}
	%
	\item %b) Rang
	Der Rang der Systemmatrix ist 2 f\"ur $a = 3$ und 3 f\"ur $a\neq 3$.
	Der Rang der erweiterten Systemmatrix ist 2 f\"ur $a=3$ und $b=1$ und 3 f\"ur $a\neq 3$ oder $b\neq1$.
	%
	\item % c) Anzahl der L\"osungen
	\begin{iii}
			\item Das Gleichungssystem hat eine eindeutige L\"osung f\"ur $a\neq 3$.
			\item \noindent Das Gleichungssystem hat unendlich viele L\"osungen f\"ur $a=3$ und $b=1$.
			\item Das Gleichungssystem hat keine L\"osung f\"ur $a=3$ und $b\neq 1$.
	\end{iii}
	\item % d) L\"osungsmenge eindeutig
	Wir setzen $a = 1$ und $b=-1$ in die Stufenform ein und erhalten
        \begin{center}
            \begin{tabular}{|l|rrc|c|} \hline
                I    & 	2   &  --1  &  1  & --2  \\
	 			II'  & 	0   &  --2  &  1  & --4   \\
	 			III''&	0   & 0     & --2 & --4   \\ \hline
			\end{tabular}
		\end{center}
    Die L\"osungsmenge erhalten wir durch R\"uckw\"artseinsetzen. Aus der dritten Gleichung erhalten
    wir $x_3 = 2$. Aus der dritten Gleichung erhalten wir einen Freiheitsgrad.
    Wir w\"ahlen $x_2 = 3$. Daraus ergibt sich $x_1 = -\frac{1}{2}t$. Die L\"osungsmenge ist dann
    $$
    \mathcal{L} = \left\{\vec x \in \mathbb{R}^3 \mid \vec x = \begin{pmatrix} -\frac{1}{2}\\3\\2 \end{pmatrix} \right\}.
    $$
	\item % e) L\"osung unendlich viele
		Wir setzen  $a = 3$  und $b=1$in die Stufenform ein und erhalten
        \begin{center}
            \begin{tabular}{|l|rrc|c|} \hline
                I    & 	2   &   --1 &   3 & 2  \\
	 			II'  & 	0   &  --2  &   1 & --4   \\
	 			III''&	0   &   0   &   0 & 0     \\ \hline
			\end{tabular}
		\end{center}
    Die L\"osungsmenge erhalten wir durch R\"uckw\"artseinsetzen. Aus der dritten Gleichung erhalten
    wir $x_3 = t$. Aus der zweiten Gleichung erhalten wir $x_2 = 2+\frac{t}{2}$ und schlie\ss lich aus der ersten
    Gleichung $x_1 = 2-\frac{5}{4}t$. Damit ist die L\"osungsmenge gegeben durch:
    $$
    \mathcal{L} = \left\{\vec x \in \mathbb{R}^3 \mid \vec x = \begin{pmatrix}2\\2\\0 \end{pmatrix} + t \begin{pmatrix} -\frac{5}{4}\\\frac{1}{2}\\1 \end{pmatrix}, t\in\mathbb{R}\right\}.
	$$
	\item % f) Bild
	Um das Bild der Systemmatrix, transponieren wir die Matrix. Das Bild kann bestimmt werden, indem wir die Zeilenstufenform
	von $A^T$ bestimmen. 
        \begin{center}
            \begin{tabular}{|l|rrc|c|} \hline
                I    & 	2   &   --4 & --2 &   \\
	 			II   &--1   &   0   & --1 & II'=2II+I  \\
	 			III  &  3   &  --5  & --2 & III'=2III-3I  \\ \hline
                I    & 	2   &   --4 & --2 &   \\
	 			II'  & 	0   &   --2 & --2 &   \\
	 			III' &  0   &   1   &  1  &   \\ \hline
                I    & 	2   &   --4 & --2 &   \\
	 			II'  & 	0   &   --2 & --2 &   \\
	 			III''&  0   &   0   &    &   \\ \hline
			\end{tabular}
		\end{center}
    Damit ergeben die Nicht-Nullzeilen von $A^T$ das Bild von $A$.
    $$
    \operatorname{Bild} = \operatorname{span}\left\{ \begin{pmatrix} 2\\-4\\-2\end{pmatrix}, \begin{pmatrix} 0\\-2\\-2\end{pmatrix} \right\}.
	$$
	\item % g) Kern
	Wir setzen $a=-1$ ein. Um den Kern zu bestimmen, l\"ost man das homogene System $\vec A \vec x = \vec 0$.
        \begin{center}
            \begin{tabular}{|l|rrc|c|} \hline
                I    & 	2   &  --1  &  3  &  0  \\
	 			II'  & 	0   &  --2  &  1  &  0   \\
	 			III''&	0   & 0     &  0  &  0     \\ \hline
			\end{tabular}
		\end{center}
	Durch R\"uckw\"artseinsetzen erhalten wir $x_3 = t$ f\"ur $t\in \mathbb R$. Aus der dritten Gleichung erhalten wir einen Freiheitsgrad und setzen
	$x_2 = \frac{1}{2}t$. Aus der ersten Gleichung erhalten wir $x_1 = -\frac{5}{4}t$.
	Daraus ergibt sich der Kern
	$$
	\operatorname{Kern} = \operatorname{span}\left\{ \begin{pmatrix}-\frac{5}{4}\\\frac{1}{2}\\1\end{pmatrix}\right\}
	$$

\end{abc}
}
