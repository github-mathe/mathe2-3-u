\Aufgabe[e]{Lineare Gleichungssysteme}
{
Finden Sie mit Hilfe des Gau\ss-Algorithmus die L\"osungsmengen der folgenden linearen Gleichungssysteme: \\
$\begin{array}{rlrl}
\text{\textbf{ a)}}&\begin{array}{rcrcrcr}
-3x_1&+&3x_2&+&x_3&=&6\\
9x_1&-& 8x_2 &+& 3x_3&=&2\\
6x_1 &+& 0x_2 &+& 3x_3 &=&15
\end{array}&
\text{\textbf{ b)}}&\begin{array}{rcrcrcr}
x_1&&&+&x_3&=&2\\
x_1&+&2x_2&+&3x_3&=&6\\
2x_1&+&2x_2&+&6x_3&=&10
\end{array}\\\\
\text{\textbf{ c)}}&\begin{array}{rcrcrcr}
x_1&&&+&x_3&=&2\\
-x_1&+&1x_2&+&3x_3&=&4\\
4x_1&+&3x_2&+&16x_3&=&26
\end{array}&
\text{\textbf{ d)}}&\begin{array}{rcrcrcr}
x_1&&&+&x_3&=&2\\
-x_1&+&1x_2&+&3x_3&=&4\\
4x_1&+&3x_2&+&16x_3&=&-1
\end{array}
\end{array}$
}
\Loesung{
\begin{abc}
\item
$$\begin{array}{rrr|r|l}
-3&3&1&6\\
9&-8&3&2&+3\times \text{ 1. Gleichung}\\
6&0&3&15& + 2\times \text{ 1. Gleichung}\\\hline
-3&3&1&6\\
0&1&6&20\\
0&6&5&27&-6\times \text{ 2. Gleichung}\\\hline
-3&3&1&6\\
0&1&6&20\\
0&0&-31&-93
\end{array}$$
R\"uckw\"artseinsetzen ergibt
$$x_3=3, \quad x_2=20-6\cdot 3=2,\quad x_1=\frac 1{-3}\left( 6-3\cdot 2+3\cdot 1\right)=1.$$
$$\mathcal{L} = \left\{ \begin{pmatrix}1\\2\\3\end{pmatrix}\right\}$$
\item 
$$\begin{array}{rrr|r|l}
1&0&1&2\\
1&2&3&6&- \text{ 1. Gleichung}\\
2&2&6&10&-2\times \text{ 1. Gleichung}\\\hline
1&0&1&2\\
0&2&2&4\\
0&2&4&6&-\text{ 2. Gleichung }\\\hline
1&0&1&2\\
0&2&2&4\\
0&0&2&2
\end{array}$$
Damit hat man 
$$x_3=1, \quad x_2=1, \quad x_1=1.$$
$$\mathcal{L} = \left\{ \begin{pmatrix}1\\1\\1 \end{pmatrix}\right\}$$
\item 
$$\begin{array}{rrr|r|l}
1&0&1&2\\
-1&1&3&4&+ \text{ 1. Gleichung}\\
4&3&16&26&-4\times\text{ 1. Gleichung}\\\hline
1&0&1&2\\
0&1&4&6\\
0&3&12&18&-3\times\text{ 2. Gleichung }\\\hline
1&0&1&2\\
0&1&4&6\\
0&0&0&0
\end{array}$$
Da eine Zeile des Gleichungssystems verschwindet, kann man $x_3=t\in\R$ beliebig w\"ahlen. Durch
R\"uckw\"artseinsetzen ergibt sich daraus f\"ur $x_2$ und $x_1$:
$$x_2=6-4t,\qquad x_1=2-t.$$
Die L\"osungsmenge ist 
$$\mathcal{L} = \left\{ \begin{pmatrix}x\\y\\z \end{pmatrix} \mid \begin{pmatrix}x\\y\\z \end{pmatrix} = \begin{pmatrix}2-t\\6-4t\\t\end{pmatrix}, t\in \R \right\}$$
\item 
$$\begin{array}{rrr|r|l}
1&0&1&2\\
-1&1&3&4&+\text{ 1. Gleichung}\\
4&3&16&-1&-4\times\text{ 1. Gleichung}\\\hline
1&0&1&2\\
0&1&4&6\\
0&3&12&-9&-3\times\text{ 2. Gleichung }\\\hline
1&0&1&2\\
0&1&4&6\\
0&0&0&-27
\end{array}$$
Die letzte Zeile des Gleichungssystems liefert die Gleichung $0=-27$. Dies ist ein Widerspruch und
die L\"osungsmenge des Gleichungssystems ist leer:
$$\mathcal{L}=\{\quad\} = \emptyset.$$
\end{abc}
}

\ErgebnisC{linalgLgsyGaus012}{
\textbf{a})$\mathcal{L} = \left\{ \begin{pmatrix}1\\2\\3\end{pmatrix}\right\}$\,
\textbf{b})$\mathcal{L} = \left\{ \begin{pmatrix}1\\1\\1 \end{pmatrix}\right\}$\\
\textbf{c})$\mathcal{L} = \left\{ \begin{pmatrix}x\\y\\z \end{pmatrix} \mid \begin{pmatrix}x\\y\\z \end{pmatrix} = \begin{pmatrix}2-t\\6-4t\\t\end{pmatrix}, t\in \R \right\}$\,
\textbf{d})$\mathcal{L}=\{\quad\} = \emptyset.$
}
