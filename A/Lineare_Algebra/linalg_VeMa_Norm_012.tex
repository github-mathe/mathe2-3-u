\Aufgabe[e]{Vektornormen}{
Gegeben sind folgende Ausdr\"ucke für
$\boldsymbol x = (x_1, x_2)^\top \in \R^2$ :
$$
\begin{array}{lcl}
\| \boldsymbol{x} \|_{a} &:=& | x_1 |\,,\\[1ex]
\| \boldsymbol{x} \|_{b} &:=& | x_1 | \cdot |x_2|\,,\\[1ex]
\| \boldsymbol{x} \|_{c} &:=& 2\, | x_1 | + \frac{2}{3}\, | x_2|\,.%\\[1ex]
% \| \boldsymbol{x} \|_{d} &:=& \sqrt{{x_1}^2 - x_1 x_2 + {x_2}^2}\,.
\end{array}
$$

\begin{abc}
\item Welche der Ausdr\"ucke definieren Normen im $\R^2$\,?\\
Begr\"unden Sie Ihre Aussage durch das \"Uberpr\"ufen der Definition einer Norm. \\
Falls es sich nicht um eine Norm handelt, geben Sie ein Beispiel an, welches zeigt, dass eine Bedingung
verletzt ist.\\
%\textbf{Hinweis}: Nehmen Sie f\"ur $\norm\cdot_d$ an, dass die Dreiecksungleichung erf\"ullt ist. 

\item Skizzieren Sie f\"ur die \textit{Normen} aus \textbf{a)} die
  Einheitskreise, d.h. die Menge
$$
\Big\{ \boldsymbol{x} \in \R^2 \,\Big|\, \| \boldsymbol x \| = 1 \Big\}\,.
$$
\end{abc}
}
\Loesung{
\textbf{a)}

%%%%%%%%%%%%%%%%%%%%%%%%%%%%%%%%%%%%%%%%%%%%%%%%%%%%%%%%%%%%%%%%%%%%%%%%%%%%%%%%
\smallskip
$\| \boldsymbol{x} \|_{a} := | x_1 |$ definiert keine Norm.
Bspw. für $\boldsymbol{u} = (0, 1)^\top$ ist
$$
\| \boldsymbol{u} \|_{a} = | 0 | = 0\,,
$$
aber $\boldsymbol{u} \neq \boldsymbol 0$. Somit ist die Bedingung der
Definitheit einer Norm verletzt.


%%%%%%%%%%%%%%%%%%%%%%%%%%%%%%%%%%%%%%%%%%%%%%%%%%%%%%%%%%%%%%%%%%%%%%%%%%%%%%%%
\bigskip
$\| \boldsymbol{x} \|_{b} := | x_1 | \cdot |x_2|$ definiert keine
Norm. Bspw. für $\boldsymbol{v} = (0, 1)^\top$ ist
$$
\| \boldsymbol{v} \|_{b} = | 0 | \cdot | 1 | = 0\,,
$$
aber $\boldsymbol{v} \neq \boldsymbol 0$. Somit ist die Bedingung der
Definitheit einer Norm verletzt.


%%%%%%%%%%%%%%%%%%%%%%%%%%%%%%%%%%%%%%%%%%%%%%%%%%%%%%%%%%%%%%%%%%%%%%%%%%%%%%%%
\bigskip
$\| \boldsymbol{x} \|_{c} := 2\, | x_1 | + \frac{2}{3}\, | x_2|$
definiert eine verallgemeinerte Betragssummennorm im $\R^2$\,.
% 
Das Nachrechnen der vier Bedingungen ergibt:
\begin{iii}
\item $\| \boldsymbol x \|_c = 2\, |x_1| + \frac{2}{3}\, |x_2| \geq 0$
ist erfüllt für alle $\boldsymbol x = (x_1, x_2)^\top \in \R^2$\,.

\item Für $\boldsymbol a = (a_1, a_2)^\top \in \R^2$ ist
$\| \boldsymbol a \|_c = 2\, |a_1| + \frac{2}{3}\, |a_2| \stackrel{!}{=} 0$
genau dann, wenn $|a_1| = |a_2| = 0$ und somit $\boldsymbol a = (0, 0)^\top$\,.

\item Für $\alpha \in \R$ gilt $\| \alpha\, \boldsymbol{x} \|_c
= 2\, | \alpha\, x_1 | + \frac{2}{3}\, | \alpha\, x_2 |
= | \alpha | \cdot (2\, |x_1| + \frac{2}{3}\, |x_2|)
= | \alpha | \cdot \| \boldsymbol x \|_c$\,.

\item Für die Dreiecksungleichung sei $\boldsymbol x = (x_1, x_2)^\top$ und
$\boldsymbol y = (y_1, y_2)^\top$, somit
$$
\begin{array}{r@{\,\,}c@{\,\,}l}
\| \boldsymbol{x} + \boldsymbol{y} \|_c
&=& 2\, | x_1 + y_1 | + \frac{2}{3}\, | x_2 + y_2 |\\[1ex]
% 
~ & \leq & 2\, | x_1 | + \frac{2}{3}\, | x_2 | + 2\, | y_1 | + \frac{2}{3}\, | y_2 |\\[1ex]
% 
~ &=& \| \boldsymbol x \|_c + \| \boldsymbol y \|_c\,.
\end{array}
$$
\end{iii}


%%%%%%%%%%%%%%%%%%%%%%%%%%%%%%%%%%%%%%%%%%%%%%%%%%%%%%%%%%%%%%%%%%%%%%%%%%%%%%%%
% \bigskip
% Die Bedingungen f\"ur $\norm\cdot_d$ sind im einzelnen: 
% \begin{iii}
% \item $ \| \boldsymbol x \|_d =
% \sqrt{ \left( x_1 - \dfrac{1}{2}\, x_2 \right)^2 + \dfrac{3}{4}\, x_2^2 }\, \geq 0 $
% f\"ur alle $\boldsymbol x \in \R^2$\,.
% 
% \item Die Bedingung $ \| \boldsymbol x \|_d = 0\,
% \Leftrightarrow \, \boldsymbol x = (0, 0)^\top$
% lässt sich überprüfen durch
% $$
% 0 \stackrel{!}{=} x_1^2 - x_1 x_2 + x_2^2
% = \left( x_1 - \frac{1}{2}\, x_2 \right)^2 + \frac{3}{4}\, x_2^2\,,
% $$
% %
% somit müssen die Gleichungen
% $\dfrac{3}{4}\, x_2^2 = 0$ und
% $\left( x_1 - \dfrac{1}{2}\, x_2 \right)^2 = 0$
% erfüllt sein, welche nur die Wahl von $x_2 = 0$ und $x_1 = 0$ zulassen.
% 
% 
% \item Für $\alpha \in \R$ ist $ \| \alpha\, \boldsymbol x \|_d
% = \sqrt{ ( \alpha\, x_1 )^2 - ( \alpha\, x_1 ) ( \alpha\, x_2 ) + ( \alpha\, x_2)^2}
% = | \alpha |\, \| \boldsymbol x \|_d$\,.
% 
% \item Mit $\vec x=(x,y)^\top$ und $\vec u=(u,v)^\top$ soll gelten: 
% $$\norm{\vec x + \vec u}_d\leq \norm{\vec x}_d + \norm{\vec u}_d$$ oder gleichbedeutend:  
% \begin{align*}
% 0\leq&\left(\norm {\vec x}_d + \norm {\vec u}_d\right)^2-\norm{\vec x + \vec u}_d^2
% = \norm{\vec x}_d^2+\norm{\vec u}_d^2 + 2\norm{\vec x}_d\norm{\vec u}_d - \norm{\vec x+\vec u}_d^2\\
% =& x^2-xy+y^2+u^2-uv+v^2+2\sqrt{x^2-xy+y^2}\sqrt{u^2-uv+v^2}+ \\
% &-\left((x+u)^2-(x+u)(y+v)+(y+v)^2\right)\\
% =& x^2-xy+y^2+u^2-uv+v^2+2\sqrt{(x^2-xy+y^2)(u^2-uv+v^2)}+ \\
% &-x^2-2xu-u^2+xy+xv+uy+uv-y^2-2yv-v^2\\
% =&2\sqrt{(x^2-xy+y^2)(u^2-uv+v^2)}  -2xu+xv+uy-2yv\\
% \end{align*}
% Diese Ungleichung wiederum l\"asst sich umformen zu 
% \begin{align*}
% 2xu-xv-uy+2yv\leq& 2\sqrt{(x^2-xy+y^2)(u^2-uv+v^2)}
% \end{align*}
% Diese ist erf\"ullt, wenn die linke Seite kleiner oder gleich Null ist. Falls die linke Seite gr\"oßer als
% Null ist, ist sie \"aquivalent zu
% \begin{align*}
% &&&(2xu-xv-uy+2yv)^2\\
% &&\leq& 4(x^2-xy+y^2)(u^2-uv+v^2)\\
% \Leftrightarrow&&
% &4x^2u^2+4xu(-xv-uy+2yv)+(-xv-uy+2yv)^2\\
% &&\leq&4(x^2u^2-x^2uv+x^2v^2+(-xy+y^2)(u^2-uv+v^2))\\
% \Leftrightarrow&&&-4xu^2y+8xuyv+(-xv-uy+2yv)^2\\
% &&\leq&4(x^2v^2+(-xy+y^2)(u^2-uv+v^2))\\
% \Leftrightarrow&&&-4xu^2y+8xuyv+x^2v^2-2xv(-uy+2yv)+(-uy+2yv)^2\\
% &&\leq&4x^2v^2-4xyu^2+4xyuv-4xyv^2+4y^2(u^2-uv+v^2)\\
% \Leftrightarrow&&&4xuyv-2xv(-uy+2yv)+(-uy+2yv)^2\\
% &&\leq&3x^2v^2-4xyv^2+4y^2(u^2-uv+v^2)\\
% \Leftrightarrow&&&6xyuv+u^2y^2-4y^2uv+4y^2v^2\\
% &&\leq& 3x^2v^2+4y^2(u^2-uv+v^2)\\
% \Leftrightarrow&&&6xyuv\leq 3x^2v^2+3y^2u^2\\
% \Leftrightarrow&&&0\leq 3((xv)^2+(yu)^2-2(xv)(yu))=3(xv-yu)^2
% \end{align*}
% Diese Bedingung ist stets erf\"ullt, also gilt auch f\"ur $\norm\cdot_d$ die Dreiecksungleichung. 
% \end{iii}


%%%%%%%%%%%%%%%%%%%%%%%%%%%%%%%%%%%%%%%%%%%%%%%%%%%%%%%%%%%%%%%%%%%%%%%%%%%%%%%%
\bigskip
\textbf{b)} Skizzen von $\| \boldsymbol x \|_c = 1$:% und
% $\| \boldsymbol x \|_d = 1$: \\
\quad\\



\begin{pspicture}(-2,-2.5)(2,2.5)
\psgrid(-2,-2)(2,2)
\psline(-.5,0)(0,-1.5)(.5,0)(0,1.5)(-.5,0)
\put(.4,.6){\colorbox{white}{$\norm{\vec x}_c=1$}}
\end{pspicture}
% \qquad
% \begin{pspicture}(-2,-2.5)(2,2.5)
% \psgrid(-2,-2)(2,2)
% \psparametricplot{0}{360}
% {t cos 1 t cos t sin mul neg add sqrt div
% t sin 1 t cos t sin mul neg add sqrt div}
% 
% \put(.4,1.5){\colorbox{white}{$\norm{\vec x}_d=1$}}
% \end{pspicture}

}

\ErgebnisC{linalg_VeMa_Norm_012}
{
$\norm\cdot_a$ und $\norm\cdot_b$ sind keine Normen. 
% Der Nachweis der Dreiecksungleichung f\"ur $\norm\cdot_d$ ist aufwendig. 
}
