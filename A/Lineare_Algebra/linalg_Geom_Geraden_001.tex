\Aufgabe[e]{Abstand von Geraden im $\R^3$}{
Gegeben sind die Geraden
$$
\begin{array}{r@{\,}c@{\,}l}
\boldsymbol g_1 &=& \left\{
\boldsymbol x \in \R^3 \,\middle|\,
\boldsymbol x = (2, 3, 3)^\top + \lambda\, (-1, 1, 2)^\top\,,\quad
\lambda \in \R \right\}\,,\quad \text{und}\\[1ex]
% 
\boldsymbol g_2 &=& \left\{
\boldsymbol x \in \R^3 \,\middle|\,
\boldsymbol x = (3, 0, 4)^\top + \lambda\, (2, -2, 1)^\top\,,\quad
\lambda \in \R \right\}\,.
\end{array}
$$

Bestimmen Sie den Abstand zwischen den Geraden $\boldsymbol g_1$ und
$\boldsymbol g_2$\,.
}
\Loesung{
Den Abstand zwischen den beiden Geraden betimmen wir, indem wir eine Ebene $\boldsymbol E$ 
bestimmen, die parallel zu beiden Geraden ist und eine der Geraden enthält. Der Abstand 
zwischen den Geraden $\boldsymbol g_1$ und $\boldsymbol g_2$ ist gleich dem Abstand zwischen
Ebene und Gerade.\\
Ein Normalenvektor $\boldsymbol n$ der gesuchten Ebene muss senkrecht auf den
Richtungsvektoren beider Geraden stehen:
$$
\langle \boldsymbol n,\, (-1, 1, 2)^\top\rangle = 0\quad
\text{und}\quad
\langle \boldsymbol n,\, (2, -2, 1)^\top \rangle =0\,.
$$ 

Also muss gelten 
$$\begin{array}{rcrcrl|l}
-n_1 &+&  n_2 &+& 2 n_3 &=0\\
2n_1 &-& 2n_2 &+&   n_3 &=0&\text{+2$\times$ 1. Gl.}\\
\hline
% 
-n_1 &+& n_2  &+& 2 n_3 &=0\\
     & &      & & 5 n_3 &=0
\end{array}$$

Also gilt $n_3=0$.
Damit folgt aus der ersten Gleichung $n_1 = n_2$.
Dieser Wert kann frei gew\"ahlt werden, wir entscheiden uns f\"ur einen normierten 
Normalenvektor:
$$
\boldsymbol n = \left( \frac{1}{\sqrt 2}, \frac{1}{\sqrt 2}, 0 \right)^\top\,.
$$

Die Hessesche Normalform der Ebene ist damit
$$
\boldsymbol E = \left\{ \boldsymbol x \in \R^3 \,\middle|\,
% 
\frac{1}{\sqrt{2}} \langle \boldsymbol x, (1, 1, 0)^\top \rangle
= \frac{1}{\sqrt{2}}
\langle(3, 0, 4)^\top, (1, 1, 0)^\top \rangle \right\}\,.
$$

Da $\boldsymbol E$ und $\boldsymbol g_1$ parallel liegen,
ist der Abstand zwischen beiden gleich dem Abstand zwischen
$E$ und dem St\"utzvektor von $\boldsymbol g_1$:
% 
$$
d(\boldsymbol g_1, \boldsymbol E)
=\left| \left\langle \boldsymbol n,
\left(
\begin{pmatrix} 3\\ 0\\ 4\end{pmatrix}
- \begin{pmatrix} 2\\ 3\\ 3\end{pmatrix}
\right) \right\rangle \right|
=\sqrt{2}\,.
$$
}

\ErgebnisC{linalg_Geom_Geraden_001}
{
$\sqrt{2}$
}
