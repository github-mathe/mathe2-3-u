\Aufgabe[e]{Determinanten}{
Bestimmen Sie die Determinanten folgender Matrizen: 
\begin{align*}
\text{ a)}&&\boldsymbol A =& \begin{pmatrix}
7 & -2 & \;3 & \;0 & \;2 \\ 
-3 & 0 & 0 & 0 & 2 \\ 
1 & 2 & 3 & 4 & 5 \\ 
1 & 0 & 1 & 0 & 1 \\ 
-2 & 0 & 5 & 0 & 3 \end{pmatrix},\quad& 
\text{ b)}&&\boldsymbol B=&\begin{pmatrix}
-2 & 3 & -7 & 8 & \;3 \\ 
3 & -4 & 1 & 6 & 0 \\ 
0 & 0 & -3 & -2 & 9 \\ 
0 & 0 & 0 & 5 & 3 \\ 
0 & 0 & 0 & 4 & 2\end{pmatrix}\\
\text{ c)}&&\boldsymbol C = &\begin{pmatrix}
   2 & -3 & \;0 &  0 \\ 
  -1 &  4 &  0 &  0 \\ 
   4 &  1 &  2 & -1 \\ 
  -5 &  2 &  2 &  3
\end{pmatrix}\,& \text{ d)}&&
\boldsymbol D = &\begin{pmatrix}
3 & 0 & 0 & 4 \\ 
5 & 2 & 3 & 2 \\ 
2 & 5 & 7 & 7 \\ 
4 & 0 & 0 & 6
\end{pmatrix}.
\end{align*}

}
\Loesung{
Man entwickelt solange nach Spalten oder Zeilen, die möglichst viele Nullen enthalten, bis die verbleibenden Restmatrizen leicht direkt ausgerechnet werden können: 

\begin{align*}
\det \boldsymbol A =& \det \begin{pmatrix}
7 & -2 & 3 & 0 & 2 \\ 
-3 & 0 & 0 & 0 & 2 \\ 
1 & 2 & 3 & 4 & 5 \\ 
1 & 0 & 1 & 0 & 1 \\ 
-2 & 0 & 5 & 0 & 3
\end{pmatrix} 
= -4\cdot \det \begin{pmatrix}
7 & -2 & 3 & 2 \\ 
-3 & 0 & 0 & 2 \\ 
1 & 0 & 1 & 1 \\ 
-2 & 0 & 5 & 3
\end{pmatrix}\\[1ex]
=& -4\cdot 2\cdot \det \begin{pmatrix}
-3 & 0 & 2 \\ 
1 & 1 & 1 \\ 
-2 & 5 & 3
\end{pmatrix}\\[1ex]
=&-8\cdot \left[ -3\cdot \det \begin{pmatrix}
1 & 1 \\ 
5 & 3
\end{pmatrix} + 2 \cdot \det \begin{pmatrix}
1 & 1 \\
-2 & 5%
\end{pmatrix} \right] \\[1ex]
=& -8\cdot \left[ -3\cdot (-2)+2\cdot 7\right]=-160\,.
\end{align*}

% F\"ur die Matrix $\boldsymbol B$ werden passende Spalten- und Zeilenoperationen 
% angewandt:
% \begin{align*}
% \det \boldsymbol B = &\det \begin{pmatrix} 
% -2 & 3 & -7 & 8 & 3 \\ 
% 3 & -4 & 1 & 6 & 0 \\ 
% 0 & 0 & -3 & -2 & 9 \\ 
% 0 & 0 & 0 & 5 & 3 \\ 
% 0 & 0 & 0 & 4 & 2
% \end{pmatrix}\text{ ( $Z_2 \cdot 2$, $Z_5\cdot 5$)}\\
%  =& \frac 12 \cdot \frac 15 \cdot \det \begin{pmatrix}
%   -2 &   3 &  -7 &   8 &   3 \\
%    6 &  -8 &   2 &  12 &   0 \\
%    0 &   0 &  -3 &  -2 &   9 \\
%    0 &   0 &   0 &   5 &   3 \\
%    0 &   0 &   0 &  20 &  10 
% \end{pmatrix}\text{ ( $Z_2+3\cdot Z_1$, $Z_5-4\cdot Z_4$)}\\
%  =& \frac 1{10} \cdot \det \begin{pmatrix}
%   -2 &   3 &  -7 &   8 &   3 \\
%    0 &   1 & -19 &  36 &   9 \\
%    0 &   0 &  -3 &  -2 &   9 \\
%    0 &   0 &   0 &   5 &   3 \\
%    0 &   0 &   0 &   0 &  -2 
% \end{pmatrix}\\
% =&\frac 1{10}\cdot (-2)\cdot 1 \cdot (-3)\cdot 5 \cdot (-2) = -6
% \end{align*}
Die Matrix $\boldsymbol B$ ist eine Block-Dreiecksmatrix:
\begin{align*} \det(\boldsymbol B) &= \begin{pmatrix}
-2 & 3 & -7 & 8 & \;3 \\ 
3 & -4 & 1 & 6 & 0 \\ 
0 & 0 & -3 & -2 & 9 \\ 
0 & 0 & 0 & 5 & 3 \\ 
0 & 0 & 0 & 4 & 2\end{pmatrix}
= \det \begin{pmatrix}-2 & 3\\ 3 & -4\end{pmatrix} \cdot \det(-3) \cdot
\begin{pmatrix} 5 & 3 \\ 4 & 2\end{pmatrix}\\[2ex]
& = -1 \cdot (-3)\cdot (-2) = -6\,.
\end{align*}

Die Matrix $\boldsymbol C$ ist ebenfalls eine Block-Dreiecksmatrix:
$$\det(\boldsymbol C) = \det \begin{pmatrix} 2 & -3 \\ -1 & 4 \end{pmatrix} \cdot
\det \begin{pmatrix} 2 & -1 \\ 2 & 3 \end{pmatrix} = 5 \cdot 8 = 40\,.$$

% Ebenso werden die Matrizen $\boldsymbol C$ und $\boldsymbol D$  behandelt:
% \begin{align*}
% \det \boldsymbol C =& \det\begin{pmatrix}
%    2 &  -3 &   0 &   0 \\
%   -1 &   4 &   0 &   0 \\
%    4 &   1 &   2 &  -1 \\
%   -5 &   2 &   2 &   3 
% \end{pmatrix} (S_3+2\cdot S_4)\\
% =& \det\begin{pmatrix}
%    2 &  -3 &   0 &   0 \\
%   -1 &   4 &   0 &   0 \\
%    4 &   1 &   0 &  -1 \\
%   -5 &   2 &   8 &   3 
% \end{pmatrix} \text{(tausche $S_3$ und $S_4$)}\\
% =& -  \det\begin{pmatrix}
%    2 &  -3 &   0 &   0 \\
%   -1 &   4 &   0 &   0 \\
%    4 &   1 &  -1 &   0 \\
%   -5 &   2 &   3 &   8 
% \end{pmatrix} (Z_1+2\cdot Z_2)\\
% =& -  \det\begin{pmatrix}
%    0 &   5 &   0 &   0 \\
%   -1 &   4 &   0 &   0 \\
%    4 &   1 &  -1 &   0 \\
%   -5 &   2 &   3 &   8 
% \end{pmatrix} \text{(tausche $S_1$ und $S_2$)}\\
% =& \det\begin{pmatrix}
%    5 &   0 &   0 &   0 \\
%    4 &  -1 &   0 &   0 \\
%    1 &   4 &  -1 &   0 \\
%    2 &  -5 &   3 &   8 
% \end{pmatrix} \\
% =&5\cdot (-1)\cdot (-1)\cdot 8=40
% \end{align*}

Durch passende Zeilen- und Spaltenvertauschungen kann die Matrix $\boldsymbol D$
auf Blockdreiecksstruktur gebracht werden.
\begin{align*}
\det \boldsymbol D =&\det\begin{pmatrix}
   3 &   0 &   0 &   4 \\
   5 &   2 &   3 &   2 \\
   2 &   5 &   7 &   7 \\
   4 &   0 &   0 &   6 
\end{pmatrix} \text{(tausche $Z_1$ und $Z_3$)}\\
=& - \det\begin{pmatrix}
   2 &   5 &   7 &   7 \\
   5 &   2 &   3 &   2 \\
   3 &   0 &   0 &   4 \\
   4 &   0 &   0 &   6 
\end{pmatrix} \text{(tausche $S_1$ und $S_3$)}\\
=&   \det\begin{pmatrix}
   7 &   5 &   2 &   7 \\
   3 &   2 &   5 &   2 \\
   0 &   0 &   3 &   4 \\
   0 &   0 &   4 &   6 
\end{pmatrix} = -1 \cdot 2 = -2 \,.
% \text{($S_1 \cdot 2$ und $S_3 \cdot 3$)}\\
% =& \frac 12\cdot \frac 13  \det\begin{pmatrix}
%   14 &   5 &   6 &   7 \\
%    6 &   2 &  15 &   2 \\
%    0 &   0 &   9 &   4 \\
%    0 &   0 &  12 &   6 
% \end{pmatrix} \text{($S_1-3\cdot S_2$, $S_3 -2\cdot S_4$)}\\
% =& \frac 12\cdot \frac 13  \det\begin{pmatrix}
%   -1 &   5 &  -8 &   7 \\
%    0 &   2 &  11 &   2 \\
%    0 &   0 &   1 &   4 \\
%    0 &   0 &   0 &   6 
% \end{pmatrix} \text{($S_1-3\cdot S_2$, $S_3-2\cdot S_4$)}\\
% =&\frac 16 \cdot (-1)\cdot 2 \cdot 1\cdot 6 = -2
\end{align*}

}

\ErgebnisC{linalg_Detn_Lapl_001}
{
$\det \boldsymbol A = -160$, $\det \boldsymbol B = -6$, $\det \boldsymbol C = 40$, $\det \boldsymbol D = -2$
}
