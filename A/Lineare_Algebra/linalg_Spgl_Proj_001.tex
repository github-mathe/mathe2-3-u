\Aufgabe[e]{Spiegelung und Projektion}{
Gegeben sei der Vektor $\Vek n= (1,\, 2,\ -3)^\top$. 
\begin{abc}
\item[e] Geben Sie die Hessesche Normalform der Ebene $\Vek H$ durch den Ursprung mit Normalenvektor $\Vek n$ an. 
\item[e] Bestimmen Sie 
\begin{iii}
\item die Matrix $\Vek S\in\R^{(3,3)}$ der orthogonalen Spiegelung an $\Vek H$ und
\item die Matrix $\Vek P\in \R^{(3,3)}$ der orthogonalen Projektion auf $\Vek H$. 
\end{iii}
\textbf{Hinweis}: Die Spiegelung des Punktes $\Vek x$ ist der Punkt, der auf der Geraden durch $\Vek x$
und $\Vek P(x)$ liegt und dieselbe Entfernung  wie $\Vek x$ von der Ebene hat. 
% \item Berechnen Sie au\ss erdem f\"ur $m\in \N$: 
% $$\Vek A_m = \Vek S^m \text{ und }\Vek B_m = (\Vek S + \Vek E)^m.$$ 
\end{abc}
}

\Loesung{
\begin{abc}
\item Der Einheitsnormalenvektor zur Ebene ist: $\Vek n_0 = 1/\sqrt{14}(1,\, 2,\, -3)^\top$, damit
ist die Hessesche Normalform:
$$\Vek H = \bigl\{\Vek x\in \R^3|\, \skalar{\Vek x,\Vek n_0}=0\bigr\}.$$
\item {\textbf{ii)}} Zun\"achst wird die Matrix der Projektion ermittelt: \\
Es ist $\Vek P x = \Vek x - \skalar{\Vek x,\, \Vek n_0}\Vek n_0=(\Vek E - \Vek n_0\otimes \Vek
n_0)\Vek x$, also ist die Matrix der Projektion gegeben durch
$$\Vek P = \Vek E - \Vek n_0\otimes\Vek n_0 = \begin{pmatrix}1&0&0\\0&1&0\\0&0&1\end{pmatrix}
- \frac 1{14}\begin{pmatrix} 1 & 2 & -3\\ 2 & 4 & -6\\ -3& -6& 9\end{pmatrix}=
\frac 1{14} \begin{pmatrix} 13&-2&3\\-2&10&6\\3&6&5\end{pmatrix}.$$
{\textbf{ i)}} Die Spiegelung ist nun gegeben durch (beachte den Hinweis)
\begin{align*}
&&\Vek S \Vek x = & \Vek x + 2\cdot (\Vek P \Vek x - \Vek x)\\
\Rightarrow&& \Vek S =& \Vek E + 2 (\Vek E - \Vek n_0\otimes \Vek n_0 - \Vek E)=\Vek E - 2\Vek
n_0\otimes \Vek n_0\\
&&=&\frac 1{14}\begin{pmatrix} 12&-4&6\\-4&6&12\\6&12&-4\end{pmatrix}. 
\end{align*}
% \item F\"ur die Projektion gilt nach Definition $\Vek P^2=\Vek P$, also erst recht
% $$\Vek P^m = \Vek P$$ 
% und damit
% \begin{align*}
% \Vek A_m =& \Vek S^m = (2\Vek P - \Vek E)^m\\
% =& \sum\limits_{j=0}^m\begin{pmatrix}m\\j\end{pmatrix}2^j\Vek P^j(-1)^{m-j}\Vek E^{m-j}\\
% =& (-1)^m \Vek E^m + \sum\limits_{j=1}^m \begin{pmatrix}m\\j\end{pmatrix} 2^j(-1)^{m-j}\Vek P\\
% =& (-1)^m\Vek E + \left( \sum\limits_{j=0}^m \begin{pmatrix}m\\j\end{pmatrix} 2^j(-1)^{m-j}
% -(-1)^m\right) \Vek P \\
% =& (-1)^m\Vek E + \left( (2-1)^m-(-1)^m\right)\Vek P=\left\{\begin{array}{ll}\Vek E & \text{ f\"ur
% gerade }m\\2\Vek P - \Vek E &\text{ f\"ur ungerade }m\end{array}\right.\\
% =&\left\{\begin{array}{ll}\Vek E & \text{ f\"ur
% gerade }m\\\Vek S  &\text{ f\"ur ungerade }m\end{array}\right.
% \end{align*}
% sowie 
% \begin{align*}
% \Vek B_m =& (\Vek S + \Vek E)^m = (2\Vek P )^m = 2^m \Vek P.
% \end{align*}

\end{abc}
}


\ErgebnisC{AufglinalgSpglProj001}
{
\textbf{ a)} $\Vek H = \bigl\{\Vek x\in \R^3|\, \skalar{\Vek x,\Vek n_0}=0\bigr\}$\\
\textbf{ b)} $\Vek S = \frac 1{14}\begin{pmatrix} 12&-4&6\\-4&6&12\\6&12&-4\end{pmatrix}$, $\Vek P = \frac 1{14} \begin{pmatrix} 13&-2&3\\-2&10&6\\3&6&5\end{pmatrix}$

}
