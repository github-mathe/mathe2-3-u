\Aufgabe[e]{Eigenwerte und -vektoren}{
Es seien
$$
\boldsymbol A = \begin{pmatrix}
8 & 10\\
8 & 10
\end{pmatrix}\,,\quad
% 
\boldsymbol B = \begin{pmatrix}
 -3 & 0 & 1\\
 -6 & 0 & 2\\
-11 & 1 & 3
\end{pmatrix}\,,\quad
% 
\boldsymbol C = \begin{pmatrix}
0 & 1 & 0\\
0 & 0 & 0\\
0 & 0 & 1
\end{pmatrix}\,.
$$
\begin{abc}
\item Berechnen Sie die Eigenwerte der Matrizen.
\item Geben Sie die zu den Eigenwerten geh\"orenden Eigenvektoren an. 
\item Bestimmen Sie jeweils die algebraischen und geometrischen Vielfachheiten aller Eigenwerte.
\item Welche der Matrizen sind diagonalisierbar?
\item Multiplizieren Sie die Matrizen mit den jeweiligen gefundenen
 Eigenvektoren. 
\end{abc}
}
\Loesung{
\begin{abc}
\item Alle Eigenwerte ergeben sich als Nullstellen des Polynoms
  $\det(\boldsymbol M-\lambda \boldsymbol E)$, $\boldsymbol M=\boldsymbol A,
  \boldsymbol B,\boldsymbol C$.
\begin{align*}
0&\overset!=
\det\begin{pmatrix}8-\lambda&10\\8&10-\lambda\end{pmatrix}=\lambda^2-18\lambda+80-80\\
&\Rightarrow \lambda_1=0, \quad \lambda_2=18\\[2ex]
0&\overset!=\det\begin{pmatrix}-3-\mu&0&1\\-6&-\mu&2\\-11&1&3-\mu\end{pmatrix}\\
&=\mu(3-\mu)(3+\mu)-6-11\mu+2(3+\mu)\\
&=\mu (9-\mu^2)-9\mu=-\mu^3\\ &\Rightarrow \mu_1=\mu_2=\mu_3 = 0\\[2ex]
0&\overset!=
\det\begin{pmatrix}-\nu&1&0\\0&-\nu&0\\0&0&1-\nu\end{pmatrix}=\nu^2(1-\nu)\quad \quad\\
&\Rightarrow \nu_1=\nu_2=0,\, \nu_3=1.
\end{align*}
\item Eigenvektoren sind L\"osungen des Gleichungssystems 
$(\boldsymbol M-\lambda \boldsymbol E)v=0$: 
\begin{align*}
&&\begin{pmatrix}8&10\\8&10\end{pmatrix}\boldsymbol u_1  
&=0&\Leftarrow&&\boldsymbol u_1&=\begin{pmatrix}5\\-4\end{pmatrix}\\
&&\begin{pmatrix}-10&10\\8&-8\end{pmatrix}\boldsymbol u_2 &=0&
\Leftarrow&&\boldsymbol u_2&=\begin{pmatrix}1\\1\end{pmatrix}\\
&&\begin{pmatrix}-3&0&1\\-6&0&2\\-11&1&3\end{pmatrix}\boldsymbol v_1&=0&
\Leftarrow&&\boldsymbol v_1&=\begin{pmatrix}1\\2\\3\end{pmatrix}\\
\end{align*}
$\boldsymbol C$ hat als Eigenvektor zum Eigenwert $0$ $\boldsymbol
w_1=\begin{pmatrix}1,0,0\end{pmatrix}^\top$ und zum
Eigenwert $1$ den Eigenvektor $\boldsymbol w_3=(0,0,1)^\top$.
\item
\item $\boldsymbol A$ hat zwei verschiedene Eigenwerte und ist daher als 
zweidimensionale Matrix diagonalisierbar. \\
$\boldsymbol B$ hat nur einen linear unabh\"angigen Eigenvektor, ist also nicht
  diagonalisierbar. \\
F\"ur $\boldsymbol C$ existieren ebenfalls nur zwei linear unabh\"angige 
Eigenvektoren,  also ist auch
diese Matrix nicht diagonalisierbar. 
\item 
\begin{align*}
\boldsymbol A\boldsymbol u_1 &=0 \cdot \boldsymbol u_1, 
&\boldsymbol A\boldsymbol u_2&=18 \cdot \boldsymbol u_2\\
\boldsymbol B\boldsymbol v_1 &= 0 \cdot \boldsymbol v_1\\
\boldsymbol C\boldsymbol v_1 &= 0\cdot \boldsymbol w_1, 
&\boldsymbol C\boldsymbol w_3&=1\cdot \boldsymbol w_3
\end{align*}
\end{abc}
}

\ErgebnisC{linalg_Eign_Wert_005}
{
{ a)} $\lambda \in \{0,18\},\, \mu\in\{0\},\, \nu \in \{0,1\}$
}
