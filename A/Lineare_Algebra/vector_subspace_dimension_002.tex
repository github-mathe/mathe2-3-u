\Aufgabe[e]{Dimensionssatz f\"ur Untervektorr\"aume}{
% Verify the dimension theorem for subspaces for
Weisen Sie den Dimensionssatz nach f\"ur den Unterraum

$\boldsymbol U =
\{\, \lambda\, \boldsymbol a \,|\, \lambda \in \mathbb{R}\, \}
\subset \boldsymbol V$
und
%
$\boldsymbol W =
\{\, \mu\, \boldsymbol b \,|\, \mu \in \mathbb{R}\,\}
\subset \boldsymbol V$

mit festen  $\boldsymbol a, \boldsymbol b \in \boldsymbol V  = \mathbb{R}^n$,
$0 < n < \infty$.

}

\Loesung{
Es m\"ussen die folgenden F\"alle betrachtet werden.
%
\begin{iii}
\item Ohne Beschr\"ankung der Allgemeinheit seien
$\boldsymbol a \neq \boldsymbol 0$ und
$\boldsymbol b = \boldsymbol 0$.
%

Dann gilt $\dim \boldsymbol U = 1$,
$\boldsymbol W = \{ \boldsymbol 0 \}$, $\dim \boldsymbol W = 0$,
und daher
$$
\boldsymbol U + \boldsymbol W
= \{ \boldsymbol x \in \boldsymbol V \,|\,
\boldsymbol x \in \operatorname{span} \{ \boldsymbol u, \boldsymbol w \}\,,\,\,
\boldsymbol u \in \boldsymbol U\,,\,\,
\boldsymbol w \in \{ \boldsymbol 0 \} \}
% 
= \boldsymbol U
$$
genauso wie
$\boldsymbol U \cap \boldsymbol W = \boldsymbol U \cap \{ \boldsymbol 0 \}
= \{ \boldsymbol 0 \}$\,.
% 
Daraus folgt, dass
$\dim(\boldsymbol U + \boldsymbol W) = \dim \boldsymbol U = 1$ und
$\dim(\boldsymbol U \cap \boldsymbol W) = 0$\,.
% 
% 
Daraus folgt die Formel des Dimensionssatzes:
$$
\dim (\boldsymbol U + \boldsymbol W)
= \dim \boldsymbol U + \dim \boldsymbol W - \dim (\boldsymbol U \cap \boldsymbol W)\,.
$$

\item Die Vektoren $\boldsymbol a$ und $\boldsymbol b$ sind linear abh\"angig mit
$\boldsymbol a, \boldsymbol b \neq \boldsymbol 0$.
% 
Daher gibt es einen $\nu \in \mathbb{R} \setminus \{ 0 \}$ so, dass
$\boldsymbol b = \nu\, \boldsymbol a$.
% 
Mit $\lambda := \nu^{-1}\, \mu$,
folgt aus
$\boldsymbol u = \mu \boldsymbol a$, 
dass
$\boldsymbol u = \lambda \boldsymbol b$.
% 
Die zwei Unterr\"aume $\boldsymbol U$ und $\boldsymbol W$ sind identisch 
und daher gilt, dass
$$
\boldsymbol U
= \boldsymbol W
= \boldsymbol U + \boldsymbol W
= \boldsymbol U \cap \boldsymbol W\,,\quad
\dim \boldsymbol U = 1\,.
$$
% 
Daraus folgt der Dimensionssatz:
$$
\dim (\boldsymbol U + \boldsymbol W)
= \dim \boldsymbol U + \dim \boldsymbol W - \dim (\boldsymbol U \cap \boldsymbol W)\,.
$$

\item Die Vektoren $\boldsymbol a$ und $\boldsymbol b$ sind linear unabh\"angig.
% 

Das impliziert, dass
$\boldsymbol U + \boldsymbol W
= \{\lambda \boldsymbol a + \mu \boldsymbol b \,|\, \lambda, \mu \in \mathbb{R} \}$.
% 

Wegen der linearen Unabh\"angigkeit von $\boldsymbol a$ und $\boldsymbol b$
folgt, dass
$\dim(\boldsymbol U + \boldsymbol W) = 2$.
% 

Sei $\boldsymbol x \in \boldsymbol U \cap \boldsymbol W$.
Dann gilt, dass  $\boldsymbol x = \lambda \boldsymbol a = \mu \boldsymbol b$
f\"ur $\lambda, \mu \in \mathbb{R}$
woraus folgt, dass
$\boldsymbol 0
= \boldsymbol x - \boldsymbol x
= \lambda \boldsymbol a - \mu \boldsymbol b$.
% 

Weil $\boldsymbol a$ und $\boldsymbol b$ linear unabh\"angig sind, folgt, dass
% are linearly independent, this implies
$\mu = \lambda \stackrel{!}{=} 0$.
% 
Daher gilt
$\boldsymbol U \cap \boldsymbol W = \{ \boldsymbol 0 \}$ und
$\dim(\boldsymbol U \cap \boldsymbol W) = 0$.

Zus\"astzlich gilt 
 $\dim \boldsymbol U = 1$ und $\dim \boldsymbol W = 1$ da
$\boldsymbol a, \boldsymbol b \neq \boldsymbol 0$,
wegen der linearen Unabh\"angigkeit.
% 
% 
Daraus erhalten wir den Dimensionssatz:
$$
\dim (\boldsymbol U + \boldsymbol W)
= \dim \boldsymbol U + \dim \boldsymbol W - \dim (\boldsymbol U \cap \boldsymbol W)\,.
$$

\item F\"ur den trivialen Fall $\boldsymbol a = \boldsymbol 0$ und
$\boldsymbol b = \boldsymbol 0$ gilt der Dimensionssatz ebenfalls.
\end{iii}

}
