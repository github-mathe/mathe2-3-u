\Aufgabe{}{

Sei $A \in \mathbb R^{(3,3)}$ die Matrix:
\[ A = \begin{pmatrix}
1 & 1 & 0 \\
0 & 2 & 0 \\
0 & 0 & 1
\end{pmatrix} \]

\begin{itemize}
    \item Ist die Matrix diagonalisierbar?
    \item Man bestimme die Potenz \( A^6 \) ohne Matrixprodukte zu benutzen.
\end{itemize}
}

\Loesung{
1. Eigenwerte und Eigenvektoren:

   Die Eigenwerte einer Matrix \( A \) erhält man durch Lösen der charakteristischen Gleichung \(\det(A - \lambda I) = 0\). Für die gegebene Matrix \( A \) berechnen wir:
   \[
   \det\begin{pmatrix}
   1 - \lambda & 1 & 0 \\
   0 & 2 - \lambda & 0 \\
   0 & 0 & 1 - \lambda
   \end{pmatrix} = (1 - \lambda)\begin{vmatrix}
   2 - \lambda & 0 \\
   0 & 1 - \lambda
   \end{vmatrix}
   = (1 - \lambda)^2 (2 - \lambda)
   \]

   Die charakteristische Gleichung ist:
   \[
   (1 - \lambda)^2 (2 - \lambda) = 0
   \]

   Die Lösungen dieser Gleichung sind die Eigenwerte:
   \[
   \lambda_1 = 1, \quad \lambda_2 = 1, \quad \lambda_3 = 2
   \]

   Jetzt bestimmen wir die entsprechenden Eigenvektoren.

   - Für \(\lambda_1 = 1\), lösen wir \((A - I)v = 0\):
     \[
     \begin{pmatrix}
     0 & 1 & 0 \\
     0 & 1 & 0 \\
     0 & 0 & 0
     \end{pmatrix}
     \begin{pmatrix}
     v_1 \\
     v_2 \\
     v_3
     \end{pmatrix} = 0
     \]
     Ein möglicher Eigenvektor ist \(v_1 = \begin{pmatrix}
     1 \\
     0 \\
     0
     \end{pmatrix}\).

   - Für \(\lambda_2 = 1\), lösen wir \((A - I)v = 0\) erneut und erhalten einen zweiten Eigenvektor:
     \[
     \begin{pmatrix}
     0 & 1 & 0 \\
     0 & 1 & 0 \\
     0 & 0 & 0
     \end{pmatrix}
     \begin{pmatrix}
     v_1 \\
     v_2 \\
     v_3
     \end{pmatrix} = 0
     \]
     Ein möglicher Eigenvektor ist \(v_2 = \begin{pmatrix}
     0 \\
     0 \\
     1
     \end{pmatrix}\).

   - Für \(\lambda_3 = 2\), lösen wir \((A - 2I)v = 0\):
     \[
     \begin{pmatrix}
     -1 & 1 & 0 \\
     0 & 0 & 0 \\
     0 & 0 & -1
     \end{pmatrix}
     \begin{pmatrix}
     v_1 \\
     v_2 \\
     v_3
     \end{pmatrix} = 0
     \]
     Ein möglicher Eigenvektor ist \(v_3 = \begin{pmatrix}
     1 \\
     1 \\
     0
     \end{pmatrix}\).

   Somit ist die Matrix \( P \) aus den Eigenvektoren:
   \[
   P = \begin{pmatrix}
   1 & 1 & 0 \\
   0 & 1 & 0 \\
   0 & 0 & 1
   \end{pmatrix}
   \]

   Die Diagonalmatrix \( D \) ist:
   \[
   D = \begin{pmatrix}
   1 & 0 & 0 \\
   0 & 2 & 0 \\
   0 & 0 & 1
   \end{pmatrix}
   \]

2. **Berechnung der Potenz**:

   Mit der Diagonalisierung können wir \( A^6 \) wie folgt berechnen:
   \[ A^6 = (PDP^{-1})^6 = PD^6P^{-1} \]

   Um \( D^6 \) zu berechnen, erheben wir jeden Diagonaleintrag zur Potenz 6:
   \[
   D^6 = \begin{pmatrix}
   1^6 & 0 & 0 \\
   0 & 2^6 & 0 \\
   0 & 0 & 1^6
   \end{pmatrix} = \begin{pmatrix}
   1 & 0 & 0 \\
   0 & 64 & 0 \\
   0 & 0 & 1
   \end{pmatrix}
   \]

   Daher ist:
   \[
   A^6 = P D^6 P^{-1} = \begin{pmatrix}
   1 & 1 & 0 \\
   0 & 1 & 0 \\
   0 & 0 & 1
   \end{pmatrix}
   \begin{pmatrix}
   1 & 0 & 0 \\
   0 & 64 & 0 \\
   0 & 0 & 1
   \end{pmatrix}
   \begin{pmatrix}
   1 & -1 & 0 \\
   0 & 1 & 0 \\
   0 & 0 & 1
   \end{pmatrix}
   \]

   Durch Multiplikation erhalten wir:
   \[
   A^6 = \begin{pmatrix}
   1 & 63 & 0 \\
   0 & 64 & 0 \\
   0 & 0 & 1
   \end{pmatrix}
   \]

}
