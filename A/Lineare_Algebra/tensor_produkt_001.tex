\Aufgabe[e]{Tensorprodukt}{
Gegeben seien die Vektoren $\vec a \in \mathbb{R}^4$ und $\vec b, \vec x \in \mathbb{R}^3$.
$$
\boldsymbol a = \begin{pmatrix} -1 \\ 4 \\ 0 \\ 1 \end{pmatrix} \, , \,
\boldsymbol b = \begin{pmatrix} 3 \\ -2 \\ 1 \end{pmatrix} \, , \,
\boldsymbol x = \begin{pmatrix} 1 \\ -1 \\ -2 \end{pmatrix}.
$$
Berechnen Sie die folgenden Ausdrücke
\begin{iii}
\item $\left( \boldsymbol a \otimes \boldsymbol b \right) \boldsymbol x$
\item $\boldsymbol a \left\langle \boldsymbol b, \boldsymbol x \right\rangle$
\end{iii}
Was fällt Ihnen auf?

}
\Loesung{
\begin{iii}
\item
\begin{align*}
(\boldsymbol a \otimes \boldsymbol b ) \boldsymbol x 
&= 
\left(\begin{pmatrix} -1 \\ 4 \\ 0 \\ 1\end{pmatrix} \otimes 
\begin{pmatrix} 3 \\-2 \\ 1 \end{pmatrix} \right) 
\begin{pmatrix} 1\\-1\\-2 \end{pmatrix}\\
&=
\begin{pmatrix} -3 & 2 & -1 \\ 12 & -8 & 4 \\ 0 & 0 & 0 \\ 3 & -2 & 1 \end{pmatrix}
\begin{pmatrix} 1\\-1\\-2 \end{pmatrix}\\ 
&= 
\begin{pmatrix} -3\\12\\0\\3 \end{pmatrix}
\end{align*}
\item
\begin{align*}
\boldsymbol a \left\langle \boldsymbol b, \boldsymbol x \right\rangle &=
\begin{pmatrix} -1\\4\\0\\1 \end{pmatrix} 
\left\langle \begin{pmatrix} 3 \\ -2 \\ 1 \end{pmatrix},
\begin{pmatrix} 1\\-1 \\ -2 \end{pmatrix} \right\rangle\\
&= 
\begin{pmatrix} -1\\4\\0\\1 \end{pmatrix} 3 \\ 
&=
\begin{pmatrix} -3\\12\\0\\3 \end{pmatrix} 
\end{align*}
\end{iii}
Allgemein gilt die Identität:
$$
\left( \boldsymbol a \otimes \boldsymbol b \right) \boldsymbol x =
\boldsymbol a \left\langle \boldsymbol b, \boldsymbol x \right\rangle
$$
Da das Skalarprodukt weniger Rechenaufwand benötigt als das Tensorprodukt empfiehlt es sich
die letztere Rechnung durchzuführen.
}

\ErgebnisC{TensorProdukt}
{

}
