\Aufgabe[e]{L\"osbarkeit Linearer Gleichungssysteme}
{
F\"ur ein festes $t\in \R$ sei das lineare Gleichungssystem  
$$\begin{array}{rrrrrrrrr}
2 x_1   & -& 2 x_2 & +& 1 x_3      &-&  3    x_4 & = &1\\
-2 x_1  & -& 2t x_2& +& t x_3      &+& (4+t) x_4 & = & 0\\
-4t x_1 & -& 4 x_2 & +& (2-2t) x_3 &+& 8t    x_4 & = & 2\\
6 x_1   & -& 6 x_2 & +& (3+t) x_3  &+& (-9+t)x_4 & = & 3-t^2
\end{array}$$
gegeben. 

\begin{abc}
 \item Wenden Sie das Gau{\ss}sche Eliminationsverfahren auf das Gleichungssystem an. \\
(\textbf{ohne Rückwärtseinsetzen, Stufenform genügt!}).
 \item Bestimmen Sie in Abhängigkeit von $t$ den Rang des Gleichungssystems.  
\item Wieviele Freiheitsgrade hat das Gleichungssystem?
\item F\"ur welche Werte von $t\in \R$ ist das inhomogene 
Gleichungssystem lösbar? 
\textbf{Die L\"osung selbst ist nicht zu berechnen!}
\end{abc}
}
\Loesung{
\textbf{Zu a)} Der Gau{\ss}sche Algorithmus (ohne 
Zeilenvertauschungen!) liefert
\begin{align*}
& \left(\begin{array}{@{}rrrr|r@{}}
2 & -2 & 1 & -3 & 1\\
-2 & -2t & t & 4+t& 0\\
-4t & -4 & 2-2t & 8t& 2 \\
6 & -6 & 3+t & -9+t & 3-t^2
\end{array}\right)\\[1ex]
\rightarrow & 
\left(\begin{array}{@{}rrrr|r@{}}
2 & -2 & 1 & -3 & 1\\
0 & -2-2t & 1+t & 1+t& 1\\
0 & -4-4t & 2 & 2t& 2+2t \\
0 &  0 & t & t & -t^2
\end{array}\right)\\[1ex]
\rightarrow & 
\left(\begin{array}{@{}rrrr|r@{}}
2 & -2 & 1 & -3 & 1\\
0 & -2-2t & 1+t & 1+t& 1\\
0 &  0 & -2t & -2& 2t \\
0 &  0 & t & t & -t^2
\end{array}\right)\\[1ex]
\rightarrow & 
\left(\begin{array}{@{}rrrr|r@{}}
2 & -2 & 1 & -3 & 1\\
0 & -2-2t & 1+t & 1+t& 1\\
0 &  0 & -2t & -2& 2t \\
0 &  0 & 0 & t-1 & -t^2+t
\end{array}\right)\,.
\end{align*}

\bigskip
\textbf{Zu b)} Aus der Stufenform liest man das folgende ab:
\begin{itemize}
\item F\"ur $t=1$ verschwindet die letzte Zeile des Gleichungssystems. 
$$\left(\begin{array}{rrrr|r}
2 & -2 & 1 & -3 & 1\\
0 & -4 & 2 & 2& 1\\
0 &  0 & -2 & -2& 2 \\
0 &  0 & 0 & 0 & 0
\end{array}\right)
$$
Es verbleiben drei Zeilen,
d. h. der Rang des Gleichsungssystems ist 3. 
\item F\"ur $t=0$ verschwindet die vorletzte Stufe des Systems und die letzte Gleichung kann mit der
vorletzten eliminiert werden:
$$\left(\begin{array}{rrrr|r}
2 & -2 & 1 & -3 & 1\\
0 & -2 & 1 & 1& 1\\
0 &  0 &  0 & -2& 0 \\
0 &  0 & 0 & -1& 0
\end{array}\right) \quad\rightarrow\quad
\left(\begin{array}{rrrr|r}
2 & -2 & 1 & -3 & 1\\
0 & -2 & 1 & 1& 1\\
0 &  0 &  0 & -2& 0 \\
0 &  0 & 0 &  0& 0
\end{array}\right)
$$
 Auch dann hat das Gleichungssystem den Rang 3. 
\item F\"ur $t=-1$ verschwindet die zweite Stufe . 
$$\left(\begin{array}{rrrr|r}
2 & -2 & 1 & -3 & 1\\
0 &  0 & 0 & 0& 1\\
0 &  0 &  2 & -2& 2 \\
0 &  0 & 0 & -2& -2
\end{array}\right)\quad\rightarrow\quad 
\left(\begin{array}{rrrr|r}
2 & -2 & 1 & -3 & 1\\
0 &  0 &  2 & -2& 2 \\
0 &  0 & 0 & -2& -2\\
0 &  0 & 0 & 0& 1
\end{array}\right)
$$
Der Rang der Koeffizientenmatrix ist abermals drei. Der Rang der erweiterten Koeffizientenmatrix ist 4.
\end{itemize}
\bigskip
\textbf{Zu c)/d)} Zur L\"osbarkeit des Gleichungssystems gelten die folgenden Aussagen.
\begin{iii}
 \item F\"ur $t\not\in\{0,1,-1\}$ besitzt das Gleichungssystem nach den Gauß-Schritten
 aus \textbf{a)} vier Zeilen, es ist also eindeutig 
lösbar und hat keine Freiheitsgrade. 
\item Für $t=0$ und $t=1$  enth\"alt das Gleichungssystem nach den Gauß-Schritten aus \textbf{b)}
 noch drei Zeilen. Es ist also l\"osbar mit einem Freiheitsgrad. 
\item Für $t=-1$ enth\"alt die letzte Zeile auf der rechten Seite noch einen nicht verschwindenden
Eintrag. Die zugeh\"orige Gleichung w\"are $0=1$. Damit ist das Gleichungssystem nicht l\"osbar. 
\end{iii}

}
\ErgebnisC{linalgLgsyPara009a}{Es treten die Spezialf\"alle $t=\pm 1$ und $t=0$ auf. }
